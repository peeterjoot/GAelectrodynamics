%
% Copyright © 2017 Peeter Joot.  All Rights Reserved.
% Licenced as described in the file LICENSE under the root directory of this GIT repository.
%
\makeproblem{Properties of the wedge of three vectors.}{problem:generalizedWedge:tripleWedgeProperties}{
%\makesubproblem{}{problem:generalizedWedge:tripleWedgeProperties:a}
Show that the wedge product of three vectors is completely antisymmetric, and
%\makesubproblem{}{problem:generalizedWedge:tripleWedgeProperties:b}
show that the
wedge product of three vectors is invariant with respect to cyclic permutation.
} % problem
\makeanswer{problem:generalizedWedge:tripleWedgeProperties}{
Writing the wedge of three vectors as a grade three selection
\begin{equation}\label{eqn:generalizedWedgeProblems:20}
\Ba \wedge \Bb \wedge \Bc
=
\gpgradethree{ \Ba \Bb \Bc },
\end{equation}
and applying the vector product identity \( \Bx \By = -\By \Bx + 2 \Bx \cdot \By \), we have
\begin{equation}\label{eqn:generalizedWedgeProblems:40}
\begin{aligned}
\Ba \wedge \Bb \wedge \Bc
&=
\gpgradethree{ \Ba \Bb \Bc } \\
&=
\gpgradethree{ \lr{ -\Bb \Ba + 2 \Ba \cdot \Bb } \Bc } \\
&=
-\gpgradethree{ \Bb \Ba \Bc } \\
&=
- \Bb \wedge \Ba \wedge \Bc.
\end{aligned}
\end{equation}
Similarly
\begin{equation}\label{eqn:generalizedWedgeProblems:60}
\begin{aligned}
\Ba \wedge \Bb \wedge \Bc
&=
\gpgradethree{ \Ba \Bb \Bc } \\
&=
\gpgradethree{ \Ba \lr{ -\Bc \Bb + 2 \Bb \cdot \Bc } } \\
&=
-\gpgradethree{ \Ba \Bc \Bb } \\
&=
- \Ba \wedge \Bc \wedge \Bb.
\end{aligned}
\end{equation}
We see that any two adjactent wedge products in the wedge of three vectors may be interchanged with a corresponding sign change, a process that can be repeated until all combinations are formed.  This includes
\begin{equation}\label{eqn:generalizedWedgeProblems:80}
\begin{aligned}
\Ba \wedge \Bb \wedge \Bc
&= -\Bb \wedge \Ba \wedge \Bc  \\
&= + \Bb \wedge \Bc \wedge \Ba  \\
&= - \Bc \wedge \Bb \wedge \Ba  \\
&= \Bc \wedge \Ba \wedge \Bb  \\
&= -\Ba \wedge \Bc \wedge \Bb.
\end{aligned}
\end{equation}
%\makesubanswer{problem:generalizedWedge:tripleWedgeProperties:a}
%\makesubanswer{problem:generalizedWedge:tripleWedgeProperties:b}
} % problem
