%
% Copyright © 2016 Peeter Joot.  All Rights Reserved.
% Licenced as described in the file LICENSE under the root directory of this GIT repository.
%

\maketheorem{K-vector dot and wedge product relations.}{thm:bladeDotWedgeSymmetryIdentities:180}{
Given a k-vector \( B \) and a vector \( \Ba \), the dot and wedge products have the following commutation relationships
\boxedEquation{eqn:bladeDotWedgeSymmetryIdentities:200}{
\begin{aligned}
B \cdot \Ba  &= (-1)^{k-1} \Ba \cdot B \\
B \wedge \Ba &= (-1)^k \Ba \wedge B,
\end{aligned}
}
and can be expressed as symmetric and antisymmetric sums depending on the grade of the blade
\boxedEquation{eqn:bladeDotWedgeSymmetryIdentities:220}{
\begin{aligned}
\Ba \wedge B &= \inv{2}\lr{ \Ba B + (-1)^k B \Ba } \\
\Ba \cdot B &= \inv{2}\lr{ \Ba B - (-1)^k B \Ba }.
\end{aligned}
}
} % theorem

For example, if \( B \) and \( \Ba \) are both vectors, we recover \cref{thm:symmetricAndAntiSymmetricVectorSums:symmetricAndAnti}.  As an other example,
if \( B \) is a 2-vector, then
\begin{equation}\label{eqn:bladeDotWedgeSymmetryIdentitiesTheorem:480}
\begin{aligned}
2 ( \Ba \wedge B ) &= \Ba B + B \Ba  \\
2 ( \Ba \cdot B ) &= \Ba B - B \Ba.
\end{aligned}
\end{equation}
Observe that the dot(wedge) of two vectors is a (anti)symmetric sum of products, whereas the wedge(dot) of a vector and a bivector is an (anti)symmetric sum.

\begin{proof}
To prove \cref{thm:bladeDotWedgeSymmetryIdentities:180}, split the blade into components that intersect with and are disjoint from \( \Ba \) as follows
\begin{equation}\label{eqn:bladeDotWedgeSymmetryIdentitiesTheorem:240}
B
=
\inv{\Ba} \Bn_1 \Bn_2 \cdots \Bn_{k-1} + \Bm_1 \Bm_2 \cdots \Bm_k,
\end{equation}
where \( \Bn_i \) orthogonal to \( \Ba \) and each other, and where \( \Bm_i \) are all orthogonal.  The products of \( B \) with \( \Ba \) are
\begin{equation}\label{eqn:bladeDotWedgeSymmetryIdentitiesTheorem:340}
\begin{aligned}
\Ba B
&= \Ba \inv{\Ba} \Bn_1 \Bn_2 \cdots \Bn_{k-1} + \Ba \Bm_1 \Bm_2 \cdots \Bm_k \\
&= \Bn_1 \Bn_2 \cdots \Bn_{k-1} + \Ba \Bm_1 \Bm_2 \cdots \Bm_k,
\end{aligned}
\end{equation}
and
\begin{equation}\label{eqn:bladeDotWedgeSymmetryIdentitiesTheorem:360}
\begin{aligned}
B \Ba
&= \inv{\Ba} \Bn_1 \Bn_2 \cdots \Bn_{k-1} \Ba + \Bm_1 \Bm_2 \cdots \Bm_k \Ba \\
&= (-1)^{k-1} \Bn_1 \Bn_2 \cdots \Bn_{k-1} + (-1)^k \Ba \Bm_1 \Bm_2 \cdots \Bm_k \\
&= (-1)^k \lr{ - \Bn_1 \Bn_2 \cdots \Bn_{k-1} + \Ba \Bm_1 \Bm_2 \cdots \Bm_k },
\end{aligned}
\end{equation}
or
\begin{equation}\label{eqn:bladeDotWedgeSymmetryIdentitiesTheorem:380}
(-1)^k B \Ba
=
- \Bn_1 \Bn_2 \cdots \Bn_{k-1} + \Ba \Bm_1 \Bm_2 \cdots \Bm_k.
\end{equation}

Respective addition and subtraction of \cref{eqn:bladeDotWedgeSymmetryIdentitiesTheorem:340} and \cref{eqn:bladeDotWedgeSymmetryIdentitiesTheorem:380} gives
\begin{equation}\label{eqn:bladeDotWedgeSymmetryIdentitiesTheorem:400}
\begin{aligned}
\Ba B + (-1)^k B \Ba
&= 2 \Ba \Bm_1 \Bm_2 \cdots \Bm_k \\
&= 2 \Ba \wedge B,
\end{aligned}
\end{equation}
and
\begin{equation}\label{eqn:bladeDotWedgeSymmetryIdentitiesTheorem:420}
\begin{aligned}
\Ba B - (-1)^k B \Ba
&= 2 \Bn_1 \Bn_2 \cdots \Bn_{k-1} \\
&= 2 \Ba \cdot B,
\end{aligned}
\end{equation}
proving \cref{eqn:bladeDotWedgeSymmetryIdentities:220}.  Grade selection from \cref{eqn:bladeDotWedgeSymmetryIdentitiesTheorem:380} gives
\begin{equation}\label{eqn:bladeDotWedgeSymmetryIdentitiesTheorem:440}
\begin{aligned}
(-1)^k B \cdot \Ba
&= - \Bn_1 \Bn_2 \cdots \Bn_{k-1} \\
&= - \Ba \cdot B,
\end{aligned}
\end{equation}
and
\begin{equation}\label{eqn:bladeDotWedgeSymmetryIdentitiesTheorem:460}
\begin{aligned}
(-1)^k B \wedge \Ba
&= \Ba \Bm_1 \Bm_2 \cdots \Bm_k \\
&= \Ba \wedge B,
\end{aligned}
\end{equation}
which proves \cref{eqn:bladeDotWedgeSymmetryIdentities:200}.
\end{proof}
