%
% Copyright � 2018 Peeter Joot.  All Rights Reserved.
% Licenced as described in the file LICENSE under the root directory of this GIT repository.
%
%{
\input{../latex/blogpost.tex}
\renewcommand{\basename}{distribution}
%\renewcommand{\dirname}{notes/phy1520/}
\renewcommand{\dirname}{notes/ece1228-electromagnetic-theory/}
%\newcommand{\dateintitle}{}
%\newcommand{\keywords}{}

\input{../latex/peeter_prologue_print2.tex}

\usepackage{peeters_layout_exercise}
\usepackage{peeters_braket}
\usepackage{peeters_figures}
\usepackage{siunitx}
\usepackage{verbatim}
%\usepackage{mhchem} % \ce{}
%\usepackage{macros_bm} % \bcM
%\usepackage{macros_qed} % \qedmarker
%\usepackage{txfonts} % \ointclockwise

\beginArtNoToc

\generatetitle{XXX}
%\chapter{XXX}
%\label{chap:distribution}
% \citep{sakurai2014modern} pr X.Y
% \citep{pozar2009microwave}
% \citep{qftLectureNotes}
% \citep{doran2003gap}
% \citep{jackson1975cew}
% \citep{griffiths1999introduction}


%
% Copyright � 2016 Peeter Joot.  All Rights Reserved.
% Licenced as described in the file LICENSE under the root directory of this GIT repository.
%
\paragraph{Theorem: Distribution of inner products}
Given two blades \(C_s, A_r\) with grades subject to \(s > r > 0\), and a vector \(\Bb\), the inner product distributes according to
\begin{equation*}
C_s \cdot \lr{ \Bb \wedge A_r } = \lr{ C_s \cdot \Bb } \cdot A_r.
\end{equation*}

\paragraph{Proof:}
%
% Copyright � 2016 Peeter Joot.  Cll Rights Reserved.
% Licenced as described in the file LICENSE under the root directory of this GIT repository.
%
The proof is straightforward, but also mechanical.
Start by expanding the wedge and dot products within a grade selection operator
\begin{dmath}\label{eqn:wedgeDistributionIdentityProblemsWedgeDistributionIdentityProof:1460}
C_s \cdot \lr{ \Bb \wedge A_r }
=
\gpgrade{C_s (\Bb \wedge A_r)}{s - (r + 1)}
=
\inv{2} \gpgrade{C_s \lr{\Bb A_r + (-1)^{r} A_r \Bb} }{s - (r + 1)}.
\end{dmath}

Solving for \(C_r \Bb\) in
\begin{dmath}\label{eqn:wedgeDistributionIdentityProblemsWedgeDistributionIdentityProof:1480}
2 \Bb \cdot A_r = \Bb A_r - (-1)^{r} A_r \Bb,
\end{dmath}
we have
\begin{dmath}\label{eqn:wedgeDistributionIdentityProblemsWedgeDistributionIdentityProof:1500}
C_s \cdot \lr{ \Bb \wedge A_r }
=
\inv{2} \gpgrade{ C_s \Bb A_r + C_s \lr{ \Bb A_r - 2 \Bb \cdot A_r } }{s - (r + 1)}
=
\gpgrade{ C_s \Bb A_r }{s - (r + 1)}
-
\cancel{\gpgrade{ C_s \lr{ \Bb \cdot A_r } }{s - (r + 1)}}.
\end{dmath}

The last term above is zero since we are selecting the \(s - r - 1\) grade element of a multivector with grades \(s - r + 1\) and \(s + r - 1\), which has no terms for \(r > 0\).
Now we can expand the \(C_s \Bb\) multivector product, for
\begin{dmath}\label{eqn:wedgeDistributionIdentityProblemsWedgeDistributionIdentityProof:1520}
C_s \cdot \lr{ \Bb \wedge A_r }
=
\gpgrade{ \lr{ C_s \cdot \Bb + C_s \wedge \Bb} A_r }{s - (r + 1)}.
\end{dmath}

The latter multivector (with the wedge product factor) above has grades \(s + 1 - r\) and \(s + 1 + r\), so this selection operator finds nothing.
This leaves
\begin{dmath}\label{eqn:wedgeDistributionIdentityProblemsWedgeDistributionIdentityProof:1540}
C_s \cdot \lr{ \Bb \wedge A_r }
=
\gpgrade{
\lr{ C_s \cdot \Bb } \cdot A_r
+ \lr{ C_s \cdot \Bb } \wedge A_r
}{s - (r + 1)}.
\end{dmath}

The first dot products term has grade \(s - 1 - r\) and is selected, whereas the wedge term has grade \(s - 1 + r \ne s - r - 1\) (for \(r > 0\)), which completes the proof.
%.  \(\qedmarker\)

%}
\EndNoBibArticle
