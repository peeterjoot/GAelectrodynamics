%
% Copyright � 2018 Peeter Joot.  All Rights Reserved.
% Licenced as described in the file LICENSE under the root directory of this GIT repository.
%
\maketheorem{Line integral of the field.}{thm:amperes:280}{
The line integral of the electromagnetic field strength is
\begin{equation*}
\ointclockwise_{\partial A} d\Bx\, F
=
I \int_A dA \lr{ \ncap J - \PD{n}{F} },
\end{equation*}
where \( \PDi{n}{F} = \lr{ \ncap \cdot \spacegrad } F \).
Expressed in terms of the conventional consistent fields and sources, this multivector relationship expands to four equations, one for each grade
\begin{equation*}
\begin{aligned}
\ointclockwise_{\partial A} d\Bx \cdot \BE &=  \int_A dA\, \ncap \cdot \BM \\
\ointclockwise_{\partial A} d\Bx \cross \BH
&=
\int_A dA
\lr{
   - \ncap \cross \BJ
   + \frac{ \ncap \rho_\txtm }{\mu}
   - \PD{n}{\BH}
} \\
\ointclockwise_{\partial A} d\Bx \cross \BE &=
\int_A dA
\lr{
     \ncap \cross \BM
   + \frac{\ncap \rho}{\epsilon}
   - \PD{n}{\BE}
} \\
\ointclockwise_{\partial A} d\Bx \cdot \BH &= -\int_A dA\, \ncap \cdot \BJ.
\end{aligned}
\end{equation*}
} % theorem
