%
% Copyright � 2023 Peeter Joot.  All Rights Reserved.
% Licenced as described in the file LICENSE under the root directory of this GIT repository.
%
%{
We know from conventional electromagnetism (given no fictitious magnetic sources) that we can represent the six components of the electric and magnetic fields in terms of four scalar fields
\index{\(\phi\)}
\index{\(\BA\)}
\begin{equation}\label{eqn:mvpotentials:80}
\begin{aligned}
\BE &= -\spacegrad \phi - \PD{t}{\BA} \\
\BH &= \inv{\mu} \spacegrad \cross \BA.
\end{aligned}
\end{equation}
where
\begin{enumerate}
\item \( \phi \) is the scalar potential \si{V} (Volts), and
\item \( \BA \) is the vector potential \si{W/m} (Webers/meter).
\end{enumerate}

The conventional way of constructing these potentials makes use of the identities,
\begin{equation}\label{eqn:mvpotentials:60}
\begin{aligned}
\spacegrad \cdot \lr{ \spacegrad \cross \BA } &= 0 \\
\spacegrad \cross \lr{ \spacegrad \phi } &= 0,
\end{aligned}
\end{equation}
which are special cases of \( \spacegrad \wedge \spacegrad \wedge \chi = 0 \) (for blades \( \chi \).)
Applying those to the source free Maxwell's equations to find representations of \( \BE, \BH \) that automatically satisfy those equations.  For that conventional analysis, see \S 18-6 \citep{feynman1963flpII:MaxwellEquations}, \S 10.1 \citep{griffiths1999introduction}, or \S 6.4 \citep{jackson1975cew}.  We can also find such a potential representation using geometric algebra methods that are cross product free (\cref{problem:mvpotentials:1}.)

For Maxwell's equations with fictitious magnetic sources, it can be shown that a potential representation of the field
\index{\(\phi_m\)}
\index{\(\BF\)}
\begin{equation}\label{eqn:mvpotentials:100}
\begin{aligned}
\BH &= -\spacegrad \phi_m - \PD{t}{\BF} \\
\BE &= -\inv{\epsilon} \spacegrad \cross \BF.
\end{aligned}
\end{equation}
satisfies the source-free grades of Maxwell's equation.
Here
\begin{enumerate}
\item \( \phi_m \) is the scalar potential for (fictitious) magnetic sources \si{A} (Amperes), and
\item \( \BF \) is the vector potential for (fictitious) magnetic sources \si{C} (Coulombs).
\end{enumerate}
See \citep{balanis2005antenna}, and \citep{pozar2009microwave} for such derivations.  As with the conventional source potentials, we can also apply our geometric algebra toolbox to easily find these results (\cref{problem:mvpotentials:2}.)

In \cref{eqn:mvpotentials:80}, and \cref{eqn:mvpotentials:100} we have a mix of time partials and curls that is reminiscent of Maxwell's equation itself.  It's obvious to wonder whether there is a more coherent integrated form for the potential.  This is in fact the case.
\makelemma{Multivector potentials.}{lemma:mvpotentials:1}{
For Maxwell's equation with electric sources, the total field \( F \) expressed in terms of the potentials of \cref{eqn:mvpotentials:80} can be expressed in multivector potential form
\begin{equation}\label{eqn:mvpotentials:520}
F = \gpgrade{ \lr{ \spacegrad - \inv{c} \PD{t}{} } \lr{ -\phi + c \BA } }{1,2}.
\end{equation}
For Maxwell's equation with only fictitious magnetic sources, the total field \( F \) expressed in terms of the potentials of \cref{eqn:mvpotentials:100} can be expressed in multivector form
\begin{equation}\label{eqn:mvpotentials:540}
F = \gpgrade{ \lr{ \spacegrad - \inv{c} \PD{t}{} } I \eta \lr{ -\phi_m + c \BF } }{1,2}.
\end{equation}
} % lemma
The reader should try to verify this themselves (\cref{problem:mvpotentials:3}.)

Using superposition, we can form a multivector potential that includes all grades.
\makedefinition{Multivector potential.}{dfn:mvpotentials:1}{
We call \( A \), a multivector with all grades, the multivector potential, defining the total field as
\begin{equation}\label{eqn:mvpotentials:600}
\begin{aligned}
F
&=
\gpgrade{ \lr{ \spacegrad - \inv{c} \PD{t}{} } A }{1,2} \\
&=
\lr{ \spacegrad - \inv{c} \PD{t}{} } A
-
\gpgrade{ \lr{ \spacegrad - \inv{c} \PD{t}{} } A }{0,3}.
\end{aligned}
\end{equation}
Imposition of the constraint
\begin{equation}\label{eqn:mvpotentials:680}
\gpgrade{ \lr{ \spacegrad - \inv{c} \PD{t}{} } A }{0,3} = 0,
\end{equation}
is called the Lorentz gauge condition, and allows us to express \( F \) in terms of the potential without any grade selection filters.
} % definition
\makelemma{Conventional multivector potential.}{lemma:mvpotentials:2}{
Let
\begin{equation}\label{eqn:mvpotentials:620}
A = -\phi + c \BA + I \eta \lr{ -\phi_m + c \BF }.
\end{equation}
With \cref{dfn:mvpotentials:1}, this results in the conventional potential representation of the electric and magnetic fields
\begin{equation}\label{eqn:mvpotentials:640}
\begin{aligned}
\BE &= -\spacegrad \phi - \PD{t}{\BA} - \inv{\epsilon} \spacegrad \cross \BF \\
\BH &= -\spacegrad \phi_m - \PD{t}{\BF} + \inv{\mu} \spacegrad \cross \BA.
\end{aligned}
\end{equation}
In terms of potentials, the Lorentz gauge condition \cref{eqn:mvpotentials:680} takes the form
\begin{equation}\label{eqn:mvpotentials:660}
\begin{aligned}
0 &= \inv{c} \PD{t}{\phi} + \spacegrad \cdot (c \BA) \\
0 &= \inv{c} \PD{t}{\phi_m} + \spacegrad \cdot (c \BF).
\end{aligned}
\end{equation}
} % lemma
\begin{proof}
See \cref{problem:mvpotentials:4}.
\end{proof}
%}
