%
% Copyright © 2017 Peeter Joot.  All Rights Reserved.
% Licenced as described in the file LICENSE under the root directory of this GIT repository.
%

Ampere's law relating the line integral of the magnetic field around a loop can be used to compute the azimuthal magnetic field around a current source.
Let's compute the magnetic field using superposition in a region of space that lies between two z-axis currents of magnitude \( I_1, I_2 \) situated respectively at \( \Bp_1, \Bp_2 \), as illustrated in
\cref{fig:amperesLawBetweenTwoCurrents:amperesLawBetweenTwoCurrentsFig1}.

\imageFigure{../figures/GAelectrodynamics/amperesLawBetweenTwoCurrentsFig1}{Magnetic field between two current sources.}{fig:amperesLawBetweenTwoCurrents:amperesLawBetweenTwoCurrentsFig1}{0.3}

We will need to add azimuthal field components along the \( \phicap_1, \phicap_2 \) directions, a task that is more tractable with GA than in vector algebra using coordinates.

First consider the field surrounding a single current source, say \( I_1 \) at point \( \Bp_1 \).
Combining the two parameter Stokes' theorem integral from 
%\cref{eqn:twoparameter:280}
\cref{thm:surfaceintegral:500}
, and Maxwell's magnetostatics equations for \( \BB \) in \cref{eqn:magnetostatics:380} we have

\begin{dmath}\label{eqn:amperesLawMagnetostaticsExample:20}
\int_A d^2 \Bx \cdot (\spacegrad \wedge \BB) = \ointclockwise d\Bx \cdot \BB = \mu \int d^2 \Bx \cdot (I \BJ).
\end{dmath}

Flipping the orientation of the Stokes' integral so that we are integrating along the \( \phicap \) direction, we have

\begin{dmath}\label{eqn:amperesLawMagnetostaticsExample:40}
\ointctrclockwise d\Bx \cdot \BB
= -\mu I \int d^2 \Bx \wedge \BJ
= \mu I_\txte,
\end{dmath}
where \( I_\txte \) is the enclosed charge (in this case \( I_1 \)).
Because of symmetry, the magnetic field due to this one line charge is

\begin{dmath}\label{eqn:amperesLawMagnetostaticsExample:60}
\BB
= \frac{\mu I_\txte \phicap}{2 \pi R},
\end{dmath}
where \( R \) is the radius of the circle centred one the current, out to the point \( \Br \) where the field is observed.
Each of the current sources at points \( \Bp_k \) contributes a magnetic field

\begin{dmath}\label{eqn:amperesLawMagnetostaticsExample:80}
\BB_k(\Br)
= \frac{\mu I_k \rcap_k \Be_{12} }{2 \pi \Norm{ \Br - \Bp_k} }
= \frac{\mu I_k \lr{ \Br - \Bp_k} \Be_{12} }{2 \pi \lr{ \Br - \Bp_k}^2 }
= \frac{\mu I_k}{2\pi} \inv{ \Br - \Bp_k} \Be_{12},
\end{dmath}
where the azimuthal angle has been determined by rotating the radial unit vector counterclockwise by 90 degrees using \( \phicap_k = \rcap_k\, \Be_{12} \).
The total magnetic field bivector between the charges is

\begin{equation}\label{eqn:amperesLawMagnetostaticsExample:100}
I \BB(\Br)
= \frac{\mu}{2\pi} \sum_{k = 1}^2 I_k \inv{\Bp_k - \Br } \Be_3.
\end{equation}

%The radius of the k-th circles are \( \Norm{ \Bp_k - \Br } \).
A product including the (inverse) vector difference \( \Bp_k - \Br \) encodes both the magnitude of the field contribution at the point \( \Br \) as well as the orientation of the magnetic field bivector components in the x-z and y-z planes.
There was no need to fall back to coordinates, which could easily become cumbersome and error prone.
