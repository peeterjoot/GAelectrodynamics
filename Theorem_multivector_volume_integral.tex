%
% Copyright � 2018 Peeter Joot.  All Rights Reserved.
% Licenced as described in the file LICENSE under the root directory of this GIT repository.
%
\maketheorem{Multivector volume integral.}{thm:volumeintegral:100}{
Given an connected volume \( V \) with a volume element \( d^3 \Bx = d\Bx_1 \wedge d\Bx_2 \wedge d\Bx_3 \), and multivector functions \( F, G \), a \textit{volume integral} can be reduced to a surface integral as follows
\begin{equation*}
\int_V F d^3\Bx \lrboldpartial G
= \ointctrclockwise_{\partial V} F d^2\Bx G,
\end{equation*}
where \( \partial V \) is the boundary of the volume \( V \), and \( d^2 \Bx \) is the counterclockwise oriented area element on the boundary of the volume.
In \R{3} with \( d^3 \Bx = I dV \), \( d^2 \Bx = I \ncap dA \), this integral can be written using a scalar volume element
\begin{equation*}
\int_V dV\, F \lrboldpartial G
= \int_{\partial V} dA\, F \ncap G.
\end{equation*}
} % theorem
