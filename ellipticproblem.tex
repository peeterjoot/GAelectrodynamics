%
% Copyright © 2023 Peeter Joot.  All Rights Reserved.
% Licenced as described in the file LICENSE under the root directory of this GIT repository.
%
%{
\makeproblem{Elliptic curvilinear and reciprocal basis.}{problem:ellipticproblem:10}{
\makesubproblem{}{problem:ellipticproblem:10:a}
Show that an ellipse can be parameterized by
\begin{equation}\label{eqn:ellipticproblem:20}
   \Bx = s \Be_1 \cosh\lr{ \mu + i \theta },
\end{equation}
where \( i = \Be_{12} \), and find the values of the semi-major and semi-minor axes.
\makesubproblem{}{problem:ellipticproblem:10:b}
Determine how \( \mu \) and the eccentricity \( \epsilon = \sqrt{1 - b^2/a^2} \) are related.
\makesubproblem{}{problem:ellipticproblem:10:c}
Compute the curvilinear and reciprocal frame vectors for the parameterization \( \Bx(s, \theta) \) above, and use this to verify
\cref{eqn:curvilinearDefined:520} and \cref{eqn:curvilinearDefined:540} respectively.
\makesubproblem{}{problem:ellipticproblem:10:d}
Check using scalar grade selection that \( \Bx^i \cdot \Bx_j = {\delta^i}_j \).
Hints: Given \( \mu = \Atanh(1/2) \),
\begin{itemize}
\item \( \cosh( \mu + i \theta ) \Be_2 = \Be_2 \cosh( \mu - i \theta ) \).
\item \( \Real\lr{ \cosh( \mu - i \theta ) \sinh( \mu + i \theta ) } = 2/3 \).
\end{itemize}
} % problem
\makeanswer{problem:ellipticproblem:10}{
\makesubanswer{}{problem:ellipticproblem:10:a}
Expanding the \( \cosh \) in terms of exponentials, we find
\begin{equation}\label{eqn:ellipticproblem:40}
\begin{aligned}
\Be_1 \cosh\lr{ \mu + i \theta }
&=
\frac{\Be_1}{2} \lr{ e^{\mu + i \theta} + e^{-\mu - i\theta} } \\
&=
\Be_1 \frac{e^\mu}{2} \lr{ \cos\theta + i \sin\theta }
+
\Be_1 \frac{e^{-\mu}}{2} \lr{ \cos\theta - i \sin\theta } \\
&=
\Be_1 \frac{ e^\mu + e^{-\mu} }{2} \cos\theta
+ \Be_2 \frac{ e^\mu - e^{-\mu} }{2} \sin\theta \\
&=
\Be_1 \cosh\mu \cos\theta + \Be_2 \sinh\mu \sin\theta,
\end{aligned}
\end{equation}
so
\begin{equation}\label{eqn:ellipticproblem:60}
\Bx = s \Be_1 \cosh\lr{ \mu + i \theta } = \Be_1 a \cos\theta + \Be_2 b \sin\theta,
\end{equation}
where
\begin{equation}\label{eqn:ellipticproblem:80}
\begin{aligned}
   a &= s \cosh\mu \\
   b &= s \sinh\mu,
\end{aligned}
\end{equation}
are the semi-major and semi-minor axis values.
\makesubanswer{}{problem:ellipticproblem:10:b}
The eccentricity (squared) is
\begin{equation}\label{eqn:ellipticproblem:100}
\begin{aligned}
   \epsilon^2 
   &= 1 - \tanh^2\mu \\
   &= \frac{\cosh^2\mu - \sinh^2\mu}{\cosh^2\mu} \\
   &= \inv{\cosh^2\mu},
\end{aligned}
\end{equation}
so the eccentricity is
\begin{equation}\label{eqn:ellipticproblem:120}
   \epsilon = \inv{\cosh\mu}.
\end{equation}
\makesubanswer{}{problem:ellipticproblem:10:c}
Our curvilinear basis vectors are
\begin{equation}\label{eqn:ellipticproblem:140}
\begin{aligned}
\Bx_s &= \Be_1 \cosh\lr{ \mu + i \theta } \\
\Bx_\theta &= \Be_2 s \sinh\lr{ \mu + i \theta } \\
\end{aligned}
\end{equation}
% finish.
\makesubanswer{}{problem:ellipticproblem:10:d}
} % answer
%}
