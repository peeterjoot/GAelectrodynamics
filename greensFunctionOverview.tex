%
% Copyright � 2016 Peeter Joot.  All Rights Reserved.
% Licenced as described in the file LICENSE under the root directory of this GIT repository.
%
%{
\index{Green's function}

\subsection{Motivation.}

We will now introduce Green's functions, which provide a general method of solving many of the linear differential equations that will be encountered in electromagnetism.

\subsubsection{Time domain problems in electromagnetism}

Examples of the PDEs that we can apply Green's function techniques to include

\begin{subequations}
\label{eqn:greensFunctionOverview:200}
\begin{equation}\label{eqn:greensFunctionOverview:220}
\lr{ \spacegrad + \inv{c} \PD{t}{} } F(\Bx, t) = J(\Bx, t)
\end{equation}
\begin{equation}\label{eqn:greensFunctionOverview:162}
\lr{ \spacegrad^2 - \inv{c^2} \PDSq{t}{} } F(\Bx, t) =
\lr{ \spacegrad - \inv{c} \PD{t}{} }
\lr{ \spacegrad + \inv{c} \PD{t}{} } F(\Bx, t) =
B(\Bx, t).
\end{equation}
\end{subequations}

The reader is no doubt familiar with the wave equation (\cref{eqn:greensFunctionOverview:162}), where \( F \) is the waving function, and \( B \) is the forcing function.
Scalar and vector valued wave equations are
encountered in scalar and vector forms in conventional electromagnetism.
We will see multivector variations of the wave equation, so it should be assumed that \( F \) and \( B \) are multivector valued.

\Cref{eqn:greensFunctionOverview:220} is actually the geometric algebra form of Maxwell's equation (singular),
where \( F \) is a 1,2 multivector, and \( J \) is a multivector containing all the charge and current density contributions.
We will call the operator in \cref{eqn:greensFunctionOverview:220} the spacetime gradient\footnote{A slightly different operator is also called the spacetime gradient in STA (Space Time Algebra) \citep{doran2003gap}, which employs a non-Euclidean basis to generate a four dimensional relativistic geometric algebra.
Our spacetime gradient is related to the STA spacetime gradient by a constant factor.}.

\subsubsection{Frequency domain problems in electromagnetism.}
It will often be convient to utilize a time harmonic (frequency domain) representation of \cref{eqn:greensFunctionOverview:200}.
This can be achieved by utilizing Fourier transform pairs or with a phasor representation.

We may define Fourier transform pairs of multivector fields and sources in the conventional fashion

\index{Fourier transform}
\makedefinition{Multivector Fourier transform pairs}{dfn:greensFunctionOverview:280}{
The Fourier transform pair for a multivector valued function \( f(\Bx, t) \) will be written as
\begin{equation*}
\begin{aligned}
f(\Bx, t) &= \int f_\omega(\Bx) e^{j \omega t} d\omega \\
f_\omega(\Bx) &= \inv{2 \pi} \int f(\Bx, t) e^{-j \omega t} dt,
\end{aligned}
\end{equation*}
where \( j \) is an arbitrary scalar imaginary that commutes with all multivectors.
} % definition

In these transform pairs, the imaginary \( j \) need not be represented by any geometrical imaginary such as \( \Be_{12} \).
In particular, we need not assume that the represention of \( j \) is the
\R{3} pseudoscalar \( I \), despite the fact that \( I \) does commute with all \R{3} multivectors.
We wish to have the freedom to
assume that non-geometric real and imaginary operations can be performed without picking or leaving out any specific grade pseudoscalar components of the multivector fields or sources, so we won't impose any a-priori restrictions on the representations of \( j \).
In particular, this provides the freedom to utilize phasor (fixed frequency) representions of our multivector functions.
%Introduction of yet another imaginary quantity in a geometric algebra context where we have so many to pick it somewhat unfortunate, but it allows us to apply Fourier transform techniques without worry about the non-commutative effects that might have to be considered should we choose to use a geometric imaginary to represent the frequency dependency.
We will use the engineering convention for our
phasor representations, where assuming a complex exponential time dependence of the following form is assumed

\index{time harmonic}
\index{frequency domain}
\makedefinition{Multivector phasor representation.}{dfn:greensFunctionOverview:300}{
The phasor representation \( f(\Bx) \) of a multivector valued (real) function \( f(\Bx, t) \) is defined implicitly as
\begin{equation*}
f(\Bx, t) = \Real\lr{ f(\Bx) e^{j \omega t} },
\end{equation*}
where \( j \) is an arbitrary scalar imaginary that commutes with all multivectors.
} % definition

The complex valued multivector \( f(\Bx) \) is still generated from the real Euclidean basis for \R{3}, so
there will be
no reason to introduce complex inner products spaces into the mix.

The reader must take care when reading any literature that utilizes Fourier transforms or phasor representation, since the conventions vary.
In particular the physics representation of a phasor typically uses the opposite sign convention
\( f(\Bx, t) = \Real\lr{ f(\Bx) e^{-i \omega t }} \), which toggles the sign of all the imaginaries in derived results.

Armed with Fourier transform or phasor representations, the frequency domain representations of
\cref{eqn:greensFunctionOverview:200} are found to be

\index{Helmholtz operator}
\begin{subequations}
\label{eqn:greensFunctionOverview:320}
\begin{equation}\label{eqn:greensFunctionOverview:240}
\lr{ \spacegrad + j k } F(\Bx) = J(\Bx)
\end{equation}
\begin{equation}\label{eqn:greensFunctionOverview:260}
\lr{ \spacegrad^2 + k^2 } F(\Bx)
=
\lr{ \spacegrad - j k } \lr{ \spacegrad + j k } F(\Bx)
= B(\Bx),
\end{equation}
\end{subequations}

where \( k = \omega/c \), and any explicit frequency dependence in our transform pairs has been suppressed.
We will call these equations the first and second order Helmholtz equations respectively.
The first order equation applies a multivector differential operator to a multivector field, which must equal the multivector forcing function (the sources).

For statics problems (\( k = 0 \)), we may work with real fields and sources, dispensing with any need to take real parts.

%}
