%
% Copyright � 2016 Peeter Joot.  All Rights Reserved.
% Licenced as described in the file LICENSE under the root directory of this GIT repository.
%
%{
\index{Green's function}

\subsection{Motivation.}

We will now introduce Green's functions, which provide a general method of solving many of the linear differential equations that will be encountered in electromagnetism.

One such linear differential equation is the inhomogeneous wave equation

\begin{dmath}\label{eqn:gradientGreensFunctionEuclidean:162}
\lr{ \spacegrad^2 - \inv{c^2} \PDSq{t}{} } F(\Bx, t) = B(\Bx, t)
\end{dmath}

The time harmonic (frequency domain) representation of the wave equation can be found by assuming a that fixed frequency solution exists.
In complex notation, that means that we can assume that all sources and fields have a complex exponential time dependence of the form
\footnote{This is the engineering convention for the time dependence.
The reader must take care when reading the literature, since some authors (notably in physics) use the opposite sign convention
\( F(\Bx, t) = \Real\lr{ F(\Bx) e^{-i \omega t }} \).}

\index{time harmonic}
\index{frequency domain}
\begin{dmath}\label{eqn:gradientGreensFunctionEuclidean:60}
F(\Bx, t) = \Real\lr{ F(\Bx) e^{j \omega t} },
\end{dmath}

where \( j \) is a scalar imaginary that need not be represented by any geometrical imaginary such as \( \Be_{123}, \Be_{12}, \cdots \).
After substitution of the time harmonic representation into \cref{eqn:gradientGreensFunctionEuclidean:162}, the problem is reduced to finding a solution that is a function of space and time to one that is purely spatial

\begin{dmath}\label{eqn:gradientGreensFunctionEuclidean:159}
\lr{ \spacegrad^2 + \frac{\omega^2}{c^2} } F(\Bx) = B(\Bx).
\end{dmath}

Superposition of discrete or continuous combinations of fixed frequency solutions, once found, can be used to determine more general solutions to the original wave equation \cref{eqn:gradientGreensFunctionEuclidean:162}.

We will writing \( \omega^2/c^2 = k^2 \), to obtain the standard form of the Helmholtz equation we wish to solve

\index{Helmholtz equation}
\index{second order Helmholtz equation}
\begin{dmath}\label{eqn:gradientGreensFunctionEuclidean:160}
\lr{ \spacegrad^2 + k^2 } F(\Bx) = B(\Bx).
\end{dmath}

This is a linear differential equation that is second order with respect to the gradient.
Despite employing a complex representations of the fields and sources, our vector basis is still a real valued Euclidean basis, and we will have no reason to introduce complex inner products spaces into the mix.
We will also encounter statics problems that have no time dependence in electromagnetism.
Some of these problems have the structure of \cref{eqn:gradientGreensFunctionEuclidean:160} with \( k = 0 \), and for those problems the fields and sources are real.

\index{Helmholtz operator}
Observe that the Helmholtz operator can be factored into operators that are first order in the gradient

\begin{dmath}\label{eqn:gradientGreensFunctionEuclidean:161}
\lr{ \spacegrad - j k }\lr{ \spacegrad + j k } F(\Bx) = B(\Bx).
\end{dmath}

We will see that the time harmonic Maxwell's equation, in its GA form, is a first order equation in the gradient of the form

\index{first order Helmholtz equation}
\begin{dmath}\label{eqn:gradientGreensFunctionEuclidean:180}
\lr{ \spacegrad + j k } F(\Bx) = J(\Bx),
\end{dmath}

where \( F \) is a (complex) 1,2 multivector, and \( J \) is a (complex) multivector containing all the charge and current density contributions.
Our initial goal is to develop the Green's function toolbox that can be used to solve first and second order Helmholtz equations of the form
\cref{eqn:gradientGreensFunctionEuclidean:180} and
\cref{eqn:gradientGreensFunctionEuclidean:160} respectively.

%}
