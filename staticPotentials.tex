%
% Copyright � 2018 Peeter Joot.  All Rights Reserved.
% Licenced as described in the file LICENSE under the root directory of this GIT repository.
%
%{
\input{../latex/blogpost.tex}
\renewcommand{\basename}{staticPotentials}
%\renewcommand{\dirname}{notes/phy1520/}
\renewcommand{\dirname}{notes/ece1228-electromagnetic-theory/}
%\newcommand{\dateintitle}{}
%\newcommand{\keywords}{}

\input{../latex/peeter_prologue_print2.tex}

\usepackage{peeters_layout_exercise}
\usepackage{peeters_braket}
\usepackage{peeters_figures}
\usepackage{siunitx}
%\usepackage{mhchem} % \ce{}
%\usepackage{macros_bm} % \bcM
%\usepackage{macros_qed} % \qedmarker
%\usepackage{txfonts} % \ointclockwise

\beginArtNoToc

\generatetitle{Potential solutions to the static Maxwell's equation using geometric algebra}
%\chapter{Potential solutions to the static Maxwell's equation using geometric algebra}
%\label{chap:staticPotentials}

Maxwell's equations for static configurations where neither the electromagnetic field strength \( F = \BE + I \eta \BH \), nor current \( J = \eta (c \rho - \BJ) + I(c\rho_m - \BM) \) is a function of time, is a first order multivector (gradient) equation of the form
\begin{dmath}\label{eqn:staticPotentials:20}
\spacegrad F = J.
\end{dmath}

While direct solutions to this equations are possible with the multivector Green's function for the gradient
\begin{dmath}\label{eqn:staticPotentials:40}
G(\Bx, \Bx') = \frac{\Bx - \Bx'}{4 \pi \Norm{\Bx - \Bx'}^3 },
\end{dmath}
the aim in this post is to explore second order (potential) solutions in a geometric algebra context.  Can we assume that it is possible to find a multivector potential \( A \) for which
% maketheorem here:
\begin{dmath}\label{eqn:staticPotentials:60}
F = \spacegrad A,
\end{dmath}
is a solution to the Maxwell statics equation?  If such a solution exists, then Maxwell's equation is simply
\begin{dmath}\label{eqn:staticPotentials:80}
\spacegrad^2 A = J,
\end{dmath}
which can be easily solved using the Green's function for the Laplacian
\begin{dmath}\label{eqn:staticPotentials:n}
G(\Bx, \Bx') = -\inv{\Norm{\Bx - \Bx'} }.
\end{dmath}

It is immediately clear that some restrictions must be imposed on the potential. In particular, since the field \( F \) has only vector and bivector grades, this gradient must have no scalar, nor pseudoscalar grades.  That is
\begin{dmath}\label{eqn:staticPotentials:100}
\gpgrade{\spacegrad A}{0,3} = 0.
\end{dmath}
This constraint on the potential can be avoided if a grade selection operation is built directly into the assumed potential solution, requiring that the field is given by
\begin{dmath}\label{eqn:staticPotentials:120}
F = \gpgrade{\spacegrad A}{1,2}.
\end{dmath}
With such a constraint, Maxwell's equation has a much less friendly form
\begin{dmath}\label{eqn:staticPotentials:140}
\spacegrad^2 A - \spacegrad \gpgrade{\spacegrad A}{0,3} = J.
\end{dmath}
Luckily, it is possible to introduce a transformation of potentials, a gauge transformation, that eliminates the ugly grade selection term, and allows the potential equation to be expressed as a plain old Laplacian.  We do so by assuming first that it is possible to find a solution of the Laplacian equation that has the desired grade restrictions.  That is
\begin{dmath}\label{eqn:staticPotentials:160}
\begin{aligned}
\spacegrad^2 A' &= J \\
\gpgrade{\spacegrad A'}{0,3} &= 0,
\end{aligned}
\end{dmath}
for which \( F = \spacegrad A' \) is a grade 1,2 solution to \( \spacegrad F = J \).  Suppose that \( A \) is any formal solution, free of any grade restrictions, to \( \spacegrad^2 A = J \), and \( F = \gpgrade{\spacegrad A}{1,2} \).  Can we find a function \( \tilde{A} \) for which \( A = A' + \tilde{A} \)?

Maxwell's equation in terms of \( A \) is
\begin{dmath}\label{eqn:staticPotentials:180}
J
= \spacegrad \gpgrade{\spacegrad A}{1,2}
= \spacegrad^2 A
- \spacegrad \gpgrade{\spacegrad A}{0,3}
= \spacegrad^2 (A' + \tilde{A}
- \spacegrad \gpgrade{\spacegrad A}{0,3}
\end{dmath}
or
\begin{dmath}\label{eqn:staticPotentials:200}
\spacegrad^2 \tilde{A} = \spacegrad \gpgrade{\spacegrad A}{0,3}.
\end{dmath}
This non-homogeneous Laplacian equation that can be solved as is for \( \tilde{A} \) using the Green's function for the Laplacian.  Alternatively, we may also solve the equivalent first order system using the Green's function for the gradient.
\begin{dmath}\label{eqn:staticPotentials:220}
\spacegrad \tilde{A} = \gpgrade{\spacegrad A}{0,3}.
\end{dmath}
Clearly \( \tilde{A} \) is not unique, as we can add any function \( \psi \) satisfying the
homogeneous Laplacian equation \( \spacegrad^2 \phi = 0 \).

In summary,
if \( A \) is any multivector solution to \( \spacegrad^2 A = J \), then
\( F = \spacegrad A' \) is a solution to Maxwell's equation, where \( A' = A - \tilde{A} \), and \( \tilde{A} \) is a solution to the non-homogeneous Laplacian equation
% \cref{eqn:staticPotentials:200}
or the non-homogeneous gradient equation
% \cref{eqn:staticPotentials:220}
above.

\paragraph{Integral potential solutions.}

%}
\EndArticle
%\EndNoBibArticle
