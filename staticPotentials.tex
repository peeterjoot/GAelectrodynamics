%
% Copyright � 2018 Peeter Joot.  All Rights Reserved.
% Licenced as described in the file LICENSE under the root directory of this GIT repository.
%
%{
%\input{../latex/blogpost.tex}
%\renewcommand{\basename}{staticPotentials}
%%\renewcommand{\dirname}{notes/phy1520/}
%\renewcommand{\dirname}{notes/ece1228-electromagnetic-theory/}
%%\newcommand{\dateintitle}{}
%%\newcommand{\keywords}{}
%
%\input{../latex/peeter_prologue_print2.tex}
%
%\usepackage{peeters_layout_exercise}
%\usepackage{peeters_braket}
%\usepackage{peeters_figures}
%\usepackage{siunitx}
%%\usepackage{mhchem} % \ce{}
%%\usepackage{macros_bm} % \bcM
%%\usepackage{macros_qed} % \qedmarker
%%\usepackage{txfonts} % \ointclockwise
%
%\beginArtNoToc
%
%\generatetitle{Potential solutions to the static Maxwell's equation using geometric algebra}
%\chapter{Potential solutions to the static Maxwell's equation using geometric algebra}
\label{chap:staticPotentials}
%\subsubsection{Gauge transformations.}

Finding the potential requires solving the equation
\begin{dmath}\label{eqn:staticPotentials:140}
\spacegrad^2 A - \spacegrad \gpgrade{\spacegrad A}{0,3} = J.
\end{dmath}
It would clearly be desirable if we could find a potential for which the field is a strict gradient
\begin{dmath}\label{eqn:staticPotentials:60}
F = \spacegrad A,
\end{dmath}
since that would reduce Maxwell's equation to simply
\begin{dmath}\label{eqn:staticPotentials:80}
\spacegrad^2 A = J,
\end{dmath}
which we know how to solve.
Given the potential specified by \cref{eqn:unpackStaticPotential:20}, the 
split of the field
given by
\cref{eqn:staticPotentials:60}
into
vector+bivector grades and scalar+pseudoscalar grades is:
\begin{dmath}\label{eqn:staticPotentials:420}
\begin{aligned}
F
&= \spacegrad A \\
&=
 \spacegrad A_0 +
 \spacegrad \cdot A_2 +
 \spacegrad \wedge A_1 +
 \spacegrad \cdot A_3 \\
&\quad + \spacegrad \cdot A_1 + \spacegrad \wedge A_2.
\end{aligned}
\end{dmath}
Only the vector and bivector grades of the potential can contribute scalar and pseudoscalar grades, so if
gradient representation of the field as in \cref{eqn:staticPotentials:60} can be found,
we require the imposition of a constraint on the vector and bivector grades of a multivector potential
\begin{dmath}\label{eqn:staticPotentials:440}
0 = \spacegrad \cdot A_1 + \spacegrad \wedge A_2.
\end{dmath}
Such constraints are conspiring to make life difficult, but
luckily, a transformation of potentials (a gauge transformation) can be used to impose these constraints.
As a side effect such transformation
reduced the problem to that of a Laplacian.

\maketheorem{Potential solution of Maxwell static's equation.}{thm:staticPotentials:420}{
If \( A \) is a multivector solution to \( \spacegrad^2 A = J \),
for which \( \gpgrade{\spacegrad A}{0,3} \ne 0 \),
%, that is
%\begin{equation*}
%A(\Bx)
%= \int dV' G(\Bx, \Bx') J(\Bx')
%= -\int dV' \frac{J(\Bx')}{\Norm{\Bx - \Bx'} },
%\end{equation*}
then
\( F = \spacegrad A' \) is a solution to Maxwell's equation, where \( A' = A - \tilde{A} \), and \( \tilde{A} \)
%is a solution to the non-homogeneous Laplacian equation
%\begin{equation*}
%\spacegrad^2 \tilde{A} = \spacegrad \gpgrade{\spacegrad A}{0,3},
%\end{equation*}
of the non-homogeneous gradient equation
\begin{equation*}
\spacegrad \tilde{A} = \gpgrade{\spacegrad A}{0,3}.
\end{equation*}
} % theorem

We can prove \cref{thm:staticPotentials:420}
by assuming that it is possible to find a solution of the Laplacian equation that has the desired grade restrictions.
That is
\begin{dmath}\label{eqn:staticPotentials:160}
\begin{aligned}
\spacegrad^2 A' &= J \\
\gpgrade{\spacegrad A'}{0,3} &= 0,
\end{aligned}
\end{dmath}
for which \( F = \spacegrad A' \) is a grade 1,2 solution to \( \spacegrad F = J \).
Suppose that \( A \) is any formal solution, free of any grade restrictions, to \( \spacegrad^2 A = J \), and \( F = \gpgrade{\spacegrad A}{1,2} \).
Can we find a function \( \tilde{A} \) for which \( A = A' + \tilde{A} \)?

Maxwell's equation in terms of \( A \) is
\begin{dmath}\label{eqn:staticPotentials:180}
J
= \spacegrad \gpgrade{\spacegrad A}{1,2}
= \spacegrad^2 A
- \spacegrad \gpgrade{\spacegrad A}{0,3}
= \spacegrad^2 (A' + \tilde{A})
- \spacegrad \gpgrade{\spacegrad A}{0,3}
\end{dmath}
or
\begin{dmath}\label{eqn:staticPotentials:200}
\spacegrad^2 \tilde{A} = \spacegrad \gpgrade{\spacegrad A}{0,3}.
\end{dmath}
%This non-homogeneous Laplacian equation that can be solved as is for \( \tilde{A} \) using the Green's function for the Laplacian.
Using the Green's function for the gradient, we can
find a solution to the non-homogeneous gradient equation
\begin{dmath}\label{eqn:staticPotentials:220}
\spacegrad \tilde{A} = \gpgrade{\spacegrad A}{0,3},
\end{dmath}
so that \( A' = A - \tilde{A} \) has the required constraints on its vector and bivector components.
If \( \spacegrad \gpgrade{\spacegrad A}{0,3} \) is zero, then the constraints are automatically satisfied, but we are
free to add any \( \tilde{A} \) to the potential, provided that it is a solution of the homogeneous Laplacian equation \( \spacegrad^2 \tilde{A} = 0\).

\subsubsection{Integral form of the gauge transformation.}

Additional insight is possible by considering the gauge transformation in integral form.
\maketheorem{Integral form of potential gauge transformation.}{thm:staticPotentials:460}{
The solution of the Laplacian \cref{eqn:staticPotentials:80} is
\begin{equation*}
A(\Bx) = -\int_V dV' \frac{J(\Bx')}{\Norm{\Bx - \Bx'} } - \tilde{A}(\Bx),
\end{equation*}
where \( \tilde{A} \) is any function for which \( \spacegrad^2 \tilde{A} = 0 \).
For \( F = \spacegrad A \) to be a solution of
\cref{eqn:statics:20}, the Maxwell statics equation, \( \tilde{A} \) must also be a solution of the gradient equation
\begin{equation*}
\spacegrad \tilde{A}(\Bx)
= \int_{\partial V} dA' \frac{ \gpgrade{\ncap' J(\Bx')}{0,3} }{\Norm{\Bx - \Bx'} } 
=
-\int_{\partial V} dA' \frac{ \eta \ncap' \cdot \BJ(\Bx') + I \ncap' \cdot \BM(\Bx')}{\Norm{\Bx - \Bx'} },
\end{equation*}
where
\( \ncap' = (\Bx' - \Bx)/\Norm{\Bx' - \Bx} \) is the unit normal pointing away from the point \( \Bx \).
} % theorem

To prove \cref{thm:staticPotentials:460} we will look
at the constraints on \( \tilde{A} \) that must be imposed for \( F = \spacegrad A \) to be a valid (i.e. grade 1,2) solution of Maxwell's equation.
\begin{dmath}\label{eqn:staticPotentials:300}
F
= \spacegrad A
=
-\int_V dV' \lr{ \spacegrad \inv{\Norm{\Bx - \Bx'} } } J(\Bx')
- \spacegrad \tilde{A}(\Bx)
=
\int_V dV' \lr{ \spacegrad' \inv{\Norm{\Bx - \Bx'} } } J(\Bx')
- \spacegrad \tilde{A}(\Bx)
=
\int_V dV' \spacegrad' \frac{J(\Bx')}{\Norm{\Bx - \Bx'} } - \int_V dV' \frac{\spacegrad' J(\Bx')}{\Norm{\Bx - \Bx'} }
- \spacegrad \tilde{A}(\Bx)
=
\int_{\partial V} dA' \ncap' \frac{J(\Bx')}{\Norm{\Bx - \Bx'} } - \int_V \frac{\spacegrad' J(\Bx')}{\Norm{\Bx - \Bx'} }
- \spacegrad \tilde{A}(\Bx),
\end{dmath}
where the fundamental theorem of geometric calculus
has been used to transform the gradient volume integral into an integral over the bounding surface.
Operating on Maxwell's equation with the gradient gives \( \spacegrad^2 F = \spacegrad J \), which has only grades 1,2 on the left hand side, meaning that \( J \) is constrained in a way that requires \( \spacegrad J \) to have only grades 1,2.
% reference to section that discussed this.
This means that \( F \) has grades 1,2 if
\begin{dmath}\label{eqn:staticPotentials:320}
\spacegrad \tilde{A}(\Bx)
= \int_{\partial V} dA' \frac{ \gpgrade{\ncap' J(\Bx')}{0,3} }{\Norm{\Bx - \Bx'} }.
\end{dmath}
The product \( \ncap' J \) expands to
\begin{dmath}\label{eqn:staticPotentials:340}
\ncap' J
=
\gpgradezero{\ncap' J_1} + \gpgradethree{\ncap' J_2}
=
\ncap' \cdot (-\eta \BJ) + \gpgradethree{\ncap' (-I \BM)}
=- \eta \ncap' \cdot \BJ -I \ncap' \cdot \BM,
\end{dmath}
so
\begin{dmath}\label{eqn:staticPotentials:360}
\spacegrad \tilde{A}(\Bx)
=
-\int_{\partial V} dA' \frac{ \eta \ncap' \cdot \BJ(\Bx') + I \ncap' \cdot \BM(\Bx')}{\Norm{\Bx - \Bx'} }.
\end{dmath}
Observe that if there is no flux of current density \( \BJ \) and (fictitious) magnetic current density \( \BM \) through the surface (i.e. the currents form a closed loop in this volume), then \( F = \spacegrad A \) is a solution to Maxwell's equation without any gauge transformation.
%Alternatively \( F = \spacegrad A \) is also a solution if \( \lim_{\Bx' \rightarrow \infty} \BJ(\Bx')/\Norm{\Bx - \Bx'} = \lim_{\Bx' \rightarrow \infty} \BM(\Bx')/\Norm{\Bx - \Bx'} = 0 \) and the bounding volume is taken to infinity.

%In the case where there are non-zero normal current components out of the volume,
%it should be noted that the question of existance has been ignored.
%In particular, while we can find a solution \( \tilde{A} \) of \cref{eqn:staticPotentials:360} using the Green's function for the gradient \cref{eqn:greensFunctionFirstOrderHelmholtz:900}, is such a solution also a solution of the homogeneous Laplacian equation?

%}
%\EndNoBibArticle
