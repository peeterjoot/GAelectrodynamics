%
% Copyright © 2018 Peeter Joot.  All Rights Reserved.
% Licenced as described in the file LICENSE under the root directory of this GIT repository.
%
%{

The linear polarization of \cref{eqn:polarization_linearPolarization:300} can be generalized from sinosoidal functions of the phase angle \cref{eqn:polarization_circular:520}, to arbitrary functions, as in
\begin{dmath}\label{eqn:polarization_phaseAndEnergyMomentum:540}
F = \lr{ 1 + \Be_3 } \Abs{\BE} \Be_1 e^{i\psi} f(\phi).
\end{dmath}

For example, \( f(\phi) = e^{i\phi} \) would result in a circularly polarized state, and
a Gaussian modulation could be added into the mix with \( f(\phi) = e^{i \phi - (\phi/\sigma)^2/2 } \).

If the phase dependence of \cref{eqn:polarization_phaseAndEnergyMomentum:540} is a scalar function, then
the energy momentum multivector for the field can be calculated simply
\begin{dmath}\label{eqn:polarization_phaseAndEnergyMomentum:560}
\calE + \frac{\BS}{v}
=
\inv{2} \epsilon
F F^\dagger
=
\inv{2} \epsilon
\lr{ 1 + \Be_3 } \Abs{\BE}^2 \Be_1 \cancel{e^{i\psi}} f^2(\phi)
\cancel{e^{-i\psi} }
\Be_1
\lr{ 1 + \Be_3 }
=
\inv{2} \epsilon
\lr{ 1 + \Be_3 } \Abs{\BE}^2 \cancel{\Be_1} f^2(\phi)
\cancel{\Be_1 }
\lr{ 1 + \Be_3 }
=
\epsilon \lr{ 1 + \Be_3 } \Abs{\BE}^2 f^2(\phi),
\end{dmath}
where the projective property \( \lr{ 1 + \Be_3 }^2 = 2 \lr{ 1 + \Be_3 } \) was used in the final simplification.
The energy, and Poynting vectors are
\begin{dmath}\label{eqn:polarization_phaseAndEnergyMomentum:580}
\begin{aligned}
\calE &= \epsilon \Abs{\BE}^2 f^2(\phi) \\
\BS &= \inv{\eta} \Be_3 \Abs{\BE}^2 f^2(\phi).
\end{aligned}
\end{dmath}
% v epsilon = sqrt( epsilon^2/ (epsilon mu) ) = 1/eta

More care for this calculation is required if the phase function \( f(\phi) \) is multivector valued, since it may not commute with the \( \Be_1 \) and \( e^{i\psi} \) factors of \( F \).
%}
