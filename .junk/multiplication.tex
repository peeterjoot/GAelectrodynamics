
%%\input{../blogpost.tex}
%%\renewcommand{\basename}{multiplication}
%%%\renewcommand{\dirname}{notes/phy1520/}
%%\renewcommand{\dirname}{notes/ece1228-electromagnetic-theory/}
%%%\newcommand{\dateintitle}{}
%%%\newcommand{\keywords}{}
%%
%%\input{../peeter_prologue_print2.tex}
%%
%%\usepackage{peeters_layout_exercise}
%%\usepackage{peeters_braket}
%%\usepackage{peeters_figures}
%%\usepackage{siunitx}
%%%\usepackage{mhchem} % \ce{}
%%%\usepackage{macros_bm} % \bcM
%%%\usepackage{txfonts} % \ointclockwise
%%
%%\beginArtNoToc
%%
%%\generatetitle{Vector multiplication}
%%%\chapter{Vector multiplication}
%%%\label{chap:multiplication}
%%
Geometric Algebra defines a multiplication operation for vectors, forming a vector space spanned by all the possible vector products.  This algebra is described by the following small set of axioms

\makeaxiom{Associative multiplication.}{axiom:multiplication:associative}{

The product of any three vectors \(\Ba,\Bb,\Bc\) is associative.

\begin{equation*}\label{eqn:multiplication:160}
\Ba (\Bb \Bc)
= (\Ba \Bb) \Bc
= \Ba \Bb \Bc.
\end{equation*}
}

\makeaxiom{Linearity.}{axiom:multiplication:linear}{
Vector products are linear with respect to addition and subtraction.

\begin{dmath*}\label{eqn:multiplication:180}
\begin{aligned}
(\Ba + 3 \Bb \Bd) \Bc &= \Ba \Bb + 3 \Bb \Bd \Bc \\
\Ba (\Bb \Bd - 2 \Bc) &= \Ba \Bb \Bd - 2 \Ba \Bc.
\end{aligned}
\end{dmath*}
}

\makeaxiom{Contraction.}{axiom:multiplication:contraction}{

The square of a vector is the squared length of the vector.

\begin{dmath*}\label{eqn:multiplication:200}
\Ba^2 = \Norm{\Ba}^2.
\end{dmath*}

The notion of length here is metric dependent.  For the problems considered in these notes
it can be assumed that there is an orthonormal Euclidean basis, where the vector length is always positive.
For special relativistic calculations, also of interest in electrodynamics, but not the focus of these notes, the length of a (four-)vector may generally be negative or positive.
}

...

This contraction axiom, justified or not, has additional implications

\begin{dmath}\label{eqn:multiplication:80}
x^2
= \Bx^2
= (x \Be)(x \Be)
= x^2 \Be^2.
\end{dmath}


