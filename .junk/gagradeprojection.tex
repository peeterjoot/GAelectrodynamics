%
% Copyright � 2019 Peeter Joot.  All Rights Reserved.
% Licenced as described in the file LICENSE under the root directory of this GIT repository.
%
%{
\input{../latex/blogpost.tex}
\renewcommand{\basename}{gagradeprojection}
%\renewcommand{\dirname}{notes/phy1520/}
\renewcommand{\dirname}{notes/ece1228-electromagnetic-theory/}
%\newcommand{\dateintitle}{}
%\newcommand{\keywords}{}

\input{../latex/peeter_prologue_print2.tex}

\usepackage{peeters_layout_exercise}
\usepackage{peeters_braket}
\usepackage{peeters_figures}
\usepackage{siunitx}
\usepackage{verbatim}
%\usepackage{mhchem} % \ce{}
%\usepackage{macros_bm} % \bcM
%\usepackage{macros_qed} % \qedmarker
%\usepackage{txfonts} % \ointclockwise

\beginArtNoToc

\generatetitle{XXX}
%\chapter{XXX}
%\label{chap:gagradeprojection}
% \citep{sakurai2014modern} pr X.Y
% \citep{pozar2009microwave}
% \citep{qftLectureNotes}
% \citep{doran2003gap}
% \citep{jackson1975cew}
% \citep{griffiths1999introduction}
For the first decomposition, 
since \( x \cdot B \) is a vector, \( x \wedge ( x \cdot B ) \) must be a bivector (grade 2).  The product of the bivector \( A \) with another bivector is a multivector with grades that may include 0, 2, 4, and the dot product of those two is the grade-0 selection.  That is
\begin{equation}\label{eqn:WHAT:n}
\begin{aligned}
   A \cdot \lr{ x \wedge (x \cdot B) }
   &=
   \gpgradezero{ A \lr{ x \wedge (x \cdot B) } } \\
   &=
   \gpgradezero{ A \lr{ x (x \cdot B) - x \cdot (x \cdot B) } }.
\end{aligned}
\end{equation}
Here we used \( x y = x \cdot y + x \wedge y \), which leaves us a with a scalar factor \( x \cdot (x\cdot B) \), that has no contribution to the grade-0 selection, which leaves us with
\begin{equation}\label{eqn:WHAT:n}
\begin{aligned}
   A \cdot \lr{ x \wedge (x \cdot B) }
   &=
   \gpgradezero{ A x (x \cdot B) } \\
   &=
   \gpgradezero{ (A \cdot x) (x \cdot B) } \\
   &=
   \gpgradezero{ (A \cdot x) (x B - x \wedge B) } \\
   &=
   \gpgradezero{ (A \cdot x) x B }.
\end{aligned}
\end{equation}
Here we made use of the fact that \( A x = A \cdot x + A \wedge x \), a vector and trivector, of which only the vector component contributes to the scalar selection.  Then we are free to rewrite \( x \cdot B \) in terms of \( x B \) and a trivector that also has no contritution to the scalar selection.  The final result follows from the fact that reversion of a scalar leaves it unchanged, so 
\begin{equation}\label{eqn:WHAT:n}
\begin{aligned}
   \gpgradezero{ (A \cdot x) x B }
   &=
   \gpgradezero{ B x (x \cdot A) } \\
   &=
   B \cdot \lr{ x \wedge (x \cdot A) },
\end{aligned}
\end{equation}
where we are able to use the very first step to write the scalar selection back in terms of dots and wedges.

The result of 4.154 probably requires similar logic.

%}
\EndArticle
%\EndNoBibArticle
