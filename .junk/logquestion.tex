%
% Copyright � 2019 Peeter Joot.  All Rights Reserved.
% Licenced as described in the file LICENSE under the root directory of this GIT repository.
%
%{
\input{../latex/blogpost.tex}
\renewcommand{\basename}{logquestion}
%\renewcommand{\dirname}{notes/phy1520/}
\renewcommand{\dirname}{notes/ece1228-electromagnetic-theory/}
%\newcommand{\dateintitle}{}
%\newcommand{\keywords}{}

\input{../latex/peeter_prologue_print2.tex}

\usepackage{peeters_layout_exercise}
\usepackage{peeters_braket}
\usepackage{peeters_figures}
\usepackage{siunitx}
\usepackage{verbatim}
%\usepackage{mhchem} % \ce{}
%\usepackage{macros_bm} % \bcM
%\usepackage{macros_qed} % \qedmarker
%\usepackage{txfonts} % \ointclockwise

\beginArtNoToc

\generatetitle{XXX}
%\chapter{XXX}
%\label{chap:logquestion}
% \citep{sakurai2014modern} pr X.Y
% \citep{pozar2009microwave}
% \citep{qftLectureNotes}
% \citep{doran2003gap}
% \citep{jackson1975cew}
% \citep{griffiths1999introduction}

Using the chain rule (in its non-commutative form), you can write
\begin{equation}\label{eqn:logquestion:20}
   d( g \log g ) = dg \log g + g d \lr{ \log g },
\end{equation}
but after this you get into trouble.  Does \( \log g \) have a derivative?  You can't assume that the chain rule applies, since \( g \) may not commute with \( dg \).  A more fundamental question is whether or not \( \log g \) itself exists.

Hestenes (\citep{hestenes1999nfc}) defines the multivector logarithm as a solution, for some \( x \), of
\begin{equation}\label{eqn:logquestion:40}
e^x = g,
\end{equation}
I suspect that is generally not possible to compute.

%}
\EndArticle
