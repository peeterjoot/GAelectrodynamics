%
% Copyright � 2019 Peeter Joot.  All Rights Reserved.
% Licenced as described in the file LICENSE under the root directory of this GIT repository.
%
%{
\input{../latex/blogpost.tex}
\renewcommand{\basename}{triint}
%\renewcommand{\dirname}{notes/phy1520/}
\renewcommand{\dirname}{notes/ece1228-electromagnetic-theory/}
%\newcommand{\dateintitle}{}
%\newcommand{\keywords}{}

\input{../latex/peeter_prologue_print2.tex}

\usepackage{txfonts} % \ointclockwise
\usepackage{peeters_layout_exercise}
\usepackage{peeters_braket}
\usepackage{peeters_figures}
\usepackage{siunitx}
\usepackage{verbatim}
%\usepackage{mhchem} % \ce{}
%\usepackage{macros_bm} % \bcM
%\usepackage{macros_qed} % \qedmarker
%\usepackage{txfonts} % \ointclockwise

\beginArtNoToc

\generatetitle{XXX}
%\chapter{XXX}
%\label{chap:triint}
% \citep{sakurai2014modern} pr X.Y
% \citep{pozar2009microwave}
% \citep{qftLectureNotes}
% \citep{doran2003gap}
% \citep{jackson1975cew}
% \citep{griffiths1999introduction}

Given mulitvector functions \( F, G\) in \R{3}, a trivector valued volume element \( d^3 \Bx = d\Bx_1 \wedge d\Bx_2 \wedge d\Bx_3 = du dv dw\, \Bx_u \wedge \Bx_v \wedge \Bx_w \)
\begin{equation*}
\int_V F d^3\Bx \lrspacegrad G
= \ointclockwise_{\partial V} F d^2 \Bx G,
\end{equation*}
where \( \partial V \) is the boundary of the volume \( V \),
and \( d^2 \Bx \) is the counterclockwise oriented area element on the boundary of the volume, that is
\begin{equation*}
\begin{aligned}
\ointclockwise_{\partial V} F d^2 \Bx G
&=
\int \evalbar{\lr{F d\Bx_1 \wedge d\Bx_2 G}}{\Delta w} \\
&\qquad +\int \evalbar{\lr{F d\Bx_2 \wedge d\Bx_3 G}}{\Delta u} \\
&\qquad +\int \evalbar{\lr{F d\Bx_3 \wedge d\Bx_1 G}}{\Delta v}.
\end{aligned}
\end{equation*}

%}
\EndArticle
%\EndNoBibArticle
