%
% Copyright � 2018 Peeter Joot.  All Rights Reserved.
% Licenced as described in the file LICENSE under the root directory of this GIT repository.
%
%{
%\input{../latex/blogpost.tex}
%\renewcommand{\basename}{unpackStaticPotential}
%%\renewcommand{\dirname}{notes/phy1520/}
%\renewcommand{\dirname}{notes/ece1228-electromagnetic-theory/}
%%\newcommand{\dateintitle}{}
%%\newcommand{\keywords}{}
%
%\input{../latex/peeter_prologue_print2.tex}
%
%\usepackage{peeters_layout_exercise}
%\usepackage{peeters_braket}
%\usepackage{peeters_figures}
%\usepackage{siunitx}
%%\usepackage{mhchem} % \ce{}
%%\usepackage{macros_bm} % \bcM
%%\usepackage{macros_qed} % \qedmarker
%%\usepackage{txfonts} % \ointclockwise
%
%\beginArtNoToc
%
%\generatetitle{Unpacking the potential equation.}
%%\chapter{Unpacking the potential equation.}
\label{chap:unpackStaticPotential}

In this section we will explore potential (second order) solutions of
\cref{eqn:statics:20}, the Maxwell statics equation.
While we can solve this for the field directly using
the Green's function for the gradient \cref{eqn:greensFunctionFirstOrderHelmholtz:900}, a second order solution can use the scalar valued Green's function \cref{eqn:greensFunctionHelmholtz:80} which may be simpler to evaluate in some circumstances.

\makedefinition{Multivector potential (statics.)}{thm:staticPotentials:380}{
We call \( A \) the \textit{multivector potential} for the (static) field if
\begin{equation*}
F = \gpgrade{\spacegrad A}{1,2},
\end{equation*}
and
\begin{equation*}
\spacegrad \gpgrade{\spacegrad A}{1,2} = J.
\end{equation*}
} % definition

Without any a-priori knowledge, we should assume that the multivector potential contains all possible grades.  If those grades are indicated with subscripts, as in
\begin{dmath}\label{eqn:unpackStaticPotential:20}
A = A_0 + A_1 + A_2 + A_3,
\end{dmath}
then
\begin{dmath}\label{eqn:unpackStaticPotential:40}
F =F =  \gpgrade{\spacegrad A}{1,2}
=
\mathLabelBox[ labelstyle={below of=m\themathLableNode, below of=m\themathLableNode} ]
{
\spacegrad A_0 + \spacegrad \cdot A_2
}
{
\(\BE\)
}
+
%\mathLabelBox[ labelstyle={yshift=0.2em}, linestyle={} ]
\mathLabelBox[ labelstyle={below of=m\themathLableNode, below of=m\themathLableNode} ]
{
\spacegrad \wedge A_1 + \spacegrad A_3
}
{
\(I \eta \BH\)
}
.
\end{dmath}
Application of the gradient to \cref{eqn:unpackStaticPotential:40} provides one equation for each grade
\begin{dmath}\label{eqn:unpackStaticPotential:60}
\begin{aligned}
\spacegrad \cdot \lr{ \spacegrad A_0 + \cancel{\spacegrad \cdot A_2} } &= \frac{\rho}{\epsilon} \\
\spacegrad \cdot \lr{ \spacegrad \wedge A_1 + \cancel{\spacegrad A_3} } &= -\eta \BJ \\
\spacegrad \wedge \lr{ \cancel{\spacegrad A_0} + \spacegrad \cdot A_2 } &= -I \BM \\
\spacegrad \wedge \lr{ \cancel{\spacegrad \wedge A_1} + \spacegrad A_3 } &= I c \rho_\txtm,
\end{aligned}
\end{dmath}
where the cancelled terms follow from \( \spacegrad \wedge \spacegrad \wedge f = 0 \), after noting that \( \spacegrad \cdot (\spacegrad \cdot A_2) = I \spacegrad \wedge (\spacegrad \wedge (-I A_2)) \).
The Maxwell statics equation uncouples completely, providing one equation for each grade, where each grade of the multivector potential is related to precisely one source.

The conventional representation of these equations may be obtained with the following representation of the multivector potential.
\makedefinition{Multivector potential representation.}{dfn:unpackStaticPotential:80}{
Let
%\label{eqn:potentialSection:40}
\begin{equation*}
A =
      - \phi
      + c \BA
      + \eta I \lr{ -\phi_m + c \BF },
\end{equation*}
where
\begin{enumerate}
\item \( \phi \) is the scalar potential \si{V} (Volts).
\item \( \BA \) is the vector potential \si{Wb/m} (Webers/meter).
\item \( \phi_m \) is the scalar potential for (fictitious) magnetic sources \si{A} (Amperes).
\item \( \BF \) is the vector potential for (fictitious) magnetic sources \si{C} (Coulombs).
\end{enumerate}
} % definition

This specific breakdown of \( A \) into scalar and vector potentials, and dual (pseudoscalar and bivector) potentials has been chosen to match SI conventions, specifically those of \citep{balanis2005antenna}.

With substitution of \cref{dfn:unpackStaticPotential:80} into \cref{eqn:unpackStaticPotential:60}, and a bit of manipulation (\cref{problem:unpackStaticPotential:120}), the multivector potential equation \( \spacegrad \gpgrade{\spacegrad A}{1,2} = J \) is seen to expand to
\begin{dmath}\label{eqn:unpackStaticPotential:100}
\begin{aligned}
\spacegrad^2 \phi &= -\frac{\rho}{\epsilon} \\
\spacegrad \cdot \lr{ \spacegrad \wedge \BA } = \spacegrad^2 \BA - \spacegrad (\spacegrad \cdot \BA) &= - \mu \BJ \\
\spacegrad \cdot \lr{ \spacegrad \wedge \BF } = \spacegrad^2 \BF - \spacegrad (\spacegrad \cdot \BF) &= - \epsilon \BM \\
\spacegrad^2 \phi_\txtm &= -\frac{\rho_\txtm}{\mu} \\
\end{aligned}
\end{dmath}

For statics problems, it is clearly desirable if we can find \( \BA, \BF \) such that \( \spacegrad \cdot \BA = 0, \spacegrad \cdot \BF = 0 \), as the potentials all follow by solution of a set of independent Laplacian equations.
This can arranged by a suitable choice of gauge.  We will look next at gauge transformations from a multivector point of view.

\makeproblem{Grade split of the potential equations.}{problem:unpackStaticPotential:120}{
Verify \cref{eqn:unpackStaticPotential:100}.
} % problem

%}
%\EndArticle
