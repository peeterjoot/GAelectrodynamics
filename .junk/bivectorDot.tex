
Alternatively, this result can be obtained compactly using tensor contraction techniques, first making a duality transformation and then expanding in coordinates

\begin{dmath}\label{eqn:bivectorDot:100}
(\Ba \wedge \Bb) \cdot (\Bc \wedge \Bd)
=
\gpgradezero{
-I (\Ba \cross \Bb) (-I) (\Bc \cross \Bd)
}
=
- (\Ba \cross \Bb) \cdot (\Bc \cross \Bd)
=
-(\epsilon_{ijk} \Be_i a_j b_k) \cdot (\epsilon_{r s t} \Be_r c_s d_t)
=
-\epsilon_{ijk} a_j b_k \epsilon_{i s t} c_s d_t
=
-\delta_{jk}^{[st]}
a_j b_k c_s d_t
=
-a_s b_t c_s d_t
+
a_t b_s c_s d_t
=
-(\Ba \cdot \Bc)(\Bb \cdot \Bd)
+
(\Ba \cdot \Bd)(\Bb \cdot \Bc).
\end{dmath}

A student of physics might consider this a natural alternative approach.
%Some GA authors may find this alternate derivation offensive, as it contains both an expansion by coordinates and requires an alternate toolbox.
