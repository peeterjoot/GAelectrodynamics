%
% Copyright © 2017 Peeter Joot.  All Rights Reserved.
% Licenced as described in the file LICENSE under the root directory of this GIT repository.
%
\makeproblem{Ninety degree rotation in coordinates.}{problem:coordinateNinetyDegreeRotation:1}{

Rotate the vector

\begin{dmath}\label{eqn:coordinateNinetyDegreeRotation:240}
   \Bx = \rho
\begin{bmatrix}
   \cos\theta \\
   \sin\theta \\
\end{bmatrix},
\end{dmath}

in the \( x,y \) plane by \(\pm \pi/2\) using the 2D rotation matrix
\begin{dmath}\label{eqn:coordinateNinetyDegreeRotation:280}
R_\theta =
\begin{bmatrix}
\cos\theta & -\sin\theta \\
\sin\theta & \cos\theta
\end{bmatrix},
\end{dmath}

and compare the result with \cref{eqn:SimpleProducts2:300} and \cref{eqn:SimpleProducts2:310}.
} % problem

\makeanswer{problem:coordinateNinetyDegreeRotation:1}{
The counterclockwise (positive) rotation is

\begin{dmath}\label{eqn:coordinateNinetyDegreeRotation:260}
\Bx'
=
\begin{bmatrix}
   0 & -1 \\
   1 & 0 \\
\end{bmatrix}
\begin{bmatrix}
   \cos\theta \\
   \sin\theta \\
\end{bmatrix}
=
\rho
\begin{bmatrix}
   -\sin\theta \\
   \cos\theta
\end{bmatrix},
\end{dmath}

consistent with \cref{eqn:SimpleProducts2:300}.  The clockwise rotation is

\begin{dmath}\label{eqn:coordinateNinetyDegreeRotation:300}
\Bx''
=
\begin{bmatrix}
   0 & 1 \\
   -1 & 0 \\
\end{bmatrix}
\begin{bmatrix}
   \cos\theta \\
   \sin\theta \\
\end{bmatrix}
=
\rho
\begin{bmatrix}
   \sin\theta \\
   -\cos\theta
\end{bmatrix},
\end{dmath}

which matches \cref{eqn:SimpleProducts2:310}.
} % answer
