The scalar portion of this product is the dot product, whereas the bivector portion is something else.


We will show that this bivector is related to the \R{3} cross product.
Low
We have found that it is possible to represent the dot product of two

What is the product of two vectors that

\begin{dmath}\label{eqn:SimpleProducts2:1180}
\Ba \cdot \Bb
=
\gpgradezero{ \Ba \Bb }
=
\gpgradezero{ \lr{ \sum_i a_i \Be_i }
\lr{ \sum_j b_j \Be_j }
}
=
\sum_{ij} a_i b_j \gpgradezero{ \Be_i \Be_j }.
\end{dmath}


In order to discuss the product of two vectors, we have to introduce two additional operators.  The first of these is a generalization of the dot product to multivectors

By this definition, the dot product of two vectors \( \Ba, \Bb \), is

\boxedEquation{eqn:SimpleProducts2:1140}{
\Ba \cdot \Bb
=
\gpgradezero{ \Ba \Bb }.
}

Before anything else, let's check that this definition agrees with our expectations for a vector dot product.  Let
so the generalized dot product is

\begin{dmath}\label{eqn:SimpleProducts2:1560}
\Ba \cdot \Bb
=
\gpgradezero{ \Ba \Bb }
=
\gpgradezero{ \lr{ \sum_i a_i \Be_i }
\lr{ \sum_j b_j \Be_j }
}
=
\sum_{ij} a_i b_j \gpgradezero{ \Be_i \Be_j }.
\end{dmath}

Here a change of index variables was used for \( \Bb \) and the sums and scalars were all factored out of the grade selection operator.  The multivectors \( \Be_i \Be_j \) are unit scalars whenever \( i = j \) and are bivectors whenever \( i \ne j \), so the scalar selection is a Kronecker delta function, leaving

\begin{dmath}\label{eqn:SimpleProducts2:1200}
\Ba \cdot \Bb
=
\sum_{ij} a_i b_j \delta_{ij}
=
\sum_{i} a_i b_i.
\end{dmath}

As claimed this is the familiar (Euclidean) dot product.
Some examples of vector-bivector, bivector-bivector, and vector-trivector dot products will be considered later.
Since immediate goal is examination of the geometric product of two vectors, we need our second special case grade selection operator, the wedge product


In particular, the wedge product of two vectors \( \Ba, \Bb \) is

\boxedEquation{eqn:SimpleProducts2:1220}{
\Ba \wedge \Bb = \gpgradetwo{ \Ba \Bb }.
}

Using the coordinate expansion \cref{eqn:SimpleProducts2:1160}, the wedge product of two vectors is

\begin{dmath}\label{eqn:SimpleProducts2:1240}
\Ba \wedge \Bb
=
\gpgradetwo{ \Ba \Bb }
=
\gpgradetwo{ \lr{ \sum_i a_i \Be_i }
\lr{ \sum_j b_j \Be_j }
}
=
\sum_{ij} a_i b_j \gpgradetwo{ \Be_i \Be_j }
\end{dmath}

As pointed out \( \Be_i \Be_j \) is a (unit) scalar whenever \( i = j \), and is a bivector (grade 2) whenever \( i \ne j \), so

\begin{dmath}\label{eqn:SimpleProducts2:1260}
\Ba \wedge \Bb
=
\sum_{i \ne j} a_i b_j \Be_i \Be_j
=
\inv{2}
\sum_{i \ne j} a_i b_j
\Be_i \Be_j
+
\inv{2}
\sum_{r \ne t} a_r b_t
\Be_r \Be_t.
\end{dmath}

The sum has been halved and doubled using a second set of indexes.  This is a sneaky trick often used in physics with indexed (tensor) equations, and worth knowing.  That second sum can now be manipulated

\begin{dmath}\label{eqn:SimpleProducts2:1280}
\inv{2}
\sum_{r \ne t} a_r b_t
\Be_r \Be_t
=
\inv{2}
\sum_{r \ne t} a_r b_t
(-\Be_t \Be_r)
=
-\inv{2}
\sum_{j \ne i} a_j b_i
\Be_i \Be_j.
\end{dmath}

The normal vectors \( \Be_r, \Be_t \) were first anticommuted (swap the order and the sign), and then a final substitution of summation indexes was made.  Adding the two halves together again gives

\begin{dmath}\label{eqn:SimpleProducts2:1300}
\Ba \wedge \Bb
=
\inv{2} \sum_{i \ne j} (a_i b_j - a_j b_i) \Be_i \Be_j,
\end{dmath}

or, in terms of determinants

The \R{2} expansion of the wedge product is just

\begin{dmath}\label{eqn:SimpleProducts2:1340}
\Ba \wedge \Bb
=
\begin{vmatrix}
a_1 & a_2 \\
b_1 & b_2
\end{vmatrix}
\Be_1 \Be_2.
\end{dmath}
\paragraph{Computing the normal 2D}

Similar to multiplication by the complex number \( i \),
left and right multiplication by the pseudoscalar also rotate vectors by \( \pi/2 \) radians, however, the sign of the rotation depends on whether the multiplication is from the left or from the right.
With

and multiplication from the right

the vector is rotated clockwise by \( \pi/2 \).
% (\cref{problem:left2dimaginarymultiplication:1}).
%Just as the imaginary rotates complex numbers, the 2D pseudoscalar rotates vectors, with the cavaet that one must be careful about the order that this multiplication is performed.
%
\paragraph{Complex numbers and rotations}

The 2D pseudoscalar has been seen to have the characteristics of the complex imaginary.
This analogy can be extended by noting that the following multivector

\begin{dmath}\label{eqn:SimpleProducts2:320}
z = x + \Be_1 \Be_2 y,
\end{dmath}

provides an isomorphic representation of a complex number.
This can be made obvious by introduces the lable \( i = \Be_1 \Be_2 \) for the pseudoscalar, since it has the desired characteristics of the imaginary.

Such a complex exponential only rotates vectors that lie in the plane of the bivector, and commute with vector (or vector components) that lie outside of the rotational plane (\cref{problem:normalMult:1}).
This means that conjugate complex exponentials applied from both sides,
rotate the components of the vector that lie in the rotational plane, but leave the normal components unaltered.
The desired expression of a \R{N} rotation
has the ``sandwich'' structure
\footnote{Dual sided rotations of this form are also found in Pauli (matrix) algebra and quaternion formalisms, both of which, like complex numbers, can be considered special cases of geometric algebras.}

\boxedEquation{eqn:SimpleProducts2:840}{
\Bx' =
e^{-i\theta/2} \Bx e^{i \theta/2},
}

where \( i \) is the unit bivector for the plane of rotation.
When the vector has no components outside of the plane, single sided rotations can be used, which is often convienient

\begin{equation}\label{eqn:SimpleProducts2:980}
\Bx' = \Bx e^{i \theta} = e^{-i\theta} \Bx.
\end{equation}

\paragraph{Vector product.}

The product of two 2D vectors is illustrative.
Let

\begin{dmath}\label{eqn:SimpleProducts2:500}
\begin{aligned}
   \Bx &= x_1 \Be_1 + x_2 \Be_2 \\
   \By &= y_1 \Be_1 + y_2 \Be_2
\end{aligned},
\end{dmath}

and consider their product
\begin{dmath}\label{eqn:SimpleProducts2:520}
\Bx \By
=
\lr{ x_1 \Be_1 + x_2 \Be_2 }
\lr{ y_1 \Be_1 + y_2 \Be_2 }
=
x_1 y_1 \Be_1 \Be_1 + x_2 y_1 \Be_2 \Be_1
+
x_1 y_2 \Be_1 \Be_2 + x_2 y_2 \Be_2 \Be_2
=
x_1 y_1
- x_2 y_1 \Be_1 \Be_2
+ x_1 y_2 \Be_1 \Be_2
+ x_2 y_2
=
x_1 y_1 + x_2 y_2
+ \lr{ x_1 y_2 - x_2 y_1 } \Be_1 \Be_2.
\end{dmath}

The vector product has a scalar and a bivector component, where the scalar component is completely symmetric, whereas the bivector component is completely antisymmetric.
Observe that the scalar selection from the vector product is the dot product.
A name is required for the bivector (grade-2) selection from this product, and it is called the wedge product \( \Bx \wedge \By \), an operation that has to be examined in more depth.
With such a definition, the vector product can be written as

%\begin{dmath}\label{eqn:SimpleProducts2:540}
\boxedEquation{eqn:SimpleProducts2:540}{
\Bx \By
= \Bx \cdot \By + \Bx \wedge \By,
}
%\end{dmath}

where

\begin{dmath}\label{eqn:SimpleProducts2:560}
\begin{aligned}
\Bx \cdot \By &\equiv \gpgradezero{\Bx\By} = x_1 y_1 + x_2 y_2 \\
\Bx \wedge \By &\equiv \gpgradetwo{\Bx\By} =
\begin{vmatrix}
   x_1 & x_2 \\
   y_1 & y_2
\end{vmatrix}
   \Be_1 \Be_2. \\
\end{aligned}
\end{dmath}

