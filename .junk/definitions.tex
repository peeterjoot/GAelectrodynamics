%
% Copyright � 2016 Peeter Joot.  All Rights Reserved.
% Licenced as described in the file LICENSE under the root directory of this GIT repository.
%
%{
%%\input{../blogpost.tex}
%%\renewcommand{\basename}{multiplication}
%%%\renewcommand{\dirname}{notes/phy1520/}
%%\renewcommand{\dirname}{notes/ece1228-electromagnetic-theory/}
%%%\newcommand{\dateintitle}{}
%%%\newcommand{\keywords}{}
%%
%%\input{../peeter_prologue_print2.tex}
%%
%%\usepackage{peeters_layout_exercise}
%%\usepackage{peeters_braket}
%%\usepackage{peeters_figures}
%%\usepackage{siunitx}
%%%\usepackage{mhchem} % \ce{}
%%%\usepackage{macros_bm} % \bcM
%%%\usepackage{txfonts} % \ointclockwise
%%
%%\beginArtNoToc
%%
%%\generatetitle{Vector multiplication}
%%%\chapter{Vector multiplication}
%%%\label{chap:multiplication}
%%

A few new GA terms have been introduced in an ad-hoc fashion as required.  Here is a systematic exposition of some of the key definitions used to refer to the types of the geometric objects that will be encountered.

The grade of a scalar, vector, bivector, and trivector are 0, 1, 2, and 3 respectively.


%%%}
%%%\EndArticle
%%\EndNoBibArticle
