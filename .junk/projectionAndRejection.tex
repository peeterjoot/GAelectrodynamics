
(cut)
The pythagorean property of these two vector components can also be checked.
Computing the squared length using \( \Norm{\By}^2 = \By \cdot \By = \By^2 \), the squared length of the projective component is

\begin{dmath}\label{eqn:SimpleProducts2:740}
\lr{ \lr{\Bx \cdot \ucap } \ucap }^2
=
\lr{\Bx \cdot \ucap }^2
=
(x_1 u_1 + x_2 u_2)^2
=
x_1^2 u_1^2 + x_2^2 u_2^2 + 2 x_1 x_2 u_1 u_2.
\end{dmath}

The squared length of the rejective component is
\begin{dmath}\label{eqn:SimpleProducts2:760}
\lr{ \lr{\Bx \wedge \ucap } \ucap }^2
=
-(\Bx \wedge \ucap) \ucap^2 (\Bx \wedge \ucap)
=
-
\lr{\begin{vmatrix}
   x_1 & x_2 \\
   u_1 & u_2
\end{vmatrix}}^2
(\Be_1 \Be_2)^2
=
x_1^2 u_2^2 + x_2^2 u_1^2 - 2 x_1 x_2 u_1 u_2.
\end{dmath}

Adding these together gives

\begin{dmath}\label{eqn:SimpleProducts2:780}
\lr{ \lr{\Bx \cdot \ucap } \ucap }^2 + \lr{ \lr{\Bx \wedge \ucap } \ucap }^2
=
x_1^2 u_1^2 + x_2^2 u_2^2
+x_1^2 u_2^2 + x_2^2 u_1^2
=
x_1^2 ( u_1^2 + u_2^2 )
+
x_2^2 ( u_1^2 + u_2^2 )
=
\Bx^2,
\end{dmath}

recovering the squared length of the vector as expected.
It is generally true in higher dimensions that the projection and rejection can be written as

\begin{dmath}\label{eqn:SimpleProducts2:800}
\begin{aligned}
\Proj_\ucap(\Bx) &= (\Bx \cdot \ucap) \ucap \\
\RejName_\ucap(\Bx) &= (\Bx \wedge \ucap) \ucap.
\end{aligned}
\end{dmath}

The Pythagorean aspect of this statement in higher degree spaces
will be demonstrated later in a coordinate free fashion after some additional identities have been derived.

The unit vector restriction defining the direction of projection and rejection can be relaxed in a compact fashion by introducing the vector \boldTextAndIndex{inverse}, which is always well defined and unique in a Euclidean space

\boxedEquation{eqn:SimpleProducts2:860}{
\inv{\Bu} \equiv \frac{\Bu}{\Bu^2}.
}

Now the projection and rejection onto the direction of \( \Bu \) are

\boxedEquation{eqn:SimpleProducts2:880}{
\begin{aligned}
\Proj_\Bu(\Bx) &= (\Bx \cdot \Bu) \inv{\Bu} \\
\RejName_\Bu(\Bx) &= (\Bx \wedge \Bu) \inv{\Bu}.
\end{aligned}
}

%\makelemma{\R{3} pseudoscalar commutation.}{dfn:projectionAndRejection:r3pcommutation}{
%The \R{3} pseudoscalar \( I = \Be_1 \Be_2 \Be_3 \) commutes with all \R{3} multivectors.
%} % lemma
%
%To prove this, it is sufficient to consider the commutation of \( I \) with each of the standard basis vectors \( \Be_1, \Be_2, \Be_3 \) (\cref{problem:projectionAndRejection:1160}).
%
Now the cross product form of the rejection equation can be determined
