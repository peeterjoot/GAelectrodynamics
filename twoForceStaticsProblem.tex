%
% Copyright © 2023 Peeter Joot.  All Rights Reserved.
% Licenced as described in the file LICENSE under the root directory of this GIT repository.
%
%{
\makeproblem{Find static load in two member configuration.}{problem:twoForceStaticsProblem:1}{
In introductory physics or mechanics classes, systems such as those illustrated in \cref{fig:twoForceProblemWithWeight:twoForceProblemWithWeightFigure}, are common.  All of these have a static load under gravity, and with supporting members (rigid beams or wire lines), which can be under compression, or tension, depending on the geometry.
\pmathImageThreeFiguresOneLine{../figures/GAelectrodynamics/}{twoForceProblemWithWeightFigure1}{twoForceProblemWithWeightFigure2}{twoForceProblemWithWeightFigure3}{Static load with two members.}{fig:twoForceProblemWithWeight:twoForceProblemWithWeightFigure}{scale=0.3}{twoForceProblemWithWeightFigure.nb}

We seek the magnitudes of the forces in the two members, where the static configuration means that we have a force balance equation of the form
\begin{equation}\label{eqn:twoForceStaticsProblem:20}
   \BF_s + \BF_r + m \Bg = 0.
\end{equation}
If each of these forces are expressed in complex exponential form
\begin{equation}\label{eqn:twoForceStaticsProblem:40}
\begin{aligned}
   \BF_r &= f_r \Be_1 e^{i\alpha} \\
   \BF_s &= f_s \Be_1 e^{i\beta} \\
   \Bg &= g \Be_1,
\end{aligned}
\end{equation}
then the system to be solved is
\begin{equation}\label{eqn:twoForceStaticsProblem:60}
f_r \Be_1 e^{i\alpha} + f_s \Be_1 e^{i\beta} + m g \Be_1 = 0.
\end{equation}
By computing a ratio of wedge products, find the scalar magnitudes \( f_r, f_s \) for such a system.
} % problem
\makeanswer{problem:twoForceStaticsProblem:1}{
To solve for \( f_r \) or \( f_s \), first take wedge products with the force direction vectors to eliminate the variable not of interest
\begin{equation}\label{eqn:twoForceStaticsProblem:80}
\begin{aligned}
   f_r \lr{ \Be_1 e^{i\alpha} } \wedge \lr{ \Be_1 e^{i\beta} } + m g \Be_1 \wedge \lr{ \Be_1 e^{i\beta} } &= 0 \\
   f_s \lr{ \Be_1 e^{i\beta} } \wedge \lr{ \Be_1 e^{i\alpha} } + m g \Be_1 \wedge \lr{ \Be_1 e^{i\alpha} } &= 0.
\end{aligned}
\end{equation}
Writing the wedges as grade two selections, and noting that \( e^{i\theta} \Be_1 = \Be_1 e^{-i\theta } \), we have
\begin{equation}\label{eqn:twoForceStaticsProblem:100}
\begin{aligned}
f_r &= - m g \frac{ \gpgradetwo{\Be_1^2 e^{i\beta}} }{ \gpgradetwo{ \Be_1^2 e^{-i\alpha} e^{i\beta} } } = - m g \frac{ i \sin\beta }{ i \sin\lr{ \beta - \alpha } } \\
f_s &= - m g \frac{ \gpgradetwo{\Be_1^2 e^{i\alpha}} }{ \gpgradetwo{ \Be_1^2 e^{-i\beta} e^{i\alpha} } } = m g \frac{ i \sin\alpha }{ i \sin\lr{ \beta - \alpha } }.
\end{aligned}
\end{equation}
The pseudoscalar factors \( i \) cancel out.  We are left with a ratio of sines, but those fell out of the grade selection directly, without requiring us to pull out our good old tables of trig identities.
} % answer
%}
