%
% Copyright © 2018 Peeter Joot.  All Rights Reserved.
% Licenced as described in the file LICENSE under the root directory of this GIT repository.
%
%{
%%%Any multivector can be expressed in terms of the curvilinear basis \( \setlr{ \Bx_{u_1}, \Bx_{u_2}, \cdots, \Bx_k} \), but computation of the curvilinear coordinates requires the reciprocal basis.
%%%
%%%For example, a vector \( \Bf \) constrained to the tangent space admits a representation
%%%
%%%\begin{dmath}\label{eqn:curvilinearDefinedCoordinates:380}
%%%\Bf = \sum_i a_i \Bx_{u_i}.
%%%\end{dmath}
%%%
%%%Dotting with \( \Bx^{u_j} \) gives
%%%
%%%\begin{dmath}\label{eqn:curvilinearDefinedCoordinates:280}
%%%\Bf \cdot \Bx^{u_j}
%%%= \sum_i a_i \Bx_{u_i} \cdot \Bx^{u_j}
%%%= \sum_i a_i {\delta^i}_j
%%%= a_j,
%%%\end{dmath}
%%%
%%%so
%%%\begin{dmath}\label{eqn:curvilinearDefinedCoordinates:300}
%%%\Bf = \sum_i \lr{ \Bf \cdot \Bx^{u_i} } \Bx_{u_i}.
%%%\end{dmath}
%%%
\subsubsection{Bivector coordinates.}
Higher grade multivector objects may also be represented in curvilinear coordinates.  Illustrating by example, we will calculate the coordinates of a
bivector constrained to a three parameter manifold \( \Span \setlr{ \Bx_1, \Bx_2, \Bx_3 } \) which can be represented as

\begin{equation}\label{eqn:curvilinearDefinedCoordinates:320}
B
= \inv{2} \sum_{i, j} B^{ij} \Bx_{i} \wedge \Bx_{j}
= \sum_{i < j} B^{ij} \Bx_{i} \wedge \Bx_{j}.
\end{equation}

The coordinates \( B^{ij} \) can be determined by dotting \( B \) with \( \Bx^{j} \wedge \Bx^{i} \), where \( i \ne j \), yielding

\begin{dmath}\label{eqn:curvilinearDefinedCoordinates:340}
B \cdot \lr{ \Bx^{j} \wedge \Bx^{i} }
=
\inv{2} \sum_{r, s} B^{rs} \lr{ \Bx_{r} \wedge \Bx_{s} } \cdot \lr{ \Bx^{j} \wedge \Bx^{i} }
=
\inv{2} \sum_{r, s} B^{rs} \lr{ \lr{ \Bx_{r} \wedge \Bx_{s} } \cdot \Bx^{j} } \cdot \Bx^{i}
=
\inv{2} \sum_{r, s} B^{rs} \lr{ \Bx_{r} {\delta_s}^j - \Bx_{s} {\delta_r}^j } \cdot \Bx^{i}
=
\inv{2} \sum_{r, s} B^{rs} \lr{ {\delta_r}^i {\delta_s}^j - {\delta_s}^i {\delta_r}^j }
=
\inv{2} \lr{ B^{i j} - B^{j i} }.
\end{dmath}

We see that the coordinates of a bivector, even with respect to a non-orthonormal basis, are antisymmetric, so
\cref{eqn:curvilinearDefinedCoordinates:340} is just \( B^{ij} \) as claimed.  That is

\begin{dmath}\label{eqn:curvilinearDefinedCoordinates:401}
B^{ij} = B \cdot \lr{ \Bx^{j} \wedge \Bx^{i} }.
\end{dmath}

Just as the reciprocal frame was instrumental for computation of the coordinates of a vector with respect to an arbitrary (i.e. non-orthonormal frame), we use the reciprocal frame to calculate the coordinates of a bivector, and could do the same for higher grade k-vectors as well.
%}
