%
% Copyright � 2016 Peeter Joot.  All Rights Reserved.
% Licenced as described in the file LICENSE under the root directory of this GIT repository.
%
\index{reverse}
\makedefinition{Reverse}{dfn:reverse:1}{

\index{\(A^\dagger\)}
Let \( A \) be a multivector with j multivector factors,
\( A = B_1 B_2 \cdots B_j \),
not necessarily orthogonal.
The reverse \( A^\dagger \), or reversion, of this multivector \( A \) is
\begin{equation*}
A^\dagger = B_j^\dagger B_{j-1}^\dagger \cdots B_1^\dagger.
\end{equation*}
Scalars and vectors are their own reverse, and
the reverse of a sum of multivectors is the sum of the reversions of its summands.
} % definition

Examples:
\begin{dmath}\label{eqn:reverseDefined:21}
\begin{aligned}
\lr{ 1 + 2 \Be_{12} + 3 \Be_{321} }^\dagger &= 1 + 2 \Be_{21} + 3 \Be_{123} \\
\lr{ (1 + \Be_1)(\Be_{23} - \Be_{12}) }^\dagger &= (\Be_{32} + \Be_{12})(1 + \Be_1).
\end{aligned}
\end{dmath}

FIXME: moved.  incorporate.

\Cref{thm:multiplication:anticommutationNormal} 
can also be applied to any pairs of orthogonal vectors in a arbitrary k-vector, for example
\begin{dmath}\label{eqn:normalVectors:300}
\Be_3 \Be_2 \Be_1
= (\Be_3 \Be_2) \Be_1
= -(\Be_2 \Be_3) \Be_1
= -\Be_2 (\Be_3 \Be_1)
= +\Be_2 (\Be_1 \Be_3)
= +(\Be_2 \Be_1) \Be_3
= -\Be_1 \Be_2 \Be_3,
\end{dmath}
showing that reversal of all the factors in a trivector such as \( \Be_1 \Be_2 \Be_3 \) toggles the sign.

