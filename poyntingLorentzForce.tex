%
% Copyright � 2018 Peeter Joot.  All Rights Reserved.
% Licenced as described in the file LICENSE under the root directory of this GIT repository.
%
%{
In this section we show that the
conservation laws \cref{eqn:poyntingF:980} associated with the tensor components \( T(\Be_k) \) relate the time rate of change of the Poynting vector to the continuum Lorentz force equation.
The Poynting and Lorentz force relations can be shown by summing \cref{eqn:poyntingF:980} over all the unit vector directions

\begin{dmath}\label{eqn:poyntingLorentzForce:20}
\sum_{k = 1}^3
c \lr{ \spacegrad \cdot \BT(\Be_k) } \Be_k
+
\sum_{k = 1}^3
\PD{t}{} \gpgradezero{ T(\Be_k) } \Be_k
=
\frac{1}{2 \eta}
\sum_{k = 1}^3
\gpgradezero{ \Be_k \lr{ F^\dagger J + J^\dagger F} } \Be_k.
\end{dmath}

From \cref{eqn:poyntingF:800} we see that

\begin{dmath}\label{eqn:poyntingLorentzForce:40}
\gpgradezero{ T(\Be_k) }
=
-\inv{c} \BS \cdot \Be_k.
\end{dmath}

The gradient terms submit to a bit of manipulation and notational sugar

\begin{dmath}\label{eqn:poyntingLorentzForce:260}
\sum_{k = 1}^3
\lr{ \spacegrad \cdot \BT(\Be_k) } \Be_k
=
\sum_{k,m = 1}^3
\Be_k \partial_m \BT(\Be_k) \cdot \Be_m
=
\sum_{k,m = 1}^3
\Be_k \partial_m T_{km}
=
\sum_{k,m = 1}^3
\partial_m T_{mk} \Be_k
=
\sum_{k,m = 1}^3
\partial_m \BT(\Be_m).
\end{dmath}

If we allow the partials to act on \( \BT \),
this shows that we can parameterize the Maxwell stress tensor by the gradient itself

\begin{dmath}\label{eqn:poyntingLorentzForce:280}
\sum_{k = 1}^3
\lr{ \spacegrad \cdot \BT(\Be_k) } \Be_k
=
\BT(\spacegrad),
\end{dmath}
which really means

\begin{dmath}\label{eqn:poyntingLorentzForce:300}
\BT(\spacegrad)
=
\frac{\epsilon}{2} \gpgradeone{ F \spacegrad F^\dagger },
\end{dmath}
where \( \spacegrad \) acts bidirectionally using the chain rule on both \( F \) and \( F^\dagger \).

To reduce the current terms, we will consider electric and magnetic sources in sequence.
Starting with an electric current density
\( \BJ = \rho \Bv \), for which the multivector current is

\begin{equation}\label{eqn:poyntingLorentzForce:60}
J = \frac{\rho}{\epsilon}\lr{ 1 - \Bv_\txte/c } = J^\dagger,
\end{equation}
the grade selection of the current term is

\begin{dmath}\label{eqn:poyntingLorentzForce:80}
\inv{2 \eta} \gpgradezero{ \Be_k \lr{ F^\dagger J + J^\dagger F} }
=
\frac{\rho}{2 \epsilon \eta} \gpgradezero{ \Be_k \lr{ (\BE - I \eta \BH)(1 - \Bv_\txte/c)  + (1 - \Bv_\txte/c)( \BE + I \eta \BH )} }
=
\frac{\rho}{2 \epsilon \eta} \Be_k \cdot \gpgradeone{ (\BE - I \eta \BH)(1 - \Bv_\txte/c)  + (1 - \Bv_\txte/c)( \BE + I \eta \BH )}
=
\frac{\rho}{\epsilon \eta} \Be_k \cdot \lr{ \BE + I \eta \BH \wedge \Bv_\txte/c }
=
\frac{\rho}{\epsilon \eta} \Be_k \cdot \lr{ \BE + \Bv_\txte \cross \BB }
=
c\rho \Be_k \cdot \lr{ \BE + \Bv_\txte \cross \BB }.
\end{dmath}

Given a magnetic current density \( \BM = \Bv_\txtm \rho_\txtm \) the multivector current is

\begin{equation}\label{eqn:poyntingLorentzForce:120}
J = I \rho_\txtm \lr{ 1 - \Bv_\txtm/c } = -J^\dagger.
\end{equation}

This time we have

\begin{dmath}\label{eqn:poyntingLorentzForce:180}
\inv{2 \eta} \gpgradezero{ \Be_k \lr{ F^\dagger J + J^\dagger F} }
=
\frac{\rho_\txtm}{2  \eta} \gpgradezero{ \Be_k \lr{ (\BE - I c \BB) I (1 - \Bv_\txtm/c)  - I (1 - \Bv_\txtm/c)( \BE + I c \BB )} }
=
\frac{\rho_\txtm}{2  \eta} \Be_k \cdot \gpgradeone{ (I \BE + c \BB)(1 - \Bv_\txtm/c)  + (1 - \Bv_\txtm/c)( -I \BE + c \BB )}
=
\frac{\rho_\txtm}{ \eta} \Be_k \cdot \lr{ c \BB - I \BE \wedge \Bv_\txtm/c }
=
c \epsilon \rho_\txtm \Be_k \cdot \lr{ c \BB - \frac{\Bv_\txtm}{c} \cross \BE }.
\end{dmath}

Putting the pieces together we have

\begin{dmath}\label{eqn:poyntingLorentzForce:100}
\rho \lr{ \BE + \Bv_\txte \cross \BB }
+ \epsilon \rho_\txtm \lr{ c \BB - \frac{\Bv_\txtm}{c} \cross \BE }
=
\BT(\spacegrad)
-
\inv{c^2}
\PD{t}{ \BS }.
\end{dmath}

When there are only electric sources, we have the continuum equivalent of the Lorentz force on the RHS, and can make the identification of \( \rho \lr{ \BE + \Bv_\txte \cross \BB } \) with the force density acting on the charge and current distribution.
Incidentally, as a side effect, this shows the desired form of the dual Lorentz force equation for magnetic sources must be

\begin{dmath}\label{eqn:poyntingLorentzForce:140}
\frac{d\Bp}{dt} = \epsilon q_\txtm \lr{ c \BB - \frac{\Bv_\txtm}{c} \cross \BE },
\end{dmath}
so for (fictious) magnetic sources the electric field does no work, whereas for electric sources the magnetic field does no work.
If we identify

\begin{dmath}\label{eqn:poyntingLorentzForce:200}
\bcP_{\textrm{em}} = \inv{c^2} \BS/c^2,
\end{dmath}
as the momentum density of the field, and identify

\begin{dmath}\label{eqn:poyntingLorentzForce:220}
\bcP_{\textrm{mech}}
=
\rho \lr{ \BE + \Bv_\txte \cross \BB }
+ \rho_\txtm \lr{ \BB - \Bv_\txtm \cross \BE },
\end{dmath}
as the mechanical momentum density associated with the particles in the volume, then the total momentum density is
%\begin{dmath}\label{eqn:poyntingLorentzForce:240}
\boxedEquation{eqn:poyntingLorentzForce:240}{
\PD{t}{} \lr{
\bcP_{\textrm{mech}}
+
\bcP_{\textrm{em}}
}
=
\BT(\spacegrad).
}
%\end{dmath}

We'd like the integrate \cref{eqn:poyntingLorentzForce:240} over a volume to determine the forces on the charge and current distributions in that volume.
Before doing so, we have to first step back and consider what the volume integral of \( \BT(\spacegrad) \) means.
Using coordinates as an intermediate, we have

\begin{dmath}\label{eqn:poyntingLorentzForce:320}
\int_V dV \BT(\spacegrad)
=
\sum_k \int_V dV \Be_k \spacegrad \cdot \BT(\Be_k)
=
\sum_k \int_{\partial V} dA \Be_k \ncap \cdot \BT(\Be_k)
=
\sum_{k,m} \int_{\partial V} dA \Be_k n_m {\BT(\Be_k) \cdot \Be_m}
=
\sum_{k,m} \int_{\partial V} dA \Be_k n_m T_{km}
=
\int_{\partial V} dA n_m T_{mk} \Be_k
=
\int_{\partial V} dA n_m \BT(\Be_m)
=
\int_{\partial V} dA \BT(\ncap).
\end{dmath}

As \( \BT(\spacegrad) \) had the structure of a set of divergence operators, it is perhaps not surprising that this should be the result.

Now we can integrate over the volume to find the force on the charged particles in that region

%\begin{dmath}\label{eqn:poyntingLorentzForce:160}
\boxedEquation{eqn:poyntingLorentzForce:160}{
\BF =
\int_V dV \rho \lr{ \BE + \Bv_\txte \cross \BB }
+ \int_V dV \rho_\txtm \lr{ \BB - \Bv_\txtm \cross \BE }
=
\int_{\partial V} dA \BT(\ncap)
-
\inv{c^2}
\int_V dV
\PD{t}{ \BS }.
}
%\end{dmath}

This way of expressing the force equation is quite nice as the Maxwell stress tensor need only be evaluated in the normal direction.
%}
