%
% Copyright © 2017 Peeter Joot.  All Rights Reserved.
% Licenced as described in the file LICENSE under the root directory of this GIT repository.
%
The workhorse operator of geometric algebra is called grade selection, defined as
\index{grade selection}
\index{\(\gpgrade{M}{k}\)}
\index{\(\gpgradezero{M}\)}
\makedefinition{Grade selection operator}{dfn:gradeselection:gradeselection}{
Given a set of k-vectors \( M_k, k \in [0,N] \), and any multivector of their sum
\begin{equation*}
M = \sum_{i = 0}^N M_i,
\end{equation*}
the grade selection operator is defined as
\begin{equation*}\label{eqn:multivector_nomenclature:40}
\gpgrade{M}{k} = M_k.
\end{equation*}
Due to its importance, selection of the (scalar) zero grade is given the shorthand
\begin{equation*}
\gpgradezero{M} = \gpgrade{M}{0} = M_0.
\end{equation*}
}

To illustrate grade selection by example, given a multivector \( M = 3 - \Be_3 + 2 \Be_1 \Be_2 + 7 \Be_1 \Be_2 \Be_4 \), then
\begin{equation}\label{eqn:multivector_nomenclature:80}
\begin{aligned}
\gpgrade{M}{0} &= \gpgradezero{M} = 3 \\
\gpgrade{M}{1} &= - \Be_3 \\
\gpgrade{M}{2} &= 2 \Be_1 \Be_2 \\
\gpgrade{M}{3} &= 7 \Be_1 \Be_2 \Be_4.
\end{aligned}
\end{equation}

Grade selection is the fundamental operation of geometric algebra.  Grade selection will be used directly as a tool, and will
also be used to define a number of other auxillary operators,
including a 
generalized multivector dot product, and the wedge product which is related to the \R{3} cross product by a multivector constant, and shares some properties of the cross product in other dimensions.
