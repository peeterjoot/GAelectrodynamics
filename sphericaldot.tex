%
% Copyright � CCYY Peeter Joot.  All Rights Reserved.
% Licenced as described in the file LICENSE under the root directory of this GIT repository.
%
\makeproblem{Spherical coordinate basis orthogonality.}{problem:sphericaldot:1}{
\index{spherical coordinates}
Using scalar selection, show that the spherical curvilinear basis of \cref{eqn:curvilinearspherical:80} are all mutually orthogonal.
} % problem

\makeanswer{problem:sphericaldot:1}{
Computing the various dot products is made easier by noting that \( \Be_3 \) and \( e^{i \phi } \) commute, whereas \( e^{j\theta } \Be_3 = \Be_3 e^{-j\theta}, \Be_1 e^{i\phi} = e^{-i\phi} \Be_1, \Be_2 e^{i\phi} = e^{-i\phi} \Be_2 \) (since \( \Be_3 j \), \( \Be_1 i \) and \( \Be_2 i \) all anticommute.)  Also note that
\begin{equation}\label{eqn:sphericaldot:240}
\begin{aligned}
j \phicap
&= \Be_{31} e^{i\phi} \Be_2 e^{i\phi} \\
&= \Be_{312} e^{-i\phi} e^{i\phi} \\
&= I.
\end{aligned}
\end{equation}
The dot products, working with the normalized vectors, are
\begin{subequations}
\label{eqn:sphericaldot:160}
\begin{equation}\label{eqn:sphericaldot:180}
\begin{aligned}
\rcap \cdot \thetacap
&=
\gpgradezero{
\rcap \rcap j
} \\
&=
\gpgradezero{
j
} \\
&= 0
\end{aligned}
\end{equation}
\begin{equation}\label{eqn:sphericaldot:200}
\begin{aligned}
\rcap \cdot \phicap
&=
\gpgradezero{
\Be_3 e^{j \theta} \phicap
} \\
&=
\gpgradezero{
\Be_3 \lr{ \cos\theta + j \sin\theta } \phicap
} \\
&=
\cos\theta
\gpgradezero{
\Be_3
\phicap
}
+
\sin\theta
\gpgradezero{
\Be_3 j \phicap
}
\\
&=
\cos\theta
\gpgradezero{
\Be_{32} \cos\phi + \Be_{13} \sin\phi
}
+
\sin\theta
\gpgradezero{
\Be_{12}
}
\\
&=
0
\end{aligned}
\end{equation}
\begin{equation}\label{eqn:sphericaldot:220}
\begin{aligned}
\thetacap \cdot \phicap
&=
\gpgradezero{
\rcap j \phicap
} \\
&=
\gpgradezero{
\rcap I
} \\
&=
0.
\end{aligned}
\end{equation}
\end{subequations}
} % answer
