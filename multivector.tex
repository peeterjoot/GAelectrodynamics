%\chapter{Multivector spaces}
\section{Multivectors}

\subsection{Multivector space.}

The first lesson that must be learned in the study in GA, is to unlearn claims
%\footnote{I heard such claims from high school math and physics teachers.}
that vectors cannot be multiplied.
Instead we start by assuming that a multiplication operation between any number of vectors can be defined,
and that sums, called multivectors, of scalars, vectors, and any products of vectors are also well defined.
These rules are formalized in the definition of a multivector space.

\makedefinition{Multivector space.}{def:multiplication:multivectorspace}{
   Given a (generating) vector space \( V \) with a basis \( \setlr{ \Bx_1, \Bx_2, \cdots } \), a multivector is any sum

   \( a_0 + \sum_i a_i \Bx_i + \sum_{ij} a_{ij} \Bx_i \Bx_j + \sum_{ijk} \Bx_i \Bx_j \Bx_k + \cdots \), where \( a_0, a_i, a_{ij}, \cdots \) are scalars,
and vector multiplication is represented by juxtaposition.

   A multivector space is a set \( M = \setlr{ x, y, z, \cdots } \) of multivectors, where the following axioms are satisfied

\begin{tcolorbox}[tab2,tabularx={X|Y},title=Multivector space axioms.,boxrule=0.5pt]
    Contraction. & \( \Bx^2 = \Bx \cdot \Bx, \forall \Bx \in V \) \\ \hline
    Addition is closed. & \( x + y \in M \) \\ \hline
    Multiplication is closed. & \( x y \in M \) \\ \hline
    Addition is associative. & \( (x + y) + z = x + (y + z) \) \\ \hline
    Addition is commutative. & \( y + x = x + y \) \\ \hline
    There exists a zero element \( 0 \in M \).  & \( x + 0 = x \) \\ \hline
    There exists a negative additive inverse \( -x \in M \). & \( x + (-x) = 0 \) \\ \hline
    Multiplication is distributive.  & \( z( x + y ) = z x + z y \), \( (z + w)x = z x + w x \) \\ \hline
    Multiplication is associative. & \( (x y) z = x ( y z ) \) \\ \hline
    There exists a multiplicative identity \( 1 \). & \( 1 x = x \) \\ \hline
\end{tcolorbox}
}

The definition of a multivector space is strikingly similar to that of a vector space
\cref{def:prerequisites:vectorspace}, with the following changes

\begin{itemize}
\item References to vectors in the axioms, are changed to multivectors.
\item Scalar multiplication is generalized to multivector multiplication (of which scalar multiplication is a special case).
\item The vector addition operation is generalized to multivector addition.
\item A vector multiplication operation is presumed.
\item A rule that specifies the meaning of a product of a vector with itself is provided (the contraction axiom).
\end{itemize}

Some work is required to systematically examine the consequences of this abstract definition.

\subsection{Nomenclature: grade, k-vector, bivector, trivector.}

To facilitate discussion, it is useful to first introduce some nomenclature.

%\makedefinition{Vector multiplication.}{dfn:multivector:vectormultiplication}{
%Given elements \( \Bx, \By, \Bz \) from a vector space with an associated dot product, define a multiplication operation,
%written using juxtaposition (\( \Bx \By, \Bx \By \Bz, \cdots \)), with properties
%
%\begin{tcolorbox}[tab2,tabularx={X|Y},title=Vector multiplication properties.,boxrule=0.5pt]
%    Contraction. &\( \Bx^2 = \Bx \cdot \Bx \) \\ \hline
%    Distribution.  & \( \Bz( \Bx + \By ) = \Bz \Bx + \Bz \By \),
%\( (\Bx + \By)\Bz = \Bx \Bz + \By \Bz \) \\ \hline
%    Associativity. & \( (\Bx \By) \Bz = \Bx ( \By \Bz ) \) \\ \hline
%    Non-commutativity (order matters). & In general, \( \Bx \By \ne \By \Bx \) \\ \hline
%\end{tcolorbox}
%} % definition

\makedefinition{Grade and k-vector.}{dfn:multivector:kvector}{
A product of
\( k \) mutually perpendicular vectors, or a sum of such products, is called a k-vector.
The number \( k \) of such products is called the grade.
Scalars are grade zero k-vectors.  
} % definition

Multivectors are therefore sums of k-vectors with different grades.  

Examples of k-vectors with grades 0, 1, 2, and 3 respectively are

\begin{dmath}\label{eqn:multivector:180}
\begin{aligned}
&1 \\
&\Be_1, \Be_2, \Be_3 \\
&\Be_1 \Be_2, \Be_2 \Be_1, \Be_1 \Be_2 + \Be_2 \Be_3, \Be_1 \Be_2 + \Be_3 \Be_4 \\
&\Be_1 \Be_2 \Be_3, \Be_1 \Be_3 \Be_2, \Be_1 \Be_4 \Be_2
\end{aligned}
\end{dmath}

\makedefinition{Bivector.}{dfn:multivector:bivector}{
A bivector, or 2-vector, is a k-vector with grade 2.
} % definition

Each of \( \Be_1 \Be_2, \Be_2 \Be_1, \Be_1 \Be_2 + \Be_2 \Be_3 \), and \( \Be_1 \Be_2 + \Be_3 \Be_4 \) are bivectors.
%All but the last of these represents an oriented plane segment.

\makedefinition{Trivector.}{dfn:multivector:trivector}{
A trivector, or 3-vector, is a k-vector with grade 3.
} % definition

Each of \( \Be_1 \Be_2 \Be_3, \Be_1 \Be_3 \Be_2, \Be_1 \Be_4 \Be_2 \) are trivectors.
% , and represent oriented volumes.

\paragraph{Mixed grade sums}
In traditional vector algebra, the 
``weird'' sum of a scalar and vector is forbidden and undefined, but is explicitly allowed in GA.  For example, 

\begin{dmath}\label{eqn:multivector:240}
1 + \Be_1,
\end{dmath}

is a simple mixed grade multivector.
Such mixed grade mathematical objects are not only well defined in GA, but are required to represent some vector products.  One of the simplest examples is the following vector product

\begin{dmath}\label{eqn:multivector:260}
\Be_1 ( \Be_1 + \Be_2 )
=
\Be_1 \Be_1 + \Be_1 \Be_2
=
\Be_1 \cdot \Be_1 + \Be_1 \Be_2
=
1 + \Be_1 \Be_2,
\end{dmath}

where the last step assumes the vector space is Euclidean.

\subsection{Unpacking the axioms}

Until otherwise stated, we assume a Euclidean vector space with an orthonormal basis \( \setlr{\Be_1, \Be_2, \cdots } \).  Generalizations for non-Euclidean spaces are left to the reader.

By the contraction axiom, the square of a unit vector is also unity

%\begin{equation}\label{eqn:multiplication:60}
\boxedEquation{eqn:multiplication:60}{
\Be_i^2 = 1.
}
%\end{equation}

With this implication noted, now consider the square of a simple two dimensional vector

\begin{dmath}\label{eqn:gaTutorial:80}
2
=
(\Be_1 + \Be_2)^2
= (\Be_1 + \Be_2)(\Be_1 + \Be_2)
= \Be_1^2 + \Be_2 \Be_1 + \Be_1 \Be_2 + \Be_2^2
= 2 + \Be_2 \Be_1 + \Be_1 \Be_2.
\end{dmath}

The sum above with both scalar terms and terms that are composed of products of vectors is called a multivector.
A product of two perpendicular vectors (or a sum of such products) is called a bivector, and can be used to represent an oriented plane.
Geometric Algebra allows for sums of scalars, vectors, bivectors, and higher degree products.

Observe that for this identity to hold, the bivector terms must sum to zero.
That is

%\begin{dmath}\label{eqn:multiplication:140}
\boxedEquation{eqn:multiplication:140}{
\Be_1 \Be_2 = -\Be_1 \Be_2.
}
%\end{dmath}

This implies that the product of two orthonormal vectors anticommutes.
In general it is also true that

\maketheorem{Normal anticommutation}{thm:multiplication:anticommutationNormal}{
The product of any two normal vectors \(\Bu\), and \(\Bv\) anticommute.
\begin{equation*}
\Bu \Bv = -\Bv \Bu.
\end{equation*}
} % theorem

%


\subsection{Problems}
\makeproblem{One dimensional multivector space.}{problem:multivector:40}{
   Verify that for \( c, d \in \bbR \) the set \( M = \setlr{ c + d \Be_1 } \) satisifies all the multivector axioms.
} % problem

\subsection{Orientation.  figure out where to put this}
Geometric algebra provides a mathematical representation for geometrical objects of each dimension in the space.
In a three dimensional space, there are representations for all of

\begin{itemize}
\item
oriented (signed) points with magnitude
\item
oriented line segments,
\item
oriented planes,
\item
oriented volumes,
\end{itemize}

and in higher dimensional spaces, it will be possible to represent higher dimensonal oriented hypervolumes.

