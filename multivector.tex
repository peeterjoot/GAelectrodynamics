\section{Multivector}
Geometric Algebra (\boldTextAndIndex{GA}) generalizes the concept of vector and a normed vector space, introducing a vector multiplication operation into the mix.  
To detail that generalization, it is helpful to first recall the definition of a vector space

\makedefinition{Vector space.}{def:multiplication:vectorspace}{
A vector space is a set \( V \), the elements of which are called vectors.
For all vectors \( \Bx, \By, \Bz \in V \) and scalars \( a, b \in \bbR \),
the addition and multiplication operations must satisfy the following axioms

\begin{tcolorbox}[tab2,tabularx={X|Y},title=Vector space axioms.,boxrule=0.5pt]
    Addition is closed. & \( \Bx + \By \in V \) \\ \hline
    (Scalar) multiplication is closed. & \( a \Bx \in V \) \\ \hline
    Addition is associative. & \( (\Bx + \By) + \Bz = \Bx + (\By + \Bz) \) \\ \hline
    Addition is commutative. & \( \By + \Bx = \Bx + \By \) \\ \hline
    There exists a zero element \( 0 \in V \).  & \( \Bx + 0 = \Bx \) \\ \hline
    For any \( \Bx \in V \) there exists a negative additive inverse \( -\Bx \in V \). & \( \Bx + (-\Bx) = 0 \) \\ \hline
    (Scalar) multiplication is distributive.  & \( a( \Bx + \By ) = a \Bx + a \By \), \( (a + b)\Bx = a \Bx + b\Bx \) \\ \hline
    (Scalar) multiplication is associative. & \( (a b) \Bx = a ( b \Bx ) \) \\ \hline
    There exists a multiplicative identity \( 1 \). & \( 1 \Bx = \Bx \) \\ \hline
\end{tcolorbox}
}

A vector space that also defines a length operation \( \Norm{\Bx} \) is called a normed vector space.  Geometric algebras are built from normed vector spaces as follows

\makedefinition{Multivector space.}{def:multiplication:multivectorspace}{
   Given a normed vector space \( V \), with elements \( \setlr{ \Bx, \By, \cdots, \Bz} \in V \),
   a multivector can be formed from any product of one or more of these vectors \( \Bx \By \cdots \Bz \), a scalar multiple of such a product, or the sum thereof.  These vector products are constrained by the contraction axiom, a requirement that,
   for any vector \( \Bx \in V \) the square of that vector is the squared length of that vector

\begin{equation*}
    \Bx^2 = \Norm{\Bx}^2.
\end{equation*}

A multivector space \( M \) is the set of all possible multivectors that can be formed from the generating vector space \( V \).  All multivectors \( x, y, z \in M \) must satisfy the following additional axioms
\begin{tcolorbox}[tab2,tabularx={X|Y},title=Multivector space axioms.,boxrule=0.5pt]
    Addition is closed. & \( x + y \in M \) \\ \hline
    Multiplication is closed. & \( x y \in M \) \\ \hline
    Addition is associative. & \( (x + y) + z = x + (y + z) \) \\ \hline
    Addition is commutative. & \( y + x = x + y \) \\ \hline
    There exists a zero element \( 0 \in M \).  & \( x + 0 = x \) \\ \hline
    There exists a negative additive inverse \( -x \in M \). & \( x + (-x) = 0 \) \\ \hline
    Multiplication is distributive.  & \( z( x + y ) = z x + z y \), \( (z + w)x = z x + w x \) \\ \hline
    Multiplication is associative. & \( (x y) z = x ( y z ) \) \\ \hline
    There exists a multiplicative identity \( 1 \). & \( 1 x = x \) \\ \hline
\end{tcolorbox}
}

Some work is required to systematically examine the consequences of this unfortunately abstract definition.  

Observe first that since a scalar multiple of the square of a vector is as scalar by the definition above,
any scalar is also a multivector.  For example, if \( \Be_1 \) is the unit vector along the x-axis and \( s \) is a scalar, then

\begin{equation}\label{eqn:multivector:n}
   x = s \Be_1^2 = s,
\end{equation}

is a multivector.  Since vectors (a product of one vector, or a scalar multiple thereof) is also a multivector, this
means that vectors are multivectors, and that ``wierd'' sums of scalars and vectors, such as

\begin{dmath}\label{eqn:multivector:n}
   x = 1 + \Be_1,
\end{dmath}

are also multivectors!

\makeproblem{One dimensional multivector space.}{problem:multivector:1}{
   Verify that for \( c, d \in \bbR \) the set \( M = \setlr{ c + d \Be_1 } \) satisifies all the multivector axioms.
} % problem

A scalar can be interpretted as a quantity with magnitude, but no direction.  A vector can be interpretted geometrically
as a 
object with direction and magnitude, so we see that a multivector can represent either or both such quantities.  We will also see that multivectors will be able to represent oriented planes (i.e. planes with a definitive notion of an ``up'' or ``down'' direction).  Multivectors will also be able to represent volumes and higher dimensional analogues of oriented planes.

\section{Real Euclidean multivectors}

To do so, consider the multivectors that can be formed from the real Euclidean vector space \R{N}.  Elements of this vector space will be expressed as linear combinations of a presumed standard basis \( \setlr{ \Be_i, i \in [1, N] } \) for that space, a complete set of mutually perpendicular unit vectors.

%%\makedefinition{Scalar}{def:multiplication:scalar}{
%%   A (real) number with no implied direction.
%%}
%%
%%Examples of scalars are \( \pi, 3, -4 \), and \( 0 \).
%%
%%\makedefinition{Vector}{def:multiplication:vector}{
%%%\href{https://www.youtube.com/watch?v=bOIe0DIMbI8}{A quantity with direction and magnitude.}
%%\href{https://youtu.be/bOIe0DIMbI8?t=19}{A quantity with direction and magnitude.}
%%}
%%
%%In this book, 
%%In order to express
%%\begin{dmath}\label{eqn:multivector:n}
%%\Bx = c_1 \Be_1 + c_2 \Be_2,
%%\end{dmath}
%%
%%where \( \Be_1 \) and \( \Be_2 \) are a pair of perpendicular vectors of length one along the x and y axis respectively, as illustrated in 
%%
%%FIXME: figure.
%%These 
%%
%%, as represented pictorially as an arrow 
%%
%%
%%
%%\section{Vector space}
%%\section{Vector multiplication}
%%\section{Multivector}
%%
%%Geometric Algebra, or \boldTextAndIndex{GA} defines a multiplication operation for vectors.
%%GA also
%%generalizes the concept of a vector, introducing a new type of mathematical object, the multivector.  
%%
%%
%%
%%In traditional vector algebra, a sum of a scalar and a vector, such as
%%
%%\begin{dmath}\label{eqn:multivector:n}
%%M = 1 + 2 \Be_1,
%%\end{dmath}
%%
%%is not considered meaningful.  This is 
%%
%%Vectors and scalars, or their sums
%%It is assumed here that the student is familiar with coordinate representations of vectors, the concepts of 
%%The reader should be familiar with 
%%Vectors will be represented algebraically as scaled sums 
%%A vector may also be represented algebraically in terms  as the sum of directed 
%%or in its algebraic sense, such as the \R{2} vector
%%
%%Vectors, and abstraction representing quantities with magnitude and orientation are types of multivectors.  Scalars (numbers), which have magnitude but no orientation, are also multivectors.  Multivectors that represented oriented areas, oriented volumes, and oriented higher dimensional spaces will also be defined.  The 
%%
%%.  Sets of multivectors can be assembled 
%%
%%s and multivector spaces, and introduces a multiplication operation for vectors
%%
%%, is built around the \boldTextAndIndex{multivector}.  The purpose of this section is to provide a definition of the multivector, and 
