%
% Copyright © 2023 Peeter Joot.  All Rights Reserved.
% Licenced as described in the file LICENSE under the root directory of this GIT repository.
%
%{
\makeproblem{Potentials for no-fictitious sources.}{problem:mvpotentials:1}{
Starting with Maxwell's equation with only conventional electric sources
\begin{equation}\label{eqn:mvpotentialsProblems:120}
\lr{ \spacegrad + \inv{c}\PD{t}{} } F = \gpgrade{J}{0,1}.
\end{equation}
Show that this may be split by grade into three equations
\begin{equation}\label{eqn:mvpotentialsProblems:140}
\begin{aligned}
\gpgrade{ \lr{ \spacegrad + \inv{c}\PD{t}{} } F}{0,1} &= \gpgrade{J}{0,1} \\
\spacegrad \wedge \BE + \inv{c}\PD{t}{} \lr{ I \eta \BH } &= 0 \\
\spacegrad \wedge \lr{ I \eta \BH } &= 0.
\end{aligned}
\end{equation}
Then use the identities \( \spacegrad \wedge \spacegrad \wedge \BA = 0 \), for vector \( \BA \) and \( \spacegrad \wedge \spacegrad \phi = 0 \), for scalar \( \phi \) to find the potential representation \cref{eqn:mvpotentials:80}.
} % problem
\makeanswer{problem:mvpotentials:1}{
Taking grade(0,1) and (2,3) selections of Maxwell's equation, we split our equations into source dependent and source free equations
\begin{equation}\label{eqn:mvpotentialsProblems:200}
\gpgrade{ \lr{ \spacegrad + \inv{c} \PD{t}{} } F }{0,1} = \gpgrade{J}{0,1},
\end{equation}
\begin{equation}\label{eqn:mvpotentialsProblems:220}
\gpgrade{ \lr{ \spacegrad + \inv{c} \PD{t}{} } F }{2,3} = 0.
\end{equation}

In terms of \( F = \BE + I \eta \BH \), the source free equation expands to
\begin{equation}\label{eqn:mvpotentialsProblems:240}
\begin{aligned}
0
&=
\gpgrade{
\lr{ \spacegrad + \inv{c} \PD{t}{} } \lr{ \BE + I \eta \BH }
}{2,3} \\
&=
\gpgradetwo{\spacegrad \BE}
+ \gpgradethree{I \eta \spacegrad \BH} + I \eta \inv{c} \PD{t}{\BH} \\
&=
\spacegrad \wedge \BE
+ \spacegrad \wedge \lr{ I \eta \BH }
+ I \eta \inv{c} \PD{t}{\BH},
\end{aligned}
\end{equation}
which can be further split into a bivector and trivector equation
\begin{equation}\label{eqn:mvpotentialsProblems:260}
0 = \spacegrad \wedge \BE + I \eta \inv{c} \PD{t}{\BH}
\end{equation}
\begin{equation}\label{eqn:mvpotentialsProblems:280}
0 = \spacegrad \wedge \lr{ I \eta \BH }.
\end{equation}
It's clear that we want to write the magnetic field as a (bivector) curl, so we let
\begin{equation}\label{eqn:mvpotentialsProblems:300}
I \eta \BH = I c \BB = c \spacegrad \wedge \BA,
\end{equation}
or
\begin{equation}\label{eqn:mvpotentialsProblems:301}
\BH = \inv{\mu} \spacegrad \cross \BA.
\end{equation}

\Cref{eqn:mvpotentialsProblems:260} is reduced to
\begin{equation}\label{eqn:mvpotentialsProblems:320}
\begin{aligned}
0
&= \spacegrad \wedge \BE + I \eta \inv{c} \PD{t}{\BH} \\
&= \spacegrad \wedge \BE + \inv{c} \PD{t}{} \spacegrad \wedge \lr{ c \BA } \\
&= \spacegrad \wedge \lr{ \BE + \PD{t}{\BA} }.
\end{aligned}
\end{equation}
We can now let
\begin{equation}\label{eqn:mvpotentialsProblems:340}
\BE + \PD{t}{\BA} = -\spacegrad \phi.
\end{equation}
We sneakily adjust the sign of the gradient so that the result matches the conventional representation.
} % answer
%
%
\makeproblem{Potentials for fictitious sources.}{problem:mvpotentials:2}{
Starting with Maxwell's equation with only fictitious magnetic sources
\begin{equation}\label{eqn:mvpotentialsProblems:160}
\lr{ \spacegrad + \inv{c}\PD{t}{} } F = \gpgrade{J}{2,3},
\end{equation}
show that this may be split by grade into three equations
\begin{equation}\label{eqn:mvpotentialsProblems:180}
\begin{aligned}
\gpgrade{ \lr{ \spacegrad + \inv{c}\PD{t}{} } I F}{0,1} &= I \gpgrade{J}{2,3} \\
-\eta \spacegrad \wedge \BH + \inv{c}\PD{t}{(I \BE)} &= 0 \\
\spacegrad \wedge \lr{ I \BE } &= 0.
\end{aligned}
\end{equation}
Then use the identities \( \spacegrad \wedge \spacegrad \wedge \BF = 0 \), for vector \( \BF \) and \( \spacegrad \wedge \spacegrad \phi_m = 0 \), for scalar \( \phi_m \) to find the potential representation \cref{eqn:mvpotentials:100}.
} % problem
\makeanswer{problem:mvpotentials:2}{
We multiply \cref{eqn:mvpotentialsProblems:160} by \( I \) to find
\begin{equation}\label{eqn:mvpotentialsProblems:360}
\lr{ \spacegrad + \inv{c}\PD{t}{} } I F = I \gpgrade{J}{2,3},
\end{equation}
which can be split into
\begin{equation}\label{eqn:mvpotentialsProblems:380}
\begin{aligned}
\gpgrade{ \lr{ \spacegrad + \inv{c}\PD{t}{} } I F }{1,2} &= I \gpgrade{J}{2,3} \\
\gpgrade{ \lr{ \spacegrad + \inv{c}\PD{t}{} } I F }{0,3} &= 0.
\end{aligned}
\end{equation}
We expand the source free equation in terms of \( I F = I \BE - \eta \BH \), to find
\begin{equation}\label{eqn:mvpotentialsProblems:400}
\begin{aligned}
0
&= \gpgrade{ \lr{ \spacegrad + \inv{c}\PD{t}{} } \lr{ I \BE - \eta \BH } }{0,3} \\
&= \spacegrad \wedge \lr{ I \BE } + \inv{c} \PD{t}{(I \BE)} - \eta \spacegrad \wedge \BH,
\end{aligned}
\end{equation}
which has the respective bivector and trivector grades
\begin{equation}\label{eqn:mvpotentialsProblems:420}
0 = \spacegrad \wedge \lr{ I \BE }
\end{equation}
\begin{equation}\label{eqn:mvpotentialsProblems:440}
0 = \inv{c} \PD{t}{(I \BE)} - \eta \spacegrad \wedge \BH.
\end{equation}
We can clearly satisfy \cref{eqn:mvpotentialsProblems:420} by setting
\begin{equation}\label{eqn:mvpotentialsProblems:460}
I \BE = -\inv{\epsilon} \spacegrad \wedge \BF,
\end{equation}
or
\begin{equation}\label{eqn:mvpotentialsProblems:461}
\BE = -\inv{\epsilon} \spacegrad \cross \BF.
\end{equation}
Here, once again, the sneaky inclusion of a constant factor \( -1/\epsilon \) is to make the result match the conventional.  Inserting this value for \( I \BE \) into our bivector equation yields
\begin{equation}\label{eqn:mvpotentialsProblems:480}
\begin{aligned}
0
&= -\inv{\epsilon} \inv{c} \PD{t}{} (\spacegrad \wedge \BF) - \eta \spacegrad \wedge \BH \\
&= -\eta \spacegrad \wedge \lr{ \PD{t}{\BF} + \BH },
\end{aligned}
\end{equation}
so we set
\begin{equation}\label{eqn:mvpotentialsProblems:500}
\PD{t}{\BF} + \BH = -\spacegrad \phi_m,
\end{equation}
and have a field representation that automatically satisfies the source free equations.
} % answer
\makeproblem{Total field in terms of potentials.}{problem:mvpotentials:3}{
Prove \cref{lemma:mvpotentials:1}, either by direct expansion, or by trying to discover the multivector form of the field by construction.
} % problem
\makeanswer{problem:mvpotentials:3}{
Proof by expansion is straightforward, and left to the reader.  Here we will start with \cref{eqn:mvpotentials:80}, and \cref{eqn:mvpotentials:100}, and form the respective total electromagnetic field \( F = \BE + I \eta H \) for each case.

Starting with \cref{eqn:mvpotentials:80}, we find
\begin{equation}\label{eqn:mvpotentialsProblems:560}
\begin{aligned}
F
&= \BE + I \eta \BH \\
&= -\spacegrad \phi - \PD{t}{\BA} + I \frac{\eta}{\mu} \spacegrad \cross \BA \\
&= -\spacegrad \phi - \inv{c} \PD{t}{(c \BA)} + \spacegrad \wedge (c\BA) \\
&= \gpgrade{ -\spacegrad \phi - \inv{c} \PD{t}{(c \BA)} + \spacegrad \wedge (c\BA) }{1,2} \\
&= \gpgrade{ -\spacegrad \phi - \inv{c} \PD{t}{(c \BA)} + \spacegrad (c\BA) }{1,2} \\
&= \gpgrade{ \spacegrad \lr{ -\phi + c \BA } - \inv{c} \PD{t}{(c \BA)} }{1,2} \\
&= \gpgrade{ \lr{ \spacegrad -\inv{c} \PD{t}{} } \lr{ -\phi + c \BA } }{1,2}.
\end{aligned}
\end{equation}

For the field for the fictitious source case, we start with \cref{eqn:mvpotentials:100}, and compute the result in the same way, inserting a no-op grade selection to allow us to simplify.  We find
\begin{equation}\label{eqn:mvpotentialsProblems:580}
\begin{aligned}
F
&= \BE + I \eta \BH \\
&= -\inv{\epsilon} \spacegrad \cross \BF + I \eta \lr{ -\spacegrad \phi_m - \PD{t}{\BF} } \\
&= \inv{\epsilon c} I \lr{ \spacegrad \wedge (c \BF)} + I \eta \lr{ -\spacegrad \phi_m - \inv{c} \PD{t}{(c \BF)} } \\
&= I \eta \lr{ \spacegrad \wedge (c \BF) + \lr{ -\spacegrad \phi_m - \inv{c} \PD{t}{(c \BF)} } } \\
&= I \eta \gpgrade{ \spacegrad \wedge (c \BF) + \lr{ -\spacegrad \phi_m - \inv{c} \PD{t}{(c \BF)} } }{1,2} \\
&= I \eta \gpgrade{ \spacegrad (c \BF) - \spacegrad \phi_m - \inv{c} \PD{t}{(c \BF)} }{1,2} \\
&= I \eta \gpgrade{ \spacegrad (-\phi_m + c \BF) - \inv{c} \PD{t}{(c \BF)} }{1,2} \\
&= I \eta \gpgrade{ \lr{ \spacegrad -\inv{c} \PD{t}{} } (-\phi_m + c \BF) }{1,2}.
\end{aligned}
\end{equation}
} % answer
\makeproblem{Fields in terms of potentials.}{problem:mvpotentials:4}{
Prove \cref{lemma:mvpotentials:2}.
} % problem
\makeanswer{problem:mvpotentials:4}{
We start by expanding \( (\spacegrad - (1/c)\partial_t) A \)
%using
%\cref{dfn:unpackStaticPotential:80}
and then group by grade to find
\begin{equation}\label{eqn:gaugeTransformation:1111}
\begin{aligned}
\conjstgrad A
&=
\conjstgrad \lr{  - \phi
      + c \BA
      + \eta I \lr{ -\phi_m + c \BF } } \\
&=
- \spacegrad \phi + c \spacegrad \cdot \BA + c \spacegrad \wedge \BA + \inv{c} \PD{t}{\phi} - \PD{t}{\BA} \\
&\quad + I \eta
\lr{
- \spacegrad \phi_\txtm + c \spacegrad \cdot \BF + c \spacegrad \wedge \BF + \inv{c} \PD{t}{\phi_\txtm} - \PD{t}{\BF}
} \\
&=
c \spacegrad \cdot \BA
+ \inv{c} \PD{t}{\phi}
\\
&
+
\mathLabelBox[ labelstyle={below of=m\themathLableNode, below of=m\themathLableNode} ]
{
   - \spacegrad \phi
   - \PD{t}{\BA}
   - \inv{\epsilon} \spacegrad \cross \BF
}
{
\(\BE\)
}
+
\mathLabelBox[ labelstyle={below of=m\themathLableNode, below of=m\themathLableNode} ]
{
   I \eta
   \lr{
      - \spacegrad \phi_\txtm
      - \PD{t}{\BF}
      + \inv{\mu} \spacegrad \cross \BA
   }
}
{\(I \eta \BH\)
} \\
&
+ I \eta\lr{
  c \spacegrad \cdot \BF
+ \inv{c} \PD{t}{\phi_\txtm}
},
\end{aligned}
\end{equation}
which shows the claimed field split.

%%%Let's expand and then group by grade
%%%\begin{equation}\label{eqn:mvpotentialsProblems:600}
%%%\begin{aligned}
%%%\lr{ \spacegrad - \inv{c} \PD{t}{} } A
%%%&=
%%%\lr{ \spacegrad - \inv{c} \PD{t}{} } \lr{ -\phi + c \BA + I \eta \lr{ -\phi_m + c \BF }} \\
%%%&=
%%%-\spacegrad \phi + c \spacegrad \BA + I \eta \lr{ -\spacegrad \phi_m + c \spacegrad \BF }
%%%-\inv{c} \PD{t}{\phi} + c \inv{c} \PD{t}{ \BA } + I \eta \lr{ -\inv{c} \PD{t}{\phi_m} + c \inv{c} \PD{t}{\BF} } \\
%%%&=
%%%- \spacegrad \phi
%%%+ I \eta c \spacegrad \wedge \BF
%%%- c \inv{c} \PD{t}{\BA}
%%%\quad + c \spacegrad \wedge \BA
%%%-I \eta \spacegrad \phi_m
%%%- c I \eta \inv{c} \PD{t}{\BF} \\
%%%&\quad + c \spacegrad \cdot \BA
%%%+\inv{c} \PD{t}{\phi}
%%%\quad + I \eta \lr{ c \spacegrad \cdot \BF
%%%+ \inv{c} \PD{t}{\phi_m} } \\
%%%&=
%%%- \spacegrad \phi
%%%- \inv{\epsilon} \spacegrad \cross \BF
%%%- \PD{t}{\BA}
%%%\quad + I \eta \lr{
%%%\inv{\mu} \spacegrad \cross \BA
%%%- \spacegrad \phi_m
%%%- \PD{t}{\BF}
%%%} \\
%%%&\quad + c \spacegrad \cdot \BA
%%%+\inv{c} \PD{t}{\phi}
%%%\quad + I \eta \lr{ c \spacegrad \cdot \BF
%%%+ \inv{c} \PD{t}{\phi_m} }.
%%%\end{aligned}
%%%\end{equation}
Observing that \( F = \gpgrade{ \lr{ \spacegrad -(1/c) \PDi{t}{} } A }{1,2} = \BE + I \eta \BH \), completes the problem.
We may write just \( F = \lr{ \spacegrad -(1/c) \PDi{t}{} } A \), if
the Lorentz gauge condition is assumed, as the scalar and pseudoscalar components above are obliterated.
} % answer
%}
