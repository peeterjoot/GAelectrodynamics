%
% Copyright © 2017 Peeter Joot.  All Rights Reserved.
% Licenced as described in the file LICENSE under the root directory of this GIT repository.
%
The dot product has been identified with scalar grade selection.  This motivates a general
multivector dot product definition.

\index{multivector dot product}
\makedefinition{Multivector dot product}{dfn:generalizedDot:100}{
The dot (or inner) product of two multivectors
\( A = \sum_{i = 0}^N \gpgrade{A}{i}, B = \sum_{i = 0}^N \gpgrade{B}{i} \)
is defined as
\begin{equation*}
A \cdot B \equiv
\sum_{i,j = 0}^N \gpgrade{ A_i B_j }{\Abs{i - j}},
\end{equation*}
}

In many cases we are interested only in dot products between pairs of k-vectors.
If \( A, B \) are a k-vectors with grades \( i, j \) respectively, then their dot product is just a single grade selection operation

\begin{dmath}\label{eqn:generalizedDot:580}
A \cdot B = \gpgrade{ A B }{\Abs{i - j}}.
\end{dmath}

In particular, when the grades of two k-vectors are equal, the dot product is just the scalar selection of their product

\begin{dmath}\label{eqn:generalizedDot:580}
A \cdot B = \gpgradezero{ A B }.
\end{dmath}

As a first example, consider a specific vector-bivector dot product

\begin{dmath}\label{eqn:generalizedDot:560}
\lr{ \Be_1 + \Be_2 } \cdot \lr{ \Be_1 \Be_2 + 3 \Be_2 \Be_3 }
=
\gpgradeone{
\lr{ \Be_1 + \Be_2 } \lr{ \Be_1 \Be_2 + 3 \Be_2 \Be_3 }
}
=
\gpgradeone{
\Be_1^2 \Be_2 + 3 \Be_1 \Be_2 \Be_3
+
\Be_2^2 \lr{ -\Be_1 + 3 \Be_3 }
}
=
\gpgradeone{
\Be_2 + 3 \cancel{\Be_1 \Be_2 \Be_3}
-
\Be_1 + 3 \Be_3
}
=
\Be_2 - \Be_1 + 3 \Be_3.
\end{dmath}

Observe that the vector-bivector dot product selects only products with common factors, since any others are trivectors.

\subsubsection{Problems.}
   \input{vectorBivectorDot.tex}
         \shipoutAnswer
   %
% Copyright © 2016 Peeter Joot.  All Rights Reserved.
% Licenced as described in the file LICENSE under the root directory of this GIT repository.
%
\makeproblem{Dot product expansion of two bivectors.}{problem:multiplication:bivectorDot}{
Show that

\boxedEquation{eqn:bivectorDot:20}{
(\Ba \wedge \Bb) \cdot (\Bc \wedge \Bd)
=
\lr{ (\Ba \wedge \Bb) \cdot \Bc} \cdot \Bd,
}

and hence
\boxedEquation{eqn:bivectorDot:40}{
(\Ba \wedge \Bb) \cdot (\Bc \wedge \Bd)
=
(\Bb \cdot \Bc) (\Ba \cdot \Bd)
-(\Ba \cdot \Bc)( \Bb \cdot \Bd).
}
} % problem

\makeanswer{problem:multiplication:bivectorDot}{
FIXME: this proof relies on triple wedge products, which haven't been introduced yet.  Can rephrase using grade-3 selection.
Also rework as non-problem.
\begin{dmath}\label{eqn:bivectorDot:60}
(\Ba \wedge \Bb) \cdot (\Bc \wedge \Bd)
=
\gpgradezero{
(\Ba \wedge \Bb) (\Bc \wedge \Bd)
}
=
\gpgradezero{
(\Ba \wedge \Bb) (\Bc \Bd - \cancel{\Bc \cdot \Bd})
}
=
\gpgradezero{
\lr{
(\Ba \wedge \Bb) \cdot \Bc
+ \cancel{(\Ba \wedge \Bb) \wedge \Bc }
}
\Bd
}
=
\gpgradezero{
((\Ba \wedge \Bb) \cdot \Bc ) \cdot \Bd
+
\cancel{((\Ba \wedge \Bb) \cdot \Bc ) \wedge \Bd}
}
=
((\Ba \wedge \Bb) \cdot \Bc ) \cdot \Bd.
\end{dmath}

Above, any product that could not possibly contribute a scalar grade has been cancelled.  The remains are now straightforward to expand

\begin{dmath}\label{eqn:bivectorDot:80}
((\Ba \wedge \Bb) \cdot \Bc ) \cdot \Bd
=
(
\Ba (\Bb \cdot \Bc)
-
\Bb (\Ba \cdot \Bc)
)
\cdot \Bd
=
(\Ba \cdot \Bd) (\Bb \cdot \Bc)
-
(\Bb \cdot \Bd) (\Ba \cdot \Bc).
\end{dmath}

} % answer

         \shipoutAnswer
