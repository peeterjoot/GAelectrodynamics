%
% Copyright � 2017 Peeter Joot.  All Rights Reserved.
% Licenced as described in the file LICENSE under the root directory of this GIT repository.
%
%{
\input{../latex/blogpost.tex}
\renewcommand{\basename}{projectnotes}
%\renewcommand{\dirname}{notes/phy1520/}
\renewcommand{\dirname}{notes/ece1228-electromagnetic-theory/}
%\newcommand{\dateintitle}{}
%\newcommand{\keywords}{}

\input{../latex/peeter_prologue_print2.tex}

\usepackage{peeters_layout_exercise}
\usepackage{peeters_braket}
\usepackage{peeters_figures}
\usepackage{siunitx}
%\usepackage{mhchem} % \ce{}
%\usepackage{macros_bm} % \bcM
%\usepackage{macros_qed} % \qedmarker
%\usepackage{txfonts} % \ointclockwise

\beginArtNoToc

\generatetitle{M.Eng project for electromagnetic applications of geometric algebra}
%\chapter{M.Eng project for electromagnetic applications of geometric algebra}
%\label{chap:projectnotes}

\paragraph{Context for the project}

An ECE professor from Universit� di Perugia, Prof. Mauro Mongiardo, has reached out to me to collaborate on a book and papers related to applications of geometric algebra (GA) in electromagnetism, particularly focused on engineering applications in the frequency domain.  As discussed, I am interested in persuing this work for two reasons:

\begin{enumerate}
\item
It is intrinsicaly interesting to me, and I have a strong impression that there is a lot of potential for interesting applications.
\item
Doing this work in the context of an M.Eng project will help satisify the ECE graduation requirements, especially since the electromagnetics group course offerings in recent years have been very limited.
\end{enumerate}

I've read and written considerably about applications of geometric algebra outside of a university context.  That writing is scattered throughout the following notes compilations (and probably other locations)

\begin{itemize}
\item Exploring Physics with Geometric Algebra, Part I \citep{gabookI}
\item Exploring Physics with Geometric Algebra, Part II \citep{gabookII}
\item Classical Mechanics \citep{classicalmechanics}
\item Continuum Mechanics \citep{phy454}
\item Advanced Antenna Theorey \citep{ece1229}
\end{itemize}

Much of the research into geometric algebra applications to electromagnetism has been in the context of relativistic electromagnetism, where Maxwell's equations take a particularly simple form

\begin{dmath}\label{eqn:projectnotes:20}
\grad F = \inv{\epsilon} J.
\end{dmath}

This consolidates the two relativisitic tensor relations for Maxwell's equations

\begin{dmath}\label{eqn:projectnotes:40}
\begin{aligned}
\partial_\mu F^{\mu\nu} &= \inv{\epsilon} J^\nu \\
\epsilon^{\alpha\beta\gamma\kappa} \partial_\beta F_{\gamma\kappa} &= 0.
\end{aligned}
\end{dmath}

into a single multivector equation, where

\begin{dmath}\label{eqn:projectnotes:60}
\begin{aligned}
F &= \BE + I c \BB \\
I &= \gamma_0 \gamma_1 \gamma_2 \gamma_3 \\
\BE &= \sum_{k = 1}^3 \gamma_k \gamma_0 E^k \\
\BB &= \sum_{k = 1}^3 \gamma_k \gamma_0 B^k \\
\BJ &= \sum_{k = 1}^3 \gamma_k \gamma_0 J^k \\
\grad &= \gamma^\mu \partial_\mu = \gamma^\mu \PD{x^\mu}{} \\
x^0 &= c t \\
\partial_0 &= \inv{c} \PD{t}{} \\
c &= 1/\sqrt{\mu\epsilon} \\
J &= \gamma_0 \lr{ c \rho - \BJ },
\end{aligned}
\end{dmath}

where \( \setlr{ \gamma_\mu } \) is a relativisitic four-vector basis satisfying \( (\gamma_0)^2 = 1 \), \( (\gamma^k)^2 = -1 \), and \( \gamma^\mu \gamma_\mu = 1 \).  In this context (referred to as the Space Time Algebra, or STA) all spatial vectors \((\BE, \BB, \BJ)\) are actually bivectors, as is the electromagnetic field \( F \).  This abstraction allows problems to be formulated
without any explicit reference to either electric or magnetic fields, quantities that are observer dependent.  Lorentz boosts that translate from an observe frame to can be formulated as easily as rotations, which is especially powerful given that rotations in geometric algebra have such a compact representation.
This power comes with a level of abstraction that makes the subject impalatible for applications in engineering.
There is also a considerable learning curve for geometric algebra, and that learning curve is made still steeper by requiring the electromagnetic practitioner to also deal with the relativisitic abstractions.

\paragraph{Maxwell's equation in a Euclidean basis}

Maxwell's equation can also be expressed in a compact geometric algebra multivector equation, without the use of
four vectors and non-Euclidean geometries, namely

\begin{dmath}\label{eqn:projectnotes:80}
\lr{ \inv{c} \PD{t}{} + \spacegrad } F = \inv{\epsilon} J
\end{dmath}

where
\begin{dmath}\label{eqn:projectnotes:120}
\begin{aligned}
F &= \BE + I c \BB \\
I &= \Be_1 \Be_2 \Be_3 \\
c &= 1/\sqrt{\mu\epsilon} \\
J &= c \rho - \BJ.
\end{aligned}
\end{dmath}

A choice of a fixed observer frame fixes the representation of the four-gradient, expressed here as a multivector operator, but result is otherwise identical to the representation of Maxwell's equation \cref{eqn:projectnotes:20}.
It is straightforward to show that this representation is equivalent to the normal vectorial form of Maxwell's equations

\begin{dmath}\label{eqn:projectnotes:100}
\begin{aligned}
\spacegrad \cdot \BE &= \inv{\epsilon} \rho \\
\spacegrad \cdot \BB &= 0 \\
\spacegrad \cross \BE &= -\PD{t}{\BB} \\
\spacegrad \cross \BB &= \mu \lr{ \BJ + \epsilon \PD{t}{\BE} }.
\end{aligned}
\end{dmath}

While the fixed observer frame GA representation of Maxwell's equation \cref{eqn:projectnotes:80} is only subtlely different from the STA representation, I believe that this representation is preferrable to the study of electromagnetism with respect to engineering applications.  Not having to deal with non-Euclidean geometries and four vectors should considerably reduce the learning curve required to exploit the compact GA representation in real world applications.  A compact representation alone is clearly not the only desirable attribute, for if that were the case, engineers would work exclusively with the tensor form of Maxwell's equations \cref{eqn:projectnotes:40}.  
As engineers, having time as an independent variable, and an assumption that the geometry we have to deal with is Euclidean, are definite prerequisites!

\paragraph{scratch notes}

Part of what I'd like to accomplish with this project is to
Engineering applications of electromagnetism keep time (or frequency) as a distinct quantity of interest, so some thought would be required to figure out how to most naturally express some of the electromagnetic concepts that have concise representation in relativistic geometric algebra where time is just one coordinate of a four vector.
 (i.e. for personal non-academic research), however, it will still take a considerable amount of work to assemble these ideas into a coherent form

  This would also require a fair amount of literature review since some of the relevant material has probably also been presented in a piecemeal fashion.

An example is the electromagnetic stress tensor, which in (relativistic) geometric algebra is just T(a) = -F a F/(2 epsilon), where a is a four vector.  Picking a timelike four vector 'a', produces the Poynting theorem relations, whereas picking spacelike four vectors 'a' produce the rest of the (lesser known) Poynting like conservation relations that bring in all 16 elements of the stress-energy tensor into the mix.  It is not obvious if there is corresponding natural (yet still concise) representation of the complete stress energy tensor representation when using the usual three-spatial+one-time coordinate geometric algebra representation of Maxwell's equations.


\paragraph{syllabus}

M.Eng project on Engineering applications of Geometric Algebra to engineering electromagnetism.

%}
\EndArticle
