%
% Copyright © 2017 Peeter Joot.  All Rights Reserved.
% Licenced as described in the file LICENSE under the root directory of this GIT repository.
%
The product of two colinear vectors is a scalar, and the product of two normal vectors is a bivector.
To understand the form for a product of two unrestricted vectors, consider their product expressed as a coordinate expansion.  Let

\begin{dmath}\label{eqn:SimpleProducts2:1160}
\begin{aligned}
\Ba &= \sum_i a_i \Be_i \\
\Bb &= \sum_i b_i \Be_i,
\end{aligned}
\end{dmath}

The product of these vectors is

\begin{dmath}\label{eqn:SimpleProducts2:1360}
\Ba \Bb
=
\lr{ \sum_i a_i \Be_i } \lr{ \sum_j b_j \Be_j }
=
\sum_{ij} a_i b_j \Be_i \Be_j
=
\sum_{i = j} a_i b_j \Be_i \Be_j
+
\sum_{i \ne j} a_i b_j \Be_i \Be_j
\end{dmath}

The first sum over \( i = j \) is just the dot product since \( \Be_i \Be_i = 1 \), so the general product of two vectors is

\begin{dmath}\label{eqn:SimpleProducts2:1480}
\Ba \Bb
=
\Ba \cdot \Bb
+
\sum_{i \ne j} a_i b_j \Be_i \Be_j.
\end{dmath}

The product of two vectors is a multivector with a scalar (grade 0) component, and a bivector (grade 2) component.  This can be written symbolically as

\boxedEquation{eqn:SimpleProducts2:1380}{
\Ba \Bb = \gpgradezero{ \Ba \Bb } + \gpgradetwo{ \Ba \Bb }.
}

As a side effect of having performed this expansion, we see that it is possible to compute the dot product of two vectors by scalar grade selection

\boxedEquation{eqn:SimpleProducts2:1400}{
\Ba \cdot \Bb = \gpgradezero{ \Ba \Bb }.
}

This form of dot product motivates a more general definition of dot product for multivectors, which is

\makedefinition{Multivector dot product}{dfn:gradeselection:100}{
The dot (or inner) product of two multivectors

\begin{equation*}
\begin{aligned}
A &= \sum_{i = 0}^N A_i = \sum_{i = 0}^N \gpgrade{A}{i}, \\
B &= \sum_{i = 0}^N B_i = \sum_{i = 0}^N \gpgrade{B}{i},
\end{aligned}
\end{equation*}

is defined as
\begin{equation*}
A \cdot B \equiv
\sum_{i,j = 0}^N \gpgrade{ A_i B_j }{\Abs{i - j}}
\end{equation*}
} % definition

The bivector term of the vector product is called the wedge product, and written as

\boxedEquation{eqn:SimpleProducts2:1420}{
\Ba \wedge \Bb \equiv \gpgradetwo{ \Ba \Bb }.
}

Similar to the multivector dot product, the multivector wedge product that generalizes
\cref{eqn:SimpleProducts2:1420} is defined as

\makedefinition{Multivector wedge product.}{dfn:gradeselection:480}{
For the multivectors \( A, B \) defined in \cref{dfn:gradeselection:100}, the wedge (or outer) product is defined as

\begin{equation*}
A \wedge B
\equiv
\sum_{i,j = 0}^N \gpgrade{ A_i B_j }{i + j}.
\end{equation*}
} % definition

The vector product of \cref{eqn:SimpleProducts2:1400} can now be expressed as a sum of dot and wedge products

\boxedEquation{eqn:SimpleProducts2:1440}{
\Ba \Bb = \Ba \cdot \Bb + \Ba \wedge \Bb.
}

This is a very important identity, and will have a number of applications.
It is premature to consider applications since
the properties and geometry of the wedge product have not been explored.

To start exploring that geometry let's consider the polar form of the vector, dot and wedge products for two vectors \( \Ba \) and \( \Bb \), with respective magnitudes \( a, b \).
Let \( \ucap \) and \( \vcap \) be an orthonormal pair of vectors in the plane of \( \Ba \) and \( \Bb \), oriented in a positive rotational sense as illustrated in
\cref{fig:Parallelogram:ParallelogramFig1}.
\imageFigure{../figures/GAelectrodynamics/ParallelogramFig1}{Two vectors in a plane.}{fig:Parallelogram:ParallelogramFig1}{0.3}
If \( i_{ab} = \ucap \vcap \) is the unit pseudoscalar for the plane containing these vectors, then the polar forms are

\begin{dmath}\label{eqn:SimpleProducts2:1660}
\begin{aligned}
\Ba &= a \ucap e^{ i_{ab} \theta_a } = a e^{ -i_{ab} \theta_a } \ucap \\
\Bb &= b \ucap e^{ i_{ab} \theta_b } = b e^{ -i_{ab} \theta_b } \ucap,
\end{aligned}
\end{dmath}

The vector product of these two vectors is

\begin{dmath}\label{eqn:SimpleProducts2:1680}
\Ba \Bb
=
\lr{ a e^{ -i_{ab} \theta_a } \ucap } \lr{ b \ucap e^{ i_{ab} \theta_b } }
=
a b
e^{ -i_{ab} \theta_a } ( \ucap \ucap ) e^{ i_{ab} \theta_b }
=
a b
e^{ i_{ab} (\theta_b - \theta_a)}.
\end{dmath}

The vector, dot and wedge products are therefore

\boxedEquation{eqn:SimpleProducts2:1700}{
\begin{aligned}
\Ba \Bb &= \Norm{\Ba} \Norm{\Bb} \exp\lr{ i_{ab} (\theta_b - \theta_a) } \\
\Ba \cdot \Bb &= \Norm{\Ba} \Norm{\Bb} \cos( \theta_b - \theta_a ) \\
\Ba \wedge \Bb &= i_{ab} \Norm{\Ba} \Norm{\Bb} \sin( \theta_b - \theta_a ).
\end{aligned}
}

The product of two vectors is the product of their magnitudes, multiplied by a ``unit'' complex exponential multivector with grades 0,2.

Since the cross product can be written as \( \Ba \cross \Bb = \ncap_{ab} \sin(\theta_b - \theta_a) \),
\cref{eqn:SimpleProducts2:1700} provides a strong hint that the
wedge and cross products are related.

One property of the wedge product follows by setting \( \Bb = \Ba \) in
\cref{eqn:SimpleProducts2:1440}, which gives

\begin{dmath}\label{eqn:SimpleProducts2:1500}
\Ba \Ba = \Ba \cdot \Ba + \Ba \wedge \Ba,
\end{dmath}

but since \( \Ba \Ba = \Ba \cdot \Ba \), we conclude that

\boxedEquation{eqn:SimpleProducts2:1520}{
\Ba \wedge \Ba = 0.
}

Like the cross product, the
wedge product of any colinear vectors is also zero, which should be clear from the polar form
of the wedge product in
\cref{eqn:SimpleProducts2:1700}.

Let's compare the coordinate expansion of the wedge product to that of the cross product, by
eliminating the redundant terms in the bivector term of \cref{eqn:SimpleProducts2:1480}

\begin{dmath}\label{eqn:SimpleProducts2:1460}
\Ba \wedge \Bb
=
\sum_{i \ne j} a_i b_j \Be_i \Be_j
=
\sum_{i < j} a_i b_j \Be_i \Be_j
+
\sum_{j < i} a_i b_j \Be_i \Be_j
=
\sum_{i < j} a_i b_j \Be_i \Be_j
+
\sum_{i < j} a_j b_i \Be_j \Be_i
=
\sum_{i < j} a_i b_j \Be_i \Be_j
+
\sum_{i < j} a_j b_i (-\Be_i \Be_j)
=
\sum_{i < j} (a_i b_j - a_j b_i) \Be_i \Be_j.
\end{dmath}

The scalar factors can be written as a determinants, yielding a tidy coordinate expansion of the wedge product of two vectors

\boxedEquation{eqn:SimpleProducts2:1320}{
\Ba \wedge \Bb
=
\sum_{i < j}
\begin{vmatrix}
a_i & a_j \\
b_i & b_j
\end{vmatrix}
\Be_i \Be_j.
}

This coordinate expansion can also be used to show that the wedge product of any colinear vectors is zero.
We also see that the wedge product is antisymmetric (exchanging the vectors toggles the sign), or

\boxedEquation{eqn:SimpleProducts2:1540}{
\Ba \wedge \Bb
=
-\Bb \wedge \Ba,
}

It's helpful to write out the coordinate expansion of
\cref{eqn:SimpleProducts2:1320} explicitly for \R{2} and \R{3}.
For \R{2} there is only one term in this sum

\begin{dmath}\label{eqn:SimpleProducts2:1720}
\Ba \wedge \Bb
=
\begin{vmatrix}
a_1 & a_2 \\
b_1 & b_2
\end{vmatrix}
\Be_1 \Be_2.
\end{dmath}

For \R{3} the wedge product has three terms, which can be combined using a cross product like determinant mnemonic

\begin{dmath}\label{eqn:SimpleProducts2:1740}
\Ba \wedge \Bb
=
\begin{vmatrix}
\Be_2 \Be_3 & \Be_3 \Be_1 & \Be_1 \Be_2 \\
a_1 & a_2 & a_3 \\
b_1 & b_2 & b_3 \\
\end{vmatrix}.
\end{dmath}

Let's summarize the wedge product properties and relations we have found so far, and compare those to the cross product

\begin{tcolorbox}[tab2,tabularx={X||Y|Y},title=Cross product and \R{3} wedge product comparison.,boxrule=0.5pt]
Property & Cross product & Wedge product
\\ \hline
Same vectors & \( \Ba \cross \Ba = 0 \) & \( \Ba \wedge \Ba = 0 \)
\\ \hline
Antisymmetry & \( \Bb \cross \Ba = -\Ba \cross \Bb \) & \( \Bb \wedge \Ba = -\Ba \wedge \Bb \)
\\ \hline
Determinant expansion
&
\(
\Ba \cross \Bb
=
\begin{vmatrix}
\Be_1 & \Be_2 & \Be_3 \\
a_1 & a_2 & a_3 \\
b_1 & b_2 & b_3 \\
\end{vmatrix}
\)
&
\(
\Ba \wedge \Bb
=
\begin{vmatrix}
\Be_2 \Be_3 & \Be_3 \Be_1 & \Be_1 \Be_2 \\
a_1 & a_2 & a_3 \\
b_1 & b_2 & b_3 \\
\end{vmatrix}
\)
\\ \hline
Polar form &
\( \ncap_{ab} \Norm{\Ba} \Norm{\Bb} \sin( \theta_b - \theta_a )  \) &
\( i_{ab} \Norm{\Ba} \Norm{\Bb} \sin( \theta_b - \theta_a )  \)
\\ \hline
\end{tcolorbox}

All the wedge properties except the determinant expansion above are valid in any dimension.
It is reasonable to guess that the \R{3} wedge product is related to the cross product by some constant multivector factor \( i_{ab} = A \ncap_{ab} \).  In coordinate form, this requires a simultaneous solution to

\begin{dmath}\label{eqn:SimpleProducts2:1580}
\begin{aligned}
\Be_2 \Be_3 &= A \Be_1 \\
\Be_3 \Be_1 &= A \Be_2 \\
\Be_1 \Be_2 &= A \Be_3.
\end{aligned}
\end{dmath}

Multiplying on the right by \( \Be_1, \Be_2, \Be_3 \) respectively, this factor seems to be

\begin{equation}\label{eqn:SimpleProducts2:1600}
A = \Be_2 \Be_3 \Be_1 = \Be_3 \Be_1 \Be_2 = \Be_1 \Be_2 \Be_3,
\end{equation}

which are all permutations of the \R{3} unit pseudoscalar \( I = \Be_1 \Be_2 \Be_3 \).
This indicates that the cyclic permutations of the \R{3} pseudoscalar must all be identical (\cref{problem:SimpleProducts2:permutationspseudoscalar}).

We now have a coordinate free relationship for the \R{3} wedge product and the cross product

\boxedEquation{eqn:SimpleProducts2:1620}{
\Ba \wedge \Bb = I ( \Ba \cross \Bb ),
}

and can also express the
\R{3} vector product as a multivector combination of the dot and cross products

\boxedEquation{eqn:SimpleProducts2:1640}{
\Ba \Bb = \Ba \cdot \Bb + I(\Ba \cross \Bb).
}

Like
\cref{eqn:SimpleProducts2:1440}, this is also a very important relationship.
In particular, this identity will be what we use to assemble all the separate scalar and vector Maxwell's equations into a single multivector equation.

\makeproblem{Wedge product of colinear vectors.}{problem:SimpleProducts2:wedgecolinear}{
Given \( \Bb = \alpha \Ba \), use
\cref{eqn:SimpleProducts2:1320} to show that the wedge product of any pair of colinear vectors is zero.
} % problem

\makeproblem{Wedge product antisymmetry.}{problem:SimpleProducts2:1560}{
Prove \cref{eqn:SimpleProducts2:1540} using \cref{eqn:SimpleProducts2:1320}.
} % problem

\makeproblem{Permutations of the \R{3} pseudoscalar}{problem:SimpleProducts2:permutationspseudoscalar}{
Show that each of the permutations of
\cref{eqn:SimpleProducts2:1600} are all equal.
} % problem
