%
% Copyright � 2018 Peeter Joot.  All Rights Reserved.
% Licenced as described in the file LICENSE under the root directory of this GIT repository.
%
%{
\input{../latex/blogpost.tex}
\renewcommand{\basename}{reply}
%\renewcommand{\dirname}{notes/phy1520/}
\renewcommand{\dirname}{notes/ece1228-electromagnetic-theory/}
%\newcommand{\dateintitle}{}
%\newcommand{\keywords}{}

\input{../latex/peeter_prologue_print2.tex}

\usepackage{peeters_layout_exercise}
\usepackage{peeters_braket}
\usepackage{peeters_figures}
\usepackage{siunitx}
%\usepackage{mhchem} % \ce{}
%\usepackage{macros_bm} % \bcM
%\usepackage{macros_qed} % \qedmarker
%\usepackage{txfonts} % \ointclockwise

\beginArtNoToc

\generatetitle{XXX}
%\chapter{XXX}
%\label{chap:reply}

\color{Blue}{
   \paragraph{C6}:

You can do it from \( \Bj = (2 \Ba \Bb - 1)/\sqrt{3} \), but you don't have the tools yet.
Expanding \( \Bj^2 \) you get
\begin{dmath}\label{eqn:reply:160}
\Bj^2 = \inv{3} \lr{ 4 \Ba \Bb \Ba \Bb - 4 \Ba \Bb + 1 },
\end{dmath}
which might look intractible, however, by rearranging
\begin{dmath}\label{eqn:reply:100}
\Ba \cdot \Bb = \frac{\Ba \Bb + \Bb \Ba}{2},
\end{dmath}
which is a vector product form of the dot product that comes later in the chapter I, you can find both of
\begin{dmath}\label{eqn:reply:120}
\begin{aligned}
\Bb \Ba &= 2 \Ba \cdot \Bb - \Ba \Bb \\
\Ba \Bb &= 2 \Ba \cdot \Bb - \Bb \Ba
\end{aligned}.
\end{dmath}
These can be used to reduce the \( \Ba \Bb \Ba \Bb \) product that you get expanding \( \Bj^2 \).  For example
\begin{dmath}\label{eqn:reply:140}
\Ba \Bb \Ba \Bb
=
\lr{ 2 \Ba \cdot \Bb - \Bb \Ba } \Ba \Bb
=
2 (\Ba \cdot \Bb) \Ba \Bb - \Bb \Ba^2 \Bb
=
2 (\Ba \cdot \Bb) \Ba \Bb - \Bb^2 \Ba^2
=
\Ba \Bb - 1,
\end{dmath}
where the last step used \( \Ba^2 = \Bb^2 = 1 \) and \( \Ba \cdot \Bb = 1/2 \).
Inserting into \cref{eqn:reply:160} you get
\begin{dmath}\label{eqn:reply:n}
\Bj^2 = \inv{3}\lr{ 4 (\Ba \Bb - 1) - 4 \Ba \Bb + 1 }
=
\frac{-3}{3}
= -1.
\end{dmath}
}

%%%\color{Red}{
%%%\paragraph{Q1}:
%%%
%%%Examples:
%%%
%%%\begin{itemize}
%%%\item
%%%$\Be_1 \Be_2 + \Be_2 \Be_3$.  This is a bivector, and is also a multivector with only a grade 2 component.
%%%\item
%%%$\Be_1 \Be_2 \Be_3$.  This is a trivector, and is also a multivector with only a grade 3 component.
%%%\item
%%%$1 + \Be_1 \Be_2$  This is not a k-vector as there is no single grade.  It is a multivector.  In this case, it is a sum of a scalar (0-vector) and a bivector (2-vector).
%%%\item
%%%$0$. This is a scalar (0-vector), and also a multivector.
%%%\end{itemize}
%%%
%%%A multivector is a sum of 0 or more k-vectors, where $k = 0,\cdots,N$, and
%%%$N$ is the dimension of the generating vector space.
%%%
%%%I will split the definition of multivector and multivector space, which I think may help clarify this.
%%%
%%%\paragraph{Q2}:
%%%Yes, this is correct, however, $-\Be_2 i = -\Be_2 \Be_1 \Be_2 = + \Be_1 \Be_2^2 = \Be_1$, so this is not really that much different.  One way to look at this is that you can factor unity in terms of any vector, so in \R{2} $1 = \Be_1^2$ or $1 = \Be_2^2$, so
%%%
%%%\begin{dmath}\label{eqn:reply:20}
%%%\Be_1 \cos(\theta) + \Be_2 \sin(\theta)
%%%= \Be_2 ( \sin(\theta) + \Be_2 \Be_1 \cos(\theta))
%%%= \Be_2 ( \sin(\theta) - i \cos(\theta) )
%%%\end{dmath}
%%%
%%%\paragraph{C3}:
%%%
%%%Exactly.  I discuss two geometric representations of plane waves in chapter three.  One using \( i = \Be_1 \Be_2 \) to represent the transverse plane, and another representing the rotational aspects using the \R{3} pseudoscalar \( I = \Be_1 \Be_2 \Be_3 \).  The total multivector field ($F = \BE + I \eta \BH$) can be put into either the form
%%%
%%%\begin{dmath}\label{eqn:reply:40}
%%%F = \BE (1 + \Be_3) f(\beta z + I c t)
%%%\end{dmath}
%%%
%%%or
%%%
%%%\begin{dmath}\label{eqn:reply:60}
%%%F = \BE (1 + \Be_3) g(\beta z + i c t)
%%%\end{dmath}
%%%
%%%where \( c \) is the group velocity of the medium, and \( \Be_3 \) is the propagation direction.
%%%
%%%\paragraph{Q/C 2}:
%%%The circle is an error.  It should be an arc that indicates the orientation.  Here's why the two factors were chosen:
%%%
%%%\begin{dmath}\label{eqn:reply:80}
%%%I \Bx
%%%=
%%%I \lr{ a \Be_1 + b \Be_2 }
%%%=
%%%\lr{ a \Be_1 \Be_2 \Be_3 \Be_1 + b \Be_1 \Be_2 \Be_3 \Be_2 }
%%%=
%%%\lr{ a \Be_2 \Be_3 - b \Be_1 \Be_3 }
%%%=
%%%\lr{ a \Be_2 - b \Be_1 } \Be_3.
%%%\end{dmath}
%%%
%%%The dual of the vector $\Bx$ is a bivector, which can be factored as a product of two orthogonal vectors.  In this case, those factors are $a \Be_2 - b \Be_1$ and $\Be_3$.
%%%}

%}
\EndArticle
%\EndNoBibArticle
