%
% Copyright © 2018 Peeter Joot.  All Rights Reserved.
% Licenced as described in the file LICENSE under the root directory of this GIT repository.
%
%{
For a multivector current with only electric sources

\begin{dmath}\label{eqn:potentialSection_electric:1880}
J = \eta \lr{ c \rho - \BJ },
\end{dmath}

we can construct a multivector potential with only scalar and vector grades
\begin{dmath}\label{eqn:potentialSection_electric:1900}
A = - \phi + c \BA.
\end{dmath}

The resulting field is

\begin{dmath}\label{eqn:potentialSection_electric:80}
F
=
\BE + I \eta \BH
=
\gpgrade{ \lr{ \spacegrad - \inv{c} \PD{t}{} }
\lr{
      - \phi
      + c \BA
}
}{1,2},
\end{dmath}

which expands to

\boxedEquation{eqn:potentialSection:2240}{
F =
-\spacegrad \phi
-\PD{t}{\BA}
+ c \spacegrad \wedge \BA.
}

The respective electric and magnetic fields can be extracted using a duality transformation for the bivector curl

\begin{dmath}\label{eqn:potentialSection_electric:1920}
F
=
-\spacegrad \phi
-\PD{t}{\BA}
+ I c \spacegrad \cross \BA,
\end{dmath}

from which we can read off the field components

\begin{dmath}\label{eqn:potentialSection_electric:100}
\begin{aligned}
\BE &= -\spacegrad \phi -\PD{t}{\BA} \\
\mu \BH &= \spacegrad \cross \BA.
\end{aligned}
\end{dmath}

Observe that the grade selection encodes the precise recipe required to produce the desired combination of gradients, curls and time partials.

The potential representation of the field \cref{eqn:potentialSection_electric:80} is only a solution if Maxwell's equation is also satisfied, or

\begin{dmath}\label{eqn:potentialSection_electric:1960}
\lr{ \spacegrad^2 - \inv{c^2} \PDSq{t}{} } \lr{ -\phi + c\BA }
= \eta \lr{ c \rho - \BJ } +
\lr{ \spacegrad + \inv{c} \PD{t}{} } \gpgrade{ \lr{ \spacegrad - \inv{c} \PD{t}{} } \lr{ -\phi + c\BA } }{0,3}
= \eta \lr{ c \rho - \BJ } +
\lr{ \spacegrad + \inv{c} \PD{t}{} } \lr{ c \spacegrad \cdot \BA + \inv{c} \PD{t}{\phi} }.
\end{dmath}

Imposing a constraint on the potential grades

\begin{dmath}\label{eqn:potentialSection_electric:2020}
\spacegrad \cdot \BA + \inv{c^2} \PD{t}{\phi} = 0,
\end{dmath}

the Lorenz gauge condition, is clearly an expedient way to simplify this relationship.
In particular,
in the frequency domain \( \PDi{t}{} \leftrightarrow j \omega = j k c \), this gauge choice allows the scalar potential to be entirely eliminated, since

\begin{dmath}\label{eqn:potentialSection_electric:2040}
\phi = \frac{j c^2}{\omega} \spacegrad \cdot \BA.
\end{dmath}

so the multivector potential is completely determined by a single vector potential

\begin{dmath}\label{eqn:potentialSection_electric:2060}
A =
-\frac{j c^2}{\omega} \spacegrad \cdot \BA + c \BA,
\end{dmath}

Maxwell's equation is reduced to a Helmholtz equation

\begin{dmath}\label{eqn:potentialSection_electric:2080}
\lr{ \spacegrad^2 + k^2} A = J,
\end{dmath}

and the field is simply
\begin{dmath}\label{eqn:potentialSection_electric:2100}
F = \lr{ \spacegrad - j k } A.
\end{dmath}

The conventional electric and magnetic field expressions can be found by substituting \cref{eqn:potentialSection_electric:2040} into
\cref{eqn:potentialSection_electric:1920} and switching to the frequency domain

\begin{dmath}\label{eqn:potentialSection_electric:2380}
F
=
-j \frac{c^2}{\omega} \spacegrad \lr{ \spacegrad \cdot \BA }
-j \omega \BA
+ I c \spacegrad \cross \BA,
\end{dmath}

%To compare this to conventional results, let's substitute \cref{eqn:potentialSection_electric:2060} into \cref{eqn:potentialSection_electric:2100},
%
%\begin{dmath}\label{eqn:potentialSection_electric:2300}
%F
%=
%\lr{ \spacegrad - j k }
%\lr{
%-\frac{j c^2}{\omega} \spacegrad \cdot \BA + c \BA
%}
%=
%-\frac{k c^2}{\omega} \spacegrad \cdot \BA -j k c \BA
%-\frac{j c^2}{\omega} \spacegrad \lr{ \spacegrad \cdot \BA } + c \spacegrad \BA
%=
%- \cancel{ c \spacegrad \cdot \BA }
%-j \omega \BA
%-\frac{j c}{k} \spacegrad \lr{ \spacegrad \cdot \BA }
%+ \cancel{ c \spacegrad \cdot \BA }
%+ c \spacegrad \wedge \BA
%=
%-j \omega \BA -\frac{j c}{k} \spacegrad \lr{ \spacegrad \cdot \BA }
%+ c \spacegrad \wedge \BA.
%\end{dmath}
%
%From this the electric and magnetic fields can be read off

so

\begin{dmath}\label{eqn:potentialSection_electric:2320}
\begin{aligned}
\BE &= -j \omega \BA -\frac{j c}{k} \spacegrad \lr{ \spacegrad \cdot \BA } \\
\mu \BH &= \spacegrad \cross \BA.
\end{aligned}
\end{dmath}

%}
