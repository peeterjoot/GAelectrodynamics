%
% Copyright © 2017 Peeter Joot.  All Rights Reserved.
% Licenced as described in the file LICENSE under the root directory of this GIT repository.
%

We've found solutions for a number of static charge and current distributions.

\begin{enumerate}[(a)]
\item For constant electric sources along the z-axis
(\cref{eqn:statics:180})
, with current \( \BJ \) moving with velocity \( \Bv = v \Be_3 \), the field had the form \( F = E \rhocap \lr{ 1 - \Bv/c } \).
\item For constant magnetic sources along the z-axis
(\cref{problem:statics:240})
, with current \( \BM \) moving with velocity \( \Bv = v \Be_3 \), the field had the form \( F = \eta H I \rhocap \lr{ 1 - \Bv/c } \).
\item For constant electric sources in the x-y plane
(\cref{eqn:statics:320})
, with current \( \BJ \) moving with velocity \( \Bv = v \Be_1 e^{i\theta}, i = \Be_{12} \), the field had the form \( F = E \Be_3 \lr{ 1 - \Bv/c } \).
\item For constant magnetic sources in the x-y plane
(\cref{problem:statics:241})
, with current \( \BM \) moving with velocity \( \Bv = v \Be_1 e^{i\theta}, i = \Be_{12} \), the field had the form \( F = \eta H i \lr{ 1 - \Bv/c } \).
\end{enumerate}

In all cases the field has the form \( F = A ( 1 - \Bv/c ) \), where \( A \) is either a vector or a bivector that anticommutes with the current velocity \( \Bv \), so the stress energy tensor \( T(1) \) has the form

\begin{dmath}\label{eqn:poyntingF:860}
T(1)
= \frac{\epsilon}{2} A ( 1 - \Bv/c )^2 A^\dagger
= \frac{\epsilon}{2} A A^\dagger ( 1 + \Bv/c )^2
= \frac{\epsilon}{2} A A^\dagger \lr{ 1 + \lr{ \frac{\Bv}{c} }^2 + 2 \frac{\Bv}{c} },
\end{dmath}

For the electric sources this is
\begin{dmath}\label{eqn:poyntingF:880}
\calE + \frac{\BS}{c} = \frac{\epsilon}{2} E^2 \lr{ 1 + \lr{ \frac{\Bv}{c} }^2 + 2 \frac{\Bv}{c} },
\end{dmath}

or
\begin{dmath}\label{eqn:poyntingF:900}
\begin{aligned}
\calE &= \frac{\epsilon}{2} E^2 \lr{ 1 + \lr{ \frac{\Bv}{c} }^2 } \\
\BS &= \epsilon E^2 \Bv.
\end{aligned}
\end{dmath}

For the magnetic sources this is
\begin{dmath}\label{eqn:poyntingF:920}
\calH + \frac{\BS}{c} = \frac{\mu}{2} H^2 \lr{ 1 + \lr{ \frac{\Bv}{c} }^2 + 2 \frac{\Bv}{c} },
\end{dmath}

or
\begin{dmath}\label{eqn:poyntingF:940}
\begin{aligned}
\calH &= \frac{\mu}{2} H^2 \lr{ 1 + \lr{ \frac{\Bv}{c} }^2 } \\
\BS &= \mu H^2 \Bv.
\end{aligned}
\end{dmath}

There are three terms in the multivector \( (1 -\Bv/c)^2 = 1 + \lr{ \ifrac{\Bv}{c} }^2 + 2 \ifrac{\Bv}{c} \).  For electric sources,
the first scalar term is due to the charge distribution, and provides the electric field contribution to the energy density.
The second scalar term is due to the current distribution, and provides the magnetic field contribution to the energy density.
The final vector term, proportional to the current velocity contributes to the Poynting vector, showing that the field momentum travels along the direction of the current in these static configurations.
