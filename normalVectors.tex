%
% Copyright © 2017 Peeter Joot.  All Rights Reserved.
% Licenced as described in the file LICENSE under the root directory of this GIT repository.
%
\index{anticommutation}
\maketheorem{Anticommutation of orthogonal vectors}{thm:multiplication:anticommutationNormal}{
Let \(\Bu\), and \(\Bv\) be two orthogonal vectors, the product of which \( \Bu \Bv \) is a bivector.
Changing the order of these products toggles the sign of the bivector.
\begin{equation*}
\Bu \Bv = -\Bv \Bu.
\end{equation*}
This sign change on interchange is called anticommutation.
} % theorem

\begin{proof}
%To prove \cref{thm:multiplication:anticommutationNormal}
Let \( \Bu, \Bv \) be a pair of orthogonal vectors, such as those of
\cref{fig:unitSum:unitSumFig1}.  The squared length of the sum \( \Bu + \Bv \) can be expressed in using
the contraction axiom, or by explicit expansion (taking care to maintain the order of products)
\begin{align*}
\lr{ \Bu + \Bv }^2 &= \lr{ \Bu + \Bv }\lr{ \Bu + \Bv } = \Bu^2 + \Bu \Bv + \Bv \Bu + \Bv^2 \\
\lr{ \Bu + \Bv }^2 &= \Norm{ \Bu + \Bv }^2 = \Bu^2 + \Bv^2.
\end{align*}
Comparing the two expansions and rearranging completes the proof%
\footnote{We will see later (\cref{thm:products:2080}) that the converse of this theorem is also true: If the product of two vectors is a bivector, those vectors are orthogonal.}.
% of the first part of this theorem.
\end{proof}
%
\pmathImageFigure{../figures/GAelectrodynamics/\subfigdir/}{unitSumFig1}{Sum of orthogonal vectors.}{fig:unitSum:unitSumFig1}{0.3}{orientedAreas.nb}

Some examples of anticommuting pairs include,
\( \Be_2 \Be_1 = -\Be_1 \Be_2 \),
\( \Be_3 \Be_2 = -\Be_2 \Be_3 \), and
\( \Be_1 \Be_3 = -\Be_3 \Be_1 \).  This theorem can also be applied to any pairs of orthogonal vectors in a arbitrary k-vector, for example
\begin{equation}\label{eqn:normalVectors:300}
\begin{aligned}
\Be_3 \Be_2 \Be_1
&= (\Be_3 \Be_2) \Be_1 \\
&= -(\Be_2 \Be_3) \Be_1 \\
&= -\Be_2 (\Be_3 \Be_1) \\
&= +\Be_2 (\Be_1 \Be_3) \\
&= +(\Be_2 \Be_1) \Be_3 \\
&= -\Be_1 \Be_2 \Be_3,
\end{aligned}
\end{equation}
showing that reversal of all the factors in a trivector such as \( \Be_1 \Be_2 \Be_3 \) toggles the sign.

