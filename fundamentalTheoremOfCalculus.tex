%
% Copyright � 2016 Peeter Joot.  All Rights Reserved.
% Licenced as described in the file LICENSE under the root directory of this GIT repository.
%
%{
%\input{../blogpost.tex}
%\renewcommand{\basename}{fundamentalTheoremOfCalculus}
%\renewcommand{\dirname}{notes/phy1520/}
%%\newcommand{\dateintitle}{}
%%\newcommand{\keywords}{}
%
%\input{../peeter_prologue_print2.tex}
%
%\usepackage{peeters_layout_exercise}
%\usepackage{peeters_braket}
%\usepackage{peeters_figures}
%\usepackage{siunitx}
%
%\beginArtNoToc
%
%\generatetitle{Fundamental theorem of geometric calculus}
%\label{chap:fundamentalTheoremOfCalculus}

\subsection{Hypervolume integral}
We wish to generalize the concepts of line, surface and volume integrals to hypervolumes and multivector functions, and define a hypervolume integral as

\makedefinition{Multivector integral.}{dfn:fundamentalTheoremOfCalculus:240}{
Given a hypervolume parameterized by \( k \) parameters, k-volume volume element \( d^k \Bx \), and
multivector functions \( F, G \), a k-volume integral with the vector derivative acting to the right on \( G \) is written as
\begin{equation*}
\int d^k\Bx \rboldpartial G,
\end{equation*}
a k-volume integral with the vector derivative acting to the left on \( F \) is written as
\begin{equation*}
\int F d^k\Bx \lboldpartial,
\end{equation*}
and a k-volume integral with the vector derivative acting bidirectionally on \( F, G \) is written as
\begin{equation*}
\int F d^k\Bx \lrboldpartial G
\equiv
\int \lr{ F d^k\Bx \lboldpartial} G
+
\int F d^k\Bx \lr{ \rboldpartial G }.
\end{equation*}
The explicit meaning of these directionally acting derivative operations is given by the chain rule coordinate expansion
\begin{dmath*}
F d^k \Bx \lrboldpartial G
=
F d^k \Bx \lr{ \sum_i \Bx^i {\stackrel{ \leftrightarrow }{\partial_i}} } G
=
(\partial_i F) d^k \Bx \sum_i \Bx^i G
+
F d^k \Bx \sum_i \Bx^i (\partial_i G)
\equiv
(F d^k \Bx \lboldpartial) G
+
F d^k \Bx (\rboldpartial G),
\end{dmath*}
with \( \boldpartial \) acting on \( F \) and \( G \), but not the volume element \( d^k \Bx \), which may also be a function of the implied parameterization.
} % definition

The vector derivative
% (and gradient)
may not commute with \( F, G \) nor the volume element \( d^k \Bx \), so we are forced to use some notation to indicate what the vector derivative (or gradient) acts on.
In conventional right acting cases, where there is no ambiguity, arrows will usually be omitted, but braces may also be used to indicate the scope of derivative operators.
This bidirectional notation will also be used for the gradient, especially for volume integrals in \R{3} where the vector derivative is identitical to the gradient.

Some authors use the Hestenes dot notation, with overdots or primes to indicatating the exact scope of multivector derivative operators, as in
\begin{dmath}\label{eqn:fundamentalTheoremOfCalculus:260}
F d^k \Bx \boldpartial G =
\dot{F} d^k \Bx \dot{\boldpartial} G
+
F d^k \Bx \dot{\boldpartial} \dot{G}.
\end{dmath}
The dot notation has the advantage of emphasizing that the action of the vector derivative (or gradient) is on the functions \( F, G \), and not on the hypervolume element \( d^k \Bx \).
However, in this book, where primed operators such as \( \spacegrad' \) are used to indicate that derivatives are taken with respect to primed \( \Bx' \) variables, a mix of dots and ticks would have been confusing.
%Over arrows also have the advantage of being visually conspicuous.

\subsection{Fundamental theorem.}
\index{fundamental theorem of geometric calculus}

The fundamental theorem of geometric calculus is a generalization of many conventional scalar and vector integral theorems, and relates a hypervolume integral to its boundary.
This is a a powerful theorem, which we will use with Green's functions to solve Maxwell's equation, but also to derive the geometric algebra form of Stokes' theorem, from which most of the familiar integral calculus results follow.

Before stating the theorem itself, we must state what we mean by the boundary integral.
\makedefinition{Hypervolume boundary integral.}{dfn:fundamentalTheoremOfCalculus:280}{
Given a parameterization \( \Bx = \Bx(u_1, u_2, \cdots) \), with volume element \( d^k \Bx = d^k u I_k \), where
\( I_k = \Bx_1 \wedge \Bx_2 \wedge \cdots \wedge \Bx_k \),
we define the boundary integral for \( k > 1 \) as
\begin{equation*}
\oint_{\partial V} F d^{k-1} \Bx G
\equiv
\sum_i \int d^{k-1} u_i \evalbar{ \lr{ F (I_k \cdot \Bx^i) \cdot \Bx G }}{\Delta u_i},
\end{equation*}
where \( d^{k-1} u_i \) is the product of all \( du_j \) except for \( du_i \).
For
\( k = 1 \) we define the boundary ``integral'' as just the evaluation of the \( F G \) multivector product over the range of the parameter
\begin{equation*}
\oint_{\partial V} F d^{0} \Bx G
\equiv
\evalbar{ F G }{\Delta u_1}.
\end{equation*}
} % definition

This will become more clear as the specific cases of \( k = 1, 2, 3 \), representing generalized line, surface, and volume integrals respectively, are considered.

\maketheorem{Fundamental theorem of geometric calculus}{thm:fundamentalTheoremOfCalculus:1}{
For multivectors \(F, G \), and a hypervolume element \(d^k \Bx\),
\begin{equation*}
\int_V F d^k \Bx \boldpartial G = \oint_{\partial V} F d^{k-1} \Bx G,
\end{equation*}
where \( d^{k-1} \Bx \) will be defined.
}

The proof follows from direct expansion, starting with
\begin{dmath}\label{eqn:fundamentalTheoremOfCalculus:300}
\int_V F d^k \Bx \boldpartial G
=
\sum_i \int_V d^k u F I_k \Bx^i \lrpartial_i G
=
\sum_i \int_V d^k u F \lr{ I_k \cdot \Bx^k +  I_k \wedge \Bx^i } \lrpartial_i G.
\end{dmath}
Because \( \Bx^k \) lies in \( \Span \setlr{ \Bx_i } \), the wedge product above is zero, leaving
\begin{dmath}\label{eqn:fundamentalTheoremOfCalculus:320}
\int_V F d^k \Bx \boldpartial G
=
\sum_i \int_V d^k u F I_k \cdot \Bx^k \lrpartial_i G,
\end{dmath}
however all the wedge factors of \( I_k \cdot \Bx^k \) are \( \PDi{u_j}{\Bx} \) taken with \( i \) held constant, which means that this integral can be written as a perfect differential
\begin{dmath}\label{eqn:fundamentalTheoremOfCalculus:340}
\int_V F d^k \Bx \boldpartial G
=
\sum_i \int_V d^{k-1} u_i \partial_i \lr{ F I_k \cdot \Bx^k G},
\end{dmath}
and integrated in turn for each \( du_i \), leaving the boundary integral of \cref{dfn:fundamentalTheoremOfCalculus:280}, proving the theorem in a limited fashion.
For a full proof of \cref{thm:fundamentalTheoremOfCalculus:1}, additional mathematical sublties must be considered.
The reader is referred to \citep{hestenes1985clifford}, \citep{doran2003gap},
and \citep{sobczyk2011fundamental}, which all which tackle different aspects of general geometric calculus required for a full proof, as well as to \citep{aMacdonaldVAGC} which covers aspects of connectivity also omitted here.

FIXME: I don't think this ``held constant'' argument is valid.  Instead I think that an argument like that of the Green's theorem proof below is actually required.  We end up with the miraculous perfect cancellation required, but not for a reason as trivial as held constant, since holding \( \Bx \) constant with respect to \( u_i \) doesn't mean that the result is not also a function of \( u_i \).  i.e. this will work because of equality of mixed partials, not because of ``held constant''.  That will make for a messier proof, and perhaps why MacDonald's nice little book skipped it all in the first place.

For clarity, the argument above will be repeated separately for each of the line, surface, and volume integral cases, but before doing so
we will state the general Stokes' theorem in its geometric algebra form.

%}
%\EndArticle
