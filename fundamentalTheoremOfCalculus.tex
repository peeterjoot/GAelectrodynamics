%
% Copyright � 2016 Peeter Joot.  All Rights Reserved.
% Licenced as described in the file LICENSE under the root directory of this GIT repository.
%
%{
%\input{../blogpost.tex}
%\renewcommand{\basename}{fundamentalTheoremOfCalculus}
%\renewcommand{\dirname}{notes/phy1520/}
%%\newcommand{\dateintitle}{}
%%\newcommand{\keywords}{}
%
%\input{../peeter_prologue_print2.tex}
%
%\usepackage{peeters_layout_exercise}
%\usepackage{peeters_braket}
%\usepackage{peeters_figures}
%\usepackage{siunitx}
%
%\beginArtNoToc
%
%\generatetitle{Fundamental theorem of geometric calculus}
%\label{chap:fundamentalTheoremOfCalculus}

Having generalized line, surface, and volume integrals to multivector functions, we wish to state the form of the integrand that is perfectly integrable.  That statement requires bidirectional integration operators, denoted using left, right, or left-right overarrows, as follows.

\index{\(\rboldpartial\)}
\index{\(\lboldpartial\)}
\index{\(\lrboldpartial\)}
%why: after:
%Because integral operators like the vector derivative and gradient do not necessarily commute with multivector functions nor hyper-volume elements, it is useful to introduce

\makedefinition{Bidirectional vector derivative operators.}{dfn:fundamentalTheoremOfCalculus:240}{
Given a hypervolume parameterized by \( k \) parameters, k-volume volume element \( d^k \Bx \), and
multivector functions \( F, G \), let
\begin{equation*}
\lroverarrow{L} = \sum_i \Bx^i \partial_i,
\end{equation*}
designate a linear differential operator (i.e. the gradient or vector derivative), where the partials act on multivector functions to the left or right (but not the reciprocal frame vectors \( \Bx^i \)).

To express unidirection action of the operator only to the left or right, we use arrows to designate the scope of the derivatives, writing respectively
\begin{equation*}
\begin{aligned}
\int_V F d^k\Bx \loverarrow{L} G &= \sum_i \int_V \lr{ \partial_i F } d^k\Bx \Bx^i G \\
\int_V F d^k\Bx \roverarrow{L} G &= \sum_i \int_V F d^k\Bx \Bx^i \lr{ \partial_i G },
\end{aligned}
\end{equation*}
and designate bidirectional action as
\begin{equation*}
\int_V F d^k\Bx \lroverarrow{L} G
\equiv
\int_V \lr{ F d^k\Bx \loverarrow{L}} G
+
\int_V F d^k\Bx \lr{ \roverarrow{L} G }.
\end{equation*}
In all such cases \( L \) operates on \( F \) and \( G \), but not the volume element \( d^k \Bx \), which may also be a function of the implied parameterization.
%%The explicit meaning of these directional acting derivative operations is given by the chain rule coordinate expansion
%%\begin{equation*}
%%F d^k \Bx \lroverarrow{L} G
%%=
%%F d^k \Bx \lr{ \sum_i \Bx^i {\stackrel{ \leftrightarrow }{\partial_i}} } G
%%=
%%(\partial_i F) d^k \Bx \sum_i \Bx^i G
%%+
%%F d^k \Bx \sum_i \Bx^i (\partial_i G)
%%\equiv
%%(F d^k \Bx \loverarrow{L}) G
%%+
%%F d^k \Bx (\roverarrow{L} G),
%%\end{equation*}
} % definition

The vector derivative
% (and gradient)
may not commute with \( F, G \) nor the volume element \( d^k \Bx \), so we are forced to use some notation to indicate what the vector derivative (or gradient) acts on.
In conventional right acting cases, where there is no ambiguity, arrows will usually be omitted, but braces may also be used to indicate the scope of derivative operators.
This bidirectional notation will also be used for the gradient, especially for volume integrals in \R{3} where the vector derivative is identical to the gradient.

Some authors use the Hestenes dot notation, with overdots or primes to indicating the exact scope of multivector derivative operators, as in
\begin{equation}\label{eqn:fundamentalTheoremOfCalculus:260}
\dot{F} d^k \Bx \dot{\boldpartial} \dot{G} =
\dot{F} d^k \Bx \dot{\boldpartial} G
+
F d^k \Bx \dot{\boldpartial} \dot{G}.
\end{equation}
The dot notation has the advantage of emphasizing that the action of the vector derivative (or gradient) is on the functions \( F, G \), and not on the hypervolume element \( d^k \Bx \).
However, in this book, where primed operators such as \( \spacegrad' \) are used to indicate that derivatives are taken with respect to primed \( \Bx' \) variables, a mix of dots and ticks would have been confusing.
%Over arrows also have the advantage of being visually conspicuous.

