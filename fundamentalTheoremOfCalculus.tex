%
% Copyright � 2016 Peeter Joot.  All Rights Reserved.
% Licenced as described in the file LICENSE under the root directory of this GIT repository.
%
%{
%\input{../blogpost.tex}
%\renewcommand{\basename}{fundamentalTheoremOfCalculus}
%\renewcommand{\dirname}{notes/phy1520/}
%%\newcommand{\dateintitle}{}
%%\newcommand{\keywords}{}
%
%\input{../peeter_prologue_print2.tex}
%
%\usepackage{peeters_layout_exercise}
%\usepackage{peeters_braket}
%\usepackage{peeters_figures}
%\usepackage{siunitx}
%
%\beginArtNoToc
%
%\generatetitle{Fundamental theorem of geometric calculus}
%\label{chap:fundamentalTheoremOfCalculus}

\subsection{Hypervolume integral}
We wish to generalize the concepts of line, surface and volume integrals to hypervolumes and multivector functions, and define a hypervolume integral as

\makedefinition{Multivector integral.}{dfn:fundamentalTheoremOfCalculus:240}{
Given a hypervolume parameterized by \( k \) parameters, k-volume volume element \( d^k \Bx \), and
multivector functions \( F, G \), a k-volume integral with the vector derivative acting to the right on \( G \) is written as
\begin{equation*}
\int_V d^k\Bx \rboldpartial G,
\end{equation*}
a k-volume integral with the vector derivative acting to the left on \( F \) is written as
\begin{equation*}
\int_V F d^k\Bx \lboldpartial,
\end{equation*}
and a k-volume integral with the vector derivative acting bidirectionally on \( F, G \) is written as
\begin{equation*}
\int_V F d^k\Bx \lrboldpartial G
\equiv
\int_V \lr{ F d^k\Bx \lboldpartial} G
+
\int_V F d^k\Bx \lr{ \rboldpartial G }.
\end{equation*}
The explicit meaning of these directionally acting derivative operations is given by the chain rule coordinate expansion
\begin{dmath*}
F d^k \Bx \lrboldpartial G
=
F d^k \Bx \lr{ \sum_i \Bx^i {\stackrel{ \leftrightarrow }{\partial_i}} } G
=
(\partial_i F) d^k \Bx \sum_i \Bx^i G
+
F d^k \Bx \sum_i \Bx^i (\partial_i G)
\equiv
(F d^k \Bx \lboldpartial) G
+
F d^k \Bx (\rboldpartial G),
\end{dmath*}
with \( \boldpartial \) acting on \( F \) and \( G \), but not the volume element \( d^k \Bx \), which may also be a function of the implied parameterization.
} % definition

The vector derivative
% (and gradient)
may not commute with \( F, G \) nor the volume element \( d^k \Bx \), so we are forced to use some notation to indicate what the vector derivative (or gradient) acts on.
In conventional right acting cases, where there is no ambiguity, arrows will usually be omitted, but braces may also be used to indicate the scope of derivative operators.
This bidirectional notation will also be used for the gradient, especially for volume integrals in \R{3} where the vector derivative is identitical to the gradient.

Some authors use the Hestenes dot notation, with overdots or primes to indicatating the exact scope of multivector derivative operators, as in
\begin{dmath}\label{eqn:fundamentalTheoremOfCalculus:260}
\dot{F} d^k \Bx \dot{\boldpartial} \dot{G} =
\dot{F} d^k \Bx \dot{\boldpartial} G
+
F d^k \Bx \dot{\boldpartial} \dot{G}.
\end{dmath}
The dot notation has the advantage of emphasizing that the action of the vector derivative (or gradient) is on the functions \( F, G \), and not on the hypervolume element \( d^k \Bx \).
However, in this book, where primed operators such as \( \spacegrad' \) are used to indicate that derivatives are taken with respect to primed \( \Bx' \) variables, a mix of dots and ticks would have been confusing.
%Over arrows also have the advantage of being visually conspicuous.

\subsection{Fundamental theorem.}
\index{fundamental theorem of geometric calculus}

The fundamental theorem of geometric calculus is a generalization of many conventional scalar and vector integral theorems, and relates a hypervolume integral to its boundary.
This is a a powerful theorem, which we will use with Green's functions to solve Maxwell's equation, but also to derive the geometric algebra form of Stokes' theorem, from which most of the familiar integral calculus results follow.

\maketheorem{Fundamental theorem of geometric calculus}{thm:fundamentalTheoremOfCalculus:1}{
Given
multivectors \(F, G \),
a parameterization \( \Bx = \Bx(u_1, u_2, \cdots) \), with hypervolume element \( d^k \Bx = d^k u I_k \), where
\( I_k = \Bx_1 \wedge \Bx_2 \wedge \cdots \wedge \Bx_k \), the hypervolume integral is related to the boundary integral by
\begin{equation*}
\int_V F d^k \Bx \lrboldpartial G = \oint_{\partial V} F d^{k-1} \Bx G,
\end{equation*}
where \( \partial V \) represents the boundary of the volume, and \( d^{k-1} \Bx \) is the hypersurface element.  The hypersurface element and boundary integral is defined for \( k > 1 \) as
\begin{equation*}
\oint_{\partial V} F d^{k-1} \Bx G
\equiv
\sum_i \int d^{k-1} u_i \evalbar{ \lr{ F \lr{ I_k \cdot \Bx^i} G }}{\Delta u_i},
\end{equation*}
where \( d^{k-1} u_i \) is the product of all \( du_j \) except for \( du_i \).
For
\( k = 1 \) the hypersurface element and associated
boundary ``integral''
is really just convienent general shorthand, and
should be taken to mean the evaluation of the \( F G \) multivector product over the range of the parameter
\begin{equation*}
\oint_{\partial V} F d^{0} \Bx G
\equiv
\evalbar{ F G }{\Delta u_1}.
\end{equation*}
} % theorem

The geometry of the hypersurface element \( d^{k-1} \Bx \) will be made more clear when we
consider the specific cases of \( k = 1, 2, 3 \), representing generalized line, surface, and volume integrals respectively.

We will sketch a proof for \( k < 4 \) that omits a number of details.  This starts with expanding the hypervolume
integral, by separating the
geometric product of the volume element and vector derivative direction vectors into dot and wedge contributions
\begin{dmath}\label{eqn:fundamentalTheoremOfCalculus:300}
\int_V F d^k \Bx \lrboldpartial G
=
\sum_i \int_V d^k u F I_k \Bx^i \lrpartial_i G
=
\sum_i \int_V d^k u F \lr{ I_k \cdot \Bx^i +  I_k \wedge \Bx^i } \lrpartial_i G.
\end{dmath}
Because \( \Bx^i \) lies in \( \Span \setlr{ \Bx_j } \), the wedge product above is zero, leaving
\begin{dmath}\label{eqn:fundamentalTheoremOfCalculus:320}
\int_V F d^k \Bx \lrboldpartial G
=
\sum_i \int_V d^k u F \lr{ I_k \cdot \Bx^i} \lrpartial_i G,
=
\sum_i \int_V d^k u (\partial_i F) I_k \cdot \Bx^i G
+
\sum_i \int_V d^k u F \lr{ I_k \cdot \Bx^i} (\partial_i G)
=
\sum_i \int_V d^k u \partial_i \lr{ F \lr{ I_k \cdot \Bx^i} G }
-
\int_V d^k u F \lr{ \sum_i \partial_i \lr{ I_k \cdot \Bx^i } } G.
\end{dmath}

The sum in the second integral turns out to be zero, but it slightly messy to show in general.
For \( k = 1 \) this sum is trivial
\begin{equation}\label{eqn:fundamentalTheoremOfCalculus:360}
\partial_1 (\Bx_1 \cdot \Bx^1) = \partial_1 1 = 0.
\end{equation}
For \( k = 2 \), this sum is
\begin{dmath}\label{eqn:fundamentalTheoremOfCalculus:380}
\sum_{i = 1}^2 \partial_i \lr{ I_3 \cdot \Bx^i }
=
\partial_1 ((\Bx_1 \wedge \Bx_2) \cdot \Bx^1)
+
\partial_2 ((\Bx_1 \wedge \Bx_2) \cdot \Bx^2)
=
\partial_1 (-\Bx_2)
+
\partial_2 \Bx_1
=
-\frac{\partial^2 \Bx}{\partial u_1 \partial_2}
+\frac{\partial^2 \Bx}{\partial u_2 \partial_1},
\end{dmath}
which is zero by equality of mixed partials.  For \( k = 2 \), this sum is
\begin{dmath}\label{eqn:fundamentalTheoremOfCalculus:400}
\begin{aligned}
\sum_{i = 1}^3 \partial_i \lr{ I_k \cdot \Bx^i }
&=
\partial_3 \lr{ (\Bx_1 \wedge \Bx_2 \wedge \Bx_3) \cdot \Bx^3 }
+
\partial_1 \lr{ (\Bx_2 \wedge \Bx_3 \wedge \Bx_1) \cdot \Bx^1 }
+
\partial_2 \lr{ (\Bx_3 \wedge \Bx_1 \wedge \Bx_2) \cdot \Bx^2 } \\
&=
\partial_3 \lr{ \Bx_1 \wedge \Bx_2 }
+
\partial_1 \lr{ \Bx_2 \wedge \Bx_3 }
+
\partial_2 \lr{ \Bx_3 \wedge \Bx_1 } \\
&=
         (\partial_3 \Bx_1) \wedge \Bx_2 + \Bx_1 \wedge (\partial_3 \Bx_2) \\
&\quad + (\partial_1 \Bx_2) \wedge \Bx_3 + \Bx_2 \wedge (\partial_1 \Bx_3) \\
&\quad + (\partial_2 \Bx_3) \wedge \Bx_1 + \Bx_3 \wedge (\partial_2 \Bx_1) \\
&=
\Bx_2 \wedge \lr{ - \partial_3 \Bx_1 + \partial_1 \Bx_3 }
+
\Bx_3 \wedge \lr{ - \partial_1 \Bx_2 + \partial_2 \Bx_1 }
+
\Bx_1 \wedge \lr{ - \partial_2 \Bx_3 + \partial_3 \Bx_2 } \\
&=
\Bx_2 \wedge \lr{ - \frac{\partial^2 \Bx}{\partial_3 \partial_1} + \frac{\partial^2 \Bx}{\partial_1 \partial_3} }
+
\Bx_3 \wedge \lr{ - \frac{\partial^2 \Bx}{\partial_1 \partial_2} + \frac{\partial^2 \Bx}{\partial_2 \partial_1} }
+
\Bx_1 \wedge \lr{ - \frac{\partial^2 \Bx}{\partial_2 \partial_3} + \frac{\partial^2 \Bx}{\partial_3 \partial_2} },
\end{aligned}
\end{dmath}
which is also zero by equality of mixed partials.

With the sum in the second integral of \cref{eqn:fundamentalTheoremOfCalculus:320} killed, we are left with a sum of perfect differentials, each separately integratable
\begin{dmath}\label{eqn:fundamentalTheoremOfCalculus:340}
\int_V F d^k \Bx \lrboldpartial G
=
\sum_i \int_{\partial V} d^{k-1} u_i \int_{\Delta u_i} du_i \PD{u_i}{} \lr{ F \lr{ I_k \cdot \Bx^i} G}
=
\sum_i \int_{\partial V} d^{k-1} u_i \evalbar{\lr{ F \lr{ I_k \cdot \Bx^i} G}}{\Delta u_i},
\end{dmath}
which completes the sketch of the proof.
Most of the theoretical heavy lifting was done by the reciprocal frame vectors, but the final result does not actually require computing those vectors.
When \( k \) equals the dimension of the space, as in \R{3} volume integrals, the vector derivative \( \boldpartial \) is identical to the \( \spacegrad \), in which case we do not even require the reciprocal frame vectors to express the gradient.

For a full proof of \cref{thm:fundamentalTheoremOfCalculus:1}, additional mathematical sublties must be considered.
Issues of connectivity of the hypervolumes (and integration theory in general) are covered very nicely in
\citep{aMacdonaldVAGC}.
For other general issues required for a complete proof, like the triagularization of the volume and its boundary, please see
\citep{hestenes1985clifford}, \citep{doran2003gap}, and \citep{sobczyk2011fundamental}.

Before moving on to the line, surface, and volume integral cases, we will state and prove the
general Stokes' theorem in its geometric algebra form.

%}
%\EndArticle
