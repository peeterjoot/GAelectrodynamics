%
% Copyright © 2018 Peeter Joot.  All Rights Reserved.
% Licenced as described in the file LICENSE under the root directory of this GIT repository.
%
%{

In this section we generalize \cref{eqn:reciprocal_R2:180} to \R{3} vectors, which will illustrate the general case by example.

Given
a subspace spanned by a three vector basis \( \setlr{ \Bx_1, \Bx_2, \Bx_3 } \) the reciprocal frame vectors can be written as dot products

\begin{dmath}\label{eqn:reciprocal_R3:320}
\begin{aligned}
\Bx^1 &= \lr{ \Bx_2 \wedge \Bx_3 } \cdot \lr{ \Bx^3 \wedge \Bx^2 \wedge \Bx^1 } \\
\Bx^2 &= \lr{ \Bx_3 \wedge \Bx_1 } \cdot \lr{ \Bx^1 \wedge \Bx^3 \wedge \Bx^2 } \\
\Bx^3 &= \lr{ \Bx_1 \wedge \Bx_2 } \cdot \lr{ \Bx^2 \wedge \Bx^1 \wedge \Bx^3 }.
\end{aligned}
\end{dmath}

Each of those trivector terms equals \( - \Bx^1 \wedge \Bx^2 \wedge \Bx^3 \) and can be related to the (known) pseudoscalar \( \Bx_1 \wedge \Bx_2 \wedge \Bx_3 \) by observing that

\begin{dmath}\label{eqn:reciprocal_R3:340}
\lr{ \Bx^1 \wedge \Bx^2 \wedge \Bx^3 } \cdot \lr{ \Bx_3 \wedge \Bx_2 \wedge \Bx_1 }
=
\Bx^1 \cdot \lr{ \Bx^2 \cdot \lr{ \Bx^3 \cdot \lr{ \Bx_3 \wedge \Bx_2 \wedge \Bx_1 } }}
=
\Bx^1 \cdot \lr{ \Bx^2 \cdot \lr{ \Bx_2 \wedge \Bx_1 } }
=
\Bx^1 \cdot \Bx_1
=
1,
\end{dmath}
which means that

\begin{dmath}\label{eqn:reciprocal_R3:360}
-\Bx^1 \wedge \Bx^2 \wedge \Bx^3
= -\inv{ \Bx_3 \wedge \Bx_2 \wedge \Bx_1 }
= \inv{ \Bx_1 \wedge \Bx_2 \wedge \Bx_3 },
\end{dmath}
and

\boxedEquation{eqn:reciprocal:380}{
\begin{aligned}
\Bx^1 &= \lr{ \Bx_2 \wedge \Bx_3 } \cdot \inv{ \Bx_1 \wedge \Bx_2 \wedge \Bx_3 } \\
\Bx^2 &= \lr{ \Bx_3 \wedge \Bx_1 } \cdot \inv{ \Bx_1 \wedge \Bx_2 \wedge \Bx_3 } \\
\Bx^3 &= \lr{ \Bx_1 \wedge \Bx_2 } \cdot \inv{ \Bx_1 \wedge \Bx_2 \wedge \Bx_3 }
\end{aligned}
}

Geometrically, dotting with this trivector is a duality transformation within the subspace spanned by the three vectors \( \Bx_1, \Bx_2, \Bx_3 \), also scaling the result so that the \( \Bx_i \cdot \Bx^j = {\delta_i}^j \) condition is satisfied.  The scaling factor is the volume of the parallelopiped spanned by \( \Bx_1, \Bx_2, \Bx_3 \).

%}
