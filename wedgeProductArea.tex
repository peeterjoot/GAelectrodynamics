The coordinate representation of the \R{2} wedge product (\cref{eqn:SimpleProducts2:1720}) had a single \( \Be_{12} \) bivector factor, whereas the expansion in coordinates for the general \R{N} wedge product was considerably messier (\cref{eqn:SimpleProducts2:1320}).
This difference is essentially just one of the choice of basis.

A simpler coordinate representation for the \R{N} wedge product follows by choosing an
orthonormal basis
for the planar subspace spanned by the wedge vectors.
Given vectors \( \Ba, \Bb \), let \( \setlr{\ucap, \vcap} \) be an orthonormal basis for the plane subspace
\( P = \Span\setlr{ \Ba, \Bb } \).
The coordinate representation in this basis is

\begin{dmath}\label{eqn:wedgeProductArea:1900}
\begin{aligned}
\Ba &= (\Ba \cdot \ucap) \ucap + (\Ba \cdot \vcap) \vcap \\
\Bb &= (\Bb \cdot \ucap) \ucap + (\Bb \cdot \vcap) \vcap.
\end{aligned}
\end{dmath}

Wedging these vectors gives

\begin{dmath}\label{eqn:SimpleProducts2:1860}
\Ba \wedge \Bb
=
\gpgradetwo{
   \lr{
   (\Ba \cdot \ucap) \ucap + (\Ba \cdot \vcap) \vcap
   }
   \lr{
   (\Bb \cdot \ucap) \ucap + (\Bb \cdot \vcap) \vcap
   }
}
=
\gpgradetwo{
\cancel{
   (\Ba \cdot \ucap) (\Bb \cdot \ucap) \ucap^2
}
+
\cancel{
   (\Ba \cdot \vcap) (\Bb \cdot \vcap) \vcap^2
}
+
\lr{
      (\Ba \cdot \ucap)
   (\Bb \cdot \vcap)
   -
   (\Ba \cdot \vcap) (\Bb \cdot \ucap)
}
\ucap \vcap
}
=
\lr{
      (\Ba \cdot \ucap)
   (\Bb \cdot \vcap)
   -
   (\Ba \cdot \vcap) (\Bb \cdot \ucap)
}
\ucap \vcap.
\end{dmath}

Such a basis allows for the most compact (single term) coordinate representation of the wedge product

\begin{dmath}\label{eqn:SimpleProducts2:1880}
\Ba \wedge \Bb
=
\begin{vmatrix}
   \Ba \cdot \ucap & \Ba \cdot \vcap \\
   \Bb \cdot \ucap & \Bb \cdot \vcap
\end{vmatrix}
\ucap \vcap.
\end{dmath}

FIXME: justify an geometric interpretation (area).  show that multiple factorizations are possible, so there is naturally no preferred geometry for a bivector.  tie this to orientation.

The wedge product is therefore the (possibly signed) area of the parallelopiped formed by the vectors \( \Ba, \Bb \), multiplied by a unit pseudoscalar for the subspace of the plane \( P \).
Provided the area of this parallelopiped is non-zero, which is always the case for non-colinear vectors, there are clearly many possible normal factorizations for the wedge product.

