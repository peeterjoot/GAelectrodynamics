%
% Copyright � 2018 Peeter Joot.  All Rights Reserved.
% Licenced as described in the file LICENSE under the root directory of this GIT repository.
%
\maketheorem{Poynting's theorem (differential form.)}{thm:poyntingTheorem:1180}{
The adjoint energy momentum tensor of the spacetime gradient satisfies the following multivector equation
\begin{equation*}
\overbar{T}(\spacegrad + (1/c)\partial_t) = \frac{\epsilon}{2} \lr{ F^\dagger J + J^\dagger F }.
\end{equation*}
The multivector \( F^\dagger J + J^\dagger F \) can only have scalar and vector grades, since it equals its reverse.
This equation can be put into a form that is more obviously a conservation law by stating it as a set of
scalar grade identities
\begin{equation*}
\spacegrad \cdot \gpgradeone{ T(a) } + \inv{c} \PD{t}{} \gpgradezero{ T(a) }
=
\frac{\epsilon}{2} \gpgradezero{ a( F^\dagger J + J \dagger F) },
\end{equation*}
or as a pair of scalar and vector grade conservation relationships
%%which expands to the multivector equation
%\begin{equation*}
%\inv{c} \PD{t}{} \lr{ \calE - \frac{\BS}{c} }
%+ \spacegrad \cdot \frac{\BS}{c}
%+ \BT(\spacegrad)
%=
%-\inv{c} \lr{ \BE \cdot \BJ + \BH \cdot \BM }
%+
%\rho \BE + \epsilon \BE \cross \BM
%+
%\rho_\txtm \BH + \mu \BJ \cross \BH,
%\end{equation*}
%or as separate scalar and vector equations
\begin{equation*}
\begin{aligned}
\inv{c} \PD{t}{\calE} + \spacegrad \cdot \frac{\BS}{c} &= -\inv{c} \lr{ \BE \cdot \BJ + \BH \cdot \BM } \\
-\inv{c^2} \PD{t}{\BS} + \BT(\spacegrad) &= \rho \BE + \epsilon \BE \cross \BM + \rho_\txtm \BH + \mu \BJ \cross \BH.
\end{aligned}
\end{equation*}
Conventionally, only the scalar grade relating the time rate of change of the energy density to the flux of the Poynting vector, is called Poynting's theorem.
Here the more general multivector (adjoint) relationship is called \textit{Poynting's theorem}, which includes conservation laws relating for the field energy and momentum densities and conservation laws relating the Poynting vector components and the Maxwell stress tensor.
} % theorem
