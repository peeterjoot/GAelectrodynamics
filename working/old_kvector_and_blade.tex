%
% Copyright © 2017 Peeter Joot.  All Rights Reserved.
% Licenced as described in the file LICENSE under the root directory of this GIT repository.
%

A fair amount of nomenclature and notation is unfortunately required before systematically examining the implications of the multivector space axioms that define geometric algebra.

Multivectors which can be factored into normal vector products, such as
\begin{dmath}\label{eqn:multiplication:220}
\Be_1 \Be_2 + 3 \Be_1 \Be_3
=
\Be_1 (\Be_2 + 3 \Be_3),
\end{dmath}

are blades.
In contrast, the following grade 2 multivectors

\begin{dmath}\label{eqn:multiplication:240}
\Be_1 \Be_2 + \Be_3 \Be_4,
\end{dmath}

and
\begin{dmath}\label{eqn:multiplication:260}
\Be_1 \Be_2 + \Be_2 \Be_3 + \Be_3 \Be_1,
\end{dmath}

which cannot be factored into two vector products, are not blades.

\index{k-vector}
\makedefinition{k-vector.}{dfn:multivector:kvector}{
A sum of k-blades is called a k-vector.
} % definition

Multivectors are therefore sums of k-vectors with different grades.

All the k-blade examples in 
\cref{eqn:multivector:180}
 are also k-vectors.
K-vectors with grades 2 and 3 are so pervasive that they are given special names.

\index{bivector}
\makedefinition{Bivector.}{dfn:multivector:bivector}{
A bivector, or 2-vector, is a k-vector with grade 2.
} % definition

Any 2-blade, such as the product \( \Be_1 \Be_2 \) is a bivector.
Any sum of 2-blades, such as \( \Be_2 \Be_3 + 3 \Be_4 \Be_1 \), is also a bivector.
%Each of \( \Be_1 \Be_2, \Be_2 \Be_1, \Be_1 \Be_2 + \Be_2 \Be_3 \), and \( \Be_1 \Be_2 + \Be_3 \Be_4 \) are bivectors.
%All but the last of these represents an oriented plane segment.

\index{trivector}
\makedefinition{Trivector.}{dfn:multivector:trivector}{
A trivector, or 3-vector, is a k-vector with grade 3.
} % definition

%Quantities with higher grades than 3 are not generally given explicit names.
The multivector \( \Be_3 \Be_1 \Be_2 \) is a trivector, as is \( \Be_1 \Be_2 \Be_3 + 3 \Be_5 \Be_4 \Be_1 \).
The latter is not a blade.
%Each of \( \Be_1 \Be_2 \Be_3, \Be_1 \Be_3 \Be_2, \Be_1 \Be_4 \Be_2 \) are trivectors.
% , and represent oriented volumes.


