%
% Copyright © 2017 Peeter Joot.  All Rights Reserved.
% Licenced as described in the file LICENSE under the root directory of this GIT repository.
%
FIXME: find a home for this.  Also have a number of figures in junk.pdf that should be brought in.

Perhaps after triple wedge products, and certainly after areas of wedge products.

Given this, it is reasonable to associate an orientation with 2-blades (like \( \ucap \vcap \) or \( \Be_{12} \)).
We will also see that an area can also be associated with a 2-blade.

Eventually this will justify a geometrical characterization of k-vector blades along the following lines

\begin{itemize}
\item A scalar (0-vector) represents an oriented (signed) point.
\item A vector (1-vector, also always a 1-blade) represents an oriented line segment.
\item A bivector, if it is also a 2-blade, represents an oriented area segment.  You could think of such a bivector as a representation of an area elements that has, say, an upper or lower side, just as a vector can be thought of as having a head and a tail.
\item A trivector, if also a 3-blade, represents an oriented volume segment.  We can associate an inwards or outward normal with the orientation of this volume element, or associate the orientation with a rotational sense as depicted in
\cref{fig:orientedVolume:orientedVolumeFig1}.
\imageFigure{../figures/GAelectrodynamics/orientedVolumeFig1}{Oriented Volume}{fig:orientedVolume:orientedVolumeFig1}{0.3}
\item A k-blade represents an oriented k-dimensional hypervolume element.
\end{itemize}

As with multivectors in general, it isn't clear whether there is a good geometric characterization for a k-vector that isn't also a blade.


