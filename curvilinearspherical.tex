%
% Copyright � 2017 Peeter Joot.  All Rights Reserved.
% Licenced as described in the file LICENSE under the root directory of this GIT repository.
%
%{
%\index{curvilinear coordinates}
\index{curvilinear bases}
\index{spherical coordinates}
We will use the physics and engineering convention for spherical polar angles, as illustrated in
\cref{fig:sphericalPolar:sphericalPolarFig1}.
The conventional way to introduce a spherical polar position vector representation is through coordinates, where inspection of the geometry shows that
\begin{equation}\label{eqn:curvilinearspherical:17}
\Bx = r \lr{ \sin\theta \cos\phi, \sin\theta \sin\phi, \cos\theta },
\end{equation}
We can find a compact GA representation from this coordinate representation without too much trouble (\cref{problem:curvilinearspherical:2}), but we can also examine the geometry of the situation directly.
\imageFigure{../figures/GAelectrodynamics/sphericalPolarFig1}{Spherical polar conventions.}{fig:sphericalPolar:sphericalPolarFig1}{0.3}

For the unit bivector for the azimuthal (x-y) plane, let's write
\begin{equation}\label{eqn:curvilinearspherical:19}
i = \Be_{12},
\end{equation}
We see that the projection of \( \Bx \) onto the azimuthal plane has direction \( \Proj_i \Bx \propto \Be_1 e^{i\phi} \).  As we rotate from the north pole \( \Be_3 \) down through \( \Bx \) to the azimuthal plane, we are rotating in the plane
\begin{equation}\label{eqn:curvilinearspherical:21}
\begin{aligned}
j &= \Be_3 \wedge \lr{ \Be_1 e^{i \phi} } \\
  &= \Be_{31} e^{i \phi}.
\end{aligned}
\end{equation}
So, just by inspection, we've found the spherical polar representation of \( \Bx \), which is
\begin{equation}\label{eqn:curvilinearspherical:60}
\Bx = r \Be_3 e^{j \theta}.
\end{equation}
Observe that all the \( \phi \) dependency sneakily hides out in the unit bivector \( j = j(\phi) \).

Given a parameterized representation of \( \Bx \), we may compute the basis elements
\begin{subequations}
\label{eqn:curvilinearspherical:80}
\begin{equation}\label{eqn:curvilinearspherical:100}
\Bx_r = \Be_3 e^{j \theta}
\end{equation}
\begin{equation}\label{eqn:curvilinearspherical:120}
\begin{aligned}
\Bx_\theta
&= r \Be_3 j e^{j \theta} \\
&= r \Be_3 e^{j \lr{ \theta + \pi/2}} \\
%&= r \Be_3 \Be_{31} e^{i\phi} e^{j \theta} \\
%&= r \Be_1 e^{i\phi} e^{j \theta}
\end{aligned}
\end{equation}
\begin{equation}\label{eqn:curvilinearspherical:140}
\begin{aligned}
\Bx_\phi
&= \PD{\phi}{} \lr{ r \Be_3 e^{j\theta} } \\
&= r \Be_3 \PD{\phi}{} \lr{ \cos\theta + j \sin\theta } \\
&= r \Be_3 \sin\theta \PD{\phi}{} \lr{ \Be_{31} e^{i \phi} } \\
&= r \sin\theta \Be_1 \Be_{12} e^{i \phi} \\
&= r \sin\theta \Be_2 e^{i \phi}.
\end{aligned}
\end{equation}
\end{subequations}
See \cref{problem:dexpquestion:1} for why it is necessary to first expand the exponential into its trigonometric constituents.
These vectors are all mutually orthogonal (\cref{problem:sphericaldot:1}).

Orthonormalization of the curvilinear basis is now possible by inspection
\index{\(\rcap\)!spherical}
\index{\(\thetacap\)!spherical}
\index{\(\phicap\)!spherical}
\begin{equation}\label{eqn:curvilinearspherical:240}
\begin{aligned}
\rcap &= \Bx_r = \Be_3 e^{j \theta} \\
%\thetacap &= \inv{r} \Bx_\theta = \Be_1 e^{i\phi} e^{j \theta} \\
\thetacap &= \inv{r} \Bx_\theta = \Be_3 e^{j \lr{ \theta + \pi/2}} = \rcap j \\
\phicap &= \inv{r \sin\theta} \Bx_\phi = \Be_2 e^{i \phi},
\end{aligned}
\end{equation}
so
\index{\(\Bx^r\)!spherical}
\index{\(\Bx^\theta\)!spherical}
\index{\(\Bx^\phi\)!spherical}
\begin{equation}\label{eqn:curvilinearspherical:260}
\begin{aligned}
\Bx^r &= \rcap = \Be_3 e^{j \theta} \\
%\Bx^\theta &= \inv{r} \thetacap = \inv{r} \Be_1 e^{i\phi} e^{j \theta} \\
\Bx^\theta &= \inv{r} \thetacap = \inv{r} \Be_3 e^{j \lr{\theta + \pi/2}} \\
\Bx^\phi &= \inv{r \sin\theta} \phicap = \inv{r \sin\theta} \Be_2 e^{i \phi}.
\end{aligned}
\end{equation}

\index{gradient!spherical}
In particular, this shows that the spherical representation of the gradient is
\begin{equation}\label{eqn:curvilinearspherical:280}
\begin{aligned}
\spacegrad
&=
\Bx^r \PD{r}{}
+ \Bx^\theta \PD{\theta}{}
+ \Bx^\phi \PD{\phi}{} \\
&=
\rcap \PD{r}{}
+\inv{r} \thetacap \PD{\theta}{}
+\inv{r \sin\theta} \phicap \PD{\phi}{}.
\end{aligned}
\end{equation}

%\iftoggle{kindle-version}{}{
It's worth pointing out that computations like finding the curvilinear basis vectors is easily performed in software
\begin{mmaCell}[moredefined={i, j, ej, x, xr, xt, xp, e, GeometricProduct},morepattern={phi_, t_, p_, r_, theta_}]{Input}
  ClearAll[i, j, ej, x, xr, xt, xp]
  i = e[1, 2];
  j[phi_] = GeometricProduct[e[3, 1], Cos[phi] + i Sin[phi]];
  ej[t_, p_] = Cos[t] + j[p] Sin[t];
  x[r_, t_, p_] = r GeometricProduct[e[3], ej[t, p]];

  xr[r_, theta_, phi_] = D[x[a, theta, phi], a] /. a\(\pmb{\to}\)r;
  xt[r_, theta_, phi_] = D[x[r, t, phi], t] /. t\(\pmb{\to}\)theta;
  xp[r_, theta_, phi_] = D[x[r, theta, p], p] /. p\(\pmb{\to}\)phi;

  \{x[r, \mmaUnd{\(\pmb{\theta}\)}, \mmaUnd{\(\pmb{\phi}\)}],
  xr[r, \mmaUnd{\(\pmb{\theta}\)}, \mmaUnd{\(\pmb{\phi}\)}],
  xt[r, \mmaUnd{\(\pmb{\theta}\)}, \mmaUnd{\(\pmb{\phi}\)}],
  xp[r, \mmaUnd{\(\pmb{\theta}\)}, \mmaUnd{\(\pmb{\phi}\)}]\} // Column
\end{mmaCell}
\begin{mmaCell}{Output}
  r (Cos[\(\theta\)] e[3] + Cos[\(\phi\)] e[1] Sin[\(\theta\)] + e[2] Sin[\(\theta\)] Sin[\(\phi\)])
  Cos[\(\theta\)] e[3] + Cos[\(\phi\)] e[1] Sin[\(\theta\)] + e[2] Sin[ \(\theta\)] Sin[\(\phi\)]
  r (Cos[\(\theta\)] Cos[\(\phi\)] e[1] - e[3] Sin[\(\theta\)] + Cos[\(\theta\)] e[2] Sin[\(\phi\)])
  r (Cos[\(\phi\)] e[2] Sin[\(\theta\)] - e[1] Sin[\(\theta\)] Sin[\(\phi\)])
\end{mmaCell}
where it also easy to show that these vectors are mutually perpendicular
%}
%\iftoggle{kindle-version}{}{
\begin{mmaCell}[moredefined={x1, x2, x3, xr, xt, xp, InnerProduct}]{Input}
  ClearAll[x1, x2, x3]
  x1 = xr[r, \mmaUnd{\(\pmb{\theta}\)}, \mmaUnd{\(\pmb{\phi}\)}];
  x2 = xt[r, \mmaUnd{\(\pmb{\theta}\)}, \mmaUnd{\(\pmb{\phi}\)}];
  x3 = xp[r, \mmaUnd{\(\pmb{\theta}\)}, \mmaUnd{\(\pmb{\phi}\)}];
  MapThread[InnerProduct, \{\{x1, x2, x3\}, \{x2, x3, x1\}\}]// Simplify
\end{mmaCell}
\begin{mmaCell}{Output}
  \{0,0,0\}
\end{mmaCell}
Unfortunately, if we perform these computations in software, we loose our compact representation.
%}

The spherical (oriented) volume element can also be computed in a compact fashion
\begin{equation}\label{eqn:curvilinearspherical:300}
\begin{aligned}
\frac{d^3 \Bx}{dr d\theta d\phi}
&= \Bx_r \wedge \Bx_\theta \wedge \Bx_\phi \\
&= \gpgradethree{ \Bx_r \Bx_\theta \Bx_\phi } \\
&= \gpgradethree{
   \rcap
   r \rcap j
   r \sin\theta \Be_2 e^{i \phi}
   } \\
&= r^2 \sin\theta
   \gpgradethree{
   \Be_{31} e^{i\phi}
   \Be_2 e^{i \phi}
   } \\
&= r^2 \sin\theta\, \Be_{123}.
\end{aligned}
\end{equation}

The scalar factor is the Jacobian with respect to the spherical parameterization
\begin{equation}\label{eqn:curvilinearspherical:320}
\begin{aligned}
\frac{dV}{
dr d\theta d\phi}
&= \frac{\partial( x_1, x_2, x_3)}{\partial(r, \theta, \phi)} \\
&= \begin{vmatrix}
\sin\theta \cos\phi & \sin\theta \sin\phi & \cos\theta \\
r \cos\theta \cos\phi & r \cos\theta \sin\phi & -r \sin\theta \\
-r \sin\theta \sin\phi & r \sin\theta \cos\phi & 0 \\
\end{vmatrix} \\
&= r^2 \sin\theta.
\end{aligned}
\end{equation}

The final reduction of \cref{eqn:curvilinearspherical:300}, and the expansion of the Jacobian \cref{eqn:curvilinearspherical:320}, are both easily verified with software.
%\iftoggle{kindle-version}{}{
\begin{mmaCell}[moredefined={OuterProduct, xr, xt, xp, e1, e2, e3, jacobian, e}]{Input}
  OuterProduct[ xr[r, \mmaUnd{\(\pmb{\theta}\)}, \mmaUnd{\(\pmb{\phi}\)}],
  xt[r, \mmaUnd{\(\pmb{\theta}\)}, \mmaUnd{\(\pmb{\phi}\)}],
  xp[r, \mmaUnd{\(\pmb{\theta}\)}, \mmaUnd{\(\pmb{\phi}\)}]]

  \{e1,e2,e3\} = IdentityMatrix[3];
  jacobian = \{xr[r, \mmaUnd{\(\pmb{\theta}\)}, \mmaUnd{\(\pmb{\phi}\)}],
  xt[r, \mmaUnd{\(\pmb{\theta}\)}, \mmaUnd{\(\pmb{\phi}\)}],
  xp[r, \mmaUnd{\(\pmb{\theta}\)}, \mmaUnd{\(\pmb{\phi}\)}]\} /. \{e[1] \(\pmb{\to}\) e1, e[2] \(\pmb{\to}\) e2, e[3]\(\pmb{\to}\)e3\};
  Det[ jacobian ] // Simplify
\end{mmaCell}
\begin{mmaCell}{Output}
  \mmaSup{r}{2} e[1,2,3] Sin[\(\theta\)]
\end{mmaCell}
\begin{mmaCell}{Output}
  \mmaSup{r}{2} Sin[\(\theta\)]
\end{mmaCell}
%}
Performing these calculations manually are left as problems
for the student (\cref{problem:curvilinearspherical:1}, \cref{problem:volumeselection:1}).
%}
