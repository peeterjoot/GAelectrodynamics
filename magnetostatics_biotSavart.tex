\index{Biot-Savart law}
\index{Green's function!gradient}
The magnetostatic Maxwell equation \cref{eqn:magnetostatics:380} can be inverted directly using the Green's function for the gradient

\begin{dmath}\label{eqn:magnetostatics_biotSavart:40}
I \BB(\Bx)
= \int_V dV' G(\Bx, \Bx') \spacegrad' I \BB(\Bx')
\end{dmath}

This expansion can be simplified by inserting a no-op grade selection operation

\begin{dmath}\label{eqn:magnetostatics_biotSavart:680}
I \BB(\Bx)
= \gpgradetwo{ \int_V dV' G(\Bx, \Bx') \spacegrad' I \BB(\Bx') }
= \int_V dV' \gpgradetwo{ G(\Bx, \Bx') (-\mu \BJ(\Bx')) }
= \inv{4\pi} \int_V dV' \frac{\Bx - \Bx'}{ \Abs{\Bx - \Bx'}^3 } \wedge (-\mu \BJ(\Bx')),
\end{dmath}
or

\boxedEquation{eqn:magnetostatics:760}{
I \BB(\Bx)
= \frac{\mu}{4\pi} \int_V dV' \BJ(\Bx') \wedge \frac{\Bx - \Bx'}{ \Abs{\Bx - \Bx'}^3 }.
}

This is the Biot-Savart law in its GA form.
The traditional expression requires only a duality transformation \( \BJ \wedge \Bf = I ( \BJ \cross \Bf) \), or

\begin{dmath}\label{eqn:magnetostatics_biotSavart:700}
\BB(\Bx)
= \frac{\mu}{4\pi} \int_V dV' \BJ(\Bx') \cross \frac{\Bx - \Bx'}{ \Abs{\Bx - \Bx'}^3 }.
\end{dmath}

The freedom to insert a no-op bivector grade selection may seem like a sneaky move.
To remove doubt about the validity of this move, here is a demonstration that
the scalar grade discarded by the grade selection operation on the integrand of \cref{eqn:magnetostatics_biotSavart:680} is explicitly zero,
provided the current density vanishes faster than \( r \) on the infinite sphere.

\begin{dmath}\label{eqn:magnetostatics_biotSavart:60}
 \int_V dV' \frac{\Bx - \Bx'}{ \Abs{\Bx - \Bx'}^3 } \cdot \BJ(\Bx')
= \frac{\mu}{4\pi} \int_V dV' \lr{ \spacegrad \inv{ \Abs{\Bx - \Bx'} }} \cdot \BJ(\Bx')
=  \int_V dV' \lr{ \spacegrad' \inv{ \Abs{\Bx - \Bx'} }} \cdot \BJ(\Bx')
=  \int_V dV' \lr{
\spacegrad' \cdot \frac{\BJ(\Bx')}{ \Abs{\Bx - \Bx'} }
-
\frac{\spacegrad' \cdot \BJ(\Bx')}{ \Abs{\Bx - \Bx'} }
}.
\end{dmath}

By \cref{eqn:magnetostatics:460}, the divergence of the current density is zero, which kills the second term.
The divergence theorem can be used to express the remaining integral as a surface integral, so

\begin{dmath}\label{eqn:magnetostatics_biotSavart:100}
 \int_V dV' \frac{\Bx - \Bx'}{ \Abs{\Bx - \Bx'}^3 } \cdot \BJ(\Bx')
=  \int_V dV' \spacegrad' \cdot \frac{\BJ(\Bx')}{ \Abs{\Bx - \Bx'} }
=  \int_{\partial V} dA' \ncap \cdot \frac{\BJ(\Bx')}{ \Abs{\Bx - \Bx'} }.
\end{dmath}

Provided the normal component of \( \BJ(\Bx')/\Abs{\Bx - \Bx'} \) vanishes on the boundary of an infinite sphere, we see that the
the scalar selection of the convolution integral is zero, justifying the (sneaky) bivector selection operation.

