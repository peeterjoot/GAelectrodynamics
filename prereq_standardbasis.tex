%
% Copyright © 2017 Peeter Joot.  All Rights Reserved.
% Licenced as described in the file LICENSE under the root directory of this GIT repository.
%
\makedefinition{Dot product.}{dfn:prerequisites:dotproduct}{
Let \( \Bx, \By \) be vectors from a vector space \( V \).
A dot product \( \Bx \cdot \By \) is a mapping \( V \cross V \rightarrow \bbR \)
with the following properties

\begin{tcolorbox}[tab2,tabularx={X|Y},title=Dot product properties.,boxrule=0.5pt]
    Symmetric in both arguments & \( \Bx \cdot \By = \By \cdot \Bx \) \\ \hline
    Bilinear & \( (a \Bx + b \By) \cdot (a' \Bx' + b' \By' ) =
a a' (\Bx \cdot \Bx') + b b' (\By \cdot \By')
+
a b' (\Bx \cdot \By') + b a' (\By \cdot \Bx') \)
\\ \hline
    (Optional) Positive definite & \( \Bx \cdot \Bx \ge 0 \) \\ \hline
\end{tcolorbox}
} % definition

In GA it can be useful to omit the requirement for a dot product to have the positive definite property.
This has specific relevance in electrodynamics, since Maxwell's equations take their simplest form when expressed in terms of four-vector (relativistic) vector spaces.

Because the dot product is bilinear, it is
specified completely by the dot products of a set of basis elements for the space.  For example,
given a basis \( \setlr{ \Be_1, \Be_2, \cdots, \Be_N} \), and two vectors

\begin{dmath}\label{eqn:prerequisites:240}
\begin{aligned}
   \Bx &= \sum_{i = 1}^N x_i \Be_i \\
   \By &= \sum_{i = 1}^N y_i \Be_i,
\end{aligned}
\end{dmath}

the dot product of the two is

\begin{dmath}\label{eqn:prerequisites:260}
\Bx \cdot \By
=
   \lr{ \sum_{i = 1}^N x_i \Be_i } \cdot
   \lr{ \sum_{j = 1}^N y_j \Be_j }
=
   \sum_{i,j = 1}^N x_i y_j \lr{ \Be_i \cdot \Be_j }.
\end{dmath}

Such an expansion in coordinates can be written in matrix form as

\begin{dmath}\label{eqn:prerequisites:280}
\Bx \cdot \By
=
\Bx^\T G \By,
\end{dmath}

where \( G \) is the symmetric matrix with elements \( g_{ij} = \Be_i \cdot \Be_j \).  This matrix \( G \), or its elements \( g_{ij} \) is also called the metric for the space.

In this book the metric is always diagonal, with all diagonal values having an absolute value of one.

\makedefinition{Length}{dfn:prerequisites:norm}{
   The squared norm of a vector \( \Bx \) is defined as

\begin{dmath}\label{eqn:prerequisites:200}
   \Norm{\Bx}^2 = \Bx \cdot \Bx,
\end{dmath}

a quantity that need not be positive.  The length of a vector \( \Bx \) is defined as

\begin{equation*}
\Norm{\Bx} =
\sqrt{\Abs{ \Norm{\Bx}^2 }}.
\end{equation*}
}

%A vector space with an associated norm based length is called a normed vector space.  Any dot product space is also a normed vector space.

\makedefinition{Unit vector}{dfn:prerequisites:unitvector}{
   A vector \( \Bx \) is called a unit vector if its absolute squared length is one, \( \Abs{\Bx \cdot \Bx} = 1 \).
} % definition

%A unit vector \( \xcap \) may be generated from any vector \( \Bx \) that has a non-zero squared norm by computing
%
%\begin{dmath}\label{eqn:prerequisites:220}
%\xcap = \frac{\Bx}{\sqrt{\Abs{\Norm{\Bx}^2}}}.
%\end{dmath}
%
\makedefinition{Normal}{dfn:prerequisites:normal}{
   Two vectors \( \Bx, \By \) are normal, or orthogonal, if their dot product is zero, \( \Bx \cdot \By = 0 \).
}

\makedefinition{Orthonormal}{dfn:prerequisites:orthonormal}{
   Two vectors \( \Bx, \By \) are orthonormal if they are both unit vectors and normal to each other, \( \Bx \cdot \By = 0 \), \( \Abs{\Bx \cdot \Bx} = \Abs{\By \cdot \By} = 1 \).

   A set of vectors \( \setlr{ \Bx, \By, \cdots, \Bz } \) is an orthonormal set if all pairs of vectors in that set are orthonormal.
}

\makedefinition{Standard basis.}{dfn:prerequisites:standardbasis}{
   A basis
\( \setlr{ \Be_1, \Be_2, \cdots, \Be_N} \) is called a standard basis if that set is orthonormal.
} % definition

\makedefinition{Euclidean space.}{dfn:prerequisites:euclideanspace}{
   A vector space with basis
   \( \setlr{ \Be_1, \Be_2, \cdots, \Be_N} \) is called Euclidean if all the dot product pairs between the basis elements are not only orthonormal, but positive definite.  That is

\begin{equation*}
\Be_i \cdot \Be_j = \delta_{ij}.
\end{equation*}
} % definition
