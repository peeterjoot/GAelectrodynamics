%{
\makeproblem{Exponential derivatives.}{problem:dexpquestion:1}{
\makesubproblem{}{problem:dexpquestion:1:a}
For real or complex \( x \), we know that
\begin{equation*}
(e^x)' = x' e^x,
\end{equation*}
but this is not generally true for multivectors \( x \) (or square matrices for that matter.)
Expand \( e^x \) in Taylor series and take derivatives, and show that this identity requires that that \( x' \) commutes with \( x \).
\makesubproblem{}{problem:dexpquestion:1:b}
Given any \( j^2 = \mathrm{constant} \), and scalar constant \( \theta \), show that
\begin{equation*}
\lr{ e^{j\theta} }' = j' j^{-1} \sinh(j\theta),
\end{equation*}
where derivatives are with respect to \( j\).

We will see an example of such an exponential later when we construct an exponential representation of spherical polar vectors, setting \( i = \mathbf{e}_{12}, \, j = \mathbf{e}_{31} e^{i\phi}, \, \mathbf{x} = r \mathbf{e}_3 e^{j\theta}\).  Such a vector representation hides all the \( \phi \) dependence away in the bivector \( j \), and computation of \( \partial \Bx/\partial \phi \) requires that we know how to correctly compute these sorts of exponential derivatives.

Hint: show that \( j j' = - j' j \), and consider the effect of this on the \( (j^k)' \) term in the Taylor series.
\makesubproblem{}{problem:dexpquestion:1:c}
Let when \( j^2 = -1 \), show that
\begin{equation*}
\lr{ e^{j\theta} }' = j' \sin(\theta).
\end{equation*}

Observe can also be found (much more easily) by first expanding the exponential as
\begin{equation*}
e^{j\theta} = \cos\theta + j \sin\theta,
\end{equation*}
and then taking derivatives.  Contrast this to an application of \( (e^x)' = x' e^x \), which would lead us to believe that \( (e^{j\theta})' = \theta j' e^{j\theta} \), which is incorrect.
%
} % problem
\makeanswer{problem:dexpquestion:1}{
\makeSubAnswer{}{problem:dexpquestion:1:a}
From the power series representation of the exponential, we compute
\begin{equation}\label{eqn:dexpquestion:60}
\lr{e^x}' = \sum_{k = 1}^\infty \frac{ (x^k)' }{k!}
\end{equation}
If \( x \) is an arbitrary algebraic entity with unknown characteristics, we may only write
\begin{equation}\label{eqn:dexpquestion:119}
\lr{ x^k }' = x' x^{k-1} + x x' x^{k-2} + \cdots
\end{equation}
It is only when we know a-priori that \( x \) and \( x' \) commute, can we reduce this in the usual combinatoric fashion, writing
\begin{equation}\label{eqn:dexpquestion:99}
\lr{ x^k }' = k x' x^{k-1}.
\end{equation}
If (and only if) that is true, do we have
\begin{equation}\label{eqn:dexpquestion:219}
\begin{aligned}
\lr{e^x}'
&= \sum_{k = 1}^\infty \frac{ k x' x^{k-1} }{k!} \\
&= x' \sum_{k = 1}^\infty \frac{ x^{k-1} }{(k-1)!} \\
&= x' \sum_{k = 0}^\infty \frac{ x^{k} }{k!} \\
&= x' e^x.
\end{aligned}
\end{equation}
Hiding in this identity is the assumption that \( x \) commutes with \( x' \), so it is not generally true for non-commutative objects with as multivectors, or square matrices.
\makeSubAnswer{}{problem:dexpquestion:1:b}
If \( j^2 \) is a constant, then we must have
\begin{equation}\label{eqn:dexpquestion:139}
\lr{ j^2 }' = j j' + j' j = 0,
\end{equation}
or
\begin{equation}\label{eqn:dexpquestion:159}
j j' = -j' j.
\end{equation}

For example, for the \( j = \Be_{31} e^{i\phi} \) in the spherical polar example that was mentioned, taking derivatives with respect to \(\phi\), we have \( j' j = -j j' = \Be_{12} \).

We now set \( x = j \theta \), and see that we have
\begin{equation}\label{eqn:dexpquestion:179}
\lr{ (j \theta)^k }' = \lr{ j' j^{k-1} + j j' j^{k-2} + \cdots } \theta^k,
\end{equation}
but since \( j \) anticommutes with \( j' \) this is zero whenever \( k \) is even, and all but one term cancels out when \( k \) is odd.
The exponential derivative is
\begin{equation}\label{eqn:dexpquestion:61}
\begin{aligned}
\lr{e^{j\theta}}'
&= \sum_{k = 1, \, k \in \mathrm{odd}}^\infty \frac{ j' j^{k-1} \theta^k }{k!} \\
&= j' j^{-1} \sum_{k = 1, \, k \in \mathrm{odd}}^\infty \frac{ j^{k} \theta^k }{k!} \\
&= j' j^{-1} \sinh\lr{j\theta}.
\end{aligned}
\end{equation}
\makeSubAnswer{}{problem:dexpquestion:1:c}
When \( j^2 = -1 \), \( \sinh\lr{j\theta} = j \sin\theta \), so
\begin{equation}\label{eqn:dexpquestion:199}
\lr{e^{j\theta}}' = j' j^{-1} j \sin\theta = j' \sin\theta.
\end{equation}
Clearly, this is what we find if we first expand the exponential in its cis form.
} % answer
%}
