%
% Copyright � 2016 Peeter Joot.  All Rights Reserved.
% Licenced as described in the file LICENSE under the root directory of this GIT repository.
%
%{
\input{../latex/blogpost.tex}
\renewcommand{\basename}{maxwells}
%\renewcommand{\dirname}{notes/phy1520/}
\renewcommand{\dirname}{notes/ece1228-electromagnetic-theory/}
%\newcommand{\dateintitle}{}
%\newcommand{\keywords}{}

\PassOptionsToPackage{dvipsnames,svgnames}{xcolor}
\input{../latex/peeter_prologue_print2.tex}

\usepackage{peeters_layout_exercise}
%\usepackage{peeters_layout}
\usepackage{peeters_braket}
\usepackage{peeters_figures}
\usepackage{siunitx}
%\usepackage{macros_qed}
%\usepackage{pdfsync}
%\usepackage{mhchem} % \ce{}
%\usepackage{txfonts} % \ointclockwise
\usepackage{macros_cal}
%\usepackage{macros_lcal}
\usepackage{macros_bm} % \bcM

\usepackage{tcolorbox}
\usepackage{tabularx}
\usepackage{array}
\usepackage{colortbl}
\usepackage{enumerate}
\tcbuselibrary{skins}

\newcommand{\dispdot}[2][.2ex]{\dot{\raisebox{0pt}[\dimexpr\height+#1][\depth]{$#2$}}}% \dispdot[<displace>]{<stuff>}
%\newcommand{\dotBJ}[0]{\mathbf{\dispdot{J}}}
\newcommand{\dotBJ}[0]{\dispdot{\mathbf{J}}}

%\newcommand{\lrspacetimegrad}[2]{\lr{\stackrel{ \leftrightarrow }{#2 \spacegrad + #1 \PD{t}{}}}}
%\newcommand{\lspacetimegrad}[2]{\lr{\stackrel{ \leftarrow }{#2 \spacegrad + #1 \PD{t}{}}}}
%\newcommand{\rspacetimegrad}[2]{\lr{\stackrel{ \rightarrow }{#2 \spacegrad + #1 \PD{t}{}}}}

\newcolumntype{Y}{>{\raggedleft\arraybackslash}X}

\tcbset{tab1/.style={fonttitle=\bfseries\large,fontupper=\normalsize\sffamily,
colback=yellow!10!white,colframe=DarkOliveGreen!75!black,colbacktitle=DarkOliveGreen!40!white,
coltitle=black,center title,freelance,frame code={
\foreach \n in {north east,north west,south east,south west}
{\path [fill=DarkOliveGreen!75!black] (interior.\n) circle (3mm); };},}}

\tcbset{tab2/.style={enhanced,fonttitle=\bfseries,fontupper=\normalsize\sffamily,
colback=yellow!10!white,colframe=DarkOliveGreen!50!black,colbacktitle=DarkOliveGreen!40!white,
coltitle=black,center title}}

%\newcounter{colorboxtable}
%\numberwithin{colorboxtable}{chapter}
%
%% layout: X||Y
%% title:
%% text:
%% label.
%\newcommand{\makecolorboxtable}[4]{%
%\refstepcounter{colorboxtable}%
%\label{#4}%
%\begin{tcolorbox}[tab2,tabularx={#1},title=Table \thecolorboxtable #2,boxrule=0.5pt]%
%#3%
%\end{tcolorbox}%
%}

\newtcolorbox[auto counter,number within=chapter]{tablelabelbox}[3][]{colback=Maroon!10!white,
colframe=DarkOliveGreen,
%fonttitle=\bfseries,
arc=0mm,
colbacktitle=white,enhanced,
attach boxed title to top left={yshift=-0.75mm,xshift=4mm},
label=#3,%
title=\textbf{{\color{DarkOliveGreen}Table\,\thetcbcounter:} {\color{Maroon}#2}},#1}
%label=#3 type=table,

\newcommand{\e}[0]{{(\mathrm{e})}}
\newcommand{\m}[0]{{(\mathrm{m})}}

%
% https://tex.stackexchange.com/a/28212/15
%
%For bold and upright, you could use the regular text-version of \imath and \jmath, which are \i and \j:
%
%\newcommand{\ihat}{\hat{\textbf{\i}}}
%\newcommand{\jhat}{\hat{\textbf{\j}}}
%bold & upright i & j in math
%
%Additionally, if you want the \hat to be bold as well, then use \boldsymbol.
%
%\newcommand{\ihat}{\boldsymbol{\hat{\textbf{\i}}}}
%\newcommand{\jhat}{\boldsymbol{\hat{\textbf{\j}}}}

\newcommand{\plain}{\color{black}}
\newcommand{\Frac}[2]{\genfrac{}{}{1pt}{}{#1}{#2}} % thicker fraction line

% ----- color -----
% https://www.overleaf.com/read/cvmtqywqgvvw#/43203532/

\definecolor{c1}{RGB}{114,0,172}   % primary
\definecolor{c2}{RGB}{45,177,93}   % true
\definecolor{c3}{RGB}{251,0,29}    % false
\definecolor{c4}{RGB}{18,110,213}  % secondary
\definecolor{c5}{RGB}{255,160,109} % tertiary
\definecolor{c6}{RGB}{219,78,158}  % alt-primary

\renewcommand{\familydefault}{\sfdefault}

\newcommand{\growth}{\color{c1}}
\newcommand{\unitQuantity}{\color{c2}}
\newcommand{\unitInterest}{\color{c3}}
\newcommand{\unitTime}{\color{c4}}
\newcommand{\perfectly}{\color{c5}}
\newcommand{\compounded}{\color{c6}}
% ----- color -----

\beginArtNoToc

%\generatetitle{Bivector, Trivector, Multivector, and Multivector space}
\generatetitle{scratch notes.}

% this is the E, H extraction example (digression)
%\begin{dmath}\label{eqn:maxwells:20}
%{\left( \frac{x}{y} \right)}^2
%\end{dmath}

%% https://betterexplained.com/articles/colorized-math-equations/
%$$\growth e
%\plain =
%\perfectly \lim_{n\to\infty}
%\plain \left(
%\unitQuantity 1 + \unitInterest \frac{1}{\compounded n}
%\plain \right)
%\unitTime^{1 \cdot \compounded n}
%$$
%
%\growth       The base for continuous growth
%\plain        is
%\\
%\unitQuantity the unit quantity
%\unitInterest earning unit interest
%\unitTime     for unit time,
%\\
%\compounded   compounded
%\perfectly    as fast as possible

%%
% Copyright © 2017 Peeter Joot.  All Rights Reserved.
% Licenced as described in the file LICENSE under the root directory of this GIT repository.
%

Ampere's law relating the line integral of the magnetic field around a loop can be used to compute the azimuthal magnetic field around a current source.
Let's compute the magnetic field using superposition in a region of space that lies between two z-axis currents of magnitude \( I_1, I_2 \) situated respectively at \( \Bp_1, \Bp_2 \), as illustrated in
\cref{fig:amperesLawBetweenTwoCurrents:amperesLawBetweenTwoCurrentsFig1}.

\imageFigure{../figures/GAelectrodynamics/amperesLawBetweenTwoCurrentsFig1}{Magnetic field between two current sources.}{fig:amperesLawBetweenTwoCurrents:amperesLawBetweenTwoCurrentsFig1}{0.3}

We will need to add azimuthal field components along the \( \phicap_1, \phicap_2 \) directions, a task that is more tractable with GA than in vector algebra using coordinates.

First consider the field surrounding a single current source, say \( I_1 \) at point \( \Bp_1 \).
Combining the two parameter Stokes' theorem integral from 
%\cref{eqn:twoparameter:280}
\cref{thm:surfaceintegral:500}
, and Maxwell's magnetostatics equations for \( \BB \) in \cref{eqn:magnetostatics:380} we have

\begin{dmath}\label{eqn:amperesLawMagnetostaticsExample:20}
\int_A d^2 \Bx \cdot (\spacegrad \wedge \BB) = \ointclockwise d\Bx \cdot \BB = \mu \int d^2 \Bx \cdot (I \BJ).
\end{dmath}

Flipping the orientation of the Stokes' integral so that we are integrating along the \( \phicap \) direction, we have

\begin{dmath}\label{eqn:amperesLawMagnetostaticsExample:40}
\ointctrclockwise d\Bx \cdot \BB
= -\mu I \int d^2 \Bx \wedge \BJ
= \mu I_\txte,
\end{dmath}
where \( I_\txte \) is the enclosed charge (in this case \( I_1 \)).
Because of symmetry, the magnetic field due to this one line charge is

\begin{dmath}\label{eqn:amperesLawMagnetostaticsExample:60}
\BB
= \frac{\mu I_\txte \phicap}{2 \pi R},
\end{dmath}
where \( R \) is the radius of the circle centred one the current, out to the point \( \Br \) where the field is observed.
Each of the current sources at points \( \Bp_k \) contributes a magnetic field

\begin{dmath}\label{eqn:amperesLawMagnetostaticsExample:80}
\BB_k(\Br)
= \frac{\mu I_k \rcap_k \Be_{12} }{2 \pi \Norm{ \Br - \Bp_k} }
= \frac{\mu I_k \lr{ \Br - \Bp_k} \Be_{12} }{2 \pi \lr{ \Br - \Bp_k}^2 }
= \frac{\mu I_k}{2\pi} \inv{ \Br - \Bp_k} \Be_{12},
\end{dmath}
where the azimuthal angle has been determined by rotating the radial unit vector counterclockwise by 90 degrees using \( \phicap_k = \rcap_k\, \Be_{12} \).
The total magnetic field bivector between the charges is

\begin{equation}\label{eqn:amperesLawMagnetostaticsExample:100}
I \BB(\Br)
= \frac{\mu}{2\pi} \sum_{k = 1}^2 I_k \inv{\Bp_k - \Br } \Be_3.
\end{equation}

%The radius of the k-th circles are \( \Norm{ \Bp_k - \Br } \).
A product including the (inverse) vector difference \( \Bp_k - \Br \) encodes both the magnitude of the field contribution at the point \( \Br \) as well as the orientation of the magnetic field bivector components in the x-z and y-z planes.
There was no need to fall back to coordinates, which could easily become cumbersome and error prone.

%%
% Copyright © 2016 Peeter Joot.  All Rights Reserved.
% Licenced as described in the file LICENSE under the root directory of this GIT repository.
%

\maketheorem{K-vector dot and wedge product relations.}{thm:bladeDotWedgeSymmetryIdentities:180}{
Given a k-vector \( B \) and a vector \( \Ba \), the dot and wedge products have the following commutation relationships
\boxedEquation{eqn:bladeDotWedgeSymmetryIdentities:200}{
\begin{aligned}
B \cdot \Ba  &= (-1)^{k-1} \Ba \cdot B \\
B \wedge \Ba &= (-1)^k \Ba \wedge B,
\end{aligned}
}
and can be expressed as symmetric and antisymmetric sums depending on the grade of the blade
\boxedEquation{eqn:bladeDotWedgeSymmetryIdentities:220}{
\begin{aligned}
\Ba \wedge B &= \inv{2}\lr{ \Ba B + (-1)^k B \Ba } \\
\Ba \cdot B &= \inv{2}\lr{ \Ba B - (-1)^k B \Ba }.
\end{aligned}
}
} % theorem

For example, if \( B \) and \( \Ba \) are both vectors, we recover \cref{thm:symmetricAndAntiSymmetricVectorSums:symmetricAndAnti}.  As an other example,
if \( B \) is a 2-vector, then
\begin{equation}\label{eqn:bladeDotWedgeSymmetryIdentitiesTheorem:480}
\begin{aligned}
2 ( \Ba \wedge B ) &= \Ba B + B \Ba  \\
2 ( \Ba \cdot B ) &= \Ba B - B \Ba.
\end{aligned}
\end{equation}
Observe that the dot(wedge) of two vectors is a (anti)symmetric sum of products, whereas the wedge(dot) of a vector and a bivector is an (anti)symmetric sum.

\begin{proof}
To prove \cref{thm:bladeDotWedgeSymmetryIdentities:180}, split the blade into components that intersect with and are disjoint from \( \Ba \) as follows
\begin{dmath}\label{eqn:bladeDotWedgeSymmetryIdentitiesTheorem:240}
B
=
\inv{\Ba} \Bn_1 \Bn_2 \cdots \Bn_{k-1} + \Bm_1 \Bm_2 \cdots \Bm_k,
\end{dmath}
where \( \Bn_i \) orthogonal to \( \Ba \) and each other, and where \( \Bm_i \) are all orthogonal.  The products of \( B \) with \( \Ba \) are
\begin{dmath}\label{eqn:bladeDotWedgeSymmetryIdentitiesTheorem:340}
\Ba B
=
\Ba \inv{\Ba} \Bn_1 \Bn_2 \cdots \Bn_{k-1} + \Ba \Bm_1 \Bm_2 \cdots \Bm_k
=
\Bn_1 \Bn_2 \cdots \Bn_{k-1} + \Ba \Bm_1 \Bm_2 \cdots \Bm_k,
\end{dmath}
and
\begin{dmath}\label{eqn:bladeDotWedgeSymmetryIdentitiesTheorem:360}
B \Ba
=
\inv{\Ba} \Bn_1 \Bn_2 \cdots \Bn_{k-1} \Ba + \Bm_1 \Bm_2 \cdots \Bm_k \Ba
=
(-1)^{k-1} \Bn_1 \Bn_2 \cdots \Bn_{k-1} + (-1)^k \Ba \Bm_1 \Bm_2 \cdots \Bm_k
=
(-1)^k \lr{ - \Bn_1 \Bn_2 \cdots \Bn_{k-1} + \Ba \Bm_1 \Bm_2 \cdots \Bm_k },
\end{dmath}
or
\begin{dmath}\label{eqn:bladeDotWedgeSymmetryIdentitiesTheorem:380}
(-1)^k B \Ba
=
- \Bn_1 \Bn_2 \cdots \Bn_{k-1} + \Ba \Bm_1 \Bm_2 \cdots \Bm_k.
\end{dmath}

Respective addition and subtraction of \cref{eqn:bladeDotWedgeSymmetryIdentitiesTheorem:340} and \cref{eqn:bladeDotWedgeSymmetryIdentitiesTheorem:380} gives
\begin{dmath}\label{eqn:bladeDotWedgeSymmetryIdentitiesTheorem:400}
\Ba B + (-1)^k B \Ba
= 2 \Ba \Bm_1 \Bm_2 \cdots \Bm_k
= 2 \Ba \wedge B,
\end{dmath}
and
\begin{dmath}\label{eqn:bladeDotWedgeSymmetryIdentitiesTheorem:420}
\Ba B - (-1)^k B \Ba
=
2
\Bn_1 \Bn_2 \cdots \Bn_{k-1}
= 2 \Ba \cdot B,
\end{dmath}
proving \cref{eqn:bladeDotWedgeSymmetryIdentities:220}.  Grade selection from \cref{eqn:bladeDotWedgeSymmetryIdentitiesTheorem:380} gives
\begin{dmath}\label{eqn:bladeDotWedgeSymmetryIdentitiesTheorem:440}
(-1)^k B \cdot \Ba
=
- \Bn_1 \Bn_2 \cdots \Bn_{k-1}
= - \Ba \cdot B,
\end{dmath}
and
\begin{dmath}\label{eqn:bladeDotWedgeSymmetryIdentitiesTheorem:460}
(-1)^k B \wedge \Ba
=
\Ba \Bm_1 \Bm_2 \cdots \Bm_k
= \Ba \wedge B,
\end{dmath}
which proves \cref{eqn:bladeDotWedgeSymmetryIdentities:200}.
\end{proof}


%%
% Copyright © 2017 Peeter Joot.  All Rights Reserved.
% Licenced as described in the file LICENSE under the root directory of this GIT repository.
%
%{
\index{boundary values}
%
% Copyright � 2018 Peeter Joot.  All Rights Reserved.
% Licenced as described in the file LICENSE under the root directory of this GIT repository.
%
\maketheorem{Boundary value relations.}{thm:boundarySurfaceSources:480}{
The difference in the normal and tangential components of the electromagnetic field spanning a surface on which there are
a surface current or surface charge or current densities \( J_\txte = J_{\textrm{es}} \delta(n), J_\txtm = J_{\textrm{ms}} \delta(n) \)
can be related to those surface sources as follows
%\label{eqn:boundarySurfaceSources:420}
\begin{equation*}
\begin{aligned}
\gpgrade{\ncap (F_2 - F_1) }{0,1} &= J_{\textrm{es}} \\
\gpgrade{\ncap (G_2 - G_1) }{2,3} &= I J_{\textrm{ms}},
\end{aligned}
\end{equation*}
where \( F_k = \BD_k + I \BH_k/c, G_k = \BE_k + I c \BB_k, k = 1,2 \) are the fields in the
where \( \ncap = \ncap_2 = -\ncap_1 \) is the outwards facing normal in the second medium.
In terms of the conventional constituent fields, these may be written
%\label{eqn:boundarySurfaceSources:460}
\begin{equation*}
\begin{aligned}
\ncap \cdot \lr{ \BD_2 - \BD_1 } &= \rho_\txts \\
\ncap \cross \lr{ \BH_2 - \BH_1 } &= \BJ_\txts \\
\ncap \cdot \lr{ \BB_2 - \BB_1 } &= \rho_{\textrm{ms}} \\
\ncap \cross \lr{ \BE_2 - \BE_1 } &= -\BM_\txts.
\end{aligned}
\end{equation*}
} % theorem


\Cref{fig:ps3Problem1Pillbox:ps3Problem1PillboxFig1} illustrates a surface where we seek to find the fields above the surface (region 2), and below the surface (region 1).
These fields will be determined by integrating Maxwell's equation over the pillbox configuration, allowing the height \( n \) of that pillbox above or below the surface to tend to zero,
and the area of the pillbox top to also tend to zero.
\pmathImageFigure{../figures/GAelectrodynamics/}{pillboxIntegrationVolumeFig1}{Pillbox integration volume.}{fig:ps3Problem1Pillbox:ps3Problem1PillboxFig1}{0.3}{pillboxIntegrationVolumeFig1.nb}

\begin{proof}
We will work with \cref{thm:dielectric:20}, Maxwell's equations in media, in their frequency domain form
\begin{equation}\label{eqn:boundarySurfaceSources:480}
\begin{aligned}
\gpgrade{ \spacegrad F }{0,1} + j k \BD &= J_{\textrm{es}} \delta(n) \\
\gpgrade{ \spacegrad G }{2,3} + j k I c \BB &= I J_{\textrm{ms}} \delta(n),
\end{aligned}
\end{equation}
and integrate these over the pillbox volume in the figure.  That is
\begin{equation}\label{eqn:boundarySurfaceSources:500}
\begin{aligned}
\int dV\, \gpgrade{ \spacegrad F }{0,1} + j k \int dV\, \BD &= \int dn dA\, J_{\textrm{es}} \delta(n) \\
\int dV\, \gpgrade{ \spacegrad G }{2,3} + j k I c \int dV\, \BB &= I \int dn dA\, J_{\textrm{ms}} \delta(n).
\end{aligned}
\end{equation}
The gradient integrals can be evaluated with \cref{thm:volumeintegral:100}.  Evaluating the delta functions picks leaves an area integral on the surface.  Additionally, we assume that we are making the pillbox volume small enough that we can employ the mean value theorem for the \( \BD, \BB \) integrals
\begin{equation}\label{eqn:boundarySurfaceSources:520}
\begin{aligned}
\int_{\partial V} dA\, \gpgrade{ \ncap F }{0,1} + j k \Delta A \lr{ n_1 \tilde{\BD}_1 + n_2 \tilde{\BD}_2 } &= \Delta A J_{\textrm{es}} \\
\int_{\partial V} dA\, \gpgrade{ \ncap G }{2,3} + j k I c \Delta A \lr{ n_1 \tilde{\BB}_1 + n_2 \tilde{\BB}_2} &= I \Delta A J_{\textrm{ms}}.
\end{aligned}
\end{equation}
We now let \( n_1, n_2 \) tend to zero, which kills off the \( \BD, \BB \) contributions, and also kills off the side wall contributions in the first pillbox surface integral.  This leaves
\begin{equation}\label{eqn:boundarySurfaceSources:540}
\begin{aligned}
\gpgrade{ \ncap_2 F_2 }{0,1} + \gpgrade{ \ncap_1 F_1 }{0,1} &= J_{\textrm{es}} \\
\gpgrade{ \ncap_2 G_2 }{2,3} + \gpgrade{ \ncap_1 G_1 }{2,3} &= J_{\textrm{ms}}.
\end{aligned}
\end{equation}
Inserting \( \ncap = \ncap_2 = -\ncap_1 \) completes the first part of the proof.

Expanding the grade selection operations, we find
\begin{equation}\label{eqn:boundarySurfaceSources:440}
\begin{aligned}
\ncap \cdot (\BD_2 - \BD_1) &= \rho_s \\
I \ncap \wedge \lr{ \BH_2/c - \BH_1/c } &= - \BJ_s/c \\
\ncap \wedge (\BE_2 - \BE_1) &= -I \BM_s \\
I c \ncap \cdot (\BB_2 - \BB_1) &= I c \rho_{ms},
\end{aligned}
\end{equation}
and expansion of the wedge's as cross's using \cref{eqn:SimpleProducts2:1620} completes the proof.
\end{proof}
%It is somewhat remarkable that the
%crazy jumble of dot products, cross products and field components in the conventional statement of the boundary conditions, follows directly from the evaluation of the product of the normal with the multivector fields.

In the special case where there are surface charge and current densities along the interface surface, but the media is uniform (\(\epsilon_1 = \epsilon_2, \mu_1 = \mu_2\)), then the field and current relationship has a particularly simple form \citep{chappell2014geometric}
\begin{equation}\label{eqn:boundarySurfaceSources:421}
\ncap (F_2 - F_1) = J_s.
\end{equation}

\makeproblem{Uniform media with currents and densities.}{problem:boundarySurfaceSources:1}{
Prove that \cref{eqn:boundarySurfaceSources:421} holds when \( \epsilon_1 = \epsilon_2, \mu_1 = \mu_2 \).
} % problem
%}

%%
% Copyright � 2018 Peeter Joot.  All Rights Reserved.
% Licenced as described in the file LICENSE under the root directory of this GIT repository.
%
%{
Some would argue that the conventional form \cref{eqn:freespace:3100} of Maxwell's equations have built in redundancy since continuity equations on the charge and current densities couple some of these equations.
We will take an opposing view, and show that such continuity equations are neccessary consequences of Maxwell's equation in its wave equation form, and derive those conditions.
This amounts to a statement that the multivector current \( J \) is not completely unconstrained.

\maketheorem{Electromagnetic wave equation and continuity conditions.}{thm:continuity:600}{
The electromagnetic field is a solution to the non-homogeneous wave equation
\begin{equation*}
\lr{ \spacegrad^2 - \inv{c^2} \PDSq{t}{} } F =
\lr{ \spacegrad - \inv{c} \PD{t}{} } J.
\end{equation*}
In source free conditions, this reduces to a homogeneous wave equation, with group velocity \( c \), the speed of the wave in the media.
When expanded explicitly in terms of electric and magnetic fields, and charge and current densities, this single equation resolves to a
non-homogeneous wave equation for each of the electric and magnetic fields
\begin{equation*}
\begin{aligned}
\lr{ \spacegrad^2 - \inv{c^2} \PDSq{t}{} } \BE
%&= \gpgrade{\lr{ \spacegrad - \inv{c} \PD{t}{} } J}{1}
&= \inv{\epsilon} \spacegrad \rho + \mu \PD{t}{\BJ} + \spacegrad \cross \BM \\
\lr{ \spacegrad^2 - \inv{c^2} \PDSq{t}{} } \BH
%&= \inv{I \eta} \gpgrade{\lr{ \spacegrad - \inv{c} \PD{t}{} } J}{2}
&= \inv{\mu} \spacegrad \rho_\txtm + \epsilon \PD{t}{\BM} - \spacegrad \cross \BJ,
\end{aligned}
\end{equation*}
as well as a pair of continuity equations coupling the respective charge and current densities
\begin{equation*}
\begin{aligned}
\spacegrad \cdot \BJ + \PD{t}{\rho} &= 0 \\
\spacegrad \cdot \BM + \PD{t}{\rho_\txtm} &= 0.
\end{aligned}
\end{equation*}
} % theorem

To prove, we operate on \cref{dfn:isotropicMaxwells:680} with \( \spacegrad - (1/c) \partial_t \), one of the factors, along with the
spacetime gradient, of the
d'Alembertian (wave equation) operator, which gives
\begin{dmath}\label{eqn:continuity:110}
\lr{ \spacegrad^2 - \inv{c^2} \PDSq{t}{} } F = \lr{ \spacegrad - \inv{c} \PD{t}{} } J.
\end{dmath}
Since the left hand side has only grades 1 and 2, \cref{eqn:continuity:110} splits naturally into two equations, one for grades 1,2 and one for grades 0,3
\begin{dmath}\label{eqn:continuity:130}
\begin{aligned}
\lr{ \spacegrad^2 - \inv{c^2} \PDSq{t}{} } F &= \gpgrade{ \lr{ \spacegrad - \inv{c} \PD{t}{} } J }{1,2} \\
                                           0 &= \gpgrade{ \lr{ \spacegrad - \inv{c} \PD{t}{} } J }{0,3}.
\end{aligned}
\end{dmath}
Unpacking these further, we find that there is information
carried in the requirement that the grade 0,3 selection of \cref{eqn:continuity:130} is zero.
In particular, grade 0 selection gives
\begin{dmath}\label{eqn:continuity:40}
0
=
\gpgradezero{ ( \spacegrad - (1/c) \partial_t ) J }
=
\gpgradezero{
\Biglr{ \spacegrad - \inv{c} \PD{t}{} }
\biglr{
   \eta \lr{ c \rho - \BJ } + I \lr{ c \rho_\txtm - \BM }
}
}
=
-\eta
\lr{ \spacegrad \cdot \BJ + \PD{t}{\rho} }
,
\end{dmath}
which demonstrates the continuity condition on the electric sources.
Similarly, grade three selection gives
\begin{dmath}\label{eqn:continuity:60}
0
=
\gpgradethree{  (\spacegrad - (1/c) \partial_t ) J }
=
\gpgradethree{
\Biglr{ \spacegrad - \inv{c} \PD{t}{} }
\lr{
   \eta \lr{ c \rho - \BJ } + I \lr{ c \rho_\txtm - \BM }
}
}
=
-I \lr{
   \spacegrad \cdot \BM + \PD{t}{\rho_\txtm}
},
\end{dmath}
which demonstrates the continuity condition on the (fictitious) magnetic sources if included in the current.

For the non-homogeneous wave equation of \cref{thm:continuity:600}, the current derivatives may be expanded explicitly.
For the wave equation for the electric field, this is
\begin{dmath}\label{eqn:continuity:150}
\lr{ \spacegrad^2 - \inv{c^2} \PDSq{t}{} } \BE
=
\gpgradeone{\lr{ \spacegrad - \inv{c} \PD{t}{} } J}
=
\gpgradeone{
   \lr{ \spacegrad - \inv{c} \PD{t}{} }
   \lr{
      \frac{\rho}{\epsilon} - \eta \BJ + I \lr{ c \rho_\txtm - \BM }
   }
}
=
\inv{\epsilon} \spacegrad \rho -I \lr{ \spacegrad \wedge \BM } + \inv{c} \eta \PD{t}{\BJ}
= \gpgrade{\lr{ \spacegrad - \inv{c} \PD{t}{} } J}{1} = \inv{\epsilon} \spacegrad \rho + \mu \PD{t}{\BJ} + \spacegrad \cross \BM,
\end{dmath}
as claimed.
The forced magnetic field equation is
\begin{dmath}\label{eqn:continuity:170}
\lr{ \spacegrad^2 - \inv{c^2} \PDSq{t}{} } \BH
=
\inv{\eta I}
\gpgradetwo{\lr{ \spacegrad - \inv{c} \PD{t}{} } J}
=
\inv{\eta I}
\gpgradetwo{
   \lr{ \spacegrad - \inv{c} \PD{t}{} }
   \lr{
      \frac{\rho}{\epsilon} - \eta \BJ + I \lr{ c \rho_\txtm - \BM }
   }
}
=
\inv{\eta I}
\lr{
   -\spacegrad \wedge \BJ + I c \spacegrad \rho_\txtm + \frac{I}{c} \PD{t}{\BM}
}
=
\inv{I}
\lr{
   -I \lr{ \spacegrad \cross \BJ} + I \inv{\mu} \spacegrad \rho_\txtm + I \epsilon \PD{t}{\BM}
}
= \inv{\mu} \spacegrad \rho_\txtm + \epsilon \PD{t}{\BM} - \spacegrad \cross \BJ,
\end{dmath}
also as claimed.

%}

%%
% Copyright © 2017 Peeter Joot.  All Rights Reserved.
% Licenced as described in the file LICENSE under the root directory of this GIT repository.
%
%{

\paragraph{Example.  field of a cylindrical distribution.}

Consider an electric multivector current distribution for static surface charge density \( \sigma \) and a current moving with constant (magnitude) current \( \Bv \) on a cylinder of radius \( R \)
\begin{equation}\label{eqn:cylinderField:20}
J = \inv{\epsilon_0} \sigma \delta( r - R ) \lr{ 1 - \frac{\Bv}{c} }.
\end{equation}

Using cylindrical coordinates for the field observation point \( \Bx \) and the volume integration point \( \Bx' \) have parameterization
\begin{equation}\label{eqn:cylinderField:40}
\begin{aligned}
\Bx &= z \Be_3 + r \rhocap \\
\Bx' &= z' \Be_3 + r' \rhocap',
\end{aligned}
\end{equation}
where \( \rhocap = \Be_1 e^{i \phi} \), and \( \rhocap' = \Be_1 e^{i \phi'} \), and \( i = \Be_{12} \).
The field is
\begin{dmath}\label{eqn:cylinderField:60}
F = \frac{\sigma}{4 \pi \epsilon_0}
\int_{-\infty}^\infty dz' \int_0^\infty r' dr' \int d\phi' \frac{
   \gpgrade{ \lr{ (z-z') \Be_3 + \Delta_r }\lr{ 1 - \Bv/c }  }{1,2}
}{ \lr{ (z -z')^2 + \Delta_r^2 }^{3/2} },
\end{dmath}
where \( \Delta_r = r \rhocap - r' \rhocap' \).
The z-axis integrals are of the form
\begin{dmath}\label{eqn:cylinderField:80}
\int_{-\infty}^\infty dz' \frac{ (1, z-z') }{ \lr{(z-z')^2 + a^2 }^{3/2}}
=
\lr{ \frac{2}{a^2} \sgn(z), 0 },
\end{dmath}
so the field is
\begin{dmath}\label{eqn:cylinderField:100}
F
=
\frac{\sigma R \sgn(z)}{2 \pi \epsilon_0}
\int_0^{2 \pi} d\phi' \frac{
   \gpgrade{ \lr{ r\rhocap - R \rhocap' } \lr{ 1 - \Bv/c } }{1,2}
}
{ (r\rhocap - R \rhocap')^2 }
=
\frac{\sigma \sgn(z)}{2 \pi \epsilon_0}
\int_0^{2 \pi} d\phi' \frac{
   \gpgrade{ \lr{ \tilde{r}\rhocap - \rhocap' } \lr{ 1 - \Bv/c } }{1,2}
}
{ (\tilde{r}\rhocap - \rhocap')^2 },
\end{dmath}
where \( \tilde{r} = r/R \).
Let's consider two cases \( \Bv = v \Be_3 \), and \( \Bv(\phi') = v \phicap' = v \Be_2 e^{i\phi'} \).  This means there are three integrals of interest
\begin{equation}\label{eqn:cylinderField:120}
\int_0^{2 \pi} d\phi' \frac{(1, e^{i\phi'}, e^{2 i\phi'})}{1 + a^2 - 2 a \cos(\phi-\phi')}.
\end{equation}
%}
 % orphaned:
%%
% Copyright © 2017 Peeter Joot.  All Rights Reserved.
% Licenced as described in the file LICENSE under the root directory of this GIT repository.
%
The charge in a volume can be related to the electric field by integrating \cref{eqn:electrostatics:380}

\begin{dmath}\label{eqn:electrostatics_enclosedCharge:420}
\int_V d^3 \Bx \spacegrad \BE = \inv{\epsilon} \int_V d^3 \Bx \rho(\Bx).
\end{dmath}

This is an oriented integral, where \( d^3 \Bx \) is a pseudoscalar volume element, such as
\( d^3 \Bx = (\Be_1 dx) \wedge (\Be_2 dy) \wedge (\Be_3 dz) = I dx dy dz \).

The LHS integral can be evaluated using
%the fundamental theorem of geometric calculus
\cref{thm:fundamentalTheoremOfCalculus:1}

\begin{dmath}\label{eqn:electrostatics_enclosedCharge:461}
\int_{\partial V} d^2 \Bx \BE = \frac{I}{\epsilon} \int_V dV \rho(\Bx).
\end{dmath}

An outward normal \( \ncap \) can be used to
parameterize the bivector surface area element \( d^2 \Bx = I \ncap dA \), which allows the pseudoscalar factors on both
sides to be cancelled

%\begin{dmath}\label{eqn:electrostatics_enclosedCharge:460}
\boxedEquation{eqn:electrostatics_enclosedCharge:460}{
\int_{\partial V} dA \ncap \BE = \frac{1}{\epsilon} \int_V dV \rho(\Bx).
}
%\end{dmath}

This is a multivector equation which must be simultaneously satisfied by its scalar and bivector components

\begin{subequations}
\label{eqn:electrostatics_enclosedCharge:481}
\begin{dmath}\label{eqn:electrostatics_enclosedCharge:501}
\int_{\partial V} dA \ncap \cdot \BE = \frac{1}{\epsilon} \int_V dV \rho(\Bx)
\end{dmath}
\begin{dmath}\label{eqn:electrostatics_enclosedCharge:521}
\int_{\partial V} dA \ncap \wedge \BE = 0.
\end{dmath}
\end{subequations}

\index{enclosed charge}
The first equation is the familiar relationship between the divergence and the enclosed charge, which could have been derived from \cref{eqn:electrostatics:140} directly.
The second provides a constraint on the tangential components of the field with respect to the enclosed volume, and could have been derived from
\cref{eqn:electrostatics:100} directly.
The multivector equation \cref{eqn:electrostatics_enclosedCharge:460} encodes both of these relationships, simultaneously incorporating the contributions of the Maxwell divergence and curl equations for the electric field, relating both to the enclosed charge.


%%
% Copyright © 2018 Peeter Joot.  All Rights Reserved.
% Licenced as described in the file LICENSE under the root directory of this GIT repository.
%
%{
We have already solved the statics equation using Green's function techniques, but we can also gain additional insight by simply integrating Maxwell's equation using \cref{thm:volumeintegral:100} which gives
\begin{equation}\label{eqn:enclosedCurrent:20}
\int_V dV \spacegrad F = \int_{\partial V} dA \ncap F = \int_V dV J
\end{equation}
The total current in the volume is related to the surface integral of \( \ncap F \) over the boundary of the volume.
This is a multivector relationship, containing a substantial amount of information.
This information can be extracted by
expanding \( \ncap F \)
\begin{dmath}\label{eqn:enclosedCurrent:40}
\ncap F
=
\ncap \lr{ \BE + I \eta \BH }
=
\ncap \cdot \BE + I (\ncap \cross \BE) + I \eta \lr{ \ncap \cdot \BH + I \ncap \cross \BH }
=
\ncap \cdot \BE - \eta ( \ncap \cross \BH ) + I (\ncap \cross \BE) + I \eta (\ncap \cdot \BH),
\end{dmath}
so
\begin{dmath}\label{eqn:enclosedCurrent:60}
\begin{aligned}
\int_{\partial V} dA\, \ncap \cdot \BE        &=  \inv{\epsilon} \int_V dV\, \rho \\
\int_{\partial V} dA\, \ncap \cross \BH       &=                 \int_V dV\, \BJ \\
\int_{\partial V} dA\, \ncap \cross \BE       &=               - \int_V dV\, \BM \\
\int_{\partial V} dA\, \ncap \cdot \BH        &=  \inv{\mu} \int_V dV\, \rho_\txtm.
\end{aligned}
\end{dmath}
Of course \cref{eqn:enclosedCurrent:60}
could have obtained directly from Maxwell's equations in their conventional form \cref{eqn:freespace:3399}.
However, had we integrated the conventional Maxwell's equations, it would not have been obvious that the crazy mix of
fields, dot and cross products in \cref{eqn:enclosedCurrent:60} had a hidden structure as simple as
\( \int_{\partial V} dA \ncap F = \int_V dV J \).
%}

%%
% Copyright � 2016 Peeter Joot.  All Rights Reserved.
% Licenced as described in the file LICENSE under the root directory of this GIT repository.
%
%{
%\input{../blogpost.tex}
%\renewcommand{\basename}{fundamentalTheoremOfCalculus}
%\renewcommand{\dirname}{notes/phy1520/}
%%\newcommand{\dateintitle}{}
%%\newcommand{\keywords}{}
%
%\input{../peeter_prologue_print2.tex}
%
%\usepackage{peeters_layout_exercise}
%\usepackage{peeters_braket}
%\usepackage{peeters_figures}
%\usepackage{siunitx}
%
%\beginArtNoToc
%
%\generatetitle{Fundamental theorem of geometric calculus}
%\label{chap:fundamentalTheoremOfCalculus}

\subsection{Hypervolume integral}
We wish to generalize the concepts of line, surface and volume integrals to hypervolumes and multivector functions, and define a hypervolume integral as

\makedefinition{Multivector integral.}{dfn:fundamentalTheoremOfCalculus:240}{
Given a hypervolume parameterized by \( k \) parameters, k-volume volume element \( d^k \Bx \), and
multivector functions \( F, G \), a k-volume integral with the vector derivative acting to the right on \( G \) is written as
\begin{equation*}
\int_V d^k\Bx \rboldpartial G,
\end{equation*}
a k-volume integral with the vector derivative acting to the left on \( F \) is written as
\begin{equation*}
\int_V F d^k\Bx \lboldpartial,
\end{equation*}
and a k-volume integral with the vector derivative acting bidirectionally on \( F, G \) is written as
\begin{equation*}
\int_V F d^k\Bx \lrboldpartial G
\equiv
\int_V \lr{ F d^k\Bx \lboldpartial} G
+
\int_V F d^k\Bx \lr{ \rboldpartial G }.
\end{equation*}
The explicit meaning of these directional acting derivative operations is given by the chain rule coordinate expansion
\begin{dmath*}
F d^k \Bx \lrboldpartial G
=
F d^k \Bx \lr{ \sum_i \Bx^i {\stackrel{ \leftrightarrow }{\partial_i}} } G
=
(\partial_i F) d^k \Bx \sum_i \Bx^i G
+
F d^k \Bx \sum_i \Bx^i (\partial_i G)
\equiv
(F d^k \Bx \lboldpartial) G
+
F d^k \Bx (\rboldpartial G),
\end{dmath*}
with \( \boldpartial \) acting on \( F \) and \( G \), but not the volume element \( d^k \Bx \), which may also be a function of the implied parameterization.
} % definition

The vector derivative
% (and gradient)
may not commute with \( F, G \) nor the volume element \( d^k \Bx \), so we are forced to use some notation to indicate what the vector derivative (or gradient) acts on.
In conventional right acting cases, where there is no ambiguity, arrows will usually be omitted, but braces may also be used to indicate the scope of derivative operators.
This bidirectional notation will also be used for the gradient, especially for volume integrals in \R{3} where the vector derivative is identical to the gradient.

Some authors use the Hestenes dot notation, with overdots or primes to indicating the exact scope of multivector derivative operators, as in
\begin{dmath}\label{eqn:fundamentalTheoremOfCalculus:260}
\dot{F} d^k \Bx \dot{\boldpartial} \dot{G} =
\dot{F} d^k \Bx \dot{\boldpartial} G
+
F d^k \Bx \dot{\boldpartial} \dot{G}.
\end{dmath}
The dot notation has the advantage of emphasizing that the action of the vector derivative (or gradient) is on the functions \( F, G \), and not on the hypervolume element \( d^k \Bx \).
However, in this book, where primed operators such as \( \spacegrad' \) are used to indicate that derivatives are taken with respect to primed \( \Bx' \) variables, a mix of dots and ticks would have been confusing.
%Over arrows also have the advantage of being visually conspicuous.

\subsection{Fundamental theorem.}
\index{fundamental theorem of geometric calculus}

The fundamental theorem of geometric calculus is a generalization of many conventional scalar and vector integral theorems, and relates a hypervolume integral to its boundary.
This is a a powerful theorem, which we will use with Green's functions to solve Maxwell's equation, but also to derive the geometric algebra form of Stokes' theorem, from which most of the familiar integral calculus results follow.

\maketheorem{Fundamental theorem of geometric calculus}{thm:fundamentalTheoremOfCalculus:1}{
Given
multivectors \(F, G \),
a parameterization \( \Bx = \Bx(u_1, u_2, \cdots) \), with hypervolume element \( d^k \Bx = d^k u I_k \), where
\( I_k = \Bx_1 \wedge \Bx_2 \wedge \cdots \wedge \Bx_k \), the hypervolume integral is related to the boundary integral by
\begin{equation*}
\int_V F d^k \Bx \lrboldpartial G = \oint_{\partial V} F d^{k-1} \Bx G,
\end{equation*}
where \( \partial V \) represents the boundary of the volume, and \( d^{k-1} \Bx \) is the hypersurface element.  The hypersurface element and boundary integral is defined for \( k > 1 \) as
\begin{equation*}
\oint_{\partial V} F d^{k-1} \Bx G
\equiv
\sum_i \int d^{k-1} u_i \evalbar{ \lr{ F \lr{ I_k \cdot \Bx^i} G }}{\Delta u_i},
\end{equation*}
where \( d^{k-1} u_i \) is the product of all \( du_j \) except for \( du_i \).
For
\( k = 1 \) the hypersurface element and associated
boundary ``integral''
is really just convenient general shorthand, and
should be taken to mean the evaluation of the \( F G \) multivector product over the range of the parameter
\begin{equation*}
\oint_{\partial V} F d^{0} \Bx G
\equiv
\evalbar{ F G }{\Delta u_1}.
\end{equation*}
} % theorem

The geometry of the hypersurface element \( d^{k-1} \Bx \) will be made more clear when we
consider the specific cases of \( k = 1, 2, 3 \), representing generalized line, surface, and volume integrals respectively.
Instead of terrorizing the reader with a general proof
\cref{thm:fundamentalTheoremOfCalculus:1},
which requires some unpleasant index gymnastics,
this book
will separately state and prove the fundamental theorem of calculus
for each of the \( k = 1, 2, 3 \) cases that are of interest for problems in \R{2} and \R{3}.
For the interested reader, a sketch of the general proof
of \cref{thm:fundamentalTheoremOfCalculus:1}
is available in \cref{chap:gagcProof}.

Before moving on to the line, surface, and volume integral cases, we will state and prove the
general Stokes' theorem in its geometric algebra form.

%}
%\EndArticle

%%
% Copyright © 2017 Peeter Joot.  All Rights Reserved.
% Licenced as described in the file LICENSE under the root directory of this GIT repository.
%
%{
The notation and nomenclature used to express Maxwell's equation in the GA literature has not been standardized.
Here are some variations that could be helpful when examining the literature.

\paragraph{Space Time Algebra (STA).  \citep{doran2003gap}}

Maxwell's equation is written
\begin{dmath}\label{eqn:galiterature:80}
\begin{aligned}
\grad F &= J \\
F &= \BE + I \BB \\
I &= \gamma_0 \gamma_1 \gamma_2 \gamma_3 \\
J &= \gamma_\mu J^\mu = \gamma_0 \lr{ \rho - \BJ } \\
\grad &= \gamma^\mu \partial_\mu = \gamma_0 \lr{ \partial_t + \spacegrad }.
\end{aligned}
\end{dmath}

STA uses a relativistic basis \( \setlr{ \gamma_\mu } \) and its dual \( \setlr{ \gamma^\mu} \) for which \( \gamma_0^2 = -\gamma_k^2 = 1, k \in 1,2,3 \), and \( \gamma^\mu \cdot \gamma_\nu = {\delta^\mu}_\nu \).
Spatial vectors are expressed in terms of the Pauli basis \( \sigma_i = \gamma_i \gamma_0 \), which are bivectors that behave as Euclidean basis vectors (squaring to unity, and all mutually anticommutative).
\( F \) is called the electromagnetic field strength (and is not a 1,2 multivector, but a bivector), \( \grad \) is called the vector derivative operator, \( \spacegrad \) called the three-dimensional vector derivative operator, and \( J \) is called the spacetime current (and is a vector, not a multivector).
The physicist's ``natural units'' \( c = \epsilon_0 = \mu_0 \) are typically used in STA.
The d'Alambertian in STA is \( \Box = \grad^2 = \partial_t^2 - \spacegrad^2 \), although the earliest
formulation of STA \citep{hestenes1966space} used \( \Box \) for the vector derivative.
Only spatial vectors are in bold, and all other multivectors are non-bold.

\paragraph{Algebra of Physical Space (APS).  \citep{baylis2004electrodynamics}}
Maxwell's equation is written as

\begin{dmath}\label{eqn:galiterature:40}
\begin{aligned}
\overbar{\partial} \BF &= \inv{\epsilon_0 c} \overline{\jmath} \\
\BF &= \BE + i c \BB \\
i &= \Be_{123} \\
\partial &= \inv{c} \partial_t - \spacegrad \\
j &= \inv{\epsilon_0 c} \lr{ \rho c + \Bj }.
\end{aligned}
\end{dmath}

\( \BF \) is called the Faraday, \( \partial \) the gradient, \( j \) the current density, and
0,1 multivectors are called paravectors.
An operation called Clifford conjugation (or spatial reversal) designated \( \overbar{A} \) is introduced that toggles the sign of any multivector components with grade \( g \mod 4 = 1,2 \).
For example given a multivector \( A = 1 + \Be_1 + \Be_{12} + \Be_{123} \), the Clifford conjugate is

\begin{dmath}\label{eqn:galiterature:60}
\overbar{A} = 1 - \Be_1 - \Be_{12} + \Be_{123},
\end{dmath}

leaving the sign of the scalar and pseudoscalar components untouched.
This Clifford conjugation operation is also used to express relativistic proper length and Lorentz transformations.
The d'Alambertian is written as \( \Box = \partial \overbar{\partial} = (1/c^2) \partial_t^2 - \spacegrad^2 \).
There are a large (arguably confusing) variety of conjugation and complex-like selection operations in APS.

\paragraph{Jancewicz.  \citep{jancewicz1988multivectors}}

% pg. 78
Maxwell's equation in linear isotropic media is written as

\begin{dmath}\label{eqn:galiterature:20}
\begin{aligned}
\calD f &= \tilde{\jmath} \\
\calD &= \spacegrad + \sqrt{\epsilon\mu} \PD{t}{} \\
f &= \Be + \Bcap \\
\Be &= \sqrt{\epsilon} \BE \\
\Bcap &= \inv{\sqrt{\mu}} I \BB \\
I &= \Be_{123} \\
\tilde{\jmath} &= \inv{\sqrt{\epsilon}}\rho - \sqrt{\mu} \Bj \\
\end{aligned}
\end{dmath}

Jancewicz works with fields that have been re-dimensionalized to the same units, uses an overhat bold notation for bivectors (which are sometimes called volutors).
\( \calD \) is called the cliffor differential operator, \( f \) the electromagnetic cliffor, and \( \tilde{\jmath} \) the density of electric sources.
The d'Alambertian is written
as \( \Box = \calD^\conj \calD = \spacegrad^2 - \epsilon\mu \partial_t^2 \), where
\( \calD^\conj = \spacegrad - \sqrt{\epsilon\mu} \partial_t \).

%}

%%
% Copyright © 2017 Peeter Joot.  All Rights Reserved.
% Licenced as described in the file LICENSE under the root directory of this GIT repository.
%
The GA formulation of Maxwell's equation has only been applied in media where it has been assumed throughout that linear constituative relationships

\begin{dmath}\label{eqn:inMatter:20}
\begin{aligned}
\BD &= \epsilon \BE \\
\BB &= \mu \BH,
\end{aligned}
\end{dmath}

have been available.  Without such assumptions the GA formalism for Maxwell's equations cannot be written as a single equation with one multivector field, but requires two equations and two multivector fields.  The two multivector fields are

\begin{dmath}\label{eqn:inMatter:40}
\begin{aligned}
F &= \BE + I v \BB \\
G &= \BD + \frac{I}{v} \BH
\end{aligned}
\end{dmath}

for which Maxwell's equations are

\begin{dmath}\label{eqn:inMatter:60}
\begin{aligned}
\gpgrade{ \lr{ \spacegrad + \inv{v} \PD{t}{} } G }{0,1} &= \rho - \frac{\BJ}{v} \\
\gpgrade{ \lr{ \spacegrad + \inv{v} \PD{t}{} } F }{2,3} &= I \lr{ v \rho_m - \BM }.
\end{aligned}
\end{dmath}

Here \( v \) is a non-dimensionalizing constant with dimensions [L/T], but is otherwise unspecified.
Direct expansion can be used to show that \cref{eqn:inMatter:60} is equivalent to Maxwell's equations.
Doing so for each of the grades in turn, we have

\begin{subequations}
\label{eqn:inMatter:80}
\begin{dmath}\label{eqn:inMatter:100}
\rho
=
\gpgradezero{ \lr{ \spacegrad + \inv{v} \PD{t}{} } G }
=
\gpgradezero{ \lr{ \spacegrad + \inv{v} \PD{t}{} } \lr{ \BD + \frac{I}{v} \BH } }
=
\spacegrad \cdot \BD
\end{dmath}
\begin{dmath}\label{eqn:inMatter:120}
- \frac{\BJ}{v}
=
\gpgradeone{ \lr{ \spacegrad + \inv{v} \PD{t}{} } G }
=
\gpgradeone{ \lr{ \spacegrad + \inv{v} \PD{t}{} } \lr{ \BD + \frac{I}{v} \BH } }
=
\inv{v} \PD{t}{\BD} + \frac{I}{v} \spacegrad \wedge \BH
=
\inv{v} \PD{t}{\BD} - \frac{1}{v} \spacegrad \cross \BH
\end{dmath}
\begin{dmath}\label{eqn:inMatter:140}
- I \BM
=
\gpgrade{ \lr{ \spacegrad + \inv{v} \PD{t}{} } F }{2}
=
\gpgrade{ \lr{ \spacegrad + \inv{v} \PD{t}{} } \lr{ \BE + I v \BB} }{2}
=
\spacegrad \wedge \BE + I \PD{t}{\BB}
\end{dmath}
\begin{dmath}\label{eqn:inMatter:160}
I v \rho_m
=
\gpgrade{ \lr{ \spacegrad + \inv{v} \PD{t}{} } F }{3}
=
\gpgrade{ \lr{ \spacegrad + \inv{v} \PD{t}{} } \lr{ \BE + I v \BB} }{3}
=
v I \spacegrad \cdot \BB.
\end{dmath}
\end{subequations}

After rearranging and cancelling common factors of \( v, I \) Maxwell's equations are recovered

\begin{dmath}\label{eqn:inMatter:n}
\begin{aligned}
\spacegrad \cdot \BD &= \rho \\
\spacegrad \cross \BH &= \BJ + \PD{t}{\BD}  \\
\spacegrad \cross \BE &= -\BM - \PD{t}{\BB} \\
\spacegrad \cdot \BB &= \rho_m.
\end{aligned}
\end{dmath}

How to work directly and effectively with \cref{eqn:inMatter:60} is not obvious, and would require additional study.

%%
% Copyright © 2016 Peeter Joot.  All Rights Reserved.
% Licenced as described in the file LICENSE under the root directory of this GIT repository.
%
\index{Maxwell's equation}
We will work with a multivector representation of the fields in isotropic media satisfying the
constituency relationships from \cref{eqn:freespace:300}, and define a multivector field that includes both electric and magnetic components

\index{\(\eta\)}
\index{\(c\)}
\index{\(F\)}
\makedefinition{Electromagnetic field strength.}{dfn:isotropicMaxwells:640}{
The \textit{electromagnetic field strength} ([\si{V/m}] (Volts/meter)) is defined as
\begin{equation*}
F = \BE + I \eta \BH \quad(= \BE + I c \BB),
\end{equation*}
where
\begin{itemize}
\item \( \eta = \sqrt{\mu/\epsilon} \) (\( [\Omega] \) Ohms), is the impedance of the media.
\item \( c = 1/\sqrt{\epsilon\mu} \) ([\si{m/s}] meters/second), is the group velocity of a wave in the media.  When \( \epsilon = \epsilon_0, \mu = \mu_0 \), \( c \) is the speed of light.
\end{itemize}
\( F \) is called the \textit{F}araday by some authors.
} % definition

The factors of \( \eta \) (or \( c \)) that multiply the magnetic fields are for dimensional consistency, since \( [\sqrt{\epsilon} \BE] = [\sqrt{\mu} \BH] = [\BB/\sqrt{\mu}]\).
The justification for imposing a dual (or complex) structure on the electromagnetic field strength can be found in the historical development of
Maxwell's equations, but we will also see such a structure arise naturally in short order.

No information is lost by imposing the complex structure of
\cref{dfn:isotropicMaxwells:640}, since we can always obtain the
electric field vector \( \BE \) and the magnetic field bivector \( I \BH \) by grade selection
from the electromagnetic field strength when desired
\begin{equation}\label{eqn:isotropicMaxwells:620}
\begin{aligned}
\BE &= \gpgradeone{ F } \\
I \BH &= \inv{\eta} \gpgradetwo{ F }.
\end{aligned}
\end{equation}

We will also
define a multivector current containing all charge densities and current densities
%
% Copyright � 2018 Peeter Joot.  All Rights Reserved.
% Licenced as described in the file LICENSE under the root directory of this GIT repository.
%
\index{\(J\)}
\makedefinition{Multivector current.}{dfn:isotropicMaxwells:660}{
The \textit{current} ([\si{V/m^2}] (Volts/square meter)) is defined as
\begin{equation*}
J = \eta \lr{ c \rho - \BJ } + I \lr{ c \rho_\txtm - \BM }.
\end{equation*}
} % definition

When the fictitious magnetic source terms \((\rho_\txtm, \BM)\) are included, the current has one grade for each possible source (scalar, vector, bivector, trivector).  With only conventional electric sources, the current is still a multivector, but contains only scalar and vector grades.

Given the multivector field and current, it is now possible to state Maxwell's equation (singular) in its geometric algebra form
%
% Copyright � 2018 Peeter Joot.  All Rights Reserved.
% Licenced as described in the file LICENSE under the root directory of this GIT repository.
%
\maketheorem{Maxwell's equation.}{dfn:isotropicMaxwells:680}{
\textit{Maxwell's equation} is a multivector equation relating the change in the electromagnetic field strength to charge and current densities and is written as
\begin{equation*}
\stgrad F = J.
\end{equation*}
} % theorem

Maxwell's equation in this form will be the starting place for all the subsequent analysis in this book.
As mentioned in \cref{chap:GreensFunctions}, the operator \( \spacegrad + (1/c) \partial_t \) will be called the \textit{spacetime gradient}\footnote{This form of spacetime gradient is given a special symbol by a number of authors, but there is no general agreement on what to use.
Instead of entering the fight, it will be written out in full in this book.}.
\begin{proof}
To prove \cref{dfn:isotropicMaxwells:680} we
first insert the
isotropic
constituency relationships from \cref{eqn:freespace:300} into
\cref{eqn:freespace:3399}, so that we are working with two field variables instead of four
\begin{equation}\label{eqn:isotropicMaxwells:500}
\begin{aligned}
\spacegrad \cdot \BE &= \frac{\rho}{\epsilon} \\
\spacegrad \cross \BE &= - \BM - \mu \PD{t}{\BH} \\
\spacegrad \cdot \BH &= \frac{\rho_\txtm}{\mu} \\
\spacegrad \cross \BH &= \BJ + \epsilon \PD{t}{\BE}
\end{aligned}
\end{equation}
Inserting \( \Ba = \spacegrad \) into \cref{eqn:SimpleProducts2:1640} the vector product of the gradient with another vector
\begin{equation}\label{eqn:isotropicMaxwells:520}
\spacegrad \Bb = \spacegrad \cdot \Bb + I \spacegrad \cross \Bb.
\end{equation}
The dot and cross products for \( \BE \) and \( \BH \) in
\cref{eqn:isotropicMaxwells:500}
can be grouped using \cref{eqn:isotropicMaxwells:520} into multivector gradient equations
\begin{equation}\label{eqn:isotropicMaxwells:540}
\begin{aligned}
\spacegrad \BE &= \frac{\rho}{\epsilon} + I \lr{ - \BM - \mu \PD{t}{\BH} } \\
\spacegrad \BH &= \frac{\rho_\txtm}{\mu} + I \lr{ \BJ + \epsilon \PD{t}{\BE} }.
\end{aligned}
\end{equation}
Multiplying the gradient equation for the magnetic field by \( \eta I \) so that both equations have the same dimensions, and so that the electric field appears in both equations as \( \BE \) and not \( I \BE \), we find
\begin{equation}\label{eqn:isotropicMaxwells:560}
\begin{aligned}
\spacegrad \BE        + \inv{c} \PD{t}{} (I \eta \BH) &= \inv{\epsilon}\rho - I \BM  \\
\spacegrad I \eta \BH + \inv{c} \PD{t}{\BE}           &= I c \rho_\txtm - \eta \BJ,
\end{aligned}
\end{equation}
where \( \mu/\eta = \eta \epsilon = 1/c \) was used to simplify things slightly, and all the field contributions have been moved to the left hand side.
The first multivector equation has only scalar and bivector grades, whereas the second has only vector and trivector grades.  This means that if we add these equations, we can recover each by grade selection, and no information is lost.  That sum is
\begin{equation}\label{eqn:isotropicMaxwells:580}
\stgrad \lr{ \BE + I \eta \BH } = \eta\lr{ c \rho - \BJ } + I \lr{ c \rho_\txtm - \BM }.
\end{equation}
Application of \cref{dfn:isotropicMaxwells:640} and \cref{dfn:isotropicMaxwells:660} to
\cref{eqn:isotropicMaxwells:580} proves the theorem, verifying the
assertion that Maxwell's equations can be consolidated into a single multivector equation.
\end{proof}
There is a lot of information packed into this single equation.
Where possible, we want to work with the multivector form of Maxwell's equation, either in the
compact form of \cref{dfn:isotropicMaxwells:680} or the explicit form of \cref{eqn:isotropicMaxwells:580},
and not decompose Maxwell's equation into the conventional representation by grade selection operations.

\subsubsection{Problems.}
\makeproblem{Dot and cross product relation to vector product.}{problem:isotropicMaxwells:700}{
Using coordinate expansion, convince yourself of the validity of \cref{eqn:isotropicMaxwells:520}.
} % problem
\makeanswer{problem:isotropicMaxwells:700}{
\begin{equation}\label{eqn:isotropicMaxwells:660}
\begin{aligned}
\spacegrad \Bb
&=
\sum_{ij=1}^3 \lr{ \Be_i \PD{x_i}{} } \lr{ \Be_j b_j } \\
&=
\sum_{ij=1}^3 \Be_i \Be_j \PD{x_i}{b_j} \\
&=
\sum_{i=1}^3 \Be_i \Be_i \PD{x_i}{b_i} + \sum_{i \ne j} \Be_i \Be_j \PD{x_i}{b_j} .
\end{aligned}
\end{equation}
Here we've decomposed the sum into symmetric and antisymmetric contributions.  The symmetric part reduces easily to the divergence
\begin{equation}\label{eqn:isotropicMaxwells:680}
\sum_{i=1}^3 \Be_i \Be_i \PD{x_i}{b_i}
=
\sum_{i=1}^3 \PD{x_i}{b_i}
= \spacegrad \cdot \Bb.
\end{equation}
Because, for \( i \ne j \), \( \Be_i \Be_j = \Be_i \wedge \Be_j = I \lr{ \Be_i \cross \Be_j } \), and both the wedge and dot products are zero for \( i = j \), we can reintroduce the sum over all \( i, j \) indexes
\begin{equation}\label{eqn:isotropicMaxwells:700}
\begin{aligned}
   \sum_{i \ne j} \Be_i \Be_j \PD{x_i}{b_j}
   &= I \sum_{i \ne j} \Be_i \cross \Be_j \PD{x_i}{b_j}  \\
   &= I \sum_{ij=1}^3 \Be_i \cross \Be_j \PD{x_i}{b_j}  \\
   &= I \sum_{ij=1}^3 \lr{ \Be_i \PD{x_i}{} } \cross \lr{ \Be_j b_j}  \\
   &= I \lr{ \spacegrad \cross \Bb }.
\end{aligned}
\end{equation}
We've demonstrated the desired result, showing that our Laissez-faire substitution \( \Ba = \spacegrad \) in \( \Ba \Bb = \Ba \cdot \Bb + I \lr{ \Ba \cross \Bb} \) was justified, despite the operator nature of the gradient.
} % answer
\makeproblem{Extracting the conventional Maxwell's equations.}{problem:isotropicMaxwells:720}{
Apply grade 0,1,2, and 3 selection operations to \cref{eqn:isotropicMaxwells:580}.  Determine the multiplicative (scalar or trivector) constants required to obtain \cref{eqn:isotropicMaxwells:500} from the equations that result from such grade selection operations.
} % problem
\makeanswer{problem:isotropicMaxwells:720}{
Our grade selection operators yield the following four equations
\begin{equation}\label{eqn:isotropicMaxwells:720}
\begin{aligned}
\gpgradezero{ \spacegrad \BE } &= \eta c \rho  \\
\gpgradeone{ \inv{c}\PD{t}{\BE} + \spacegrad I \eta \BH } &= -\eta \BJ \\
\gpgradetwo{ \spacegrad \BE + \inv{c} \PD{t}{I \eta \BH} } &= -I \BM \\
\gpgradethree{ \spacegrad I \eta \BH } &= I c \rho_m
\end{aligned}
\end{equation}
Observe that
%\begin{equation}\label{eqn:isotropicMaxwells:800}
%\lr{\eta c}^2
%= \frac{\mu}{\epsilon} \frac{1}{\epsilon\mu}
%= \inv{\epsilon^2},
%\end{equation}
%or
\( \eta c = 1/\epsilon \), so the first equation recovers Gauss's law
\begin{equation}\label{eqn:isotropicMaxwells:740}
\spacegrad \cdot \BE = \frac{\rho}{\epsilon}.
\end{equation}
Dividing the vector equation through by \( -\eta \), we have
\begin{equation}\label{eqn:isotropicMaxwells:760}
\frac{-1}{c \eta}\PD{t}{\BE} - I \lr{ \spacegrad \wedge \BH } = \BJ,
\end{equation}
or
\begin{equation}\label{eqn:isotropicMaxwells:780}
-\PD{t}{\epsilon \BE} + \spacegrad \cross \BH = \BJ,
\end{equation}
the Amp\'ere-Maxwell equation (with \( \BD = \epsilon \BE \), and \( \BH = \BB/\mu \).)
Multiplying the bivector equation through by \( -I \), and noting that \( \eta/c = \mu \), we convert it to a vector equation
\begin{equation}\label{eqn:isotropicMaxwells:820}
-I^2 \lr{ \spacegrad \cross \BE } -I^2 \PD{t}{\mu \BH} = I^2 \BM,
\end{equation}
which is the Maxwell-Faraday equation (augmented with the fictious magnetic current density.)  Finally, dividing the pseudoscalar equation through by \( I c \), we find
\begin{equation}\label{eqn:isotropicMaxwells:840}
\rho_m = \frac{\eta}{c} \spacegrad \cdot \BH = \spacegrad \cdot \lr{ \mu \BH },
\end{equation}
which is Gauss's law for magnetism (with the fictious ``engineering'' magnetic charge density term.)
} % answer

%%
% Copyright � 2018 Peeter Joot.  All Rights Reserved.
% Licenced as described in the file LICENSE under the root directory of this GIT repository.
%
%{
\input{../latex/blogpost.tex}
\renewcommand{\basename}{jefimenkosEquations}
%\renewcommand{\dirname}{notes/phy1520/}
\renewcommand{\dirname}{notes/ece1228-electromagnetic-theory/}
%\newcommand{\dateintitle}{}
%\newcommand{\keywords}{}

\input{../latex/peeter_prologue_print2.tex}

\usepackage{peeters_layout_exercise}
\usepackage{peeters_braket}
\usepackage{peeters_figures}
\usepackage{siunitx}
%\usepackage{mhchem} % \ce{}
%\usepackage{macros_bm} % \bcM
%\usepackage{macros_qed} % \qedmarker
%\usepackage{txfonts} % \ointclockwise

\newcommand{\dotBJ}[0]{\mathbf{\dot{J}}}

\beginArtNoToc

\generatetitle{Inverting Maxwell's equation}
%\chapter{Inverting Maxwell's equation}
%\label{chap:jefimenkosEquations}

Maxwell's equation (\cref{eqn:maxwellsEquations:460}) is invertable using the Green's function for the spacetime gradient \cref{thm:greensFunctionSpacetimeGradient:120}.

The full solution is
\begin{dmath}\label{eqn:jefimenkosEquations:20}
F(\Bx, t)
= F_0(\Bx, t)
+ \int dV' dt' G(\Bx - \Bx', t - t') J(\Bx', t'),
\end{dmath}

where \( F_0(\Bx, t) \) is any specific solution of the homogenoous equation \( \lr{ \spacegrad + (1/c) \partial_t } F_0 = 0 \).  With the help of \cref{eqn:derivativeOfDeltaFunction:140} we find for the time integral of \cref{eqn:jefimenkosEquations:20}

\begin{dmath}\label{eqn:jefimenkosEquations:40}
\int dt' G(\Bx - \Bx', t - t') J(\Bx', t')
=
\inv{4 \pi}
\evalbar{
\lr{
\frac{\rcap}{r^2} J(\Bx', t_r)
-
\frac{\rcap}{r} \lr{ -\inv{c} } \frac{d}{dt_r} J(\Bx', t_r)
+
\inv{c r} \frac{d}{dt_r} J(\Bx', t_r)
}
}{t_r = t - r/c}.
\end{dmath}

Denoting the time derivatives with overdots, and evaluating all terms at the retarded time \( t_r = t - r/c \), the full solution of Maxwell's equation is
given by
\boxedEquation{eqn:jefimenkosEquations:60}{
F(\Bx, t)
=
F_0(\Bx, t)
+
\inv{4 \pi}
\int dV'
\lr{
   \frac{\rcap}{r^2} J(\Bx', t_r)
   +
   \inv{c r} \lr{ 1 + \rcap } \dot{J}(\Bx', t_r)
}.
}

There have been lots of opportuntites to mess up with signs and factors of \( c \), so let's expand this out explicitly for a non-magnetic current source \( J = \rho/\epsilon - \eta \BJ \), and check the results against Jefimenko's equations found in \citep{griffiths1999introduction}.
Let's neglect the contribution of the homogeneous solution \( F_0 \), and also utilize our freedom to
insert a no-op grade 1,2 selection operation.
Such a grade selection removes any scalar and pseudoscalar terms that are neccessarily killed over the full integration range, giving

\begin{dmath}\label{eqn:jefimenkosEquations:80}
F =
\inv{4 \pi}
\int dV'
\gpgrade{
   \frac{\rcap}{r^2}
\lr{ \frac{\rho}{\epsilon} - \eta \BJ }
   +
   \inv{c r} \lr{ 1 + \rcap } \lr{ \frac{\dot{\rho}}{\epsilon} - \eta \dotBJ }
}{1,2}
=
\inv{4 \pi}
\int dV'
\lr{
   \frac{\rcap}{\epsilon r^2} \rho
   - \eta \frac{\rcap}{r^2} \wedge \BJ
   - \frac{\eta}{ c r } \dotBJ
   + \inv{c r} \rcap \frac{\dot{\rho}}{\epsilon}
   - \frac{\eta}{c r} \rcap \wedge \dotBJ
}
=
\inv{4 \pi}
\int dV'
\lr{
   \frac{\rcap}{\epsilon r^2} \rho
   + \frac{\rcap \dot{\rho}}{\epsilon c r}
   - \frac{\eta \dotBJ}{ c r }
   - I \frac{\eta }{c r} \rcap \cross \dotBJ
   - I \frac{\eta}{r^2} \rcap \cross \BJ
}.
\end{dmath}

As \( F = \BE + I \eta \BH \), we can read of the respective electric and magnetic fields by inspection.  Inserting the space and time parameters that we suppressed temporarily, we find that Jefimenko's equations for the electric and magnetic fields are

\begin{dmath}\label{eqn:jefimenkosEquations:100}
\begin{aligned}
\BE &=
\inv{4 \pi}
\int dV'
\lr{
   \frac{\rho}{\epsilon r^2} \rcap(\Bx', t_r)
   + \frac{\dot{\rho}}{\epsilon c r} \rcap(\Bx', t_r)
   - \frac{\eta }{ c r } \dotBJ(\Bx', t_r)
} \\
\BH &=
\inv{4 \pi}
\int dV'
\lr{
   \frac{1}{c r} \dotBJ(\Bx', t_r)
+
   \frac{1}{r^2} \BJ(\Bx', t_r)
} \cross \rcap,
\end{aligned}
\end{dmath}

which checks against Griffiths.

%}
\EndArticle
%\EndNoBibArticle

%%
% Copyright © 2017 Peeter Joot.  All Rights Reserved.
% Licenced as described in the file LICENSE under the root directory of this GIT repository.
%
The Lorentz force equation \cref{eqn:freespace:200}
can be restated
in terms of \( F = \BE + I \eta \BH = \BE + I c \BB \) as

\begin{dmath}\label{eqn:lorentzForce:20}
%\boxedEquation{eqn:lorentzForce:20}{
\frac{d\Bp}{dt} = q \gpgradeone{ F \lr{ 1 + \frac{\Bv}{c} } },
%}
\end{dmath}
which puts the electric and magnetic fields on equal footing.
This can be demonstrated by splitting the \( F \lr{ 1 + \ifrac{\Bv}{c} } \) multivector into its constituent grades

\begin{dmath}\label{eqn:lorentzForce:40}
q F \lr{ 1 + \frac{\Bv}{c} }
=
q
\lr{ \BE + I c \BB }
\lr{ 1 + \frac{\Bv}{c} }
=
q \BE
+ q I \BB \Bv
+ \frac{q}{c} \BE \Bv
+ q c I \BB
=
  \frac{q}{c} \BE \cdot \Bv
+ q \lr{ \BE + \Bv \cross \BB }
+ q \lr{ c I \BB + \inv{c} \BE \wedge \Bv }
+ q (I \BB) \wedge \Bv.
\end{dmath}

The grade 0 component of this product hints of \cref{eqn:freespace:220}, and substitution into the vector grade selection operation of \cref{eqn:lorentzForce:20} recovers \cref{eqn:freespace:200} as desired.

Looking to the energy-momentum tensor for the continuum equivalent of the dual Lorentz force equation \cref{eqn:poyntingLorentzForce:140}, we can introduce a generalized multivector charge

\begin{dmath}\label{eqn:lorentzForce:240}
Q =
q_\txte \lr{ 1 + \Bv_\txte/c }
-I \epsilon q_\txtm \lr{ 1 + \Bv_\txtm/c },
\end{dmath}
where \( q_\txte, q_\txtm \) are the electric and magnetic charges, and \( \Bv_\txte, \Bv_\txtm \) are their respective velocities.
The Lorentz force equation, including both electric and fictious magnetic charges, can now be written as
%\begin{dmath}\label{eqn:lorentzForce:260}
\boxedEquation{eqn:lorentzForce:260}{
\frac{d\Bp}{dt} = \gpgradeone{ F Q }.
}
%\end{dmath}


%%
% Copyright © 2016 Peeter Joot.  All Rights Reserved.
% Licenced as described in the file LICENSE under the root directory of this GIT repository.
%
\section{Conventional differential form}

The differential form of Maxwell's equations, with extensions for magnetic sources, is the starting point for all the analysis in these notes.  Those equations are

\input{../ece1229-antenna/MaxwellsStatement.tex}

The magnetic sources can be considered fictional, and are included because they are useful in antenna theory to model real phenomina such as infinitesimal current loops.

\input{../ece1229-antenna/MaxwellsFieldAndSourceDescription.tex}

These fields and sources are all real valued.  In many situations it will be desirable to work with a time harmonic (frequency-domain phasor) form of Maxwell's equations.  In engineering, a time harmonic representation presumes that all sources and fields have a frequency dependence of the form
\index{time harmonic}

\begin{dmath}\label{eqn:maxwellsEquations:20}
\bcY(\Bx, t) = \Real( \BY(\Bx, \omega) e^{j\omega t} ),
\end{dmath}

where the field (or source) \( \BY(\Bx, \Bomega) \) is allowed to be complex valued, whereupon Maxwell's equations take the form

\input{../ece1229-antenna/MaxwellsTimeHarmonic.tex}

Note that the time harmonic convention typically used in physics literature presumes a frequency dependence of the form

\begin{dmath}\label{eqn:maxwellsEquations:40}
\bcY(\Bx, t) = \Real( \BY(\Bx, \omega) e^{-i\omega t} ),
\end{dmath}

which alters the sign of any imaginary originating from a time derivative.  Care is required by the reader to understand which form of frequency dependence has been assumed.

\section{GA differential form}

Geometric Algebra admits a number of alternative representations of Maxwell's equations.  The first follows from expressing the cross products all as wedge products, leaving a pair of bivector and a pair of scalar equations

\begin{subequations}
\begin{dmath}\label{eqn:maxwellsEquations:60}
\spacegrad \wedge \bcE = - I \bcM - \PD{t}{I\bcB}
\end{dmath}
\begin{dmath}\label{eqn:maxwellsEquations:80}
\spacegrad \wedge \bcH = I \bcJ + I \PD{t}{\bcD}
\end{dmath}
\begin{dmath}\label{eqn:maxwellsEquations:100}
\spacegrad \cdot \bcD = q_\txte
\end{dmath}
\begin{dmath}\label{eqn:maxwellsEquations:120}
\spacegrad \cdot \bcB = q_\txtm.
\end{dmath}
\end{subequations}

Alternatively, the duality transformation \( \Ba \wedge \Bb = -I \Ba \cdot (I \Bb) \) allows Maxwell's equations to be all written as dot products

\begin{subequations}
\begin{dmath}\label{eqn:maxwellsEquations:140}
\spacegrad \cdot (I \bcE) = \bcM + \PD{t}{\bcB}
\end{dmath}
\begin{dmath}\label{eqn:maxwellsEquations:160}
\spacegrad \cdot (I \bcH) = -\bcJ - \PD{t}{\bcD}
\end{dmath}
\begin{dmath}\label{eqn:maxwellsEquations:180}
\spacegrad \cdot \bcD = q_\txte
\end{dmath}
\begin{dmath}\label{eqn:maxwellsEquations:200}
\spacegrad \cdot \bcB = q_\txtm,
\end{dmath}
\end{subequations}

or, using the duality transformation \( \Ba \cdot \Bb = -I (\Ba \wedge (I \Bb) \), Maxwell's equations can all be written as wedge products

\begin{subequations}
\begin{dmath}\label{eqn:maxwellsEquations:220}
\spacegrad \wedge \bcE = - I \bcM - \PD{t}{I\bcB}
\end{dmath}
\begin{dmath}\label{eqn:maxwellsEquations:240}
\spacegrad \wedge \bcH = I \bcJ + I \PD{t}{\bcD}
\end{dmath}
\begin{dmath}\label{eqn:maxwellsEquations:260}
\spacegrad \wedge (I\bcD) = I q_\txte
\end{dmath}
\begin{dmath}\label{eqn:maxwellsEquations:280}
\spacegrad \wedge (I\bcB) = I q_\txtm.
\end{dmath}
\end{subequations}

Each of these forms can be useful in different circumstances, however the real power of GA in electromagnetism follows from presuming constituative relationships between the pairs of electric and magnetic fields

\begin{subequations}
\label{eqn:maxwellsEquations:300}
\begin{dmath}\label{eqn:maxwellsEquations:320}
\bcB = \mu \bcH
\end{dmath}
\begin{dmath}\label{eqn:maxwellsEquations:340}
\bcD = \epsilon \bcE,
\end{dmath}
\end{subequations}

where \( \epsilon \) is the permitivitity of the medium [\si{F/m}] (Farads/meter), and \( \mu \) is the permeability of the medium [\si{H/m}] (Henries/meter).
The permitivitity and permeability may be functions of both time and position, and model the materials that the fields are propagating through.  In general, the these may be non-isotropic tensor operators, however, unless otherwise specified, isotropic media will be assumed in these notes.

With this constitutative relationship assumed (and a bit of rescaling), the dot and wedge products of \cref{eqn:maxwellsEquations:60}, \cref{eqn:maxwellsEquations:100} can be added, as can those of \cref{eqn:maxwellsEquations:80}, \cref{eqn:maxwellsEquations:120}.  This reduces Maxwell's equations to a pair of first order coupled gradient equations

\begin{subequations}
\begin{dmath}\label{eqn:maxwellsEquations:360}
\spacegrad \bcE = \inv{\epsilon} q_\txte - I \bcM - \mu \PD{t}{(I\bcH)}
\end{dmath}
\begin{dmath}\label{eqn:maxwellsEquations:380}
\spacegrad (I \bcH) = \frac{I q_\txtm}{\mu} - \bcJ - \epsilon \PD{t}{\bcE}.
\end{dmath}
\end{subequations}



%%
% Copyright © 2017 Peeter Joot.  All Rights Reserved.
% Licenced as described in the file LICENSE under the root directory of this GIT repository.
%
%{
Geometric algebra takes a vector space and adds two additional operations, a vector multiplication operation, and a generalized addition operation that extends vector addition to include addition of scalars and products of vectors.
Multiplication of vectors is indicated by juxtaposition, for example, if \( \Bx, \By, \Be_1, \Be_2, \Be_3, \cdots \) are vectors, then some vector products are
\begin{dmath}\label{eqn:multivector:20}
\begin{aligned}
&\Bx \By, \Bx \By \Bx, \Bx \By \Bx \By, \\
&\Be_1 \Be_2, \Be_2 \Be_1, \Be_2 \Be_3, \Be_3 \Be_2, \Be_3 \Be_1, \Be_1 \Be_3, \\
&\Be_1 \Be_2 \Be_3, \Be_3 \Be_1 \Be_2, \Be_2 \Be_3 \Be_1, \Be_3 \Be_2 \Be_1, \Be_2 \Be_1 \Be_3, \Be_1 \Be_3 \Be_2, \\
&\Be_1 \Be_2 \Be_3 \Be_1, \Be_1 \Be_2 \Be_3 \Be_1 \Be_3 \Be_2, \cdots
\end{aligned}
\end{dmath}

Vector multiplication is constrained by a rule, called the contraction axiom, that gives a meaning to the square of a vector
\boxedEquation{eqn:multivector:120}{
\Bx \Bx \equiv \Bx \cdot \Bx.
}

The square of a vector, by this definition, is the squared length of the vector, and is a scalar.
This may not appear to be a useful way to assign meaning to the simplest of vector products, since the product and the vector live in separate spaces.
If we want a closed algebraic system that includes both vectors and their products, we have to allow for the addition of scalars, vectors, or any products of vectors.  Such a sum is called a multivector, an example of which is
\begin{dmath}\label{eqn:multivector:40}
1 + 2 \Be_1 + 3 \Be_1 \Be_2 + 4 \Be_1 \Be_2 \Be_3.
\end{dmath}
In this example, we have added a
scalar (or 0-vector) \( 1 \), to a
vector (or 1-vector) \( 2 \Be_1 \), to a
bivector (or 2-vector) \( 3 \Be_1 \Be_2 \), to a
trivector (or 3-vector) \( 4 \Be_1 \Be_2 \Be_3 \).
Geometric algebra uses vector multiplication to build up a hierarchy of geometrical objects, representing points, lines, planes, volumes and hypervolumes (in higher dimensional spaces.)

\index{scalar}
\index{0-vector}
\paragraph{Scalar.}
A scalar, which we will also call a 0-vector, is a zero-dimensional object with sign, and a magnitude.
We may geometrically interpret a scalar as a (signed) point in space.
%The sign of a scalar can be represented graphically as an arrow with a head and a tail pointing into the paper (or chalkboard),
%as illustrated in
%\cref{fig:scalarOrientation:scalarOrientationFig1} where a crossed circle represents the tail, and a solid dot represents the head.
%%\footnote{We don't usually try to represent the magnitude of a scalar graphically, but could do so by scaling the size of the cross or dot.}
%\imageFigure{../figures/GAelectrodynamics/scalarOrientationFig1}{Scalar illustration.}{fig:scalarOrientation:scalarOrientationFig1}{0.05}

\index{vector}
\index{1-vector}
\paragraph{Vector.}
A vector, which we will also call a 1-vector, is a one-dimensional object with a sign, a magnitude, and a rotational attitude within the space it is embedded.
XX
\index{bivector}
\index{2-vector}
\paragraph{Bivector.}

We now wish to define a bivector, or 2-vector, as a 2 dimensional object representing a signed plane segment with magnitude and orientation.  Formally,
assuming a vector product, the algebraic description of a bivector is

\makedefinition{Bivector.}{dfn:multivector:60}{
A bivector, or 2-vector, is a sum of products of pairs of orthogonal vectors.
Given an \( N \) dimensional vector space \( V \) with an orthonormal basis \( \setlr{ \Be_1, \Be_2, \cdots, \Be_N } \),
a general bivector can be expressed as
\begin{equation*}
\sum_{1 \le i < j \le N} B_{ij} \Be_i \Be_j,
\end{equation*}
where \( B_{ij} \) is a scalar.
The vector basis \( V \) is said to be a generator of a bivector space.
} % definition

The bivector provides a structure that can encode plane oriented quantities such as torque, angular momentum, or a general plane of rotation.
A quantity like angular momentum can be represented as a magnitude times a quantity that represents the orientation of the plane of rotation.
In conventional vector algebra we use the normal of the plane to describe this orientation, but that is problematic in higher dimensional spaces where there is no unique normal.
Use of the normal to represent a plane is also logically problematic in two dimensional spaces, which have to be extended to three dimensions to use normal centric constructs like the cross product.
A bivector representation of a plane can eliminate the requirement to utilize a third (normal) dimension, which may not be relevant in the problem, and can allow some concepts (like the cross product) to be generalized to dimensions other than three when desirable.

One of the implications of the contraction axiom \cref{eqn:multivector:120}, to be discussed in more detail a bit later, is a linear dependence between bivectors formed from orthogonal products.  For example, given any pair of unit bivectors, where \( i \ne j \) we have
\begin{dmath}\label{eqn:multivector:140}
\Be_i \Be_j + \Be_j \Be_i = 0,
\end{dmath}
This is why the sum in \cref{dfn:multivector:60} was over only half the possible pairs of \( i \ne j \) indexes.
The reader can check that the set of all bivectors is a vector space per
\cref{def:prerequisites:vectorspace}, so we will call the set of all bivectors a bivector space.
In \R{2} a basis for the bivector space is \( \setlr{ \Be_1 \Be_2 } \), whereas in \R{3} a basis for the bivector space is
\( \setlr{ \Be_1 \Be_2, \Be_2 \Be_3, \Be_3 \Be_1 } \).  The unit bivectors for two possible \R{3} bivector space bases are illustrated in
\cref{fig:unitBivectors:unitBivectorsFig}.
\imageTwoFigures
{../figures/GAelectrodynamics/unitBivectorsFig1}
{../figures/GAelectrodynamics/unitBivectorsFig2}
{Unit bivectors for \R{3}}
{fig:unitBivectors:unitBivectorsFig}
{scale=0.35}

We interpret the sign of a vector as an indication of the sense of the vector's ``head'' vs ``tail''.
For a bivector, we can interpret the sign as a representation of a
a ``top'' vs. ``bottom'', or equivalently a left or right ``handedness'', as illustrated using arrows around a plane segment in
\cref{fig:circularBivectorsIn3D:circularBivectorsIn3DFig1}.
\imageFigure{../figures/GAelectrodynamics/circularBivectorsIn3DFig1}{Circular representation of two bivectors.}{fig:circularBivectorsIn3D:circularBivectorsIn3DFig1}{0.3}
For a product like \( \Be_1 \Be_2 \), the sense of the handedness follows the path \( 0 \rightarrow \Be_1 \rightarrow \Be_1 + \Be_2 \rightarrow \Be_2 \rightarrow 0 \) around the unit square in the x-y plane.
This is illustrated for all the unit bivectors in \cref{fig:unitBivectors:unitBivectorsFig}.
In \R{3} we can use the right hand rule to visualize such a handedness.  You could say that we are using the direction of the fingers around the normal to indicate the sign of the bivector, but without actually drawing that normal.

Similar to the interpretation of the magnitude of a vector as the length of that vector, we interpret the magnitude of a bivector (to be defined more exactly later), as the area of the bivector.
Other than having a boundary that surrounds a given area, a graphical bivector representation as a plane segment need not have any specific geometry, which is illustrated in
\cref{fig:bivectorRepresentationsInPlane:bivectorRepresentationsInPlaneFig1} for a set of bivectors all representing \( \Be_1 \Be_2 \).
\imageFigure{../figures/GAelectrodynamics/bivectorRepresentationsInPlaneFig1}{Graphical representations of \( \Be_1 \Be_2 \).}{fig:bivectorRepresentationsInPlane:bivectorRepresentationsInPlaneFig1}{0.3}

An oriented plane segment can always be represented as a bivector in any number of dimensions, however, when the generating vector space has dimension \( N \ge 4 \) not all bivectors defined by \cref{dfn:multivector:60} necessarily represent oriented plane segments.
The restrictions required for a bivector to have an associated oriented plane segment interpretation in higher dimensional spaces will be defined later.

Vector addition can be performed graphically by connecting vectors head to tail, and joining the first tail to the last head.  A similar procedure can be used for bivector addition as well, but gets complicated if the bivectors lie in different planes.  Here is a simple bivector sum
\begin{dmath}\label{eqn:multivector:160}
3 \Be_1 \Be_2 - 2 \Be_1 \Be_2 + 5 \Be_1 \Be_2 = 6 \Be_1 \Be_2,
\end{dmath}
which can be interpreted as taking a 3 unit area, subtracting a 2 unit area, and adding a 5 unit area.  This sum is illustrated in
\cref{fig:bivectorAdditionInPlane:bivectorAdditionInPlaneFig1}.
An visualization of arbitrarily oriented bivector addition can be found in
\cref{fig:AdditionOfBivectors:AdditionOfBivectorsFig2}, where \( \text{red} + \text{blue} = \text{green} \).  This visualization shows that the
moral of the story is that we will almost exclusively be adding bivectors algebraically, but can interpret the sum geometrically after the fact.
\imageFigure{../figures/GAelectrodynamics/bivectorAdditionInPlaneFig1}{Graphical representation of bivector addition in plane.}{fig:bivectorAdditionInPlane:bivectorAdditionInPlaneFig1}{0.2}
\imageFigure{../figures/GAelectrodynamics/AdditionOfBivectorsFig2}{Bivector addition.}{fig:AdditionOfBivectors:AdditionOfBivectorsFig2}{0.3}
%The same can be done with bivectors, where the bivectors are also connected with compatible orientation to construct a sum.
%This is illustrated graphically in \cref{fig:AdditionOfBivectors:AdditionOfBivectorsFig1}, where a blue bivector with a right handed orientation is added to a red bivector with right handed orientation, to form a green bivector also with right handed orientation, where all orientations are with respect to the exterior of the bounding surface formed by the three bivectors.
%\imageFigure{../figures/GAelectrodynamics/AdditionOfBivectorsFig1}{Bivector addition.}{fig:AdditionOfBivectors:AdditionOfBivectorsFig1}{0.3}

\index{trivector}
\index{3-vector}
\paragraph{Trivector.}

Again, assuming a vector product

\makedefinition{Trivector.}{dfn:multivector:80}{
A trivector, or 3-vector, is a sum of products of triplets of mutually orthogonal vectors.
Given an \( N \) dimensional vector space \( V \) with an orthonormal basis \( \setlr{ \Be_1, \Be_2, \cdots, \Be_N } \), a trivector is any value
\begin{equation*}
\sum_{1 \le i < j < k \le N} T_{ijk} \Be_i \Be_j \Be_k,
\end{equation*}
where \( \T_{ijk} \) is a scalar.
The vector space \( V \) is said to generate a trivector space.
} % definition

In \R{3} all trivectors are scalar multiples of \( \Be_1 \Be_2 \Be_3 \).
Like scalars, there is no direction to such a quantity, but like scalars trivectors may be signed.  The magnitude of a trivector may be interpreted as a volume.
We will defer interpreting the sign of a trivector geometrically until we tackle integration theory.
%%%, which requires some interpretation.
%%%We can interpret the magnitude of a trivector as a volume, but what is a signed volume?
%%%One answer to this question is that we can interpret the sign of the volume as the exterior or the interior of the surface on the boundry of the volume.
%%%We will see another answer when we study integration theory, since geometric integration theory uses signed volume elements, and
%%%swapping the order of two adjacent products in the volume element toggles the sign.
%%%\footnote{In conventional integration theory,
%%%this sign change occurs when swapping rows or columns in the Jacobian, but this is masked by taking the absolute value of the Jacobian after coordinate transformation.}
%%%One possible interpretation of this sign is the interior or the exterior of the bounding surface of a volume.
%%%%This orientation can be visualized with a normal pointing into or out of the volume, or like bivectors, with a cyclic direction on the surface of the volume as in illustrated with the spherical volume of \cref{fig:orientedVolume:orientedVolumeFig1}.
%%%%\imageFigure{../figures/GAelectrodynamics/orientedVolumeFig1}{Oriented Volume}{fig:orientedVolume:orientedVolumeFig1}{0.3}
%%%%In greater than three dimensions, a trivector can have a ``direction'' in the higher dimensional space, as well as a sidedness.
%%%%As was the case with the bivector, because not all the products \( \Be_i \Be_j \Be_k \) for any set of indexes \( i, j, k \) are independent, it is possible to form a trivector as a sum over a more restricted set, such as \( \sum_{1 \le i < j < k \le N} T_{ijk} \Be_i \Be_j \Be_k \).
%%%%In particular, in three dimensions, all trivectors can be expressed as scalar multiples of \( \Be_1 \Be_2 \Be_3 \).
%%%%
\index{k-vector}
\index{grade}
\paragraph{K-vector.}
\makedefinition{K-vector and grade.}{dfn:multivector:100}{
A k-vector is a sum of products of \( k \) mutually orthogonal vectors.
Given an \( N \) dimensional vector space with an orthonormal basis \( \setlr{ \Be_1, \Be_2, \cdots, \Be_N } \),
a general k-vector can be expressed as
\begin{equation*}
\sum_{1 \le i < j \cdots < m \le N} K_{i j \cdots m} \Be_{i} \Be_{j} \cdots \Be_{m},
\end{equation*}
where \( K_{i j \cdots m} \) is a scalar, indexed by \( k \) indexes \( i, j, \cdots, m \).

The number \( k \) of orthogonal vectors that generate a k-vector is called the grade.

A 0-vector is a scalar.

The vector space \( V \) is said to generate the k-vector space.
} % definition

Illustrating by example, \( 1 \) is a 0-vector with grade 0, \( \Be_1 \) is a 1-vector with grade 1, \( \Be_1 \Be_2, \Be_2 \Be_3 \), and \( \Be_3 \Be_1 \) are 2-vectors with grade 2, and \( \Be_1 \Be_2 \Be_3 \) is a 3-vector with grade 3.

We will see that the highest grade for a k-vector in an N dimensional vector space is \( N \).

\index{multivector}
\index{multivector space}
\paragraph{Multivector space.}
\makedefinition{Multivector space.}{def:multiplication:multivectorspace}{
   Given an N dimensional (generating) vector space \( V \) 
and a vector multiplication operation represented by juxtaposition,
a multivector is a sum of k-vectors, \( k \in [ 1, N ] \).

The multivector space generated by \( V \) is a set \( M = \setlr{ x, y, z, \cdots } \) of multivectors, where the following axioms are satisfied

\begin{tablebox}[tabularx={X|Y}]{Multivector space axioms.}
    Contraction & \( \Bx^2 = \Bx \cdot \Bx, \,\forall \Bx \in V \) \\ \hline
    \( M \) is closed under addition & \( x + y \in M \) \\ \hline
    \( M \) is closed under multiplication & \( x y \in M \) \\ \hline
    Addition is associative & \( (x + y) + z = x + (y + z) \) \\ \hline
    Addition is commutative & \( y + x = x + y \) \\ \hline
    There exists a zero element \( 0 \in M \)  & \( x + 0 = x \) \\ \hline
    For all \( x \in M \) there exists a negative additive inverse \( -x \in M \) & \( x + (-x) = 0 \) \\ \hline
    Multiplication is distributive  & \( x( y + z ) = x y + x z \), \( (x + y)z = x z + y z \) \\ \hline
    Multiplication is associative & \( (x y) z = x ( y z ) \) \\ \hline
    There exists a multiplicative identity \( 1 \in M \) & \( 1 x = x \) \\ \hline
\end{tablebox}
}

(CUT)

with an orthonormal basis \( \setlr{ \Be_1, \Be_2, \cdots, \Be_N } \),
%a basis \( \setlr{ \Bx_1, \Bx_2, \cdots } \),
, such as
   \( a_0 + \sum_i a_i \Be_i + \sum_{i \ne j} a_{ij} \Be_i \Be_j + \sum_{i \ne j \ne k} a_{ijk} \Be_i \Be_j \Be_k + \cdots \), where \( a_0, a_i, a_{ij}, \cdots \) are scalars.

Compared to the vector space, \cref{def:prerequisites:vectorspace}, the multivector space

\begin{itemize}
\item specifies a rule providing the meaning of a squared vector (the contraction axiom).
\item presumes a vector multiplication operation, which is not assumed to be commutative (order matters),
\item generalizes vector addition to multivector addition,
\item generalizes scalar multiplication to multivector multiplication (of which scalar multiplication and vector multiplication are special cases),
\end{itemize}

The contraction axiom is arguably the most important of the multivector space axioms, as it allows for multiplicative closure without an infinite dimensional multivector space.
The remaining set of non-contraction axioms of a multivector space are almost that of a field, however,
%\footnote{A mathematician would call a multivector space a non-commutative ring with identity \citep{van1943modern}, and could state the multivector space definition much more compactly without listing all the properties of a ring explicitly as done above.}
%(as encountered in the study of complex inner products),
%as they describe most of the properties one
%would expect of a ``well behaved'' set of number-like quantities.
a field also requires a multiplicative inverse element for all elements of the space, which exists for some multivector subspaces, but not in general.

%These axioms may seem simple enough, especially since they are not that different from the familiar axioms of the vector space,
%but it will take considerable work to extract all their consequences.
%The subject of Geometric Algebra can be viewed as the study of the impliciations of the axioms
%of the multivector space.

%}

%%
% Copyright © 2017 Peeter Joot.  All Rights Reserved.
% Licenced as described in the file LICENSE under the root directory of this GIT repository.
%

A fair amount of nomenclature and notation is unfortunately required before systematically examining the implications of the multivector space axioms that define geometric algebra.

Multivectors which can be factored into normal vector products, such as
\begin{dmath}\label{eqn:multiplication:220}
\Be_1 \Be_2 + 3 \Be_1 \Be_3
=
\Be_1 (\Be_2 + 3 \Be_3),
\end{dmath}

are blades.
In contrast, the following grade 2 multivectors

\begin{dmath}\label{eqn:multiplication:240}
\Be_1 \Be_2 + \Be_3 \Be_4,
\end{dmath}

and
\begin{dmath}\label{eqn:multiplication:260}
\Be_1 \Be_2 + \Be_2 \Be_3 + \Be_3 \Be_1,
\end{dmath}

which cannot be factored into two vector products, are not blades.

\index{k-vector}
\makedefinition{k-vector.}{dfn:multivector:kvector}{
A sum of k-blades is called a k-vector.
} % definition

Multivectors are therefore sums of k-vectors with different grades.

All the k-blade examples in 
\cref{eqn:multivector:180}
 are also k-vectors.
K-vectors with grades 2 and 3 are so pervasive that they are given special names.

\index{bivector}
\makedefinition{Bivector.}{dfn:multivector:bivector}{
A bivector, or 2-vector, is a k-vector with grade 2.
} % definition

Any 2-blade, such as the product \( \Be_1 \Be_2 \) is a bivector.
Any sum of 2-blades, such as \( \Be_2 \Be_3 + 3 \Be_4 \Be_1 \), is also a bivector.
%Each of \( \Be_1 \Be_2, \Be_2 \Be_1, \Be_1 \Be_2 + \Be_2 \Be_3 \), and \( \Be_1 \Be_2 + \Be_3 \Be_4 \) are bivectors.
%All but the last of these represents an oriented plane segment.

\index{trivector}
\makedefinition{Trivector.}{dfn:multivector:trivector}{
A trivector, or 3-vector, is a k-vector with grade 3.
} % definition

%Quantities with higher grades than 3 are not generally given explicit names.
The multivector \( \Be_3 \Be_1 \Be_2 \) is a trivector, as is \( \Be_1 \Be_2 \Be_3 + 3 \Be_5 \Be_4 \Be_1 \).
The latter is not a blade.
%Each of \( \Be_1 \Be_2 \Be_3, \Be_1 \Be_3 \Be_2, \Be_1 \Be_4 \Be_2 \) are trivectors.
% , and represent oriented volumes.



%%
% Copyright � 2018 Peeter Joot.  All Rights Reserved.
% Licenced as described in the file LICENSE under the root directory of this GIT repository.
%
%{
%%%\input{../latex/blogpost.tex}
%%%\renewcommand{\basename}{planewavesMultivector}
%%%%\renewcommand{\dirname}{notes/phy1520/}
%%%\renewcommand{\dirname}{notes/ece1228-electromagnetic-theory/}
%%%%\newcommand{\dateintitle}{}
%%%%\newcommand{\keywords}{}
%%%
%%%\input{../latex/peeter_prologue_print2.tex}
%%%
%%%\usepackage{peeters_layout_exercise}
%%%\usepackage{peeters_braket}
%%%\usepackage{peeters_figures}
%%%\usepackage{siunitx}
%%%%\usepackage{mhchem} % \ce{}
%%%%\usepackage{macros_bm} % \bcM
%%%%\usepackage{macros_qed} % \qedmarker
%%%%\usepackage{txfonts} % \ointclockwise
%%%
%%%\beginArtNoToc
%%%
%%%\generatetitle{Multivector plane wave representation}
%\chapter{Multivector plane wave representation}
\label{chap:planewavesMultivector}

With all sources zero,
the free space Maxwell's equation as given by \cref{dfn:isotropicMaxwells:680} for the
electromagnetic field strength reduces to just
\begin{equation}\label{eqn:planewavesMultivector:300}
\stgrad F(\Bx, t) = 0.
\end{equation}

Utilizing a phasor representation of the form \cref{dfn:greensFunctionOverview:300},
we will define the
phasor representation of the field as
%
% Copyright � 2018 Peeter Joot.  All Rights Reserved.
% Licenced as described in the file LICENSE under the root directory of this GIT repository.
%
\makedefinition{Plane wave.}{dfn:planewavesMultivector:680}{
We represent the
electromagnetic field strength
plane wave solution of Maxwell's equation in phasor form as
\begin{equation*}
F(\Bx, t) = \Real \lr{ F(\Bk) e^{ j \omega t }  },
\end{equation*}
where the complex valued multivector \( F(\Bk) \) also has a presumed exponential dependence
\begin{equation*}
F(\Bk)
=
\tilde{F}
e^{ -j \Bk \cdot \Bx }.
\end{equation*}
} % definition


We will now show that solutions of the electromagnetic field wave equation have the form

\index{\(\omega\)}
\index{\(\Bk\)}
\index{\(\kcap\)}
%
% Copyright � 2018 Peeter Joot.  All Rights Reserved.
% Licenced as described in the file LICENSE under the root directory of this GIT repository.
%
\maketheorem{Plane wave solutions to Maxwell's equation.}{thm:planewavesMultivector:620}{
Single frequency \textit{plane wave solutions of Maxwell's equation} have the form
\begin{equation*}
F(\Bx, t)
=
\Real \lr{
\lr{ 1 + \kcap }
\kcap \wedge \BE\,
e^{-j \Bk \cdot \Bx + j \omega t}
}
,
\end{equation*}
where \( \Norm{\Bk} = \omega/c \), \( \kcap = \Bk/\Norm{\Bk} \) is the unit vector pointing along the propagation direction, and \( \BE \) is any complex-valued vector variable.
When a \( \BE \cdot \Bk = 0 \) constraint is imposed on the vector variable \( \BE \), that variable can be interpreted as the electric field, and the solution reduces to
\begin{equation*}
F(\Bx, t)
=
\Real \lr{
\lr{ 1 + \kcap }
\BE\,
e^{-j \Bk \cdot \Bx + j \omega t}
}
,
\end{equation*}
showing that the field phasor \( F(\Bk) = \BE(\Bk) + I \eta \BH(\Bk) \) splits naturally into electric and magnetic components
\begin{equation*}
\begin{aligned}
\BE(\Bk) &= \BE\, e^{-j \Bk \cdot \Bx} \\
\eta \BH(\Bk) &= \kcap \cross \BE \, e^{-j \Bk \cdot \Bx},
\end{aligned}
\end{equation*}
where the directions \( \kcap, \BE, \BH \) form a right handed triple.
} % theorem

\begin{proof}
We wish to act on \( F(\Bk) e^{-j \Bk \cdot \Bx + j \omega t } \) with the spacetime gradient \( \spacegrad + (1/c)\partial_t \), but
must take care of order when applying the gradient to a non-scalar valued function.  In particular, if \( A \) is a multivector, then
\begin{equation}\label{eqn:planewavesMultivector:660}
\begin{aligned}
\spacegrad A e^{-j \Bk \cdot \Bx}
&= \sum_{m = 1}^3 \Be_m \partial_m A e^{-j \Bk \cdot \Bx} \\
&= \sum_{m = 1}^3 \Be_m A \lr{ -j k_m } e^{-j \Bk \cdot \Bx} \\
&= -j \Bk A.
\end{aligned}
\end{equation}
Therefore, insertion of the presumed phasor solution of the field from
\cref{dfn:planewavesMultivector:680} into
\cref{eqn:planewavesMultivector:300} gives
\begin{equation}\label{eqn:planewavesMultivector:60}
0 = -j \lr{ \Bk - \frac{\omega}{c} } F(\Bk).
\end{equation}

If \( F(\Bk) \) has a left multivector factor
\begin{equation}\label{eqn:planewavesMultivector:80}
F(\Bk) = \lr{ \Bk + \frac{\omega}{c} } \tilde{F},
\end{equation}
where \( \tilde{F} \) is a multivector to be determined, then
\begin{equation}\label{eqn:planewavesMultivector:100}
\begin{aligned}
\lr{ \Bk - \frac{\omega}{c} } F(\Bk)
&= \lr{ \Bk - \frac{\omega}{c} } \lr{ \Bk + \frac{\omega}{c} } \tilde{F} \\
&= \lr{ \Bk^2 - \lr{\frac{\omega}{c}}^2 } \tilde{F},
\end{aligned}
\end{equation}
which is zero if \( \Norm{\Bk} = \ifrac{\omega}{c} \).
Let \( \Norm{\Bk} \tilde{F} = F_0 + F_1 + F_2 + F_3 \), where
\( F_0, F_1, F_2, \) and \( F_3 \) respectively have grades 0,1,2,3, so that
\begin{equation}\label{eqn:planewavesMultivector:120}
\begin{aligned}
F(\Bk)
&= \lr{ 1 + \kcap } \lr{ F_0 + F_1 + F_2 + F_3 } \\
&= F_0 + F_1 + F_2 + F_3 + \kcap F_0 + \kcap F_1 + \kcap F_2 + \kcap F_3 \\
&= F_0 + F_1 + F_2 + F_3
+
\kcap F_0 + \kcap \cdot F_1 + \kcap \cdot F_2 + \kcap \cdot F_3 \\
&\quad +
\kcap \wedge F_1 + \kcap \wedge F_2 \\
&=
\lr{
   F_0 + \kcap \cdot F_1
}
+
\lr{
   F_1 + \kcap F_0 + \kcap \cdot F_2
} \\
&\quad +
\lr{
   F_2 + \kcap \cdot F_3 + \kcap \wedge F_1
}
+
\lr{
   F_3 + \kcap \wedge F_2
}.
\end{aligned}
\end{equation}
Since the field \( F \) has only vector and bivector grades, the grades zero and three components of the expansion above must be zero, or
\begin{equation}\label{eqn:planewavesMultivector:140}
\begin{aligned}
   F_0 &= - \kcap \cdot F_1 \\
   F_3 &= - \kcap \wedge F_2,
\end{aligned}
\end{equation}
so
\begin{equation}\label{eqn:planewavesMultivector:160}
\begin{aligned}
F(\Bk)
&=
\lr{ 1 + \kcap } \lr{
   F_1 - \kcap \cdot F_1 +
   F_2 - \kcap \wedge F_2
} \\
&=
\lr{ 1 + \kcap } \lr{
   F_1 - \kcap F_1 + \kcap \wedge F_1 +
   F_2 - \kcap F_2 + \kcap \cdot F_2
}.
\end{aligned}
\end{equation}
The multivector \( 1 + \kcap \) has the projective property of gobbling any leading factors of \( \kcap \)
\begin{equation}\label{eqn:planewavesMultivector:180}
\begin{aligned}
(1 + \kcap)\kcap
&= \kcap + 1 \\
&= 1 + \kcap,
\end{aligned}
\end{equation}
so for \( F_i \in F_1, F_2 \)
\begin{equation}\label{eqn:planewavesMultivector:200}
(1 + \kcap) ( F_i - \kcap F_i )
=
(1 + \kcap) ( F_i - F_i )
= 0,
\end{equation}
leaving
\begin{equation}\label{eqn:planewavesMultivector:220}
F(\Bk)
=
\lr{ 1 + \kcap } \lr{
   \kcap \cdot F_2 +
   \kcap \wedge F_1
}.
\end{equation}

For \( \kcap \cdot F_2 \) to be non-zero \( F_2 \) must be a bivector that lies in a plane containing \( \kcap \), and
\( \kcap \cdot F_2 \) is a vector in that plane that is perpendicular to \( \kcap \).
On the other hand \( \kcap \wedge F_1 \) is non-zero only if \( F_1 \) has a non-zero component that does not lie in along the \( \kcap \) direction, but \( \kcap \wedge F_1 \), like \( F_2 \) describes a plane that containing \( \kcap \).
This means that having both bivector and vector free variables \( F_2 \) and \( F_1 \) provide more degrees of freedom than required.
For example, if \( \BE \) is any vector, and \( F_2 = \kcap \wedge \BE \), then
\begin{equation}\label{eqn:planewavesMultivector:240}
\begin{aligned}
\lr{ 1 + \kcap }
   \kcap \cdot F_2
&=
\lr{ 1 + \kcap }
   \kcap \cdot \lr{ \kcap \wedge \BE } \\
&=
\lr{ 1 + \kcap }
\lr{
   \BE
-
\kcap \lr{ \kcap \cdot \BE }
} \\
&=
\lr{ 1 + \kcap }
\kcap \lr{ \kcap \wedge \BE } \\
&=
\lr{ 1 + \kcap }
\kcap \wedge \BE,
\end{aligned}
\end{equation}
which has the form \( \lr{ 1 + \kcap } \lr{ \kcap \wedge F_1 } \), so the electromagnetic field strength phasor may be generally written
\begin{equation}\label{eqn:planewavesMultivector:280}
F(\Bk)
=
\lr{ 1 + \kcap }
\kcap \wedge \BE \, e^{-j \Bk \cdot \Bx}
,
\end{equation}
%\end{boxed}
Expanding the multivector factor \( \lr{ 1 + \kcap } \kcap \wedge \BE \) we find
\begin{equation}\label{eqn:planewavesMultivector:720}
\begin{aligned}
\lr{ 1 + \kcap }
\kcap \wedge \BE
&=
\kcap \cdot \lr{ \kcap \wedge \BE }
+\cancel{\kcap \wedge \lr{ \kcap \wedge \BE }}
+
\kcap \wedge \BE \\
&=
\BE - \kcap \lr{ \kcap \wedge \BE }
+
\kcap \wedge \BE.
\end{aligned}
\end{equation}
The vector grade has the component of \( \BE \) along the propagation direction removed (i.e. it is the rejection), so there is no loss of generality should a
\( \BE \cdot \Bk = 0 \) constraint be imposed.  Such as constraint let's us write the bivector as a vector product \( \kcap \wedge \BE = \kcap \BE \), and then use the projective property \cref{eqn:planewavesMultivector:180} to gobble the leading \( \kcap \) factor, leaving
\begin{equation}\label{eqn:planewavesMultivector:700}
F(\Bk)
=
\lr{ 1 + \kcap }
\BE \, e^{-j \Bk \cdot \Bx}
=
\lr{ \BE + I \kcap \cross \BE }
\, e^{-j \Bk \cdot \Bx}.
\end{equation}

It is also noteworthy that
the directions \( \kcap, \Ecap, \Hcap \) form a right handed triple, which can be seen by computing their product
\begin{equation}\label{eqn:planewavesMultivector:740}
\begin{aligned}
(\kcap \Ecap) \Hcap
&= (-\Ecap \kcap) (-I \kcap \Ecap) \\
&= +I \Ecap^2 \kcap^2 \\
&= I.
\end{aligned}
\end{equation}
These vectors must all be mutually orthonormal for their product to be a pseudoscalar multiple.
Should there be doubt, explicit dot products may be computed with ease using grade selection operations
\begin{equation}\label{eqn:planewavesMultivector:760}
\begin{aligned}
\kcap \cdot \Hcap &= \gpgradezero{ \kcap (-I \kcap \Ecap) } = -\gpgradezero{ I \Ecap } = 0 \\
\Ecap \cdot \Hcap &= \gpgradezero{ \Ecap (-I \kcap \Ecap) } = -\gpgradezero{ I \kcap } = 0,
\end{aligned}
\end{equation}
where the zeros follow by noting that \( I \Ecap, I \kcap \) are both bivectors.
The conventional representation of the right handed triple relationship between the propagation direction and fields is stated as a cross product,
not as a
pseudoscalar relationship as in
\cref{eqn:planewavesMultivector:740}. These are easily seen to be equivalent
\begin{equation}\label{eqn:planewavesMultivector:340}
\begin{aligned}
\kcap
&= I \Hcap \Ecap \\
&= I (\Hcap \wedge \Ecap) \\
&= I^2 (\Hcap \cross \Ecap) \\
&= \Ecap \cross \Hcap.
\end{aligned}
\end{equation}
\end{proof}
%}
%\EndNoBibArticle

%%
% Copyright © 2017 Peeter Joot.  All Rights Reserved.
% Licenced as described in the file LICENSE under the root directory of this GIT repository.
%
\subsection{Polarization.}
\index{plane wave}
\index{polarization}
In a discussion of polarization, it is convenient to align the propagation direction along a fixed direction, usually the z-axis.
Setting \( \kcap = \Be_3, \beta z = \Bk \cdot \Bx \) in \cref{eqn:frequencydomainCore:200} the plane wave representation of the field is

\begin{dmath}\label{eqn:polarization:20}
\begin{aligned}
F(\Bx, \omega) &= (1 + \Be_3) \BE e^{-j \beta z} \\
F(\Bx, t) &= \Real\lr{ F(\Bx, \omega) e^{j \omega t} }.
\end{aligned}
\end{dmath}

Here the imaginary \( j \) has no intrinsic geometrical interpretation, \( \BE = \BE_\txtr + j \BE_\txti \) is allowed to have complex values, and all components of \( \BE \) is perpendicular to the propagation direction (\( \Be_\txtr \cdot \Be_3 = \BE_\txti \cdot \Be_3 = 0 \)).
\index{Jones vector}
A common representation of the electric field components is the Jones vector \( (c_1, c_2) \), which specifies complex coefficients for the electric field phasor in each of the possible directions

\begin{dmath}\label{eqn:polarization:120}
\BE = c_1 \Be_1 + c_2 \Be_2,
\end{dmath}

where \( c_1, c_2 \) are complex valued, say

\begin{dmath}\label{eqn:polarization:140}
\begin{aligned}
c_1 &= \alpha_1 + j \beta_1 \\
c_2 &= \alpha_2 + j \beta_2.
\end{aligned}
\end{dmath}

The tuple \( (c_1, c_2) \) is called the Jones vector, and compactly encodes the geometry of the pattern that the electric field traces out in the transverse plane.

\subsection{Circular polarization basis.}
\index{circular polarization}
\index{left circular polarization}
\index{right circular polarization}

The time domain field when written out explicitly in terms of the Jones vector components is

\begin{dmath}\label{eqn:polarization:160}
F(\Bx, t) = (1 + \Be_3) \lr{
\lr{ \alpha_1 \Be_1 + \alpha_2 \Be_2 } \cos\lr{ \omega t - \beta z }
-\lr{ \beta_1 \Be_1 + \beta_2 \Be_2 } \sin\lr{ \omega t - \beta z }
}.
\end{dmath}

Linear, circular, and elliptical polarization patterns can be obtained using specific values for \( \alpha_1, \alpha_2, \beta_1, \beta_2 \).
In particular,
a field for which the
change in phase \( \phi = \omega t - \beta z \) results in the electric field tracing out a (counterclockwise,clockwise) circle

\begin{dmath}\label{eqn:polarization:180}
\begin{aligned}
\BE_\txtR &= \Abs{\BE} \lr{ \Be_1 \cos\phi + \Be_2 \sin\phi } = \Abs{\BE} \Be_1 \exp\lr{  \Be_{12} \phi } \\
\BE_\txtL &= \Abs{\BE} \lr{ \Be_1 \cos\phi - \Be_2 \sin\phi } = \Abs{\BE} \Be_1 \exp\lr{ -\Be_{12} \phi },
\end{aligned}
\end{dmath}

is referred to as having
(left,right) circular polarization.
There are different conventions for the polarization orientation, and here the IEEE antenna convention discussed in \citep{balanis1989advanced} are used.

Fixme: check that I have this orientation right.  Noticed after the fact that the figures in Balanis use an orientation with x-axis up and y-axis right!

The bivector exponential representation of the circularly polarized electric fields in \cref{eqn:polarization:180} indicates that it is possible to represent arbitrary field polarization in a GA form that does not require any real part operation, as follows

\begin{dmath}\label{eqn:polarization:200}
F = \lr{ 1 + \Be_3 } \Be_1 \lr{ \alpha_\txtR e^{i\phi} + \alpha_\txtL e^{-i\phi} },
\end{dmath}

where the constants \( \alpha_\txtR, \alpha_\txtL \) are both complex with respect to the unit bivector imaginary \( i = \Be_{12} \) representing the plane transverse to the propagation direction

\begin{dmath}\label{eqn:polarization:220}
\begin{aligned}
\alpha_\txtR &= \alpha_{\txtR 1} + i \alpha_{\txtR 2} \\
\alpha_\txtL &= \alpha_{\txtL 1} + i \alpha_{\txtL 2}.
\end{aligned}
\end{dmath}

If a transformation from scalar to bivector imaginary \( j \rightarrow \Be_{12} = i \) is made in the Jones vector component representation of \cref{eqn:polarization:140},
then
the coefficients \cref{eqn:polarization:220} of the circular polarization states are related to the Jones vector by (\cref{problem:polarization:1})

\begin{dmath}\label{eqn:polarization:260}
\begin{aligned}
\alpha_\txtR &= \inv{2}\lr{ c_1 - i c_2 } \\
\alpha_\txtL &= \inv{2}\lr{ c_1 + i c_2 }^\dagger.
\end{aligned}
\end{dmath}

\subsection{Linear polarization.}

Linear polarization is described by

\begin{dmath}\label{eqn:polarization:280}
\begin{aligned}
\alpha_\txtR &= \inv{2}\Abs{\BE} \Be_1 e^{i(\psi + \theta)} \\
\alpha_\txtL &= \inv{2}\Abs{\BE} \Be_1 e^{i(\psi - \theta)},
\end{aligned}
\end{dmath}

or
\begin{dmath}\label{eqn:polarization:300}
F = \lr{ 1 + \Be_3 } \Abs{\BE} \Be_1 e^{i\psi} \cos( \omega t - \beta z + \theta ).
\end{dmath}

The electric field \( \BE \) traces out all the points along the line spanning the points between \( \pm \Be_1 e^{i\psi} \Abs{\BE} \), whereas the magnetic field \( \BH \) traces
out all the points along \( \pm \Be_2 e^{i\psi} \Abs{\BE}/\eta \) as illustrated (with \( \eta = 1 \)) in
\cref{fig:linearPolarization:linearPolarizationFig1}.

\imageFigure{../figures/GAelectrodynamics/linearPolarizationFig1}{Linear polarization.}{fig:linearPolarization:linearPolarizationFig1}{0.3}

\subsection{Elliptical parameterization.}

An ellipical polarized electric field can be parameterized as
\begin{dmath}\label{eqn:ellipticalWaves:340}
\BE
=
E_a \Be_1 \cos\theta + E_b \Be_2 \sin\theta,
\end{dmath}

which corresponds to a Jones vector \( (E_a, -i E_b) \), or circular polarization coefficients with values

\begin{dmath}\label{eqn:polarization:400}
\begin{aligned}
\alpha_\txtR &= \inv{2}\lr{ E_a - E_b } \\
\alpha_\txtL &= \inv{2}\lr{ E_a + E_b }.
\end{aligned}
\end{dmath}

Therefore an elliptically polarized field can be represented as

\begin{dmath}\label{eqn:polarization:420}
F = \inv{2} (1 + \Be_3) \Be_1 \lr{ (E_a + E_b) e^{i\phi} + (E_a - E_b) e^{-i\phi} }.
\end{dmath}

An interesting variation of the elliptical polarization uses a hyperbolic parameterization.
If \( a, b \) are the semi-major/minor axes of the ellipse (i.e. \( a > b \)),
and \( \Ba = a \Be_1 e^{i\psi} \) is the vectoral representation of the semimajor axis (not necessarily placed along \( \Be_1 \)),
and \( e = \sqrt{1 - (b/a)^2} \) is the eccentricity of the ellipse,
then it can be shown (\citep{hestenes1999nfc})
that an elliptic parameterization can be written
in the compact form

\begin{dmath}\label{eqn:ellipticalWaves:360}
\Br(\phi)
=
e \Ba \cosh( \Atanh(b/a) + i \phi).
\end{dmath}

Using the bivector
This is also real and has only vector grades, so the electromagnetic field for a general elliptic wave has the form

\begin{dmath}\label{eqn:ellipticalWaves:380}
\begin{aligned}
F &= e E_a \lr{ 1 + \Be_3 } \Be_1 e^{ i \psi } \cosh\lr{ \mu + i \phi} \\
\mu &= \Atanh\lr{ E_b/E_a } \\
e &= \sqrt{1 - {(E_b/E_a)}^2 },
\end{aligned}
\end{dmath}

where \( E_a(E_b) \) are the magnitudes of the electric field components lying along the semi-major(minor) axes, and the propagation direction \( \Be_3 \) is normal to both the major and minor axis directions, as illustrated in \cref{fig:ellipticalPolarization:ellipticalPolarizationFig1}.
Observe that setting \( E_b = 0 \) results in the linearly polarized field of \cref{eqn:polarization:300}.
\imageFigure{../figures/GAelectrodynamics/ellipticalPolarizationFig1}{Electric field with elliptical polarization.}{fig:ellipticalPolarization:ellipticalPolarizationFig1}{0.3}

\subsection{Pseudoscalar imaginary.}

...

\subsection{Problems.}

\makeproblem{Circular polarization coefficients relationship to the Jones vector.}{problem:polarization:1}{
By substituting \cref{eqn:polarization:220} into \cref{eqn:polarization:200}, and comparing to \cref{eqn:polarization:160},
show that the circular state coefficients have the following relationship to the Jones vector coordinates
\begin{equation*}
\begin{aligned}
\alpha_\txtR &= \lr{ \alpha_1 + \beta_2 }/2 + i \lr{ -\alpha_2 + \beta_1 }/2 \\
\alpha_\txtL &= \lr{ \alpha_1 - \beta_2 }/2 + i \lr{ -\alpha_2 - \beta_1 }/2,
\end{aligned}
\end{equation*}
and use this to prove \cref{eqn:polarization:260}.
} % problem

%%
% Copyright © 2017 Peeter Joot.  All Rights Reserved.
% Licenced as described in the file LICENSE under the root directory of this GIT repository.
%
%{

\subsection{Example.  Field of a ring of charge or current density.}

Let's now compute the field due to a static charge or current density on a ring of radius \( R \) as illustrated in
\cref{fig:chargeAndCurrentOnRing:chargeAndCurrentOnRingFig1}.

\imageFigure{../figures/GAelectrodynamics/chargeAndCurrentOnRingFig1}{Field due to a circular distribution.}{fig:chargeAndCurrentOnRing:chargeAndCurrentOnRingFig1}{0.3}

A static charge distribution on a ring at \( z = 0 \) has the form

\begin{dmath}\label{eqn:ringField:20}
\rho(\Bx) = \lambda \delta(z) \delta(r - R).
\end{dmath}

As always the current distribution is of the form \( \BJ = \Bv \rho \), and in this case the velocity is azimuthal \( \Bv = \Be_2 e^{i\phi}, i = \Be_{12} \).
The total multivector current is

\begin{dmath}\label{eqn:ringField:40}
J = \inv{\epsilon_0} \lambda \delta(z) \delta(r - R) \lr{ 1 - \frac{\Bv}{c} }.
\end{dmath}

Let the point that we observe the field, and the integration variables be

\begin{dmath}\label{eqn:ringField:60}
\begin{aligned}
\Bx &= z \Be_3 + r \rhocap \\
\Bx' &= z' \Be_3 + r' \rhocap'.
\end{aligned}
\end{dmath}

The field is

\begin{dmath}\label{eqn:ringField:80}
F(\Bx)
= \frac{\lambda}{4 \pi \epsilon_0} \iiint dz' r' dr' d\phi' \delta(z') \delta(r' - R) \frac{\gpgrade{ \lr{ (z - z') \Be_3 + r \rhocap - r' \rhocap' } \lr{ 1 - \frac{v}{c} \Be_2 e^{i\phi'} } }{1,2} } { \lr{ (z-z')^2 + (r \rhocap - r' \rhocap')^2}^{3/2} }
= \frac{\lambda}{4 \pi \epsilon_0} \int R d\phi' \frac{\gpgrade{ \lr{ z \Be_3 + r \rhocap - R \rhocap' } \lr{ 1 - \frac{v}{c} \Be_2 e^{i\phi'} } }{1,2} } { \lr{ z^2 + (r \rhocap - R \rhocap')^2}^{3/2} }
\end{dmath}

Without loss of generality, we can align the axes so that \( \rhocap = \Be_1 \), and
introduce dimensionless variables

\begin{dmath}\label{eqn:ringField:100}
\begin{aligned}
\alpha &= z/R \\
\beta &= r/R,
\end{aligned}
\end{dmath}

which gives
\begin{dmath}\label{eqn:ringField:120}
F
= \frac{\lambda}{4 \pi \epsilon_0 R} \int_0^{2\pi} d\phi' \frac{\gpgrade{ \lr{ \alpha \Be_3 + \beta \Be_1 - \Be_1 e^{i\phi'} } \lr{ 1 - \frac{v}{c} \Be_2 e^{i\phi'} } }{1,2} } { \lr{ \alpha^2 + (\beta \Be_1 - \Be_1 e^{i\phi'})^2}^{3/2} }.
\end{dmath}

In the denominator, the vector square expands as
\begin{dmath}\label{eqn:ringField:140}
(\beta \Be_1 - \Be_1 e^{i\phi'})^2
=
(\beta - e^{-i\phi'}) \Be_1^2 (\beta - e^{i\phi'})
=
\beta^2 + 1 - 2 \beta \cos(\phi'),
\end{dmath}

and the grade selection in the numerator is

\begin{dmath}\label{eqn:ringField:160}
\gpgrade{ \lr{ \alpha \Be_3 + \beta \Be_1 - \Be_1 e^{i\phi'} } \lr{ 1 - \frac{v}{c} \Be_2 e^{i\phi'}}}{1,2}
=
\alpha \Be_3 + \beta \Be_1 - \Be_1 e^{i\phi'}
-\frac{v}{c}\lr{ \alpha \Be_{31} e^{i\phi'} + \beta i \cos(\phi') + i }.
\end{dmath}

Any of the exponential integrals terms
are of the form

\begin{dmath}\label{eqn:ringField:180}
\int_0^{2\pi} d\phi' e^{i\phi'} f(\cos(\phi')) = \int_0^{2\pi} d\phi' \cos(\phi') f(\cos(\phi')),
\end{dmath}

since
the \( i \sin\phi' f(\cos(\phi') \) contributions are odd functions around \( \phi' = \pi \).

In general, for all \( z, r \) we can't symbolically evaluate the integrals.
Let
\begin{subequations}
\label{eqn:ringField:260}
\begin{equation}\label{eqn:ringField:200}
A
%= A(\alpha, \beta)
= \int_0^{2\pi} d\phi' \frac{1}{\lr{ 1 + \alpha^2 + \beta^2 - 2 \beta \cos(\phi') }^{3/2}}
\end{equation}
\begin{equation}\label{eqn:ringField:280}
B
%= B(\alpha, \beta)
= \int_0^{2\pi} d\phi' \frac{\cos(\phi')}{\lr{ 1 + \alpha^2 + \beta^2 - 2 \beta \cos(\phi') }^{3/2}}.
\end{equation}
\end{subequations}

The solutions of \cref{eqn:ringField:260} both require elliptic integrals, but are also amenable to direct numeric evaluation.  Taking these as given, the field is

\begin{dmath}\label{eqn:ringField:220}
F
=
\frac{\lambda}{4 \pi \epsilon_0 R}
\lr{
\lr{ \alpha \Be_3 + \beta \Be_1 -\frac{v }{c}\Be_{12} } A
- \lr{
\Be_1 + \frac{v}{c} \lr{ \alpha \Be_{31} + \beta \Be_{12} } } B
}.
\end{dmath}

The field directions are nicely parameterized as multivector expresssions, with the relative weightings in different directions scaled by the position dependent integral coefficients of \cref{eqn:ringField:260}.
The multivector field can be separated into its respective electric and magnetic components by inspection

\begin{dmath}\label{eqn:ringField:240}
\begin{aligned}
\BE &=
\gpgradeone{F}
=
\frac{\lambda}{4 \pi R \epsilon_0} \lr{ \alpha A \Be_3 + \Be_1( \beta A - B) } \\
\BH &=
\inv{\eta_0} \gpgradeone{-I F}
=
\frac{\lambda v}{4 \pi R } \lr{ -\Be_3 \lr{ A + \beta B } - \Be_2 \alpha },
\end{aligned}
\end{dmath}

which shows that the static charge distribution \( \rho \propto \lambda \) only contributes to the electric field, and the static current distribution \( \BJ \propto v \lambda \) only contributes to the magnetic field.

Finally, to restore generality,
should we wish to compute the field or its electric or magnetic components at an arbitrary observation azimuthal angle \( \phi \), we need only transform \( \Be_1 \rightarrow \Be_1 e^{i \phi} = \rhocap, \Be_2 \rightarrow \Be_2 e^{i\phi} = \phicap \).

\makeproblem{Magnetic sources on a ring.}{problem:ringField:300}{
Given a constant (magnitude) multivector current on a ring \( J = I \lambda_m \delta(z) \delta(r - R) ( c - v \Be_2 e^{i\phi}), i = \Be_{12} \), show that the field is
\begin{equation*}
F = \frac{\lambda_m c}{4 \pi R} \lr{
\lr{ \alpha \Be_{12} + \beta \Be_{23} + \frac{v}{c} \Be_3 } A
+
\lr{ \Be_{32} + \frac{v}{c} \lr{ \alpha \Be_1 - \beta \Be_3 } } B
}.
\end{equation*}
} % problem

%}

%%
% Copyright © 2017 Peeter Joot.  All Rights Reserved.
% Licenced as described in the file LICENSE under the root directory of this GIT repository.
%
%{
\subsection{Inverting the Maxwell statics equation.}

Similar to electrostatics and magnetostatics, we can restrict attention to time invariant fields (\( \partial_t F = 0\)) and time invariant sources (\(\partial_t J = 0\)), but consider both electric and magnetic sources.  In that case Maxwell's equation is reduced to a first order gradient equation

\begin{dmath}\label{eqn:statics:20}
\spacegrad F(\Bx) = J(\Bx)
\end{dmath}

Using the Green's function for the (first order) gradient \cref{eqn:electrostatics:260}, this can be inverted as

\begin{dmath}\label{eqn:statics:40}
F(\Bx)
= \int_V dV' G(\Bx, \Bx') \spacegrad' J(\Bx')
= \gpgrade{\int_V dV' G(\Bx, \Bx') \spacegrad' J(\Bx')}{1,2}
= \inv{4\pi} \int_V dV' \gpgrade{\frac{(\Bx - \Bx') J(\Bx')}{\Norm{\Bx - \Bx'}^3} }{1,2}.
\end{dmath}

Like we did in magnetostatics, a no-op grade selection has been inserted to simplify subsequent manipulation
\footnote{If this grade selection filter is omitted, it is possible to show that the scalar and pseudoscalar contributions to the \( (\Bx -\Bx') J \) product are zero on the boundary of the Green's integration volume. \citep{jancewicz1988multivectors:appendixI}},
yielding a compact solution to the Maxwell statics equation

\boxedEquation{eqn:statics:80}{
F(\Bx)
= \inv{4\pi} \int_V dV' \frac{\gpgrade{(\Bx - \Bx') J(\Bx')}{1,2}}{\Norm{\Bx - \Bx'}^3} + F_0,
}

where \( F_0 \) is any function for which \( \spacegrad F_0 = 0 \).

It will be preferable and simpler to work with combined fields and densities, however, some insight can be gained by
explicit expansion of \cref{eqn:statics:80} in terms of charge and current densities.
The grade selection, in terms of terms of \( \Bs = \Bx -\Bx' \) expands as

\begin{dmath}\label{eqn:statics:60}
\gpgrade{\Bs J}{1,2}
=
\eta \gpgrade{\Bs (v \rho - \BJ)}{1,2}
+
\gpgrade{\Bs I(v \rho_m - \BM)}{1,2}
=
\inv{\epsilon} \Bs \rho + \eta \BJ \wedge \Bs + \Bs v \rho_m I + \Bs \cross \BM,
\end{dmath}

so the solution \( F = \BE + \eta I \BH \) to the Maxwell statics equation in terms of the charge and current densities, and electric and magnetic fields is

\begin{dmath}\label{eqn:statics:100}
\begin{aligned}
\BE(\Bx)
&=
\inv{4\pi} \int_V dV' \inv{\Norm{\Bx - \Bx'}^3}
\lr{
   \inv{\epsilon}(\Bx - \Bx') \rho(\Bx')
   +
   (\Bx - \Bx') \cross \BM(\Bx')
} \\
\BH(\Bx)
&=
\inv{4\pi} \int_V dV' \inv{\Norm{\Bx - \Bx'}^3}
\lr{
  \BJ(\Bx') \cross (\Bx - \Bx')
+ \inv{\mu} (\Bx - \Bx') \rho_m(\Bx')
}.
\end{aligned}
\end{dmath}

We see that the combined solution \cref{eqn:statics:80} incorporates both a Coulomb's law contribution and a Biot-Savart law contribution.
When desired, any fictitious magnetic sources, also contribute to the electric and magnetic fields.

\makeproblem{}{problem:statics:81}{
Fill in any steps left out of the derivations of \cref{eqn:statics:60} and \cref{eqn:statics:100}.
} % problem

\subsection{Example.  Infinite line charge and current.}

Given a static line charge density and current density along the z-axis in free space

\begin{dmath}\label{eqn:statics:120}
\begin{aligned}
\rho(\Bx) &= \lambda \delta(x) \delta(y) \\
\BJ(\Bx) &= \Bv \rho(\Bx) = v \lambda \Be_3 \delta(x) \delta(y),
\end{aligned}
\end{dmath}

the total multivector current is
\begin{dmath}\label{eqn:statics:140}
J
= \eta_0 ( c \rho - \BJ )
= \eta_0 ( c - v \Be_3 ) \lambda \delta(x) \delta(y)
= \frac{\lambda}{\epsilon_0} \lr{ 1 - \frac{v}{c} \Be_3 } \delta(x) \delta(y)
\end{dmath}

We can find the field for this current by substitution into \cref{eqn:statics:80}.
To do so, let the field observation point be \( \Bx = \Bx_\perp + z \Be_3 \), so the total field is
\begin{dmath}\label{eqn:statics:160}
F(\Bx)
= \frac{\lambda}{4\pi \epsilon_0} \int_V dx'dy'dz' \frac{\gpgrade{(\Bx - \Bx') (1 - (v/c) \Be_3 )}{1,2}}{\Norm{\Bx - \Bx'}^3} \delta(x') \delta(y')
= \frac{\lambda}{4\pi \epsilon_0} \int_{-\infty}^\infty dz' \frac{\gpgrade{(\Bx_\perp + (z - z') \Be_3) (1 - (v/c) \Be_3 )}{1,2}}{\lr{\Bx_\perp^2 + (z-z')^2}^{3/2}}
=
\frac{\lambda \lr{ \Bx_\perp - (v/c) \Bx_\perp \Be_3}}{4\pi \epsilon_0} \int_{-\infty}^\infty \frac{dz'}{\lr{\Bx_\perp^2 + (z-z')^2}^{3/2}}
+
\frac{\lambda \Be_3}{4\pi \epsilon_0} \int_{-\infty}^\infty \frac{(z - z') dz'}{\lr{\Bx_\perp^2 + (z-z')^2}^{3/2}}.
\end{dmath}

The first integral is \( 2/\Bx_\perp^2 \), whereas the second is zero (odd function, over even interval).
Using cylindrical coordinates \( \Bx = R \rhocap + z \Be_3 \), and since
\( \Bx_\perp \cdot \Be_3 = 0 \), the wedge can be replaced with the vector product \( \Bx_\perp \Be_3 = R \rhocap \Be_3 \), leaving

\begin{equation}\label{eqn:statics:180}
F(\Bx)
=
\frac{\lambda}{2\pi \epsilon_0 R} \rhocap \lr{ 1 - \Bv/c} = \BE \lr{ 1 - \Bv/c }
= \BE + I \lr{ \frac{\Bv}{c} \cross \BE },
\end{equation}

where \( \Bv = v \Be_3 \).
The vector component of this is the electric field, which is therefore directed radially, whereas the (dual) magnetic field \( \eta_0 I \BH \)
is a set of oriented planes spanning the radial and z-axis directions.
We can also see that there is a constant proportionality factor that relates the electric and magnetic field components, namely

\begin{dmath}\label{eqn:statics:200}
I \eta_0 \BH = -\BE \Bv/c,
\end{dmath}

or

\begin{dmath}\label{eqn:statics:220}
\BH = \Bv \cross \BD.
\end{dmath}

\makeproblem{Statics solution for linear magnetic density and currents.}{problem:statics:240}{
Given magnetic charge density \( \rho_m = \lambda_m \delta(x) \delta(y) \), and current density \( \BM = v \Be_3 \rho_m = \Bv \rho_m \), show that the field is given by
\begin{equation*}
F(\Bx) = \frac{\lambda_m c}{4 \pi R} I \rhocap \lr{ 1 - \frac{\Bv}{c} },
\end{equation*}
or with \( \BB = \lambda_m \rhocap/(4 \pi R) \),
\begin{equation*}
F = \BB \cross \Bv + c I \BB.
\end{equation*}
} % problem

\subsection{Example.  Infinite planar charge and current.}

A variation on the above example puts a uniform charge density \( \rho(\Bx) = \sigma \delta(z) \) in a plane, along with an associated current density \( \BJ(\Bx) = v \Be_1 e^{i\theta} \rho(\Bx), \quad i = \Be_{12} \).  Letting \( \Bv = v \Be_1 e^{i\theta} \), the multivector current for a free space field is

\begin{dmath}\label{eqn:statics:240}
J(\Bx) = \sigma \eta_0 \lr{ c - \Bv } \delta(z),
\end{dmath}

so the field off the plane is
\begin{dmath}\label{eqn:statics:260}
F(\Bx)
=
\frac{\sigma}{4 \pi \epsilon_0 } \iiint \frac{dz' dA'}{ \Norm{ \Bx - \Bx' }^3 }
\gpgrade{ (\Bx - \Bx') (1 - \Bv/c) }{1,2} \delta(z').
\end{dmath}

If \( \Bx_\parallel = (\Bx \wedge \Be_3) \Be_3 \), and \( \Bx'_\parallel = (\Bx' \wedge \Be_3) \Be_3 \), are the components of the vectors \( \Bx, \Bx' \) in the x-y plane, then integration over \( z' \) and a change of variables \( \Bx'_\parallel - \Bx_\parallel = r' \Be_1 e^{i \theta'} \) yields

\begin{dmath}\label{eqn:statics:280}
F(\Bx)
=
\frac{\sigma}{4 \pi \epsilon_0 } \iint \frac{r' dr' d\theta'}{ ( z^2 + {r'}^2 )^{3/2} }
\gpgrade{ \lr{ z \Be_3 - r' \Be_1 e^{i\theta'} } (1 - \Bv/c) }{1,2}.
\end{dmath}

The \( e^{i\theta'} \) integrands are killed, so for \( z \ne 0 \), the field is

\begin{dmath}\label{eqn:statics:300}
F(\Bx)
=
\frac{\sigma z}{4 \pi \epsilon_0 \Abs{z} } \gpgrade{ \Be_3 (1 - \Bv/c) }{1,2}.
\end{dmath}

Since \( \Bv \in \Span \setlr{ \Be_1, \Be_2 } \) the product \( \Be_3 \Bv \) is a bivector and the grade selection can be dropped, leaving

\begin{dmath}\label{eqn:statics:320}
F(\Bx)
=
\frac{\sigma \sgn(z)}{4 \pi \epsilon_0 } \Be_3 \lr{ 1 - \frac{\Bv}{c}}.
\end{dmath}

This field toggles sign when crossing the plane, but is constant otherwise.  The electric and magnetic field components are once again related by \cref{eqn:statics:220}.

\makeproblem{Statics solution for planar magnetic density and currents.}{problem:statics:241}{
Given magnetic charge density \( \rho_m = \sigma_m \delta(z) \), and current density \( \BM = \Bv \rho_m, \Bv = v \Be_1 e^{i\theta}, i = \Be_{12}\), show that the field is given by
\begin{equation*}
F(\Bx) = \frac{\sigma_m c \sgn(z)}{4 \pi } i \lr{ 1 - \frac{\Bv}{c} }.
\end{equation*}
} % problem

%}

%%
% Copyright � 2018 Peeter Joot.  All Rights Reserved.
% Licenced as described in the file LICENSE under the root directory of this GIT repository.
%
%{
%%%\input{../latex/blogpost.tex}
%%%\renewcommand{\basename}{surfaceintegral}
%%%%\renewcommand{\dirname}{notes/phy1520/}
%%%\renewcommand{\dirname}{notes/ece1228-electromagnetic-theory/}
%%%%\newcommand{\dateintitle}{}
%%%%\newcommand{\keywords}{}
%%%
%%%\input{../latex/peeter_prologue_print2.tex}
%%%
%%%\usepackage{peeters_layout_exercise}
%%%\usepackage{peeters_braket}
%%%\usepackage{peeters_figures}
%%%\usepackage{siunitx}
%%%%\usepackage{mhchem} % \ce{}
%%%%\usepackage{macros_bm} % \bcM
%%%%\usepackage{macros_qed} % \qedmarker
%%%\usepackage{txfonts} % \ointclockwise
%%%
%%%\beginArtNoToc
%%%
%%%\generatetitle{Multivector surface integral.}
%\section{Surface integral.}
%\label{chap:surfaceintegral}

%%As mentioned in a line integral context,
%%multivectors may not commute with the vector derivative or the differential, so we allow the vector derivative to act bidirectionally using the chain rule.
%%The scope of the action of the vector derivative when acting only to the left or right is indicated using braces above.
%%Should we wish to only integrate single functions, we can set either of the other to \( 1 \), yielding integrals of the form
%%\( \int_S F d^2\Bx \lboldpartial, \) or \( \int_S d^2\Bx \boldpartial G \).

The surface integral specialization of \cref{thm:fundamentalTheoremOfCalculus:1} is

%
% Copyright � 2018 Peeter Joot.  All Rights Reserved.
% Licenced as described in the file LICENSE under the root directory of this GIT repository.
%
\maketheorem{Fundamental theorem for surface integrals.}{thm:surfaceintegral:100}{
Given a
% piecewise-smooth
continuous and connected surface
 \( S = \Bx(u, v) \)
parameterized by parameter \( u \in [u_0, u_1], v \in [v_0, v_1] \), multivector functions \( F(\Bx), G(\Bx) \) that are differentable over \( S \), and
an (bivector-valued) area element \( d^2 \Bx = d\Bx_1 \wedge d\Bx_2 = du dv\, \Bx_u \wedge \Bx_v \)
\begin{equation*}
\int_S F d^2\Bx \lrboldpartial G
= \ointclockwise_{\partial S} F d\Bx G,
\end{equation*}
where \( \partial S \) is the boundary of the surface \( S \).
} % theorem

To prove \cref{thm:surfaceintegral:100} we start by expanding the multivector product \( d^2\Bx\, \boldpartial \) in curvilinear coordinates, where we discover
that this product has only a vector grade.
The vector derivative, the projection of the gradient onto the surface at the point of integration (also called the tangent space), now has two components
\begin{dmath}\label{eqn:surfaceintegral:200}
\boldpartial
=
\sum_i \Bx^i (\Bx_i \cdot \spacegrad)
=
\Bx^u \PD{u}{}
+
\Bx^v \PD{v}{}
\equiv
\Bx^u \partial_u
+
\Bx^v \partial_v.
\end{dmath}
To see why the product of the area elements and the vector derivative
\begin{dmath}\label{eqn:surfaceintegral:300}
d^2\Bx\, \boldpartial
=
du dv\, \lr{ \Bx_u \wedge \Bx_v } \lr{ \Bx^u \partial_u + \Bx^v \partial_v },
\end{dmath}
has only a vector grade, observe that \( \Bx^u \in \Span \setlr{ \Bx_u, \Bx_v } \), so
\begin{dmath}\label{eqn:surfaceintegral:320}
\lr{ \Bx_u \wedge \Bx_v } \Bx^u
=
\lr{ \Bx_u \wedge \Bx_v } \cdot \Bx^u
+
\cancel{ \lr{ \Bx_u \wedge \Bx_v } \wedge \Bx^u }
=
\lr{ \Bx_u \wedge \Bx_v } \cdot \Bx^u
=
\Bx_u \lr{ \Bx_v \cdot \Bx^u } -
\Bx_v \lr{ \Bx_u \cdot \Bx^u }
=
-\Bx_v.
\end{dmath}
Similarly
\begin{dmath}\label{eqn:surfaceintegral:340}
\lr{ \Bx_u \wedge \Bx_v } \Bx^v
=
\lr{ \Bx_u \wedge \Bx_v } \cdot \Bx^v
+
\cancel{ \lr{ \Bx_u \wedge \Bx_v } \wedge \Bx^v }
=
\lr{ \Bx_u \wedge \Bx_v } \cdot \Bx^v
=
\Bx_u \lr{ \Bx_v \cdot \Bx^v } -
\Bx_v \lr{ \Bx_u \cdot \Bx^v }
=
\Bx_u.
\end{dmath}
Not only does \cref{eqn:surfaceintegral:300} have only a vector grade, that product reduces to just
\begin{dmath}\label{eqn:surfaceintegral:360}
d^2\Bx\, \boldpartial
=
\Bx_u \partial_v
-\Bx_v \partial_u.
\end{dmath}
Inserting \cref{eqn:surfaceintegral:360} into the surface integral, we find
\begin{dmath}\label{eqn:surfaceintegral:380}
\int_S F d^2\Bx\, \boldpartial G
=
\int_S \lr{ F d^2\Bx\, \lboldpartial} G
+
\int_S F d^2\Bx \lr{ \rboldpartial G }
=
\int_S du dv\, \lr{ \partial_v F \Bx_u - \partial_u F \Bx_v } G
+
\int_S du dv\, F \lr{ \Bx_u \partial_v G - \Bx_v \partial_u G }
=
\int_S du dv\, \lr{ \PD{v}{F} \PD{u}{\Bx} - \PD{u}{F} \PD{v}{\Bx} } G
+
\int_S du dv\, F \lr{ \PD{u}{\Bx} \PD{v}{G} - \PD{v}{\Bx} \PD{u}{G} }
=
\int_S du dv\, \PD{v}{} \lr{ F \PD{u}{\Bx} G } - \int_S du dv\, \PD{u}{} \lr{ F \PD{v}{\Bx} G }
-
\int_S du dv\, F \lr{ \PD{v}{} \PD{u}{\Bx} - \PD{u}{} \PD{v}{\Bx} } G
=
\int_S du dv\, \PD{v}{} \lr{ F \PD{u}{\Bx} G } - \int_S du dv\, \PD{u}{} \lr{ F \PD{v}{\Bx} G }.
\end{dmath}
This leaves two perfect differentials, which can both be integrated separately
\begin{dmath}\label{eqn:surfaceintegral:400}
\int_S F d^2\Bx\, \boldpartial G
=
\int_{\Delta u} du\, \evalbar{\lr{ F \PD{u}{\Bx} G }}{\Delta v} - \int_{\Delta v} dv\, \evalbar{\lr{ F \PD{v}{\Bx} G }}{\Delta u}
=
\int_{\Delta u} \evalbar{\lr{ F d\Bx_u G }}{\Delta v} - \int_{\Delta v} \evalbar{\lr{ F d\Bx_v G }}{\Delta u}.
\end{dmath}
\Cref{eqn:surfaceintegral:400} is an explicit algebraic expression of the boundary integral of \cref{thm:surfaceintegral:100}.
To complete the proof, we are left with the task of geometrically interpretting this integrand.
Suppose we are integrating over the unit parameter volume space \( [u, v] \in [0,1] \otimes [0,1] \) as illustrated in
\cref{fig:twoParameterDifferentialBoundary:twoParameterDifferentialBoundaryFig2}.
\imageFigure
%\imageTwoFigures
{../figures/GAelectrodynamics/twoParameterDifferentialBoundaryFig2}
%{../figures/GAelectrodynamics/twoParameterDifferentialBoundaryEditedFig3}
%{Contour for two parameter surface boundary, and subdivision into finer loop mesh.}
{Contour for two parameter surface boundary.}
%{Contour for two parameter surface boundary, and subdivision into finer loop mesh.}
{fig:twoParameterDifferentialBoundary:twoParameterDifferentialBoundaryFig2}
{0.4}
%{scale=0.4}
Comparing to the figure we see that we've ended up with a clockwise line integral around the boundary of the surface.
For a given subset of the surface, the bivector area element can be chosen small enough that it lies in the tangent space
to the surface at the point of integration.
In that case, a larger bounding loop can be conceptualized as the sum of a number of smaller ones, as sketched
in \cref{fig:loopIntegralInfinitesimalSum:loopIntegralInfinitesimalSumFig2},
in which case the
contributions of the interior loop paths (red and blue) cancel out, leaving only the exterior loop contributions (green.)  When that subdivision is made small enough (assuming that the surface is continuous and differentiable along each of the parameterization paths) then each area element approximates the tangent space at the point of evaluation.

%\imageFigure{../figures/gabook/loopIntegralInfinitesimalSumFig2}{Sum of infinitesimal loops.}{fig:loopIntegralInfinitesimalSum:loopIntegralInfinitesimalSumFig2}{0.35}
\imageFigure{../figures/GAelectrodynamics/twoParameterDifferentialBoundaryEditedFig3}{Sum of infinitesimal loops.}{fig:loopIntegralInfinitesimalSum:loopIntegralInfinitesimalSumFig2}{0.35}

\subsubsection{Two parameter Stokes' theorem.}

Two 
special cases of \cref{thm:surfaceintegral:100}, both variations of Stokes' theorem, result by considering scalar and vector functions.  For the scalar case we have
%
% Copyright � 2018 Peeter Joot.  All Rights Reserved.
% Licenced as described in the file LICENSE under the root directory of this GIT repository.
%
\maketheorem{Surface integral of scalar function (Stokes').}{thm:surfaceintegral:420}{
Given a scalar function \( f(\Bx) \) its surface integrals is given by
\begin{equation*}
\int_S d^2 \Bx \cdot \boldpartial f =
\int_S d^2 \Bx \cdot \spacegrad f = \ointclockwise_{\partial S} d\Bx f.
\end{equation*}
In \R{3} this can be written as
\begin{equation*}
\int_S dA \ncap \cross \spacegrad f = \ointctrclockwise_{\partial S} d\Bx f,
\end{equation*}
where \( \ncap \) is the outwards normal specified by \( d^2 \Bx = I \ncap dA \).
} % theorem


To show the first part, we can split the (multivector) surface integral into vector and trivector grades
\begin{dmath}\label{eqn:surfaceintegral:440}
\int_S d^2\Bx\, \boldpartial f
=
\int_S d^2\Bx \cdot \boldpartial f
+
\int_S d^2\Bx \wedge \boldpartial f.
\end{dmath}

Since \( \Bx^u, \Bx^v \) both lie in the span of \( \setlr{ \Bx_u, \Bx_v } \),
\( d^2\Bx \wedge \boldpartial = 0 \), killing the second integral in \cref{eqn:surfaceintegral:440}.
If the gradient is decomposed into its projection along the tangent
space (the vector derivative) and its perpendicular components, only the vector derivative components of the
gradient contribute to its dot product with the area element.  That is
\begin{dmath}\label{eqn:surfaceintegral:460}
d^2 \Bx \cdot \spacegrad
=
d^2 \Bx \cdot \lr{ \Bx^u \partial_u + \Bx^v \partial_v + \cdots }
=
d^2 \Bx \cdot \lr{ \Bx^u \partial_u + \Bx^v \partial_v }
=
d^2 \Bx \cdot \boldpartial.
\end{dmath}

This means that for a scalar function
\begin{dmath}\label{eqn:surfaceintegral:480}
\int_S d^2\Bx\, \boldpartial f
=
\int_S d^2\Bx \cdot \spacegrad f.
\end{dmath}

The second part of the theorem follows by grade selection, and application of a duality transformation for the area element
\begin{dmath}\label{eqn:surfaceintegral:500}
d^2 \Bx \cdot \spacegrad f
=
\gpgradeone{ d^2 \Bx \spacegrad f }
=
dA\, \gpgradeone{ I \ncap \spacegrad f }
=
dA\, \gpgradeone{ I \lr{ \ncap \cdot \spacegrad f + I \ncap \cross \spacegrad f} }
=
-dA\, \ncap \cross \spacegrad f.
\end{dmath}

back substitution of \cref{eqn:surfaceintegral:500} completes the proof of \cref{thm:surfaceintegral:420}.

For vector functions we have

%
% Copyright � 2018 Peeter Joot.  All Rights Reserved.
% Licenced as described in the file LICENSE under the root directory of this GIT repository.
%
\maketheorem{Surface integral of a vector function (Stokes').}{thm:surfaceintegral:500}{
Given a vector function \( \Bf(\Bx) \) the \textit{surface integral} is given by
\begin{equation*}
\int_S d^2 \Bx \cdot (\spacegrad \wedge \Bf) = \ointclockwise_{\partial S} d\Bx \cdot \Bf.
\end{equation*}
In \R{3} this can be written as
\begin{equation*}
\int_S dA \ncap \cdot \lr{ \spacegrad \cross \Bf} = \ointctrclockwise_{\partial S} d\Bx \cdot \Bf,
\end{equation*}
where \( \ncap \) is the normal specified by \( d^2 \Bx = I \ncap dA \).
} % theorem


%%This follows by setting \( F = 1, G = \Bf \) in \cref{thm:surfaceintegral:100} and selecting the scalar grade.  In particular we may form the
%%scalar selection of \( d^2 \Bx \boldpartial \Bf \) in two different ways.  The first is
%%\begin{dmath}\label{eqn:surfaceintegral:520}
%%\gpgradezero{ d^2 \Bx \boldpartial \Bf }
%%=
%%\gpgradezero{ (d^2 \Bx \cdot \boldpartial + d^2 \Bx \wedge \boldpartial ) \Bf }.
%%\end{dmath}
%%
%%The \( d^2 \Bx \wedge \boldpartial \) product with \( \Bf \) has only bivector and quad-vector components (the latter is zero in \R{3}), so its scalar grade selection is zero, and we are left with
%%\begin{dmath}\label{eqn:surfaceintegral:540}
%%\gpgradezero{ d^2 \Bx \boldpartial \Bf }
%%=
%%(d^2 \Bx \cdot \boldpartial) \cdot \Bf
%%=
%%(d^2 \Bx \cdot \spacegrad) \cdot \Bf,
%%\end{dmath}
%%where we have used \cref{eqn:surfaceintegral:460} again.  This product can also be written as
%%\begin{dmath}\label{eqn:surfaceintegral:560}
%%(d^2 \Bx \cdot \spacegrad) \cdot \Bf
%%=
%%\gpgradezero{ (d^2 \Bx \cdot \spacegrad) \Bf }
%%=
%%\gpgradezero{ (d^2 \Bx \spacegrad - d^2 \Bx \wedge \spacegrad) \Bf }
%%=
%%\gpgradezero{ d^2 \Bx \spacegrad \Bf }
%%=
%%\gpgradezero{ d^2 \Bx \lr{ \cancel{ \spacegrad \cdot \Bf } + \spacegrad \wedge \Bf } }
%%=
%%d^2 \Bx \cdot \lr{ \spacegrad \wedge \Bf }.
%%\end{dmath}
%%
%%\begin{dmath}\label{eqn:surfaceintegral:580}
%%\ointclockwise_{\partial S} d\Bx \cdot \Bf
%%=
%%\gpgradezero{ \int_S d^2\Bx \boldpartial \Bf }
%%=
%%\int_S \lr{ d^2\Bx \cdot \spacegrad } \cdot \Bf
%%=
%%\int_S d^2\Bx \cdot \lr{ \spacegrad \wedge \Bf },
%%\end{dmath}
%%as claimed.  In particular in \R{3}, we have
%%\begin{dmath}\label{eqn:surfaceintegral:600}
%%d^2\Bx \cdot \lr{ \spacegrad \wedge \Bf }
%%=
%%dA \gpgradezero{ I \ncap I \lr{ \spacegrad \cross \Bf } }
%%=
%%-dA \ncap \cdot \lr{ \spacegrad \cross \Bf }.
%%\end{dmath}
%%
%%Substitution into \cref{eqn:surfaceintegral:580} proves the last part of \cref{thm:surfaceintegral:500}.
%%
\subsubsection{Green's theorem.}

\Cref{thm:surfaceintegral:500}, when stated in terms of coordinates, is another well known result.
%
% Copyright � 2018 Peeter Joot.  All Rights Reserved.
% Licenced as described in the file LICENSE under the root directory of this GIT repository.
%
\maketheorem{Green's theorem.}{thm:surfaceintegral:620}{
Given a vector \( \Bf = \sum_i f_i \Bx^i \) in \R{N}, and a surface parameterized by \( \Bx = \Bx(u_1, u_2) \), \textit{Green's theorem}
states
\begin{equation*}
\int_S du_1 du_2 \lr{ \PD{u_2}{f_1} - \PD{u_1}{f_2} }
=
\ointclockwise_{\partial S} du_1 f_1 + du_2 f_2.
\end{equation*}
This is
often stated for vectors
\( \Bf = P \Be_1 + Q \Be_2 \in \mathbb{R}^2 \) with a Cartesian \(x,y\) parameterization as
\begin{equation*}
\int_S dx dy \lr{ \PD{y}{P} - \PD{x}{Q} }
=
\ointclockwise_{\partial S} P dx + Q dy.
\end{equation*}
} % theorem


\todo{Add an example (lots to pick from in any 3rd term calc text).}

The first equality in \cref{thm:surfaceintegral:620} holds in \R{N} for vectors expressed in terms of an arbitrary curvilinear basis.
Only the (curvilinear) coordinates of the vector \( \Bf \) contribute to this integral, and only those that lie in the tangent space.
The reciprocal basis vectors \( \Bx^i \) are also nowhere to be seen.  This is because they are either obliterated in dot products with \( \Bx_j \), or cancel due to mixed partial equality.

To see how this occurs let's look at the
area integrand of \cref{thm:surfaceintegral:500}
\begin{dmath}\label{eqn:surfaceintegral:660}
d^2 \Bx \cdot \lr{ \spacegrad \wedge \Bf }
=
du_1 du_2\, \lr{ \Bx_1 \wedge \Bx_2 } \cdot \lr{ \sum_{ij} \lr{ \Bx^i \partial_i } \wedge \lr{ f_j \Bx^j } }
=
du_1 du_2\, \sum_{ij} \lr{ \lr{ \Bx_1 \wedge \Bx_2 } \cdot \Bx^i } \cdot \lr{ \partial_i (f_j \Bx^j) }
=
du_1 du_2\, \sum_{ij} \lr{ \lr{ \Bx_1 \wedge \Bx_2 } \cdot \Bx^i } \cdot \Bx^j \partial_i f_j
+
du_1 du_2\, \sum_{ij} f_j \lr{ \lr{ \Bx_1 \wedge \Bx_2 } \cdot \Bx^i } \cdot (\partial_i \Bx^j).
\end{dmath}

With a bit of trouble, we will see that the second integrand is zero.  On the other hand, the first integrand
simplifies
without too much trouble
\begin{dmath}\label{eqn:surfaceintegral:680}
\sum_{ij} \lr{ \lr{ \Bx_1 \wedge \Bx_2 } \cdot \Bx^i } \cdot \Bx^j \partial_i f_j
=
\sum_{ij} \lr{ \Bx_1 \delta_{2i} - \Bx_2 \delta_{1i} } \cdot \Bx^j \partial_i f_j
=
\sum_{j} \Bx_1 \cdot \Bx^j \partial_2 f_j -\Bx_2 \cdot \Bx^j \partial_1 f_j
=
\partial_2 f_1 - \partial_1 f_2.
\end{dmath}

For the second integrand, we have
\begin{dmath}\label{eqn:surfaceintegral:700}
\begin{aligned}
\sum_{ij} &f_j \lr{ \lr{ \Bx_1 \wedge \Bx_2 } \cdot \Bx^i } \cdot (\partial_i \Bx^j) \\
&=
\sum_{j} f_j \sum_i \lr{ \Bx_1 \delta_{2i} - \Bx_2 \delta_{1i} } \cdot (\partial_i \Bx_j) \\
&=
\sum_{j} f_j
\lr{
\Bx_1 \cdot (\partial_2 \Bx^j)
-
\Bx_2 \cdot (\partial_1 \Bx^j)
}
\end{aligned}
\end{dmath}

We can apply the chain rule (backwards) to the portion in brackets to find
\begin{dmath}\label{eqn:surfaceintegral:720}
\Bx_1 \cdot (\partial_2 \Bx^j)
-
\Bx_2 \cdot (\partial_1 \Bx^j)
=
\cancel{\partial_2 \lr{ \Bx_1 \cdot \Bx^j }}
-
(\partial_2 \Bx_1) \cdot \Bx^j
-
\cancel{\partial_1 \lr{ \Bx_2 \cdot \Bx^j }}
+
(\partial_1 \Bx_2) \cdot \Bx^j
=
\Bx_j \cdot \lr{ \partial_1 \Bx_2 - \partial_2 \Bx_1 }
=
\Bx_j \cdot \lr{ \PD{u_1}{} \PD{u_2}{\Bx} - \PD{u_2}{} \PD{u_1}{\Bx} }
= 0.
\end{dmath}

In this reduction the derivatives of \( \Bx_i \cdot \Bx^j = \delta_{ij} \) were killed since those are constants (either zero or one).  The final step relies on the fact that we assume our vector parameterization is well behaved enough that the mixed partials are zero.

Substituting these results into
\cref{thm:surfaceintegral:500} we find
\begin{dmath}\label{eqn:surfaceintegral:740}
\ointclockwise_{\partial S} d\Bx \cdot \Bf
=
\ointclockwise_{\partial S} \lr{ du_1 \Bx_1 + du_2 \Bx_2 } \cdot \lr{ \sum_i f_i \Bx^i }
=
\ointclockwise_{\partial S} du_1\, f_1 + du_2\, f_2
=
\int_S du_1 du_2\, \lr{ \partial_2 f_1 - \partial_1 f_2 },
\end{dmath}
which completes the proof.

%}
%%%\EndArticle

%%
% Copyright © 2017 Peeter Joot.  All Rights Reserved.
% Licenced as described in the file LICENSE under the root directory of this GIT repository.
%
%original ideas from gabookII/electrodynamics/transverseField.tex:
We now wish to consider more general solutions to the source free Maxwell's equation than the plane wave solutions derived in \cref{chap:planewavesMultivector}.
One way of tackling this problem is to assume the solution exists, but ask how the field components that lie strictly along the propagation direction are related to the transverse components of the field.
Without loss of generality, it can be assumed that the propagation direction is along the z-axis.

\maketheorem{Transverse and propagation field components.}{thm:transverseField:288}{
If \( \Be_3 \) is the
propagation direction, the components of a field \( F \) in the propagation direction and in the transverse plane are respectively
\begin{equation*}
\begin{aligned}
F_z &= \inv{2} \lr{ F + \Be_3 F \Be_3 } \\
F_t &= \inv{2} \lr{ F - \Be_3 F \Be_3 },
\end{aligned}
\end{equation*}
where \( F = F_z + F_t \).
} % theorem

To determine the components of the field that lie in the propagation direction and transverse planes, we state the field in the propagation direction, building it from the electric and magnetic field projections along the z-axis
\begin{dmath}\label{eqn:transverseField:108}
F_z
=
\lr{ \BE \cdot \Be_3 }
 \Be_3
+ I \eta \lr{ \BH \cdot \Be_3 } \Be_3
=
\inv{2}
\lr{ \BE \Be_3 + \Be_3 \BE }
 \Be_3
+ \inv{2} I \eta \lr{ \BH \Be_3 + \Be_3 \BH } \Be_3
=
\inv{2}
\lr{ \BE + \Be_3 \BE \Be_3 }
+ \inv{2} I \eta \lr{ \BH + \Be_3 \BH \Be_3 }
=
\inv{2} \lr{ F + \Be_3 F \Be_3 }.
\end{dmath}
The difference \( F - F_z \) is the transverse component
\begin{dmath}\label{eqn:transverseField:308}
F_t
= F - F_z
=
F -
\inv{2} \lr{ F + \Be_3 F \Be_3 }
=
\inv{2} \lr{ F - \Be_3 F \Be_3 },
\end{dmath}
as claimed.

We wish to split the gradient into transverse and propagation direction components.

\makedefinition{Transverse and propagation direction gradients.}{dfn:transverseField:328}{
Define the \textit{propagation direction gradient} as \( \Be_3 \partial_z \), and
\textit{transverse gradient} by
\begin{equation*}
\spacegrad_t = \spacegrad - \Be_3 \partial_z.
\end{equation*}
} % definition

Given this definition, we seek to show that

\maketheorem{Transverse and propagation field solutions.}{thm:transverseField:348}{
Given a field propagating along the z-axis (either forward or backwards), with angular frequency \( \omega \), represented by the real part of
\begin{equation*}
F(x, y, z, t) = F(x, y) e^{j \omega t \mp j k z},
\end{equation*}
the field components that solve the source free Maxwell's equation are related by
\begin{equation*}
\begin{aligned}
F_t &= j \inv{ \frac{\omega}{c} \mp k \Be_3 } \spacegrad_t F_z \\
F_z &= j \inv{ \frac{\omega}{c} \mp k \Be_3 } \spacegrad_t F_t.
\end{aligned}
\end{equation*}
Written out explicitly, the transverse field component expands as
\begin{equation*}
\begin{aligned}
\BE_t &=
\frac{j}{{\frac{\omega}{c}}^2 - k^2}
\lr{
   \pm k \spacegrad_t E_z
   + \frac{\omega \eta}{c} \Be_3 \cross \spacegrad_t H_z
}
\\
\eta \BH_t &=
\frac{j}{{\frac{\omega}{c}}^2 - k^2}
\lr{
   \pm k \eta \spacegrad_t H_z
   -
   \frac{\omega}{c}
   \Be_3 \cross \spacegrad_t E_z
}.
\end{aligned}
\end{equation*}
} % theorem

To prove we first insert the assumed phasor representation into Maxwell's equation, which gives
\begin{equation}\label{eqn:transverseField:summaryMax2}
\lr{\spacegrad_t + j \lr{ \frac{\omega}{c} \mp k \Be_3 } } F(x,y) = 0.
\end{equation}

Dropping the \( x, y \) dependence for now (i.e.  \( F(x, y) \rightarrow F \), we find a relation between the transverse gradient of \( F \) and the propagation direction gradient of \( F \)

\begin{dmath}\label{eqn:transverseField:148}
\spacegrad_t F = - j \lr{ \frac{\omega}{c} \mp k \Be_3 } F.
\end{dmath}
From this we now seek to determine the relationships between \( F_t \) and \( F_z \).

Since \( \spacegrad_t \) has no \( \xcap, \ycap \) components, \( \Be_3 \) anticommutes with the transverse gradient
\begin{dmath}\label{eqn:transverseField:168}
\Be_3 \spacegrad_t = - \spacegrad_t \Be_3,
\end{dmath}
but commutes with \( 1 \mp \Be_3 \).
%In \cref{eqn:transverseField:168} it is implied that the action of \( \spacegrad_t \) is on everything to its right.
This means that
\begin{dmath}\label{eqn:transverseField:188}
\inv{2} \lr{ \spacegrad_t F \pm \Be_3 \lr{ \spacegrad_t F } \Be_3 }
=
\inv{2} \lr{ \spacegrad_t F \mp \spacegrad_t \Be_3 F \Be_3 }
=
\spacegrad_t
\inv{2} \lr{ F \mp \Be_3 F \Be_3 },
\end{dmath}
or
\begin{dmath}\label{eqn:transverseField:208}
\begin{aligned}
\inv{2} \lr{ \spacegrad_t F + \Be_3 \lr{ \spacegrad_t F } \Be_3 } &= \spacegrad_t F_t \\
\inv{2} \lr{ \spacegrad_t F - \Be_3 \lr{ \spacegrad_t F } \Be_3 } &= \spacegrad_t F_z,
\end{aligned}
\end{dmath}
so Maxwell's equation \cref{eqn:transverseField:148} becomes
\begin{dmath}\label{eqn:transverseField:228}
\begin{aligned}
\spacegrad_t F_t &= - j \lr{ \frac{\omega}{c} \mp k \Be_3 } F_z \\
\spacegrad_t F_z &= - j \lr{ \frac{\omega}{c} \mp k \Be_3 } F_t.
\end{aligned}
\end{dmath}

Provided \( \omega^2 \ne (k c)^2 \), these can be inverted.
Such an inversion allows an application of the transverse gradient to whichever one
of \( F_z, F_t \) is known, to compute the other, as stated in
\cref{thm:transverseField:348}.

The relation for \( F_t \) in
\cref{thm:transverseField:348}
is usually stated in terms of the electric and magnetic fields.
To perform that expansion, we must first evaluate the multivector inverse explicitly
\begin{dmath}\label{eqn:transverseField:348}
\begin{aligned}
F_z &= j \frac{ \frac{\omega}{c} \pm k \Be_3 }{ \lr{\frac{\omega}{c}}^2 - k^2 } \spacegrad_t F_t \\
F_t &= j \frac{ \frac{\omega}{c} \pm k \Be_3 }{ \lr{\frac{\omega}{c}}^2 - k^2 } \spacegrad_t F_z.
\end{aligned}
\end{dmath}
so that we are in position to expand most of the terms in the numerator
\begin{dmath}\label{eqn:transverseField:268}
\lr{ \frac{\omega}{c} \pm k \Be_3 } \spacegrad_t F_z
=
-\lr{ \Be_3 \frac{\omega}{c} \pm k } \spacegrad_t \Be_3 F_z
=
\lr{ \pm k - \Be_3 \frac{\omega}{c} } \spacegrad_t \lr{ E_z + I \eta H_z }
=
\lr{
   \pm k \spacegrad_t E_z
   + \frac{\omega \eta}{c} \Be_3 \cross \spacegrad_t H_z
}
+ I \lr{
   \pm k \eta \spacegrad_t H_z
   -
   \frac{\omega}{c}
   \Be_3 \cross \spacegrad_t E_z
},
\end{dmath}
from which the transverse electric and magnetic fields stated in
\cref{thm:transverseField:348} can be read off.
A similar expansion for \( \BE_z, \BH_z \) in terms of \( \BE_t, \BH_t \) is also possible.

%There is considerably more complexity required to express the transverse field in terms of separate electric and magnetic components
%compared to the equivalent total transverse field expression of...

\makeproblem{Transverse electric and magnetic field components.}{problem:transverseField:1}{
Fill in the missing details in the steps of \cref{eqn:transverseField:268}.
} % problem

\makeproblem{Propagation direction components.}{problem:transverseField:2}{
Perform an expansion like \cref{eqn:transverseField:268} to find
\( \BE_z, \BH_z \) in terms of \( \BE_t, \BH_t \).
} % problem

%%
% Copyright � 2018 Peeter Joot.  All Rights Reserved.
% Licenced as described in the file LICENSE under the root directory of this GIT repository.
%
%{
%%%\input{../latex/blogpost.tex}
%%%\renewcommand{\basename}{volumeintegral}
%%%%\renewcommand{\dirname}{notes/phy1520/}
%%%\renewcommand{\dirname}{notes/ece1228-electromagnetic-theory/}
%%%%\newcommand{\dateintitle}{}
%%%%\newcommand{\keywords}{}
%%%
%%%\input{../latex/peeter_prologue_print2.tex}
%%%
%%%\usepackage{peeters_layout_exercise}
%%%\usepackage{peeters_braket}
%%%\usepackage{peeters_figures}
%%%\usepackage{siunitx}
%%%%\usepackage{mhchem} % \ce{}
%%%%\usepackage{macros_bm} % \bcM
%%%%\usepackage{macros_qed} % \qedmarker
%%%\usepackage{txfonts} % \ointclockwise
%%%
%%%\beginArtNoToc
%%%
%%%\generatetitle{Volume integral.}
%%%%\chapter{Volume integral.}
\label{chap:volumeintegral}

%\subsection{Volume integral.}
\index{volume parameterization}
\index{volume element}
\index{differential form}
A three parameter curve, and the corresponding differentials with respect to those parameters, is sketched in
\cref{fig:normalsOnVolumeAreaElement:normalsOnVolumeAreaElementFig11}.

\imageFigure{../figures/gabook/normalsOnVolumeAreaElementFig11}{Three parameter volume element.}{fig:normalsOnVolumeAreaElement:normalsOnVolumeAreaElementFig11}{0.4}

Given parameters \( u_1, u_2, u_3 \), we can denote the differentials along each of the parameterization directions as
\begin{dmath}\label{eqn:volumeintegral:100}
\begin{aligned}
d\Bx_1 &= \PD{u_1}{\Bx} du_1 = \Bx_1 du_1 \\
d\Bx_2 &= \PD{u_2}{\Bx} du_2 = \Bx_2 du_2 \\
d\Bx_3 &= \PD{u_3}{\Bx} du_3 = \Bx_3 du_3.
\end{aligned}
\end{dmath}

The trivector valued volume element for this parameterization is
\begin{equation}\label{eqn:volumeintegral:120}
d^3 \Bx
=
d\Bx_1 \wedge
d\Bx_1 \wedge
d\Bx_1
=
d^3 u\, (\Bx_1 \wedge \Bx_2 \wedge \Bx_3),
\end{equation}
where \( d^3 u = du_1 du_2 du_3 \).
The vector derivative, the projection of the gradient onto the volume at the point of integration (also called the tangent space), now has three components
\begin{dmath}\label{eqn:volumeintegral:200}
\boldpartial
=
\sum_i \Bx^i (\Bx_i \cdot \spacegrad)
=
\Bx^1 \PD{u_1}{}
+
\Bx^2 \PD{u_2}{}
+
\Bx^3 \PD{u_3}{}
\equiv
\Bx^1 \partial_1
+
\Bx^2 \partial_2
+
\Bx^3 \partial_3.
\end{dmath}

The volume integral specialization of \cref{dfn:fundamentalTheoremOfCalculus:240} can now be stated

\makedefinition{Multivector volume integral.}{dfn:volumeintegral:100}{
Given an connected volume \( V \) parameterized by two parameters, and multivector functions \( F, G \), we define the volume integral as
\begin{equation*}
\int_V F d^3\Bx \lrboldpartial G
\equiv
\int_V \lr{ F d^3\Bx \lboldpartial} G
+
\int_V F d^3\Bx \lr{ \rboldpartial G },
\end{equation*}
where the three parameter differential form \( d^3 \Bx = d^3 u\, \Bx_1 \wedge \Bx_2 \wedge \Bx_3, d^3 u = du_1 du_2 du_3 \) varies over the volume, and \( \lrboldpartial \) acts on \( F, G \), but not the volume element \( d^2 \Bx \).
} % definition

The volume integral specialization of \cref{thm:fundamentalTheoremOfCalculus:1} is

\maketheorem{Multivector volume integral.}{thm:volumeintegral:100}{
Given an connected volume \( V \) parameterized by three parameters for which \( d\Bx_1, d\Bx_2, d\Bx_3 \) is a right handed triple, and multivector functions \( F, G \), a volume integral can be reduced to a surface integral as follows
\begin{equation*}
\int_V F d^3\Bx \lrboldpartial G
= \ointctrclockwise_{\partial V} F d^2\Bx G,
\end{equation*}
where \( \partial V \) is the boundary of the volume \( V \), and \( d^2 \Bx \) is the counterclockwise oriented area element on the boundary of the volume.
} % theorem

To see why this works, and define \( d^2 \Bx \) more precisely, we would first like to reduce the product of the volume element and the vector derivative
\begin{dmath}\label{eqn:volumeintegral:300}
d^3\Bx \boldpartial
=
d^3 u\, \lr{ \Bx_1 \wedge \Bx_2 \wedge \Bx_3 } \lr{ \Bx^1 \partial_1 + \Bx^2 \partial_2 + \Bx^3 \partial_3 }.
\end{dmath}

Since all \( \Bx^i \) lie within \( \Span \setlr{ \Bx_1, \Bx_2, \Bx_3 } \), this multivector product has only a vector grade.  That is
\begin{dmath}\label{eqn:volumeintegral:320}
\lr{ \Bx_1 \wedge \Bx_2 \wedge \Bx_3 } \Bx^i
=
\lr{ \Bx_1 \wedge \Bx_2 \wedge \Bx_3 } \cdot \Bx^i
+
\cancel{ \lr{ \Bx_1 \wedge \Bx_2 \wedge \Bx_3 } \wedge \Bx^i },
\end{dmath}
for all \( \Bx^i \).  These products reduces to
\begin{dmath}\label{eqn:volumeintegral:1621}
\begin{aligned}
\lr{ \Bx_2 \wedge \Bx_3 \wedge \Bx_1 } \Bx^1 &= \Bx_2 \wedge \Bx_3 \\
\lr{ \Bx_3 \wedge \Bx_1 \wedge \Bx_2 } \Bx^2 &= \Bx_3 \wedge \Bx_1 \\
\lr{ \Bx_1 \wedge \Bx_2 \wedge \Bx_3 } \Bx^3 &= \Bx_1 \wedge \Bx_2.
\end{aligned}
\end{dmath}

Inserting \cref{eqn:volumeintegral:1621}
into the volume integral, we find
\begin{dmath}\label{eqn:volumeintegral:380}
\int_V F d^3\Bx \boldpartial G
=
\int_V \lr{ F d^3\Bx \lboldpartial} G
+
\int_V F d^3\Bx \lr{ \rboldpartial G }
=
\int_V d^3 u\, \lr{
   (\partial_1 F) \Bx_2 \wedge \Bx_3 G
   +
   (\partial_2 F) \Bx_3 \wedge \Bx_1 G
   +
   (\partial_3 F) \Bx_1 \wedge \Bx_2 G
}
+
\int_V d^3 u\, \lr{
   F \Bx_2 \wedge \Bx_3 (\partial_1 G)
   +
   F \Bx_3 \wedge \Bx_1 (\partial_2 G)
   +
   F \Bx_1 \wedge \Bx_2 (\partial_3 G)
}
=
\int_V d^3 u\, \lr{
   \partial_1 (F \Bx_2 \wedge \Bx_3 G)
   +
   \partial_2 (F \Bx_3 \wedge \Bx_1 G)
   +
   \partial_3 (F \Bx_1 \wedge \Bx_2 G)
}
-
\int_V d^3 u\, \lr{
   F (\partial_1 (\Bx_2 \wedge \Bx_3)) G
   +
   F (\partial_2 (\Bx_3 \wedge \Bx_1)) G
   +
   F (\partial_3 (\Bx_1 \wedge \Bx_2)) G
}
=
\int_V d^3 u\, \lr{
   \partial_1 (F \Bx_2 \wedge \Bx_3 G)
   +
   \partial_2 (F \Bx_3 \wedge \Bx_1 G)
   +
   \partial_3 (F \Bx_1 \wedge \Bx_2 G)
}
-
\int_V d^3 u\, F
\lr{
   \partial_1 (\Bx_2 \wedge \Bx_3)
   +
   \partial_2 (\Bx_3 \wedge \Bx_1)
   +
   \partial_3 (\Bx_1 \wedge \Bx_2)
}
G
.
\end{dmath}

The sum within the second integral is
\begin{dmath}\label{eqn:fundamentalTheoremOfCalculus:400}
\begin{aligned}
\sum_{i = 1}^3 \partial_i \lr{ I_k \cdot \Bx^i }
&=
\partial_3 \lr{ (\Bx_1 \wedge \Bx_2 \wedge \Bx_3) \cdot \Bx^3 }
+
\partial_1 \lr{ (\Bx_2 \wedge \Bx_3 \wedge \Bx_1) \cdot \Bx^1 }
+
\partial_2 \lr{ (\Bx_3 \wedge \Bx_1 \wedge \Bx_2) \cdot \Bx^2 } \\
&=
\partial_3 \lr{ \Bx_1 \wedge \Bx_2 }
+
\partial_1 \lr{ \Bx_2 \wedge \Bx_3 }
+
\partial_2 \lr{ \Bx_3 \wedge \Bx_1 } \\
&=
         (\partial_3 \Bx_1) \wedge \Bx_2 + \Bx_1 \wedge (\partial_3 \Bx_2) \\
&\quad + (\partial_1 \Bx_2) \wedge \Bx_3 + \Bx_2 \wedge (\partial_1 \Bx_3) \\
&\quad + (\partial_2 \Bx_3) \wedge \Bx_1 + \Bx_3 \wedge (\partial_2 \Bx_1) \\
&=
\Bx_2 \wedge \lr{ - \partial_3 \Bx_1 + \partial_1 \Bx_3 }
+
\Bx_3 \wedge \lr{ - \partial_1 \Bx_2 + \partial_2 \Bx_1 }
+
\Bx_1 \wedge \lr{ - \partial_2 \Bx_3 + \partial_3 \Bx_2 } \\
&=
\Bx_2 \wedge \lr{ - \frac{\partial^2 \Bx}{\partial_3 \partial_1} + \frac{\partial^2 \Bx}{\partial_1 \partial_3} }
+
\Bx_3 \wedge \lr{ - \frac{\partial^2 \Bx}{\partial_1 \partial_2} + \frac{\partial^2 \Bx}{\partial_2 \partial_1} }
+
\Bx_1 \wedge \lr{ - \frac{\partial^2 \Bx}{\partial_2 \partial_3} + \frac{\partial^2 \Bx}{\partial_3 \partial_2} },
\end{aligned}
\end{dmath}
which is zero by equality of mixed partials.
This leaves three perfect differentials, which can integrated separately, giving
\begin{dmath}\label{eqn:volumeintegral:400}
\int_V F d^3\Bx \boldpartial G
=
\int du_2 du_3
\evalbar{ \lr{ F \Bx_2 \wedge \Bx_3 G } }{\Delta u_1}
+
\int du_3 du_1
\evalbar{ \lr{ F \Bx_3 \wedge \Bx_1 G } }{\Delta u_2}
+
\int du_1 du_2
\evalbar{ \lr{ F \Bx_1 \wedge \Bx_2 G } }{\Delta u_3}
=
\int
\evalbar{ \lr{ F d\Bx_2 \wedge d\Bx_3 G } }{\Delta u_1}
+
\int
\evalbar{ \lr{ F d\Bx_3 \wedge d\Bx_1 G } }{\Delta u_2}
+
\int
\evalbar{ \lr{ F d\Bx_1 \wedge d\Bx_2 G } }{\Delta u_3}.
\end{dmath}

This proves the theorem from an algebraic point of view.
With the aid of a geometrical model, such as that of \cref{fig:differentialVolume:differentialVolumeFig}, if
assuming that \( d\Bx_1, d\Bx_2, d\Bx_3 \) is a right handed triple).
it is possible to convince oneself that the two parameter integrands describe an integral over a counterclockwise oriented surface (
\imageTwoFigures{../figures/GAelectrodynamics/differentialVolumeFig1}{../figures/GAelectrodynamics/differentialVolumeFig2}{Differential surface of a volume.}{fig:differentialVolume:differentialVolumeFig}{scale=0.05}

We obtain the RHS of \cref{thm:volumeintegral:100} if we
introduce a mnemonic for the bounding oriented surface of the volume
\begin{dmath}\label{eqn:volumeintegral:1641}
d^2 \Bx \equiv d\Bx_1 \wedge d\Bx_2 + d\Bx_2 \wedge d\Bx_3 + d\Bx_3 \wedge d\Bx_1,
\end{dmath}
where it is implied that each component of this area element and anything that it is multiplied with is evaluated on the boundaries of the integration volume (for the parameter omitted) as detailed explicitly in
\cref{eqn:volumeintegral:400}.

\subsection{Three parameter Stokes' theorem.}

Three special cases of \cref{thm:volumeintegral:100} can be obtained by integrating scalar, vector or bivector functions over the volume, as follows

\maketheorem{Volume integral of scalar function (Stokes').}{thm:volumeintegral:420}{
Given a scalar function \( f(\Bx) \) its volume integral is given by
\begin{equation*}
\int_V d^3 \Bx \cdot \boldpartial f =
\int_V d^3 \Bx \cdot \spacegrad f = \ointctrclockwise_{\partial V} d^2\Bx f.
\end{equation*}
In \R{3} this can be written as
\begin{equation*}
\int_V dV \spacegrad f = \int_{\partial V} dA \ncap f
\end{equation*}
where \( \ncap \) is the outwards normal specified by \( d^2 \Bx = I \ncap dA, \) and \( d^3 \Bx = I dV \).
} % theorem

\maketheorem{Volume integral of vector function (Stokes').}{thm:volumeintegral:1661}{
Given a vector function \( \Bf(\Bx) \) the volume
integral of the (bivector) curl is related to a surface integral by
\begin{equation*}
\int_V d^3 \Bx \cdot \lr{ \boldpartial \wedge \Bf } =
\int_V d^3 \Bx \cdot \lr{ \spacegrad \wedge \Bf } = \ointctrclockwise_{\partial V} d^2\Bx \cdot \Bf.
\end{equation*}
In \R{3} this can be written as
\begin{equation*}
\int_V dV \spacegrad \cross \Bf = \int_{\partial V} dA \ncap \cross \Bf,
\end{equation*}
or with a duality transformation \( \Bf = I B \), where \( B \) is a bivector
\begin{equation*}
\int_V dV \spacegrad \cdot B = \int_{\partial V} dA \ncap \cdot \Bf,
\end{equation*}
where \( \ncap \) is the outwards normal specified by \( d^2 \Bx = I \ncap dA, \) and \( d^3 \Bx = I dV \).
} % theorem

\maketheorem{Volume integral of bivector function (Stokes', divergence).}{thm:volumeintegral:1681}{
Given a bivector function \( B(\Bx) \), the volume
integral of the (trivector) curl is related to a surface integral by
\begin{equation*}
\int_V d^3 \Bx \cdot \lr{ \boldpartial \wedge B } =
\int_V d^3 \Bx \cdot \lr{ \spacegrad \wedge B } = \ointctrclockwise_{\partial V} d^2\Bx \cdot B.
\end{equation*}
In \R{3} this can be written as
\begin{equation*}
\int_V dV \spacegrad \wedge B = \int_{\partial V} dA \ncap \wedge B,
\end{equation*}
or, making a duality transformation \( B(\Bx) = I \Bf(\Bx) \), where \( \Bf \) is a vector, by
\begin{equation*}
\int_V dV \spacegrad \cdot \Bf = \int_{\partial V} dA \ncap \cdot \Bf,
\end{equation*}
where \( \ncap \) is the outwards normal specified by \( d^2 \Bx = I \ncap dA, \) and \( d^3 \Bx = I dV \).
} % theorem

\subsection{Divergence theorem.}

Observe that for \R{3} we there are dot product relations in each of
\cref{thm:volumeintegral:420},
\cref{thm:volumeintegral:1661} and
\cref{thm:volumeintegral:1681} which can be summarized as
\index{divergence theorem}
\maketheorem{Divergence theorem.}{thm:volumeintegral:2661}{
Given an \R{3} multivector \( M \) containing only grades 0,1, or 2
\begin{equation*}
\int_V dV \spacegrad \cdot M = \int_{\partial V} dA \ncap \cdot M,
\end{equation*}
where \( \ncap \) is the outwards normal to the surface bounding \( V \).
} % theorem

%}
%\EndNoBibArticle

%%
% Copyright © 2017 Peeter Joot.  All Rights Reserved.
% Licenced as described in the file LICENSE under the root directory of this GIT repository.
%
\subsection{Field energy and momentum density and the stress energy tensor.}

In conventional electromagnetism the energy and momentum density of the fields are

\index{energy density}
\index{momentum density}
\index{Poynting vector}
\index{energy flux}
\index{stress tensor}
\begin{dmath}\label{eqn:poyntingF:20}
\begin{aligned}
\calE &= \inv{2} \lr{ \BD \cdot \BE + \BB \cdot \BH } \\
\bcP v &= \inv{v} \BS = \inv{v} \BE \cross \BH = \inv{v} \lr{ I \BH } \cdot \BE.
\end{aligned}
\end{dmath}

where \( \calE \) is the energy density, \( \BS \) is the Poynting vector representing energy flux through a surface per unit time, and \( \bcP \) is the momentum density of the fields.

In GA formalism, the energy and momentum densities are related by grade selection from a single multivector formed from the electrodynamic field \( F \)

\boxedEquation{eqn:poyntingF:60}{
T(1) \equiv \inv{2} \epsilon F F^\dagger = \calE + \bcP v = \calE + \frac{\BS}{v},
}

or
\begin{dmath}\label{eqn:poyntingF:40}
\begin{aligned}
\calE &= \inv{2} \epsilon \gpgradezero{ F F^\dagger } \\
\bcP v &= \inv{2} \epsilon \gpgradeone{ F F^\dagger } \\
\BS &= \inv{2 \eta} \gpgradeone{ F F^\dagger }.
\end{aligned}
\end{dmath}

Expanding \( T(1) \) in terms of \( \BE, \BH \) gives

\begin{dmath}\label{eqn:poyntingF:80}
T(1)
=
\inv{2} \epsilon F F^\dagger
=
\inv{2} \epsilon \lr{ \BE + I \eta \BH } \lr{ \BE - I \eta \BH }
=
\inv{2} \epsilon \lr{ \BE^2 + \eta^2 \BH^2 }
+
\inv{2} I \epsilon \eta \lr{ \BH \BE - \BE \BH }
=
\inv{2} \lr{ \BD \cdot \BE + \BH \cdot \BB }
+
\frac{I}{v} \BH \wedge \BE
=
\inv{2} \lr{ \BD \cdot \BE + \BH \cdot \BB }
+
\frac{1}{v} \BE \cross \BH
=
\calE + \frac{\BS}{v},
\end{dmath}

as claimed in \cref{eqn:poyntingF:20} and \cref{eqn:poyntingF:60}.  \( T(1) \) has one scalar component, and three vector components, and represents four
four of the sixteen components of the complete stress energy tensor.  The geometric algebra form of the complete stress energy tensor is

\begin{dmath}\label{eqn:poyntingF:760}
T(a) = \inv{2} \epsilon F a F^\dagger,
\end{dmath}

where \( a \) is one of \( 1, \Be_1, \Be_2 \) or \( \Be_3 \).
Each \( T(a) \) is a 0,1 multivector containing one scalar component and three vector components.
An expansion of \( T(\Be_k) \) is harder to do in closed form than \cref{eqn:poyntingF:80}, but a
brute force expansion using Mathematica of all the \( T(a) \) multivectors gives:

\begin{dmath}\label{eqn:poyntingF:800}
\begin{aligned}
T(1) &=
\frac{\epsilon}{2} \lr{E_1^2 + E_2^2 + E_3^2} + \frac{\epsilon \eta^2}{2} \lr{H_1^2 + H_2^2 + H_3^2} \\
&+ \Be_1 \eta \epsilon \lr{E_2 H_3 - E_3 H_2}
 + \Be_2 \eta \epsilon \lr{E_3 H_1 - E_1 H_3}
 + \Be_3 \eta\epsilon \lr{E_1 H_2 - E_2 H_1}
\\
T(\Be_1) &=
\eta \epsilon \lr{E_3 H_2 - E_2 H_3} \\
& + \frac{1}{2} \Be_1 \epsilon \lr{E_1^2 - E_2^2 - E_3^2} + \frac{\epsilon \eta^2}{2} \lr{ H_1^2 -  H_2^2 -  H_3^2}
  + \Be_2 \epsilon \lr{E_1 E_2 + \eta^2 H_1 H_2}
  + \Be_3 \epsilon \lr{E_1 E_3 + \eta^2 H_1 H_3}
\\
T(\Be_2) &=
\eta \epsilon \lr{E_1 H_3 - E_3 H_1} \\
& + \Be_1 \epsilon \lr{E_1 E_2 + \eta^2 H_1 H_2}
  + \frac{1}{2} \Be_2 \epsilon \lr{-E_1^2 + E_2^2 - E_3^2 } + \frac{\epsilon \eta^2}{2} \lr{-H_1^2 +  H_2^2 -  H_3^2}
  + \Be_3 \epsilon \lr{E_2 E_3 + \eta^2 H_2 H_3}
\\
T(\Be_3) &=
\eta \epsilon \lr{E_2 H_1 - E_1 H_2} \\
& + \Be_1 \epsilon \lr{E_1 E_3 + \eta^2 H_1 H_3}
  + \Be_2 \epsilon \lr{E_2 E_3 + \eta^2 H_2 H_3}
  + \frac{1}{2} \Be_3 \epsilon \lr{-E_1^2 - E_2^2 + E_3^2 } + \frac{\epsilon \eta^2}{2} \lr{ -H_1^2 -  H_2^2 + H_3^2}
\end{aligned}
\end{dmath}

The components of the multivectors \( T(a) \) that we are calling the stress energy tensor, are more conventionally written out
as a symmetric tensor \( \Theta^{ij} \) as follows
\footnote{Following conventions from \citep{jackson1975cew}, but translating to SI units.}
\begin{dmath}\label{eqn:poyntingF:840}
\begin{aligned}
\Theta^{00} &= \frac{\epsilon}{2} \lr{ \BE^2 + \eta^2 \BH^2 } \\
\Theta^{0i} &= \inv{v} \lr{ \BE \cross \BH } \cdot \Be_i \\
\Theta^{ij} &= -\epsilon \lr{ E_i E_j + \eta^2 H_i H_j - \inv{2} \delta_{ij} \lr{ \BE^2 + \eta^2 \BH^2 } }.
\end{aligned}
\end{dmath}

The symmetric stress tensor components of \cref{eqn:poyntingF:840}
are related to the multivector representation expanded in \cref{eqn:poyntingF:800} by

\begin{dmath}\label{eqn:poyntingF:820}
\begin{aligned}
&\gpgradezero{ T(1) }
%= \calE
=
{\Theta_0}^0 = \Theta^{00} \\
&\gpgradeone{ T(1) } \cdot \Be_i
%= \frac{\BS}{v} \cdot \Be_i
= {\Theta_0}^i = \Theta^{0i} \\
&\gpgradezero{ T(\Be_i) }
%= -\frac{\BS}{v} \cdot \Be_i
= {\Theta_i}^0 = -\Theta^{i0} \\
&\gpgradeone{ T(\Be_i) } \cdot \Be_j = {\Theta_i}^j = -\Theta^{ij}.
\end{aligned}
\end{dmath}

\subsection{Poynting's theorem.}

Poynting's theorem, the conservation relationship between energy and momentum density (or more generally, the stress tensor) and the sources, can be stated in terms of the multivector field \( F \) and the multivector current \( J \).
To derive this relationship we can act on (all terms of) \( F a F^\dagger \) with the space+time derivative operator \( \spacegrad + (1/v) \partial_t \).
We do so within a scalar selection operation, which simplifies things, and allows for cyclic permutation of the multivector factors (i.e. \(\gpgradezero{ABC} = \gpgradezero{CAB}\)).

\begin{dmath}\label{eqn:poyntingF:100}
\frac{\epsilon}{2} \gpgradezero{ \lr{ v \spacegrad + \PD{t}{} } F a F^\dagger }
=
\frac{\epsilon}{2} \gpgradezero{ \lr{ v \spacegrad + \PD{t}{} } \dot{F} a F^\dagger }
+
\frac{\epsilon}{2} \gpgradezero{ \lr{ v \spacegrad + \PD{t}{} } F a \dot{F}^\dagger }
=
\frac{\epsilon}{2} \gpgradezero{ v J F^\dagger a }
+
\frac{\epsilon}{2} \gpgradezero{ \dot{F}^\dagger \lr{ v \spacegrad + \PD{t}{} } F a }
=
\frac{\epsilon}{2} \gpgradezero{ v J F^\dagger a }
+
\frac{\epsilon}{2} \gpgradezero{ \lr{ \lr{ v \spacegrad + \PD{t}{} } F }^\dagger a F }
=
v \frac{\epsilon}{2} \gpgradezero{ F^\dagger J a + J^\dagger a F }
=
\frac{1}{2 \eta} \gpgradezero{ J a F^\dagger + J^\dagger a F }
%=
%\inv{\eta} \gpgradezero{ J^\dagger F a },
\end{dmath}

The over-dot notation of
\citep{hestenes1999nfc} was used to indicate the desired action of the derivative operators in the
chain rule expansion of
\cref{eqn:poyntingF:100}
, as the gradient may not commute with \( F \).

For \( a = 1 \), noting that scalars are reversion invariant (i.e. \( \gpgradezero{ J^\dagger F } = \gpgradezero{ F^\dagger J } = \gpgradezero{ J F^\dagger } \)), we obtain the
multivector form of Poynting's theorem

\index{Poynting theorem}
\boxedEquation{eqn:poyntingF:220}{
v \gpgradezero{ \spacegrad \frac{\epsilon}{2} F F^\dagger }
+ \PD{t}{} \gpgradezero{ \frac{\epsilon}{2} F F^\dagger }
=
\inv{\eta} \gpgradezero{ J^\dagger F }.
}

The conventional statement of this theorem in terms of \( \BD, \BE, \BB, \BH, \BJ, \BM \) follows by direct substitution.
The multivector current \( J \) and its reverse are

\begin{dmath}\label{eqn:poyntingF:160}
\begin{aligned}
J &= \eta \lr{ v \rho - \BJ } + I \lr{ v \rho_m - \BM } \\
J^\dagger &= \eta \lr{ v \rho - \BJ } - I \lr{ v \rho_m - \BM },
\end{aligned}
\end{dmath}

so

\begin{dmath}\label{eqn:poyntingF:180}
0 =
\spacegrad \cdot \BS
-
\inv{\eta}
\lr{
- \eta \BJ \cdot \BE
- \eta \BM \cdot \BH
}
+ \PD{t}{\calE},
\end{dmath}

or
\boxedEquation{eqn:poyntingF:200}{
\spacegrad \cdot \BS + \BJ \cdot \BE + \BM \cdot \BH
%+ \PD{t}{\calE} = 0.
+ \PD{t}{\BB} \cdot \BH
+ \PD{t}{\BD} \cdot \BE = 0.
}

The sum of the last two terms is the time rate of change of the energy density.
In particular,
with neither electric nor magnetic current sources in a region of space,
the change of energy density through a volume is matched by a corresponding flux through the bounding surface

\begin{dmath}\label{eqn:maxwellsEquations:740}
\PD{t}{} \int_V
\inv{2} dV \lr{
\BB \cdot \BH
+ \BD \cdot \BE
}
=
-\int_{\partial V} dA \ncap \cdot \BS.
\end{dmath}

Here \( \ncap \) is the outward normal, so if the energy contained in the volume is decreasing, then \( \BS \) must represent the energy per unit area that leaves the volume.
The direction of the Poynting vector is the direction that the energy is leaving the volume.
Only the components of the Poynting vector that are colinear with the surface normal will result in energy leaving or entering the volume.


%%
% Copyright © 2017 Peeter Joot.  All Rights Reserved.
% Licenced as described in the file LICENSE under the root directory of this GIT repository.
%

Poynting's theorem, the conservation relationship between energy and momentum density (or more generally, the energy-momentum tensor) and the sources, can be stated in terms of the multivector field \( F \) and the multivector current \( J \).
To derive this relationship we can act on (all terms of) \( F a F^\dagger \) with the space+time derivative operator \( \spacegrad + (1/c) \partial_t \).
We do so within a scalar selection operation, which simplifies things, and allows for cyclic permutation of the multivector factors (i.e. \(\gpgradezero{ABC} = \gpgradezero{CAB}\)).
\begin{dmath}\label{eqn:poyntingTheorem:100}
\frac{\epsilon}{2} \gpgradezero{ \lr{ c \spacegrad + \PD{t}{} } F a F^\dagger }
=
\frac{\epsilon}{2} \gpgradezero{ \lr{ c \spacegrad + \PD{t}{} } \dot{F} a F^\dagger }
+
\frac{\epsilon}{2} \gpgradezero{ \lr{ c \spacegrad + \PD{t}{} } F a \dot{F}^\dagger }
=
\frac{\epsilon}{2} \gpgradezero{ c J a F^\dagger }
+
\frac{\epsilon}{2} \gpgradezero{ \dot{F}^\dagger \lr{ c \spacegrad + \PD{t}{} } F a },
\end{dmath}
where
the over-dot notation of
\citep{hestenes1999nfc} was used to indicate the desired action of the derivative operators in the
chain rule expansion of
\cref{eqn:poyntingTheorem:100}
, as the gradient may not commute with \( F \).  Another application of Maxwell's equation reduces this further
\begin{dmath}\label{eqn:poyntingTheorem:960}
\frac{\epsilon}{2} \gpgradezero{ \lr{ c \spacegrad + \PD{t}{} } F a F^\dagger }
=
\frac{\epsilon}{2} \gpgradezero{ c J F^\dagger a }
+
\frac{\epsilon}{2} \gpgradezero{ \lr{ \lr{ c \spacegrad + \PD{t}{} } F }^\dagger F a }
=
c \frac{\epsilon}{2} \gpgradezero{ F^\dagger J a + J^\dagger F a },
\end{dmath}
or
\boxedEquation{eqn:poyntingF:980}{
c \spacegrad \cdot \gpgradeone{ \frac{\epsilon}{2} F a F^\dagger }
+ \PD{t}{} \gpgradezero{ \frac{\epsilon}{2} F a F^\dagger }
=
\frac{1}{2 \eta} \gpgradezero{ a \lr{ F^\dagger J + J^\dagger F} }.
}

For \( a = 1 \), since scalars are reversion invariant (\(\alpha^\dagger = \alpha\) for any scalars \( \alpha \))
\begin{equation}\label{eqn:poyntingTheorem:1000}
\gpgradezero{ F^\dagger J }
=
\gpgradezero{ F^\dagger J }^\dagger
=
\gpgradezero{ J^\dagger F },
\end{equation}
so the
multivector form of Poynting's theorem with respect to the time variation of the energy of the field is
\index{Poynting theorem}
\boxedEquation{eqn:poyntingF:220}{
c \spacegrad \cdot \gpgradeone{ \frac{\epsilon}{2} F F^\dagger }
+ \PD{t}{} \gpgradezero{ \frac{\epsilon}{2} F F^\dagger }
=
\inv{\eta} \gpgradezero{ J^\dagger F }.
}

The conventional statement of this theorem in terms of \( \BD, \BE, \BB, \BH, \BJ, \BM \) follows by direct substitution.
The multivector current \( J \) and its reverse are
\begin{dmath}\label{eqn:poyntingTheorem:160}
\begin{aligned}
J &= \eta \lr{ c \rho - \BJ } + I \lr{ c \rho_m - \BM } \\
J^\dagger &= \eta \lr{ c \rho - \BJ } - I \lr{ c \rho_m - \BM },
\end{aligned}
\end{dmath}
so
\begin{dmath}\label{eqn:poyntingTheorem:180}
0 =
\spacegrad \cdot \BS
-
\inv{\eta}
\lr{
- \eta \BJ \cdot \BE
- \eta \BM \cdot \BH
}
+ \PD{t}{\calE},
\end{dmath}
or
\boxedEquation{eqn:poyntingF:200}{
\spacegrad \cdot \BS + \BJ \cdot \BE + \BM \cdot \BH
%+ \PD{t}{\calE} = 0.
+ \PD{t}{\BB} \cdot \BH
+ \PD{t}{\BD} \cdot \BE = 0.
}

The sum of the last two terms is the time rate of change of the energy density.
In particular,
with neither electric nor magnetic current sources in a region of space,
the change of energy density through a volume is matched by a corresponding flux through the bounding surface
\begin{dmath}\label{eqn:poyntingTheorem:740}
\PD{t}{} \int_V
\inv{2} dV \lr{
\BB \cdot \BH
+ \BD \cdot \BE
}
=
-\int_{\partial V} dA \ncap \cdot \BS.
\end{dmath}

Here \( \ncap \) is the outward normal, so if the energy contained in the volume is decreasing, then \( \BS \) must represent the energy per unit area that leaves the volume.
The direction of the Poynting vector is the direction that the energy is leaving the volume.
Only the components of the Poynting vector that are colinear with the surface normal will result in energy leaving or entering the volume.


%%
% Copyright © 2017 Peeter Joot.  All Rights Reserved.
% Licenced as described in the file LICENSE under the root directory of this GIT repository.
%
%{
%\subsubsection{Poynting theorem prerequistes.}
Poynting's theorem is a set of conservation relationships between relating space and time change of energy density and momentum density, or more generally between related components the energy-momentum tensor.
The most powerful way of stating Poynting's theorem using geometric algebra requires a few new concepts, differential operator valued linear functions, and the adjoint.

\makedefinition{Differential operator valued multivector functions.}{dfn:poyntingTheoremRewrite:1180}{
Given a multivector valued linear functions of the form \( f(x) = A x B \), where \( A, B, x \) are multivectors, and a linear operator \( D \) such as \( \spacegrad, \partial_t \), or \( \spacegrad + (1/c) \partial_t \), the operator valued linear function \( f(D) \) is defined as
\begin{equation*}
f(D)
= A \lroverarrow{D} B
= (A \loverarrow{D}) B + A (\roverarrow{D} B),
\end{equation*}
where \( \lroverarrow{D} \) indicates that \( D \) is acting bidirectionally to the left and to the right.
} % definition

Perhaps counterintuitively, using operator valued parameters for the energy momentum tensor \( T \) or the Maxwell stress tensor \( \BT \) will be particularly effective to express the derivatives of the tensor.  There are a few cases of interest, all related to evaluation of the tensor with a parameter value of the spacetime gradient.
\maketheorem{Energy momentum tensor operator parameters.}{thm:poyntingTheoremRewrite:1200}{
\begin{equation*}
\begin{aligned}
T((1/c)\partial_t) &= \inv{c} \PD{t}{T(1)} = \inv{c} \PD{t}{} \lr{ \calE + \frac{\BS}{c} } \\
\gpgradezero{T(\spacegrad)} &= - \spacegrad \cdot \frac{\BS}{c} \\
\gpgradeone{T(\spacegrad)} &= \BT(\spacegrad) = \sum_{k = 1}^3 \lr{ \spacegrad \cdot \BT(\Be_k) } \Be_k.
\end{aligned}
\end{equation*}
} % theorem

We will proceed to prove each of the results of
\cref{thm:poyntingTheoremRewrite:1200} in sequence, starting with the time partial, which is a scalar operator
\begin{dmath}\label{eqn:poyntingTheoremRewrite:1220}
T(\partial_t)
=
\frac{\epsilon}{2}
 F \lroverarrow{\partial_t} F^\dagger
=
\frac{\epsilon}{2}
\lr{
 (\partial_t F) F^\dagger
+
 F (\partial_t F^\dagger)
}
=
\frac{\epsilon}{2}
\partial_t
 F F^\dagger
=
\partial_t T(1).
\end{dmath}

To evaluate the tensor at the gradient we have to take care of order.  This is easiest in a scalar selection where we may cyclically permute any multivector factors
\begin{dmath}\label{eqn:poyntingTheoremRewrite:1240}
\gpgradezero{T(\spacegrad)}
=
\frac{\epsilon}{2}
\gpgradezero{
 F \lrspacegrad F^\dagger
}
=
\frac{\epsilon}{2}
\gpgradezero{
 \spacegrad F^\dagger F
}
=
\frac{\epsilon}{2} \spacegrad \gpgradeone{ F^\dagger F },
\end{dmath}
but
\begin{dmath}\label{eqn:poyntingTheoremRewrite:1260}
F^\dagger F
=
\lr{ \BE - I \eta \BH } \lr{ \BE + I \eta \BH }
=
\BE^2 + \eta^2 \BH^2 + I \eta \lr{ \BE \BH - \BH \BE }
=
\BE^2 + \eta^2 \BH^2 - 2 \eta \BE \cross \BH.
\end{dmath}
Plugging \cref{eqn:poyntingTheoremRewrite:1260} into \cref{eqn:poyntingTheoremRewrite:1240} proves the result.

Finally, we want to evaluate the Maxwell stress tensor of the gradient
\begin{dmath}\label{eqn:poyntingTheoremRewrite:260}
\BT(\spacegrad)
=
\sum_{k = 1}^3 \Be_k \lr{ \BT(\spacegrad) } \cdot \Be_k
=
\sum_{k,m = 1}^3 \Be_k \partial_m \lr{ \BT(\Be_m) \cdot \Be_k }
=
\sum_{k,m = 1}^3 \Be_k \partial_m \lr{ \BT(\Be_k) \cdot \Be_m }
=
\sum_{k = 1}^3 \Be_k \lr{ \spacegrad \cdot \BT(\Be_k) },
\end{dmath}
as claimed.

Will want to integrate \( \BT(\spacegrad) \) over a volume, which is essentially a divergence operation.
\maketheorem{Divergence integral for the Maxwell stress tensor.}{thm:poyntingTheoremRewrite:1281}{
\begin{equation*}
\int_V dV \BT(\spacegrad)
=
\int_{\partial V} dA \BT(\ncap).
\end{equation*}
} % theorem

To prove \cref{thm:poyntingTheoremRewrite:1281}, we make use of the symmetric property of the Maxwell stress tensor
\begin{dmath}\label{eqn:poyntingTheoremRewrite:320}
\int_V dV \BT(\spacegrad)
=
\sum_k \int_V dV \Be_k \spacegrad \cdot \BT(\Be_k)
=
\sum_k \int_{\partial V} dA \Be_k \ncap \cdot \BT(\Be_k)
=
\sum_{k,m} \int_{\partial V} dA \Be_k n_m {\BT(\Be_k) \cdot \Be_m}
=
\sum_{k,m} \int_{\partial V} dA \Be_k n_m {\BT(\Be_m) \cdot \Be_k}
=
\sum_{k} \int_{\partial V} dA \Be_k {\BT(\ncap) \cdot \Be_k}
=
\int_{\partial V} dA \BT(\ncap),
\end{dmath}
as claimed.

Finally, before stating Poynting's theorem, we want to introduce the concept of an adjoint.

\makedefinition{Adjoint.}{dfn:poyntingTheorem:1120}{
The adjoint \( \overbar{A}(x) \) of a linear operator \( A(x) \) is defined implicitly by the scalar selection
\begin{equation*}
\gpgradezero{ y \overbar{A}(x) } =
\gpgradezero{ x A(y) }.
\end{equation*}
} % definition
The adjoint of the energy momentum tensor is particularly easy to calculate.
\maketheorem{Adjoint of the energy momentum tensor.}{thm:poyntingTheorem:1140}{
The \textit{adjoint of the energy momentum tensor} is
\begin{equation*}
\overbar{T}(x) =
\frac{\epsilon}{2} F^\dagger x F.
\end{equation*}
The adjoint \( \overbar{T} \) and \( T \) satisfy the following relationships
\begin{equation*}
\begin{aligned}
\gpgradezero{\overbar{T}(1)} &= \gpgradezero{T(1)} = \calE \\
\gpgradeone{\overbar{T}(1)} &= -\gpgradeone{T(1)} = -\frac{\BS}{c} \\
\gpgradezero{\overbar{T}(\Ba)} &= -\gpgradezero{T(\Ba)} = \Ba \cdot \frac{\BS}{c} \\
\gpgradeone{\overbar{T}(\Ba)} &= \gpgradeone{T(\Ba)} = \BT(\Ba).
\end{aligned}
\end{equation*}
} % theorem

Using
the cyclic scalar selection permutation property \(\gpgradezero{ABC} = \gpgradezero{CAB}\) we form
\begin{dmath}\label{eqn:poyntingTheoremRewrite:1160}
\gpgradezero{ x T(y) }
=
\frac{\epsilon}{2} \gpgradezero{ x F y F^\dagger }
=
\frac{\epsilon}{2} \gpgradezero{ y F^\dagger x F }.
\end{dmath}
Referring back to \cref{dfn:poyntingTheorem:1120} we see that the adjoint must have the stated form.
Proving the grade selection relationships of \cref{eqn:poyntingTheoremRewrite:1160} has been left as
an exersize for the reader.  A brute force symbolic algebra proof using Mathematica is also available in \cref{path:GAelectrodynamics:stressEnergyTensorValues.nb}.
%We can now state the adjoint form of \cref{thm:poyntingTheorem:1040}.

As in \cref{thm:poyntingTheoremRewrite:1200},
the adjoint may also be evaluated with differential operator parameters.

\maketheorem{Adjoint energy momentum tensor operator parameters.}{thm:poyntingTheoremRewrite:1280}{
\begin{equation*}
\begin{aligned}
\gpgradezero{\overbar{T}((1/c)\partial_t)} &= \inv{c} \PD{t}{T(1)} = \inv{c} \PD{t}{\calE} \\
\gpgradeone{\overbar{T}((1/c)\partial_t)} &= -\inv{c^2} \PD{t}{\BS} \\
\gpgradezero{\overbar{T}(\spacegrad)} &= \spacegrad \cdot \frac{\BS}{c} \\
\gpgradeone{\overbar{T}(\spacegrad)} &= \BT(\spacegrad).
\end{aligned}
\end{equation*}
} % theorem

The proofs of \cref{thm:poyntingTheoremRewrite:1280} are all fairly simple
\begin{dmath}\label{eqn:poyntingTheoremRewrite:1280}
\overbar{T}((1/c)\partial_t)
=
\inv{c} \frac{\epsilon}{2} \PD{t}{} \lr{ F^\dagger F }
=
\inv{c} \PD{t}{} \lr{ \calE - \frac{\BS}{c} }.
\end{dmath}
\begin{dmath}\label{eqn:poyntingTheoremRewrite:1300}
\gpgradezero{ \overbar{T}(\spacegrad) }
=
\gpgradezero{ 1 \overbar{T}(\spacegrad) }
=
\gpgradezero{ \spacegrad T(1) }
=
\spacegrad \cdot \gpgradeone{ T(1) }
=
\spacegrad \cdot \frac{\BS}{c}.
\end{dmath}
\begin{dmath}\label{eqn:poyntingTheoremRewrite:1320}
\gpgradeone{ \overbar{T}(\spacegrad) }
=
\sum_k \Be_k \lr{ \Be_k \cdot \gpgradeone{ \overbar{T}(\spacegrad) } }
=
\sum_k \Be_k \gpgradezero{ \Be_k \overbar{T}(\spacegrad) }
=
\sum_k \Be_k \gpgradezero{ \spacegrad T(\Be_k) }
=
\sum_k \Be_k \spacegrad \cdot \gpgradeone{ T(\Be_k) }
=
\sum_k \Be_k \spacegrad \cdot \BT(\Be_k)
=
\BT(\spacegrad).
\end{dmath}

\subsection{Poynting theorem.}

All the prerequisites for stating Poynting's theorem are now finally complete.
\maketheorem{Poynting's theorem (differential form.)}{thm:poyntingTheorem:1180}{
The adjoint energy momentum tensor of the spacetime gradient satisfies the following multivector equation
\begin{equation*}
\overbar{T}(\spacegrad + (1/c)\partial_t) = \frac{\epsilon}{2} \lr{ F^\dagger J + J^\dagger F }.
\end{equation*}
Note that the multivector \( F^\dagger J + J^\dagger F \) can only have scalar and vector grades, since it equals its reverse.
This equation can be put into a form that is more obviously a conservation law by stating it as a set of
scalar grade identities
\begin{equation*}
\spacegrad \cdot \gpgradeone{ T(a) } + \inv{c} \PD{t}{} \gpgradezero{ T(a) }
=
\frac{\epsilon}{2} \gpgradezero{ a( F^\dagger J + J \dagger F) }.
\end{equation*}
These may be written as respective
scalar and vector grades equations
%%which expands to the multivector equation
%\begin{equation*}
%\inv{c} \PD{t}{} \lr{ \calE - \frac{\BS}{c} }
%+ \spacegrad \cdot \frac{\BS}{c}
%+ \BT(\spacegrad)
%=
%-\inv{c} \lr{ \BE \cdot \BJ + \BH \cdot \BM }
%+
%\rho \BE + \epsilon \BE \cross \BM
%+
%\rho_\txtm \BH + \mu \BJ \cross \BH,
%\end{equation*}
%or as separate scalar and vector equations
\begin{equation*}
\begin{aligned}
\inv{c} \PD{t}{\calE} + \spacegrad \cdot \frac{\BS}{c} &= -\inv{c} \lr{ \BE \cdot \BJ + \BH \cdot \BM } \\
-\inv{c^2} \PD{t}{\BS} + \BT(\spacegrad) &= \rho \BE + \epsilon \BE \cross \BM + \rho_\txtm \BH + \mu \BJ \cross \BH.
\end{aligned}
\end{equation*}
Conventionally, only the scalar grade relating the time rate of change of the energy density to the flux of the Poynting vector, is called Poynting's theorem.
} % theorem

The conservation relationship of \cref{thm:poyntingTheorem:1180} follows from
\begin{dmath}\label{eqn:poyntingTheoremRewrite:1340}
F^\dagger \lr{ \lrspacegrad + \inv{c}\lroverarrow{\partial_t} } F
=
\lr{ \lr{ \spacegrad + \inv{c} \partial_t } F }^\dagger F
+
F^\dagger \lr{ \lr{ \spacegrad + \inv{c} \partial_t } F }
=
J^\dagger F + F^\dagger J.
\end{dmath}
The scalar form of
\cref{thm:poyntingTheorem:1180}
follows from
\begin{dmath}\label{eqn:poyntingTheoremRewrite:1360}
\gpgradezero{ a \overbar{T}(\spacegrad + (1/c) \partial_t) }
=
\gpgradezero{ (\spacegrad + (1/c) \partial_t) T(a) }
=
\spacegrad \cdot \gpgradeone{ T(a) } + \inv{c} \PD{t}{} \gpgradezero{ T(a) }.
\end{dmath}

We may use the scalar form of the theorem to extract the scalar grade, by setting \( a = 1 \), for which the right hand side
can be reduced to a single term
since scalars are reversion invariant
\begin{equation}\label{eqn:poyntingTheoremRewrite:1000}
\gpgradezero{ F^\dagger J }
=
\gpgradezero{ F^\dagger J }^\dagger
=
\gpgradezero{ J^\dagger F },
\end{equation}
so
\begin{dmath}\label{eqn:poyntingTheoremRewrite:960}
\spacegrad \cdot \gpgradeone{ T(1) }
+ \inv{c} \PD{t}{} \gpgradezero{ T(1) }
=
\spacegrad \cdot \frac{\BS}{c} + \inv{c} \PD{t}{\calE}
=
\frac{\epsilon}{2} \gpgradezero{ F^\dagger J + J^\dagger F }
=
\epsilon
\gpgradezero{ F^\dagger J }
=
\epsilon
\gpgradezero{
   \lr{ \BE -I \eta \BH }\lr{
      \eta \lr{ c \rho - \BJ } + I \lr{ c \rho_m - \BM }
   }
}
=
\epsilon
\lr{
   -\eta \BE \cdot \BJ -\eta \BH \cdot \BM
}
=
- \inv{c} \BE \cdot \BJ - \inv{c} \BH \cdot \BM,
\end{dmath}
which proves the claimed explicit expansion of the scalar grade selection of Poynting's theorem.

The left hand side of the vector grade selection follows by linearity using \cref{thm:poyntingTheoremRewrite:1280}
\begin{dmath}\label{eqn:poyntingTheoremRewrite:1380}
\gpgradeone{ \overbar{T}(\spacegrad + (1/c)\partial_t) }
=
\gpgradeone{ \overbar{T}(\spacegrad) + \overbar{T}((1/c)\partial_t) }
=
\BT(\spacegrad) - \inv{c^2} \PD{t}{\BS}.
\end{dmath}
The right hand side is a bit messier to simplify.
Let's do this in pieces by superposition, first considering just electric sources
\begin{dmath}\label{eqn:poyntingTheoremRewrite:80}
\frac{\epsilon}{2} \gpgradezero{ \Be_k \lr{ F^\dagger J + J^\dagger F} }
=
\frac{\epsilon \eta}{2}
\gpgradezero{ \Be_k \lr{ (\BE - I \eta \BH)(c \rho - \BJ)  + (c \rho - \BJ)( \BE + I \eta \BH )} }
=
\frac{1}{2c } \Be_k \cdot
\gpgradeone{ (\BE - I \eta \BH)(c \rho - \BJ)  + (c \rho - \BJ)( \BE + I \eta \BH ) }
=
\inv{c} \Be_k \cdot \lr{ c \rho \BE + I \eta \BH \wedge \BJ }
=
\inv{c} \Be_k \cdot \lr{ c \rho \BE - \eta \BH \cross \BJ }
=
\Be_k \cdot \lr{ \rho \BE + \mu \BJ \cross \BH },
\end{dmath}
and then magnetic sources
\begin{dmath}\label{eqn:poyntingTheoremRewrite:180}
\frac{\epsilon}{2 } \gpgradezero{ \Be_k \lr{ F^\dagger J + J^\dagger F} }
=
\frac{\epsilon }{2} \gpgradezero{ \Be_k \lr{ (\BE - I \eta \BH) I (c \rho_\txtm - \BM)  - I (c \rho_\txtm - \BM)( \BE + I \eta \BH )} }
=
\frac{\epsilon }{2} \Be_k \cdot \gpgradeone{ (I \BE + \eta \BH)(c \rho_\txtm - \BM)  + (c \rho_\txtm - \BM)( -I \BE + \eta \BH )}
=
\epsilon \Be_k \cdot \lr{ \eta c \rho_\txtm \BH - I \BE \wedge \BM }
=
\Be_k \cdot \lr{ \rho_\txtm \BH + \epsilon \BE \cross \BM }.
\end{dmath}
Jointly,
\cref{eqn:poyntingTheoremRewrite:1380}, \cref{eqn:poyntingTheoremRewrite:80}, \cref{eqn:poyntingTheoremRewrite:180} complete the proof.

The integral form of \cref{thm:poyntingTheorem:1180} submits nicely to physical interpretation.
\maketheorem{Poynting's theorem (integral form.)}{thm:poyntingTheoremRewrite:1420}{
\begin{dmath}\label{eqn:poyntingTheoremRewrite:1400}
\begin{aligned}
&\PD{t}{}
\int_V \calE dV
=
-\int_{\partial V} dA \ncap \cdot \BS
-
\int_V dV \lr{
   \BJ \cdot \BE
   +
   \BM \cdot \BH
} \\
&
\int_V dV \lr{ \rho \BE + \BJ \cross \BB }
+ \int_V dV \lr{ \rho_\txtm \BH - \epsilon \BM \cross \BE }
=
-
\PD{t}{ }
\int_V dV \bcP
+
\int_{\partial V} dA \BT(\ncap).
\end{aligned}
\end{dmath}
} % theorem

Proof of \cref{thm:poyntingTheoremRewrite:1420} requires only the divergence theorem, \cref{thm:poyntingTheoremRewrite:1281}, and
\cref{dfn:poyntingF:1220}.

The scalar integral in \cref{thm:poyntingTheoremRewrite:1420}
relates the rate of change of total energy in a volume to the flux of the Poynting through the surface bounding the volume.
If the energy in the volume increases(decreases), then in a current free region, there must be Poynting flux into(out-of) the volume.
The direction of the Poynting vector is the direction that the energy is leaving the volume, but only the projection of the Poynting vector along the normal direction contributes to this energy loss.

The right hand side of the vector integral in \cref{thm:poyntingTheoremRewrite:1420} is a continuous representation of the Lorentz force
(or dual Lorentz force for magnetic charges),
the mechanical force on the charges in the volume.  This can be seen by setting \( \BJ = \rho \Bv \) (or \( \BM = \rho_\txtm \BM \))
\begin{dmath}\label{eqn:poyntingTheoremRewrite:1420}
\int_V dV \lr{ \rho \BE + \BJ \cross \BB }
=
\int_V dV \rho \lr{ \BE + \Bv \cross \BB }
=
\int_V dq \lr{ \BE + \Bv \cross \BB }.
\end{dmath}

As the field in the volume is carrying the (electromagnetic) momentum \( \Bp_{\textrm{em}} = \int_V dV \bcP \), we can identify the sum of the Maxwell stress tensor's normal component over the bounding integral as time rate of change of the mechanical and electromagnetic momentum
\boxedEquation{eqn:poyntingTheoremRewrite:1440}{
\frac{d\Bp_{\textrm{mech}}}{dt} + \frac{d\Bp_{\textrm{em}}}{dt} = \int_{\partial V} dA \BT(\ncap).
}

%}

%%
% Copyright © 2017 Peeter Joot.  All Rights Reserved.
% Licenced as described in the file LICENSE under the root directory of this GIT repository.
%
\subsection{Complex power.}
TODO.
%\index{complex power}

%%
% Copyright © 2018 Peeter Joot.  All Rights Reserved.
% Licenced as described in the file LICENSE under the root directory of this GIT repository.
%
%{
\index{far field}
%
% Copyright � 2018 Peeter Joot.  All Rights Reserved.
% Licenced as described in the file LICENSE under the root directory of this GIT repository.
%
\maketheorem{Far field magnetic vector potential.}{thm:potentialSection_farfield:1}{
Given a vector potential with a radial spherical wave representation
%\label{eqn:potentialSection_farfield:2400}
\begin{equation*}
\BA = \frac{e^{-j k r}}{r} \bcA( \theta, \phi ),
\end{equation*}
the far field (\(r \gg 1 \)) electromagnetic field is
%\label{eqn:potentialSection_farfield:2520}{
\begin{equation*}
F = -j \omega \lr{ 1 + \rcap } \lr{ \rcap \wedge \BA}.
\end{equation*}
If \( \BA_\perp = \rcap \lr{ \rcap \wedge \BA} \) represents the
non-radial component of the potential, the respective electric and magnetic field components are
%\begin{dmath}\label{eqn:potentialSection_farfield:2560}
\begin{equation*}
\begin{aligned}
\BE &= -j \omega \BA_\perp \\
\BH &= \inv{\eta} \rcap \cross \BE.
\end{aligned}
\end{equation*}
%\end{dmath}
} % theorem

\begin{proof}
To prove \cref{thm:potentialSection_farfield:1}, we will utilize a
spherical representation of the gradient
\begin{equation}\label{eqn:potentialSection_farfield:2420}
\begin{aligned}
\spacegrad &= \rcap \partial_r + \spacegrad_\perp \\
\spacegrad_\perp &= \frac{\thetacap}{r} \partial_\theta + \frac{\phicap}{r\sin\theta} \partial_\phi.
\end{aligned}
\end{equation}

The gradient of the vector potential is
\begin{equation}\label{eqn:potentialSection_farfield:2440}
\begin{aligned}
\spacegrad \BA
&= \biglr{ \rcap \partial_r + \spacegrad_\perp } \frac{e^{-j k r}}{r} \bcA \\
&= \rcap \lr{ -j k - \inv{r} } \frac{e^{-j k r}}{r} \bcA
   +
   \frac{e^{-j k r}}{r} \spacegrad_\perp \bcA \\
&= - \lr{ j k + \inv{r} } \rcap \BA + O(1/r^2) \\
&\approx - j k \rcap \BA.
\end{aligned}
\end{equation}

Here, all the \( O(1/r^2) \) terms, including the action of the non-radial component of the gradient on the \( 1/r \) potential, have been neglected.
From \cref{eqn:potentialSection_farfield:2440} the far field divergence and the (bivector) curl of \( \BA \) are
\begin{equation}\label{eqn:potentialSection_farfield:2460}
\begin{aligned}
\spacegrad \cdot \BA &= - j k \rcap \cdot \BA \\
\spacegrad \wedge \BA &= - j k \rcap \wedge \BA.
\end{aligned}
\end{equation}

Finally, the far field gradient of the divergence of \( \BA \) is
\begin{equation}\label{eqn:potentialSection_farfield:2480}
\begin{aligned}
\spacegrad \lr{ \spacegrad \cdot \BA }
&=
\biglr{ \rcap \partial_r + \spacegrad_\perp } \lr{ - j k \rcap \cdot \BA } \\
&\approx
-j k \rcap \partial_r \lr{ \rcap \cdot \BA } \\
&=
-j k \rcap \lr{ -j k - \inv{r} } \lr{ \rcap \cdot \BA } \\
&\approx
-k^2 \rcap \lr{ \rcap \cdot \BA },
\end{aligned}
\end{equation}
again neglecting any \( O(1/r^2) \) terms.  The field is
\begin{equation}\label{eqn:potentialSection_farfield:2500}
\begin{aligned}
F
&= - j \omega \BA  -j \frac{c^2}{\omega} \spacegrad \lr{ \spacegrad \cdot \BA } + c \spacegrad \wedge \BA \\
&= - j \omega \BA  +j \omega \rcap \lr{ \rcap \cdot \BA } - j k c \rcap \wedge \BA \\
&= - j \omega \lr{ \BA - \rcap \lr{ \rcap \cdot \BA }} - j \omega \rcap \wedge \BA \\
&= -j \omega \rcap \lr{ \rcap \wedge \BA} - j \omega \rcap \wedge \BA \\
&= -j \omega \lr{ \rcap + 1 } \lr{ \rcap \wedge \BA},
\end{aligned}
\end{equation}
which completes the first part of the proof.  Extraction of the electric and magnetic fields can be done by inspection and is left to the reader to prove.
\end{proof}

One interpretation of this is that the (bivector) magnetic field is represented by the plane perpendicular to the direction of propagation, and the electric field by a vector in that plane.

\maketheorem{Far field electric vector potential.}{thm:potentialSection_farfield:2}{
Given a vector potential with a radial spherical wave representation
%\label{eqn:potentialSection_farfield:2400}
\begin{equation*}
\BF = \frac{e^{-j k r}}{r} \bcF( \theta, \phi ),
\end{equation*}
the far field (\(r \gg 1 \)) electromagnetic field is
%\label{eqn:potentialSection_farfield:2520}
\begin{equation*}
F = -j \omega \eta I \lr{ \rcap + 1 } \lr{ \rcap \wedge \BF }.
\end{equation*}
If \( \BF_\perp = \rcap \lr{ \rcap \wedge \BF} \) represents the
non-radial component of the potential, the respective electric and magnetic field components are
%\begin{dmath}\label{eqn:potentialSection_farfield:2560}
\begin{equation*}
\begin{aligned}
\BE &= j \omega \eta \rcap \cross \BF \\
\BH &= -j \omega \BF_\perp.
\end{aligned}
\end{equation*}
%\end{dmath}
} % theorem

The proof of \cref{thm:potentialSection_farfield:2} is left to the reader.

\makeexample{Vertical dipole potential.}{example:potentialSection:1}{
We will calculate the far field along the propagation direction vector \( \kcap \) in the z-y plane
\begin{equation}\label{eqn:potentialSection_farfield:2620}
\begin{aligned}
\kcap &= \Be_3 e^{i \theta} \\
i &= \Be_{32},
\end{aligned}
\end{equation}
for the infinitesimal dipole potential
\begin{equation}\label{eqn:potentialSection_farfield:2640}
\BA = \frac{\Be_3 \mu I_0 l}{4 \pi r} e^{-j k r},
\end{equation}
as illustrated in \cref{fig:vectorPotential:vectorPotentialFig1}.

\mathImageFigure{../figures/GAelectrodynamics/vectorPotentialFig1}{Vertical infinitesimal dipole and selected propagation direction.}{fig:vectorPotential:vectorPotentialFig1}{0.3}{zcapPotential.nb}

The wedge of \( \kcap \) with \( \BA \) is proportional to
\begin{dmath}\label{eqn:potentialSection_farfield:2660}
\begin{aligned}
\kcap \wedge \Be_3
&= \gpgradetwo{ \kcap \Be_3 } \\
&= \gpgradetwo{ \Be_3 e^{i \theta} \Be_3 } \\
&= \gpgradetwo{ \Be_3^2 e^{-i \theta} } \\
&= -i \sin\theta,
\end{aligned}
\end{dmath}
so from \cref{thm:potentialSection_farfield:2} the field is
\begin{equation}\label{eqn:potentialSection_farfield:2680}
F = j \omega \lr{ 1 + \Be_3 e^{i\theta} } i \sin\theta \frac{\mu I_0 l}{4 \pi r} e^{-j k r}.
\end{equation}

The electric and magnetic fields can be found from the respective vector and bivector grades of \cref{eqn:potentialSection_farfield:2680}
\begin{equation}\label{eqn:potentialSection_farfield:2700}
\begin{aligned}
\BE
&= \frac{j \omega \mu I_0 l}{4 \pi r} e^{-j k r} \Be_3 e^{i\theta} i \sin\theta \\
&= \frac{j \omega \mu I_0 l}{4 \pi r} e^{-j k r} \Be_2 e^{i\theta} \sin\theta \\
&= \frac{j k \eta I_0 l \sin\theta}{4 \pi r} e^{-j k r} \lr{ \Be_2 \cos\theta - \Be_3 \sin\theta },
\end{aligned}
\end{equation}
and
\begin{equation}\label{eqn:potentialSection_farfield:2720}
\begin{aligned}
\BH
&= \inv{I \eta} j \omega i \sin\theta_0 \frac{\mu I_0 l}{4 \pi r} e^{-j k r} \\
&= \inv{\eta} \Be_{321} \Be_{32} j \omega \sin\theta_0 \frac{\mu I_0 l}{4 \pi r} e^{-j k r} \\
&= -\Be_1 \frac{ j k \sin\theta_0 I_0 l}{4 \pi r} e^{-j k r}.
\end{aligned}
\end{equation}

The multivector electrodynamic field expression
\cref{eqn:potentialSection_farfield:2680} for
\( F \) is more algebraically compact than the separate electric and magnetic field expressions, but this comes with the complexity of dealing with different types of imaginaries.
There are two explicit unit imaginaries in \cref{eqn:potentialSection_farfield:2680}, the scalar imaginary \( j \) used to encode the time harmonic nature of the field, and \( i = \Be_{32} \) used to represent the plane that the far field propagation direction vector lay in.
Additionally, when the magnetic field component was extracted, the pseudoscalar \( I = \Be_{123} \) entered into the mix.
Care is required to keep these all separate, especially since \( I, j \) commute with all grades, but \( i \) does not.
} % example

%}

%%
% Copyright © 2017 Peeter Joot.  All Rights Reserved.
% Licenced as described in the file LICENSE under the root directory of this GIT repository.
%
%\makeexample{Line charge.}{example:linecharge:linecharge}{
\index{line charge}
In this example the (electric) field is calculated at a point on the z-axis, due to a finite line charge density of \( \lambda \) along a segment \( [a,b] \) of the x-axis.
The geometry of the problem is illustrated in \cref{fig:linecharge:linechargeFig1}.
\pmathImageFigure{../figures/GAelectrodynamics/}{linechargeFig1}{Line charge density.}{fig:linecharge:linechargeFig1}{0.3}{LineChargeIntegralsAndFigure.nb}

This is a fairly simple problem, and can be found in most introductory electromagnetic texts, usually set with the field observation point on the z-axis, and with a symmetric interval \( [-l/2, l/2] \), which has the side effect of killing off all but the x-axis component of the field.  For comparision purposes, this problem will be tackled first using conventional algebra, and then using geometric algebra.

\paragraph{Conventional approach.}

The integral we wish to evaluate is
\begin{equation}\label{eqn:linecharge:160}
\BE(\Bx) = \frac{\lambda}{4 \pi \epsilon} \int_a^b dx \frac{ (r \cos\theta - x) \Be_1 + r \sin\theta \Be_3 }{ \lr{ r^2 + x^2 - 2 r x \cos\theta }^{3/2} }.
\end{equation}

This can be non-dimensionalized with a \( u = x/r \) change of variables, and yields an integral for the x component and the z component of the field
\begin{equation}\label{eqn:linecharge:180}
\begin{aligned}
E_x &= \frac{\lambda}{4 \pi \epsilon r} \int_{a/r}^{b/r} du \frac{ \cos\theta - u }{ \lr{ 1 + u^2 - 2 u \cos\theta }^{3/2} } \\
E_y &= \frac{\lambda \sin\theta}{4 \pi \epsilon r} \int_{a/r}^{b/r} du \lr{ 1 + u^2 - 2 u \cos\theta }^{-3/2}.
\end{aligned}
\end{equation}
There is a common integral in the \(x\) and \(y\) components of the field.  We can tidy this up a bit by writing
\begin{equation}\label{eqn:linecharge:200}
\begin{aligned}
A &= \int_{a/r}^{b/r} du \lr{ 1 + u^2 - 2 u \cos\theta }^{-3/2} \\
B &= \int_{a/r}^{b/r} u du \lr{ 1 + u^2 - 2 u \cos\theta }^{-3/2},
\end{aligned}
\end{equation}
and then put the pieces back together again for the total field
\begin{equation}\label{eqn:linecharge:220}
\BE = \frac{\lambda}{4 \pi \epsilon r} \lr{ (A \cos\theta - B) \Be_1 + A \sin\theta \Be_3 }.
\end{equation}

Some additional structure can be imposed by introducing a rotation matrix to express the field observation point
\begin{equation}\label{eqn:linecharge:240}
\Bx = r \BR_\theta \Be_1,
\end{equation}
where
\begin{equation}\label{eqn:linecharge:260}
\BR_\theta =
\begin{bmatrix}
\cos\theta & 0 & -\sin\theta \\
0 & 1 & 0 \\
\sin\theta & 0 & \cos\theta \\
\end{bmatrix}.
\end{equation}
Writing \( \BOne \) for the \R{3} identity matrix, the field is
\begin{equation}\label{eqn:linecharge:280}
\BE = \frac{\lambda}{4 \pi \epsilon r} \lr{ A \BR_\theta - B \BOne } \Be_1.
\end{equation}
In retrospect we could have started using \cref{eqn:linecharge:240} and obtained this result more directly.
The \( A \) integral above results in both scaling and rotation of the field, depending on the observation point and the limits of the integration.  The \( B \) integral contributes only to the x-axis oriented component of the field.

\paragraph{Using geometric algebra.}

Introducing a unit imaginary \( i = \Be_{13} \) for the rotation from the x-axis to the z-axis, the field point observation point is
\begin{equation}\label{eqn:linecharge:120}
\Bx = r \Be_1 e^{i \theta}.
\end{equation}

The charge element point is \( \Bx' = x \Be_1 \), so the difference can now be written with \( \Be_1 \) factored to the left or to the right
\begin{equation}\label{eqn:linecharge:20}
\Bx - \Bx'
= \Be_1\lr{ r e^{i\theta} - x }
= \lr{ r e^{-i\theta} - x } \Be_1.
\end{equation}
These left and right factors can be used to convert the squared length of \( \Bx - \Bx' \) into from a vector product into a product of conventional looking complex conjugates
\begin{equation}\label{eqn:linecharge:40}
\begin{aligned}
\lr{ \Bx - \Bx' }^2
&= \lr{ r e^{-i\theta} - x } \Be_1 \Be_1\lr{ r e^{i\theta} - x } \\
&= \lr{ r e^{-i\theta} - x } \lr{ r e^{i\theta} - x },
\end{aligned}
\end{equation}
so the squared length of the difference is
\begin{equation}\label{eqn:linecharge:300}
\begin{aligned}
\lr{ \Bx - \Bx' }^2
&= r^2 + x^2 - r x \lr{ e^{i\theta} + e^{-i\theta} } \\
&= r^2 + x^2 - 2 r x \cos\theta,
\end{aligned}
\end{equation}
and the total (electric) field is
\begin{equation}\label{eqn:linecharge:60}
\begin{aligned}
F
&= \frac{\lambda}{4 \pi \epsilon} \int_a^b dx \frac{ r \Be_1 e^{i\theta} - x \Be_1 }{ \lr{ r^2 + x^2 - 2 x r \cos\theta }^{3/2} } \\
&= \frac{\lambda \Be_1}{4 \pi \epsilon r} \int_{a/r}^{b/r} du \frac{ e^{i\theta} - u }{ \lr{ 1 + u^2 - 2 u \cos\theta }^{3/2} }.
\end{aligned}
\end{equation}
We have replaced the matrix representation that had nine components, four zeros, and a lot of redundancy with a simple multivector result.
Moreover, the integral factor has the appearance of a conventional complex integral, and we can toss it as is into any numerical or symbol integration systems capable of complex number integrals for evaluation.
The end result is a single vector valued inverse radial factor \( \lambda \Be_1/ (4 \pi \epsilon r) \), multiplying by an integral that served to either scale or rotate-and-scale.

In particular, for \( \theta = \pi/2 \), plugging this integral into Mathematica, we find
\begin{equation}\label{eqn:linecharge:80}
\int
du \frac{ e^{i\theta} - u }{ \lr{ 1 + u^2 - 2 u \cos\theta }^{3/2} }
= \frac{1 + i u}{\sqrt{1 + u^2}},
\end{equation}
and for other angles \( \theta \neq n \pi/2 \)
\begin{equation}\label{eqn:linecharge:100}
\int
du \frac{ e^{i\theta} - u }{ \lr{ 1 + u^2 - 2 u \cos\theta }^{3/2} }
= \frac{(1 -u e^{-i\theta}) \sqrt{1 + u^2 - 2 u \cos\theta}}{(1 + u^2) \sin(2\theta)}.
\end{equation}

The numerator factors like \( \Be_1 (1 + i u) \) and \( \Be_1(1 - u e^{-i\theta}) \)
compactly describe the direction of the vector field at the observation point.
Either of these can be expanded explicitly in sines and cosines if desired
\begin{equation}\label{eqn:linecharge:140}
\begin{aligned}
\Be_1 (1 + i u) &= \Be_1 + u \Be_3 \\
\Be_1(1 - u e^{-i\theta}) &= \Be_1(1 - u \cos\theta) + u \Be_3 \sin\theta.
\end{aligned}
\end{equation}

\index{complex plane}
Perhaps more interesting than the precise form of the solution is the fact that geometric algebra allows for the introduction of a ``complex plane'' for many problems that have only two degrees of freedom.
When such a complex plane is introduced, existing Computer Algebra Systems (CAS), like Mathematica, can be utilized for the grunt work of the evaluation.

%} % example

%%
% Copyright © 2017 Peeter Joot.  All Rights Reserved.
% Licenced as described in the file LICENSE under the root directory of this GIT repository.
%
The multiplication table for the \R{2} geometric algebra can be computed with relative ease.
Many of the interesting products involve \( i = \Be_1 \Be_2 \), the unit pseudoscalar.
%\cref{eqn:normalVectors:140} % reference dead
The imaginary nature of the pseudoscalar can be demonstrated using \cref{thm:multiplication:anticommutationNormal}
\begin{dmath}\label{eqn:2dMultiplication:220}
   \lr{ \Be_1 \Be_2 }^2
   =
   (\Be_1 \Be_2)(\Be_1 \Be_2)
   =
   -(\Be_1 \Be_2)(\Be_2 \Be_1)
   =
   -\Be_1 (\Be_2^2 ) \Be_1
   =
   -\Be_1^2
   = -1.
\end{dmath}

\index{complex imaginary}
Like the (scalar) complex imaginary, the bivector \( \Be_1 \Be_2 \) also squares to \( -1 \).
The only non-trivial products left to fill in the \R{2} multiplication table are those of the unit vectors with \( i \), products that are order dependent
\begin{dmath}\label{eqn:2dMultiplication:180}
\begin{aligned}
   \Be_1 i &= \Be_1 \lr{ \Be_1 \Be_2 } \\
           &= \lr{ \Be_1 \Be_1 } \Be_2 \\
           &= \Be_2 \\
   i \Be_1 &= \lr{ \Be_1 \Be_2 } \Be_1 \\
           &= \lr{ -\Be_2 \Be_1 } \Be_1 \\
           &= -\Be_2 \lr{ \Be_1 \Be_1 } \\
           &= -\Be_2 \\
   \Be_2 i &= \Be_2 \lr{ \Be_1 \Be_2 } \\
           &= \Be_2 \lr{ -\Be_2 \Be_1 } \\
           &= -\lr{ \Be_2 \Be_2 }\Be_1 \\
           &= -\Be_1 \\
   i \Be_2 &= \lr{ \Be_1 \Be_2 } \Be_2 \\
           &= \Be_1 \lr{ \Be_2 \Be_2 } \\
           &= \Be_1.
\end{aligned}
\end{dmath}

The multiplication table for the \R{2} multivector basis can now be tabulated

%FIXME: how to reference a tcolorbox table?
% examples in http://ctan.mirrors.hoobly.com/macros/latex/contrib/tcolorbox/tcolorbox.pdf section 5.1
% requires setting up a counter variable like some of the others (theorem environments)

% various options for prettier than default table:
% https://tex.stackexchange.com/a/135421/15
% https://tex.stackexchange.com/a/298109/15
% https://tex.stackexchange.com/a/112359/15
%\captionedTable{2D Multiplication table.}{tab:2dMultiplication:10}{
%\begin{tabular}{|l||l|l|l|l|}
%\hline
%        & \( 1 \) & \( \Be_1 \) & \( \Be_2 \) & \( \Be_1 \Be_2 \) \\ \hline
%\( 1 \) & \( 1 \) & \( \Be_1 \) & \( \Be_2 \) & \( \Be_1 \Be_2 \) \\ \hline
%\( \Be_1\) & \( \Be_1 \) & \( 1 \) & \( \Be_1 \Be_2 \) & \( \Be_2 \)\\ \hline
%\( \Be_2\) & \( \Be_2 \) & \( -\Be_1 \Be_2 \) & \( 1 \) & \( -\Be_1 \)\\ \hline
%\( \Be_1 \Be_2\) & \( \Be_1 \Be_2 \) & \( -\Be_2 \) & \( \Be_1 \) & \( -1 \) \\ \hline
%\end{tabular}
%}

\begin{tablelabelbox}[tabularx={X||Y|Y|Y|Y}]{2D Multiplication table.}{label=tab:2dMultiplication:10}
        & \( 1 \) & \( \Be_1 \) & \( \Be_2 \) & \( \Be_1 \Be_2 \) \\ \hline
\( 1 \) & \( 1 \) & \( \Be_1 \) & \( \Be_2 \) & \( \Be_1 \Be_2 \) \\ \hline
\( \Be_1\) & \( \Be_1 \) & \( 1 \) & \( \Be_1 \Be_2 \) & \( \Be_2 \)\\ \hline
\( \Be_2\) & \( \Be_2 \) & \( -\Be_1 \Be_2 \) & \( 1 \) & \( -\Be_1 \)\\ \hline
\( \Be_1 \Be_2\) & \( \Be_1 \Be_2 \) & \( -\Be_2 \) & \( \Be_1 \) & \( -1 \) \\ \hline
\end{tablelabelbox}
%\begin{tcolorbox}[tab2,tabularx={X||Y|Y|Y|Y},title=2D Multiplication table.,boxrule=0.5pt]
%        & \( 1 \) & \( \Be_1 \) & \( \Be_2 \) & \( \Be_1 \Be_2 \) \\ \hline
%\( 1 \) & \( 1 \) & \( \Be_1 \) & \( \Be_2 \) & \( \Be_1 \Be_2 \) \\ \hline
%\( \Be_1\) & \( \Be_1 \) & \( 1 \) & \( \Be_1 \Be_2 \) & \( \Be_2 \)\\ \hline
%\( \Be_2\) & \( \Be_2 \) & \( -\Be_1 \Be_2 \) & \( 1 \) & \( -\Be_1 \)\\ \hline
%\( \Be_1 \Be_2\) & \( \Be_1 \Be_2 \) & \( -\Be_2 \) & \( \Be_1 \) & \( -1 \) \\ \hline
%\end{tcolorbox}

\index{pseudoscalar}
It is important to point out that the
pseudoscalar \( i \) does not commute with either basis vector, but anticommutes with both, since \( i \Be_1 = - \Be_1 i \), and \( i \Be_2 = - \Be_2 i \).
By superposition \( i \) anticommutes with any vector in the x-y plane.

More generally, if \( \Bu \) and \( \Bv \) are orthonormal, and \( \Bx \in \Span\setlr{\Bu, \Bv} \) then the bivector \( \Bu \Bv \) anticommutes with \( \Bx \), or any other vector in this plane.

%\ref{tab:2dMultiplication:10}.


%
\section{Junk?}

\subsection{Problems}
%\makeproblem{Explicit squared norm}{problem:multivector:60}{
%   Given a coordinate representation of a vector with respect to a standard basis
%\begin{dmath}\label{eqn:multivector:240}
%   \Bx = \sum_{i = 1}^N x_i \Be_i,
%\end{dmath}
%
%show that the squared norm is
%\begin{dmath}\label{eqn:multivector:260}
%   \Norm{\Bx}^2 = \Bx \cdot \Bx = \sum_{i = 1}^N x_i^2 (\Be_i \cdot \Be_i).
%\end{dmath}
%
%Observe that for a Euclidean vector space this is the squared length in the Pythagorean sense.
%}
%
\makeproblem{Null vector}{problem:multivector:80}{
Given a two dimensional non-Euclidean vector space with basis elements satisfying
\( \gamma_0 \cdot \gamma_0 = 1 = -\gamma_1 \cdot \gamma_1 \), construct a vector that has a squared
norm of 0.  Such a vector is called a null vector.
%   \Bx = \gamma_0 + \gamma_1,
}


\subsection{basis, norm, ...}

%We will use a representation such as \( \Bv = x \Be_1 + y \Be_2 + z \Be_3 \) for such vectors, where the
%coordinates are always paired with their respective direction vectors, and will not use
%column vector of coordinates or tuples such as \( \Bv = (x, y, z)\).
%, \Bv = x \xcap + y \ycap + z \zcap, or \Bv = x \ahat_x + y \ahat_y + z \ahat_z.
\makedefinition{Coordinates.}{dfn:prerequisites:coordinates}{
%Given a basis \( B =
FIXME: define
} % definition

%\makedefinition{Basis and coordinates}{dfn:multivector:basis}{
%   If \( N \) is the dimension of a vector space \( V \), a set of \( N \) vectors \( B = \setlr{ \Ba_1, \Ba_2, \cdots , \Ba_N } \) is a basis for that vector space, if it is possible to form any vector \( \Bx \in V \) as a linear combination of those vectors \( \Ba_k \).  That is, there exists scalars \( c_k \) such that for any \( \Bx \in V \)
%
%\begin{equation*}
%   \Bx = \sum_{k = 1}^N c_k \Ba_k.
%\end{equation*}
%
%The numbers \( (c_1, c_2, \cdots, c_N ) \) are referred to as the coordinates of the vector \( \Bx \) with respect to the basis \( B \).
%}

\makedefinition{Standard basis, and dot product properties.}{dfn:multivector:standardbasis}{
   Any vector space \( V \) used in this book will be assumed to have been generated from a basis \( \setlr{ \Be_1, \Be_2, \cdots, \Be_N } \), associated with a dot product that has the properties

\begin{enumerate}
   \item \( \Be_i \cdot \Be_i = \pm 1 \).
   \item \( \Be_i \cdot \Be_j = 0 \) for any \( i \ne j \).
\end{enumerate}

Such a basis will called a standard basis.  When these dot products are always positive, the vector space is referred to as a Euclidean vector space.
}

\paragraph{FIXME: remove?}
There are many possible standard bases sets.  In \R{3}, it is conventional to refer to \( \Be_1, \Be_2, \Be_3 \) as the standard bases elements if these represent the directions of the x, y, and z directions respectively.  Unless otherwise noted \( \Be_k \) refers to the direction vector for the k-th direction in a standard basis for that space.
The only non-Euclidean vector space of interest in this book (for relativistic material), has a Minkowski dot product.  For such a space, the standard basis elements will be labeled \( \setlr{ \gamma_0, \gamma_1, \gamma_2, \gamma_3 } \), where for \( i \in [1,3] \), \( \gamma_0 \cdot \gamma_0 = \pm 1 = -\gamma_i \cdot \gamma_i \).  The positive sign convention will be used.

%GA requires the vector space to have an associated
%dot product \( \Bx \cdot \By \) that
%defines the notion of perpendicularity for the space.  We will want to extend the scalar multiplication operation of the vector
%space to complex numbers, but
%will not require a (complex) order dependent inner product \( \innerprod{\Bx}{\By} \) for our vector space.
%

\paragraph{The metric, length and normality.}

An abstract vector need not have an associated notion of length, nor a notion of perpendicularity (normality).
In abstract vector algebra, length and normality are provided by defining an associated dot product \(\Bx \cdot \By\), or inner product \(\innerprod{\Bx}{\By}\).
In GA, length and normality of two vectors are provided by a metric \(g(\Bx, \By)\).
Like the dot product where \( \Bx \cdot \By = \By \cdot \Bx\), this metric is independent of order, a property that is not generally required of the inner product.
However, unlike both the dot and inner products of abstract vector algebra, where \( \Bx \cdot \Bx \ge 0\), and \( \innerprod{\Bx}{\Bx} \ge 0\), the metric \(g(\Bx, \Bx)\) may be negative (i.e. for spacetime vectors).
If \(c \) is any real or complex number, the metric in GA is \( g(c \Bx, c \Bx) = c^2 g(\Bx, \Bx)\), unlike the inner product in complex spaces, where \( \innerprod{c \Bx}{c \Bx} = \Abs{c}^2 \innerprod{c \Bx}{c \Bx} \).
Effectively, this means that our underlying direction vectors are always real.

\subsection{Orientation}
We are familiar with the idea of an oriented line segment (a vector), a quantity that can be visualized as an arrow with direction and magnitude.
The idea of an oriented plane, volume, or hypervolume is probably less familiar.
An oriented plane segment, in addition to having a specific area and a direction in space, can be visualized as having a
circulation direction, or handedness.
In a three dimensional space, this circulation direction can be associated with one of the two possible normal directions for the plane.
An oriented volume, in addition to having a given magnitude, is considered to have an associated circulation direction along its surface.
In a three dimensional space, an oriented volume can be conceptualized as a volume with either an inwards or outwards normal.

\subsection{dot and metric original text}

Vectors are often represented with an implied basis, with tuples like \( \Bx = (x,y,z) \), or with column (or row) vectors like
\(
   \Bx =
\begin{bmatrix}
x \\
y \\
z
\end{bmatrix}
\).
The values \( x, y, z \) in these representations are called the coordinates of the vectors, but only have specific meaning once a direction and magnitude is associated with each coordinate (i.e. a basis is chosen).
In three dimensions, the simplest such basis choice (the standard basis), associates the respective coordinates with a set of mutually perpendicular (normal) directions.
This is conventionally a right handed triple of direction vectors of unit length, perhaps designated \( \xcap, \ycap, \zcap \) or \( \Be_1, \Be_2, \Be_3 \).

In GA, when working with coordinates, we generally prefer to make the basis explicit, so instead of writing a vector as a set of coordinates, these coordinates
will be explicitly paired with their associated basis vectors.
For example in \R{3} a vector with coordinates \( x, y, z \) will be written as

\begin{dmath}\label{eqn:prerequisites:280}
x \Be_1 + y \Be_2 + z \Be_3.
\end{dmath}

By convention, we understand that \( \Be_1, \Be_2, \Be_3 \) in \cref{eqn:prerequisites:280} are unit length vectors, and are all mutually perpendicular (orthonormal).
The vector space must be augmented with a dot product (or inner product) to provide a measure of length and normality.  

%\makedefinition{Inner product.}{dfn:prerequisites:innerproduct}{
%The inner product 
%} % definition

For \R{3}, the dot product satisfies the following conditions

\begin{equation}\label{eqn:prerequisites:320}
\Be_i \cdot \Be_j = \delta_{ij} \, \forall i, j \in [1,3],
\end{equation}

where \( \delta_{ij} \) is the Kronecker delta \( \delta_{ij} = 1 \) for \( i = j \) and \( \delta_{ij} = 0 \) for \( i \ne j \).
Specifying the action of the dot product on all the unit vectors, completely specifies the action of the dot product on any two vectors, provided one assumes that the dot product is a bilinear operator.
For example, given

\begin{dmath}\label{eqn:prerequisites:340}
\begin{aligned}
\Ba &= a_1 \Be_1 + a_2 \Be_2 + a_3 \Be_3 \\
\Bb &= b_1 \Be_1 + b_2 \Be_2 + b_3 \Be_3,
\end{aligned}
\end{dmath}

or \( \Ba = \sum_i a_i \Be_i, \Bb = \sum_j b_j \Be_j \), we recover the familiar coordinate description of the dot product

\begin{dmath}\label{eqn:prerequisites:360}
\Ba \cdot \Bb
=
\lr{ \sum_i a_i \Be_i } \cdot \lr{ \sum_j b_j \Be_j }
=
\sum_{i,j} a_i b_j \lr{ \Be_i \cdot \Be_j }
=
\sum_{i,j} a_i b_j \delta_{ij}
=
\sum_{i} a_i b_i.
\end{dmath}

Electromagnetism is intrinsically relativistic, and there will be circumstances where vectors with both space and time components are required.
In physics, these are called four-vectors, but we will call them spacetime vectors here to avoid confusion with \( k = 4 \) k-vectors.
Following \citep{doran2003gap}, the Dirac (matrix) notation will be used as the relativistic basis, so a spacetime vector might be written like

\begin{dmath}\label{eqn:prerequisites:300}
A = c t \gamma_0 + x \gamma_1 + y \gamma_2 + z \gamma_3.
\end{dmath}

It will be seen later that our spacetime vector representation has similar properties to Dirac matrices, but we need not refer to any specific matrix representation.

For spacetime vectors, we can also assume a dot product operation between the basis vectors.  For example, given two spacetime vectors

\begin{dmath}\label{eqn:prerequisites:380}
\begin{aligned}
A &= c t \gamma_0 + x \gamma_1 + y \gamma_2 + z \gamma_3 \\
B &= c t' \gamma_0 + x' \gamma_1 + y' \gamma_2 + z' \gamma_3,
\end{aligned}
\end{dmath}

if the action of a ``dot-product'' is known between all basis vectors \( \gamma_\mu, \mu \in [0,3] \), then it will be possible to compute the dot-product of any pair of four vectors as done above for the \R{3} example.  Special relativity constrains the properties of four-vector dot products, requiring the following of the four-vector basis

\begin{dmath}\label{eqn:prerequisites:400}
\left\{
\begin{array}{l l}
\gamma_\mu \cdot \gamma_\nu = 0 & \quad \mbox{ \( \mu \ne \nu ; \mu, \nu \in [0,3] \) } \\
\gamma_0 \cdot \gamma_0 = -\gamma_i \cdot \gamma_i = \pm 1 & \quad \mbox{ \( i \in [1,3] \) }
\end{array}
\right.
\end{dmath}

Strictly speaking, this is a specification of a metric, not a dot product, since this four vector dot product specification does not satisfy the positive definite property required by most dot product definitions (i.e. \( A \cdot A \ge 0 \)).
There is a sign ambiguity in the metric specification above.  The physics of relativity is independent of the sign convention used, but we will use the positive sign convention, consistent with field theory and most matrix representations of the Dirac matrices.
\footnote{In general relativitity, many authors will use the opposite sign convention.}

Stated explicitly, we use a metric where the basis vectors satisfy the following properties

\begin{dmath}\label{eqn:prerequisites:420}
\left\{
\begin{array}{l l}
\gamma_\mu \cdot \gamma_\nu = 0 & \quad \mbox{ \( \mu \ne \nu ; \mu, \nu \in [0,3] \) } \\
\gamma_i \cdot \gamma_i = -1& \quad \mbox{ \( i \in [1,3] \) } \\
\gamma_0 \cdot \gamma_0 = 1. &\\
\end{array}
\right.
\end{dmath}


%%%\makeproblem{}{problem:multivector:50}{
%%%The most general definition of an Euclidean norm satisfies all of the properties
%%%
%%%\begin{enumerate}
%%%   \item \( \Norm{\Bx} \ge 0 \), and \( \Norm{\Bx} = 0 \iff \Bx = 0 \).
%%%   \item \( \Norm{a \Bx} = \Abs{a} \Norm{\Bx} \).
%%%   \item \( \Norm{\Bx + \By} \le \Norm{\Bx} + \Norm{\By} \).
%%%\end{enumerate}
%%%
%%%If the coordinates of a vector with respect to the standard basis are \( x_i \) then show that the Euclidean norm defined in
%%%that the Pythagorean norm
%%%\begin{equation*}
%%%\Norm{\Bx}^2 = \sum_{i = 1}^N x_i^2,
%%%\end{equation*}
%%%
%%%satisfies these properties.
%%%} % problem
%%%

%
% Copyright � 2016 Peeter Joot.  All Rights Reserved.
% Licenced as described in the file LICENSE under the root directory of this GIT repository.
%

%
%\chapter{Preface}
% this suppresses an explicit chapter number for the preface.
\chapter*{Preface}%\normalsize
  \thispagestyle{empty}
  \addcontentsline{toc}{chapter}{Preface}

\paragraph{Why you want to read this book.}
When you first learned vector algebra you learned how to add and subtract vectors, and probably asked your instructor if it was possible to multiply vectors.  Had you done so, you would have been told either ``No'', or a qualified ``No, but we can do multiplication like operations, the dot and cross products.''  This book is based on a different answer, ``Yes.''  A set of rules that define a coherent multiplication operation are provided.

Were you ever bothered by the fact that the cross product was only defined in three dimensions, or had a nagging intuition that the dot and cross products were related somehow?  The dot product and cross product seem to be complimentary, with the dot product encoding a projection operation (how much of a vector lies in the direction of another), and the magnitude of the cross product providing a rejection operation (how much of a vector is perpendicular to the direction of another).  These projection and rejection operations should be perfectly well defined in 2, 4, or N dimensions, not just 3.  In this book you will see how to generalize the cross product to N dimensions, and how this more general product (the wedge product) is useful even in the two and three dimensional problems that are of interest for physical problems (like electromagnetism.)  You will also see how the dot, cross (and wedge) products are all related to the vector multiplication operation of geometric algebra.

When you studied vector calculus, did the collection of Stokes's, Green's and Divergence theorems available seem too random, like there ought to be a higher level structure that described all these similar operations?  It turns out that such structure is available in the both the language of differential forms, and that of tensor calculus.  We'd like a toolbox that doesn't require expressing vectors as differentials, or resorting to coordinates.  Not only does geometric calculus provides such a toolbox, it also provides the tools required to operate on functions of vector products, which has profound applications for electromagnetism.

Were you offended by the crazy mix of signs, dots and cross products in Maxwell's equations?  The geometric algebra form of Maxwell's equation resolves that crazy mix, expressing Maxwell's equations as a single equation.  The formalism of tensor algebra and differential forms also provide simpler ways of expressing Maxwell's equations, but are arguably harder to relate to the vector algebra formalism so familiar to electric engineers and physics practitioners.  In this book, you will see how to work with the geometric algebra form of Maxwell's equation, and how to relate these new techniques to familiar methods.
\paragraph{Overview.}
Geometric algebra generalizes vectors, providing algebraic representations of not just directed line segments, but also points, plane segments, volumes, and higher degree geometric objects (hypervolumes.). The geometric algebra representation of planes, volumes and hypervolumes requires a vector dot product, a vector multiplication operation, and a generalized addition operation. The dot product provides the length of a vector and a test for whether or not any two vectors are perpendicular. The vector multiplication operation is used to construct directed plane segments (bivectors), and directed volumes (trivectors), which are built from the respective products of two or three mutually perpendicular vectors. The addition operation allows for sums of scalars, vectors, or any products of vectors. Such a sum is called a multivector.

The power to add scalars, vectors, and products of vectors can be exploited to simplify much of electromagnetism. In particular, Maxwell's equations for isotropic media can be merged into a single multivector equation
\begin{equation}\label{eqn:ece2500report:40}
\lr{ \spacegrad + \inv{c} \PD{t}{}} F = J,
\end{equation}
where
\begin{itemize}
\item \( \spacegrad \) is the gradient,
\item \( c = 1/\sqrt{\mu\epsilon}\) is the group velocity for waves in the media (i.e. the speed of light),
\item \( F = \BE + I c \BB \) is the multivector electromagnetic field that combines the electric (\(\BE\)) and magnetic field (\(\BB\)) into a single entity,
\item \( J = \eta\lr{ c \rho - \BJ } \) is the multivector current, combining the charge density (\(\rho\)) and the current density (\(\BJ\)) into a single entity,
\item \( I = \Be_1 \Be_2 \Be_3 \) is the ordered product of the three \R{3} basis vectors, and
\item \( \eta = \sqrt{\mu/\epsilon} \) is the impedance of the media.
\end{itemize}
%\begin{equation}\label{eqn:quaternion2maxwellWithGA:20}
%\lr{ \spacegrad + \inv{c} \PD{t}{}} \lr{ \BE + I c \BB } = \eta\lr{ c \rho - \BJ },
%\end{equation}
Encountering Maxwell's equation in its geometric algebra form leaves the student with more questions than answers. Yes, it is a compact representation, but so are the tensor and differential forms (or even the quaternionic) representations of Maxwell's equations. The student needs to know how to work with the representation if it is to be useful. It should also be clear how to use the existing conventional mathematical tools of applied electromagnetism, or how to generalize those appropriately. Individually, there are answers available to many of the questions that are generated attempting to apply the theory, but they are scattered and in many cases not easily accessible.

Much of the geometric algebra literature for electrodynamics is presented with a relativistic bias, or assumes high levels of mathematical or physics sophistication. The aim of this work was an attempt to make the study of electromagnetism using geometric algebra more accessible, especially to other dumb engineers\footnote{Sheldon: ``Engineering. Where the noble semiskilled labourers execute the vision of those who think and dream. Hello, Oompa-Loompas of science.''} like myself.
%In particular, this project explored non-relativistic applications of geometric algebra to electromagnetism. The end product of this project was a fairly small self contained book, titled "Geometric Algebra for Electrical Engineers". This book includes an introduction to Euclidean geometric algebra focused on \R{2} and \R{3} (64 pages), an introduction to geometric calculus and multivector Green's functions (64 pages), applications to electromagnetism (82 pages), and some appendices. Many of the fundamental results of electromagnetism are derived directly from the multivector Maxwell's equation, in a streamlined and compact fashion. This includes some new results, and many of the existing non-relativistic results from the geometric algebra literature. As a conceptual bridge, the book includes many examples of how to extract familiar conventional results from simpler multivector representations. Also included in the book are some sample calculations exploiting unique capabilities that geometric algebra provides. In particular, vectors in a plane may be manipulated much like complex numbers, which has a number of advantages over working with coordinates explicitly.

\paragraph{What's in this book.}

This book introduces the fundamentals of geometric algebra and calculus, and applies those tools to the study of electromagnetism.
Geometric algebra extends vector algebra by
introducing a vector multiplication operation, the vector product, incorporating aspects of both the dot and cross products.
Products or sums of products of vectors are called multivectors, and
are capable of representing oriented point, line, plane, and volume segments.

This book is divided into three parts.

\paragraph{Chapter-1.  An introduction to geometric algebra (GA).}
%\begin{enumerate}[{Chapter}-1]
%\item An introduction to geometric algebra (GA).
Topics covered include vectors, vector spaces, vector multiplication, bivectors, trivectors, multivectors, multivector spaces, dot and wedge products, multivector representation of complex numbers, rotation, reflection, projection and rejection, and linear system solution.

The focus of this book are geometric algebras generated from 2 or 3 dimensional Euclidean vector spaces.
In some cases higher dimensional spaces will be used in examples and theorems.
Some, but not all, of the places requiring generalizations for mixed signature (relativistic) spaces will be pointed out.
\paragraph{Chapter-2.  Geometric calculus, Green's function solutions of differential equations, and multivector Green's functions.}
%\item Geometric calculus, Green's function solutions of differential equations, and multivector Green's functions.
A multivector generalization of vector calculus, the fundamental theorem of geometric calculus,
is required to apply geometric algebra to electromagnetism.
Special cases of the fundamental theorem of geometric calculus include
the fundamental theorem of calculus,
Green's (area) theorem, the divergence theorem, and Stokes' theorems.
Multivector calculus also provides the opportunity to define a few unique and powerful (multivector) Green's functions of particular relevance to electromagnetism.

\paragraph{Chapter-3.  Application of Geometric Algebra to electromagnetism.}
%\item Application of Geometric Algebra to electromagnetism.
Instead of working separately with electric and magnetic fields, we will work with a hybrid multivector field, \( F \), that includes both electric and magnetic field contributions, and with a
multivector current, \( J \), that includes both charge and current densities.

Starting with the conventional form of Maxwell's equation, the multivector Maxwell's equation (singular) is derived.
This is a single multivector equation that is easier to solve and manipulate than the conventional mess of divergence and curl equations that are familiar to the reader.
The multivector Maxwell's equation is the starting point for the remainder of the analysis of the book, and from it the
wave equation, plane wave solutions, and static and dynamic solutions are derived.
The multivector form of energy density, Poynting force, and the Maxwell stress tensor, and all the associated conservation relationships are derived.
The transverse and propagation relationships for waveguide solutions are derived in their multivector form.
Polarization is discussed in a multivector context, and multivector potentials and gauge transformations are introduced.

No attempt to motivate Maxwell's equations, nor most of the results derived from them is made in this book.
%\end{enumerate}

\paragraph{Prerequisites:}

The target audience for this book is advanced undergraduate or graduate students of electrical engineering or physics.
Such an audience is assumed to be intimately familiar with vectors,
vector algebra, dot and cross products, determinants, coordinate representation, linear system solution, complex numbers, matrix algebra, and linear transformations.
It is also assumed that the reader understands and can apply conventional vector calculus concepts including the divergence and curl operators, the divergence and Stokes' theorems,
line, area and volume integrals, Greens' functions, and the Dirac delta function.
Finally, it is assumed that the reader is intimately familiar with conventional electromagnetism, including Maxwell's and the Lorentz force equations, scalar and vector potentials, plane wave solutions, energy density and Poynting vectors, and more.

\paragraph{Thanks:}

Portions of this book were reviewed or corrected by
Steven De Keninck,
Dr. Wolfgang Lindner,
Prof. Mo Mojahedi,
Prof. Alan Macdonald,
Prof. Quirino Sugon Jr.,
Miroslav Josipovi\'{c},
Bruce Gould,
Tim Put,
David Bond,
Bill Ignatiuk,
Sigmundur,
Zhengbang Zhou,
Jack Paladin,
Nicky, D, Foreest,
Peter Eriksen,
Christopher,
Wrenn Wooten,
Prof. Norman Derby,
prlw1 (on github,)
and Pippy (on discord.)
I'd like to thank everybody who provided me with any feedback (or merge-requests!)  This
feedback has significantly improved the quality of the text.

Peeter Joot \quad peeterjoot@pm.me


%}
%\EndArticle
\EndNoBibArticle
