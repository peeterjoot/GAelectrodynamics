%
% Copyright � 2016 Peeter Joot.  All Rights Reserved.
% Licenced as described in the file LICENSE under the root directory of this GIT repository.
%
%{
\input{../latex/blogpost.tex}
\renewcommand{\basename}{maxwells}
%\renewcommand{\dirname}{notes/phy1520/}
\renewcommand{\dirname}{notes/ece1228-electromagnetic-theory/}
%\newcommand{\dateintitle}{}
%\newcommand{\keywords}{}

\input{../latex/peeter_prologue_print2.tex}

\usepackage{peeters_layout_exercise}
%\usepackage{peeters_layout}
\usepackage{peeters_braket}
\usepackage{peeters_figures}
\usepackage{siunitx}
\usepackage{macros_qed}
%\usepackage{mhchem} % \ce{}

\usepackage{txfonts} % \ointclockwise
%\usepackage{macros_cal}
%\usepackage{macros_bm} % \bcM

\usepackage{tcolorbox}
\usepackage{tabularx}
\usepackage{array}
\usepackage{colortbl}
\tcbuselibrary{skins}

%\newcommand{\lrspacetimegrad}[2]{\lr{\stackrel{ \leftrightarrow }{#2 \spacegrad + #1 \PD{t}{}}}}
%\newcommand{\lspacetimegrad}[2]{\lr{\stackrel{ \leftarrow }{#2 \spacegrad + #1 \PD{t}{}}}}
%\newcommand{\rspacetimegrad}[2]{\lr{\stackrel{ \rightarrow }{#2 \spacegrad + #1 \PD{t}{}}}}

\newcolumntype{Y}{>{\raggedleft\arraybackslash}X}

\tcbset{tab1/.style={fonttitle=\bfseries\large,fontupper=\normalsize\sffamily,
colback=yellow!10!white,colframe=red!75!black,colbacktitle=Salmon!40!white,
coltitle=black,center title,freelance,frame code={
\foreach \n in {north east,north west,south east,south west}
{\path [fill=red!75!black] (interior.\n) circle (3mm); };},}}

\tcbset{tab2/.style={enhanced,fonttitle=\bfseries,fontupper=\normalsize\sffamily,
colback=yellow!10!white,colframe=red!50!black,colbacktitle=Salmon!40!white,
coltitle=black,center title}}

\newcommand{\e}[0]{{(\mathrm{e})}}
\newcommand{\m}[0]{{(\mathrm{m})}}

%
% https://tex.stackexchange.com/a/28212/15
%
%For bold and upright, you could use the regular text-version of \imath and \jmath, which are \i and \j:
%
%\newcommand{\ihat}{\hat{\textbf{\i}}}
%\newcommand{\jhat}{\hat{\textbf{\j}}}
%bold & upright i & j in math
%
%Additionally, if you want the \hat to be bold as well, then use \boldsymbol.
%
%\newcommand{\ihat}{\boldsymbol{\hat{\textbf{\i}}}}
%\newcommand{\jhat}{\boldsymbol{\hat{\textbf{\j}}}}

\beginArtNoToc

%\generatetitle{Bivector, Trivector, Multivector, and Multivector space}
\generatetitle{scratch notes.}

% this is the E, H extraction example (digression)
%%
% Copyright © 2016 Peeter Joot.  All Rights Reserved.
% Licenced as described in the file LICENSE under the root directory of this GIT repository.
%
\section{Conventional differential form}

The differential form of Maxwell's equations, with extensions for magnetic sources, is the starting point for all the analysis in these notes.  Those equations are

\input{../ece1229-antenna/MaxwellsStatement.tex}

The magnetic sources can be considered fictional, and are included because they are useful in antenna theory to model real phenomina such as infinitesimal current loops.

\input{../ece1229-antenna/MaxwellsFieldAndSourceDescription.tex}

These fields and sources are all real valued.  In many situations it will be desirable to work with a time harmonic (frequency-domain phasor) form of Maxwell's equations.  In engineering, a time harmonic representation presumes that all sources and fields have a frequency dependence of the form
\index{time harmonic}

\begin{dmath}\label{eqn:maxwellsEquations:20}
\bcY(\Bx, t) = \Real( \BY(\Bx, \omega) e^{j\omega t} ),
\end{dmath}

where the field (or source) \( \BY(\Bx, \Bomega) \) is allowed to be complex valued, whereupon Maxwell's equations take the form

\input{../ece1229-antenna/MaxwellsTimeHarmonic.tex}

Note that the time harmonic convention typically used in physics literature presumes a frequency dependence of the form

\begin{dmath}\label{eqn:maxwellsEquations:40}
\bcY(\Bx, t) = \Real( \BY(\Bx, \omega) e^{-i\omega t} ),
\end{dmath}

which alters the sign of any imaginary originating from a time derivative.  Care is required by the reader to understand which form of frequency dependence has been assumed.

\section{GA differential form}

Geometric Algebra admits a number of alternative representations of Maxwell's equations.  The first follows from expressing the cross products all as wedge products, leaving a pair of bivector and a pair of scalar equations

\begin{subequations}
\begin{dmath}\label{eqn:maxwellsEquations:60}
\spacegrad \wedge \bcE = - I \bcM - \PD{t}{I\bcB}
\end{dmath}
\begin{dmath}\label{eqn:maxwellsEquations:80}
\spacegrad \wedge \bcH = I \bcJ + I \PD{t}{\bcD}
\end{dmath}
\begin{dmath}\label{eqn:maxwellsEquations:100}
\spacegrad \cdot \bcD = q_\txte
\end{dmath}
\begin{dmath}\label{eqn:maxwellsEquations:120}
\spacegrad \cdot \bcB = q_\txtm.
\end{dmath}
\end{subequations}

Alternatively, the duality transformation \( \Ba \wedge \Bb = -I \Ba \cdot (I \Bb) \) allows Maxwell's equations to be all written as dot products

\begin{subequations}
\begin{dmath}\label{eqn:maxwellsEquations:140}
\spacegrad \cdot (I \bcE) = \bcM + \PD{t}{\bcB}
\end{dmath}
\begin{dmath}\label{eqn:maxwellsEquations:160}
\spacegrad \cdot (I \bcH) = -\bcJ - \PD{t}{\bcD}
\end{dmath}
\begin{dmath}\label{eqn:maxwellsEquations:180}
\spacegrad \cdot \bcD = q_\txte
\end{dmath}
\begin{dmath}\label{eqn:maxwellsEquations:200}
\spacegrad \cdot \bcB = q_\txtm,
\end{dmath}
\end{subequations}

or, using the duality transformation \( \Ba \cdot \Bb = -I (\Ba \wedge (I \Bb) \), Maxwell's equations can all be written as wedge products

\begin{subequations}
\begin{dmath}\label{eqn:maxwellsEquations:220}
\spacegrad \wedge \bcE = - I \bcM - \PD{t}{I\bcB}
\end{dmath}
\begin{dmath}\label{eqn:maxwellsEquations:240}
\spacegrad \wedge \bcH = I \bcJ + I \PD{t}{\bcD}
\end{dmath}
\begin{dmath}\label{eqn:maxwellsEquations:260}
\spacegrad \wedge (I\bcD) = I q_\txte
\end{dmath}
\begin{dmath}\label{eqn:maxwellsEquations:280}
\spacegrad \wedge (I\bcB) = I q_\txtm.
\end{dmath}
\end{subequations}

Each of these forms can be useful in different circumstances, however the real power of GA in electromagnetism follows from presuming constituative relationships between the pairs of electric and magnetic fields

\begin{subequations}
\label{eqn:maxwellsEquations:300}
\begin{dmath}\label{eqn:maxwellsEquations:320}
\bcB = \mu \bcH
\end{dmath}
\begin{dmath}\label{eqn:maxwellsEquations:340}
\bcD = \epsilon \bcE,
\end{dmath}
\end{subequations}

where \( \epsilon \) is the permitivitity of the medium [\si{F/m}] (Farads/meter), and \( \mu \) is the permeability of the medium [\si{H/m}] (Henries/meter).
The permitivitity and permeability may be functions of both time and position, and model the materials that the fields are propagating through.  In general, the these may be non-isotropic tensor operators, however, unless otherwise specified, isotropic media will be assumed in these notes.

With this constitutative relationship assumed (and a bit of rescaling), the dot and wedge products of \cref{eqn:maxwellsEquations:60}, \cref{eqn:maxwellsEquations:100} can be added, as can those of \cref{eqn:maxwellsEquations:80}, \cref{eqn:maxwellsEquations:120}.  This reduces Maxwell's equations to a pair of first order coupled gradient equations

\begin{subequations}
\begin{dmath}\label{eqn:maxwellsEquations:360}
\spacegrad \bcE = \inv{\epsilon} q_\txte - I \bcM - \mu \PD{t}{(I\bcH)}
\end{dmath}
\begin{dmath}\label{eqn:maxwellsEquations:380}
\spacegrad (I \bcH) = \frac{I q_\txtm}{\mu} - \bcJ - \epsilon \PD{t}{\bcE}.
\end{dmath}
\end{subequations}




%%
% Copyright © 2017 Peeter Joot.  All Rights Reserved.
% Licenced as described in the file LICENSE under the root directory of this GIT repository.
%
\subsection{Complex power.}
TODO.
%\index{complex power}

%%
% Copyright © 2017 Peeter Joot.  All Rights Reserved.
% Licenced as described in the file LICENSE under the root directory of this GIT repository.
%

A fair amount of nomenclature and notation is unfortunately required before systematically examining the implications of the multivector space axioms that define geometric algebra.

Multivectors which can be factored into normal vector products, such as
\begin{dmath}\label{eqn:multiplication:220}
\Be_1 \Be_2 + 3 \Be_1 \Be_3
=
\Be_1 (\Be_2 + 3 \Be_3),
\end{dmath}

are blades.
In contrast, the following grade 2 multivectors

\begin{dmath}\label{eqn:multiplication:240}
\Be_1 \Be_2 + \Be_3 \Be_4,
\end{dmath}

and
\begin{dmath}\label{eqn:multiplication:260}
\Be_1 \Be_2 + \Be_2 \Be_3 + \Be_3 \Be_1,
\end{dmath}

which cannot be factored into two vector products, are not blades.

\index{k-vector}
\makedefinition{k-vector.}{dfn:multivector:kvector}{
A sum of k-blades is called a k-vector.
} % definition

Multivectors are therefore sums of k-vectors with different grades.

All the k-blade examples in 
\cref{eqn:multivector:180}
 are also k-vectors.
K-vectors with grades 2 and 3 are so pervasive that they are given special names.

\index{bivector}
\makedefinition{Bivector.}{dfn:multivector:bivector}{
A bivector, or 2-vector, is a k-vector with grade 2.
} % definition

Any 2-blade, such as the product \( \Be_1 \Be_2 \) is a bivector.
Any sum of 2-blades, such as \( \Be_2 \Be_3 + 3 \Be_4 \Be_1 \), is also a bivector.
%Each of \( \Be_1 \Be_2, \Be_2 \Be_1, \Be_1 \Be_2 + \Be_2 \Be_3 \), and \( \Be_1 \Be_2 + \Be_3 \Be_4 \) are bivectors.
%All but the last of these represents an oriented plane segment.

\index{trivector}
\makedefinition{Trivector.}{dfn:multivector:trivector}{
A trivector, or 3-vector, is a k-vector with grade 3.
} % definition

%Quantities with higher grades than 3 are not generally given explicit names.
The multivector \( \Be_3 \Be_1 \Be_2 \) is a trivector, as is \( \Be_1 \Be_2 \Be_3 + 3 \Be_5 \Be_4 \Be_1 \).
The latter is not a blade.
%Each of \( \Be_1 \Be_2 \Be_3, \Be_1 \Be_3 \Be_2, \Be_1 \Be_4 \Be_2 \) are trivectors.
% , and represent oriented volumes.



%\section{Polarization.}
%%
% Copyright © 2017 Peeter Joot.  All Rights Reserved.
% Licenced as described in the file LICENSE under the root directory of this GIT repository.
%
\subsection{Polarization.}
\index{plane wave}
\index{polarization}
In a discussion of polarization, it is convenient to align the propagation direction along a fixed direction, usually the z-axis.
Setting \( \kcap = \Be_3, \beta z = \Bk \cdot \Bx \) in \cref{eqn:frequencydomainCore:200} the plane wave representation of the field is

\begin{dmath}\label{eqn:polarization:20}
\begin{aligned}
F(\Bx, \omega) &= (1 + \Be_3) \BE e^{-j \beta z} \\
F(\Bx, t) &= \Real\lr{ F(\Bx, \omega) e^{j \omega t} }.
\end{aligned}
\end{dmath}

Here the imaginary \( j \) has no intrinsic geometrical interpretation, \( \BE = \BE_\txtr + j \BE_\txti \) is allowed to have complex values, and all components of \( \BE \) is perpendicular to the propagation direction (\( \Be_\txtr \cdot \Be_3 = \BE_\txti \cdot \Be_3 = 0 \)).
\index{Jones vector}
A common representation of the electric field components is the Jones vector \( (c_1, c_2) \), which specifies complex coefficients for the electric field phasor in each of the possible directions

\begin{dmath}\label{eqn:polarization:120}
\BE = c_1 \Be_1 + c_2 \Be_2,
\end{dmath}

where \( c_1, c_2 \) are complex valued, say

\begin{dmath}\label{eqn:polarization:140}
\begin{aligned}
c_1 &= \alpha_1 + j \beta_1 \\
c_2 &= \alpha_2 + j \beta_2.
\end{aligned}
\end{dmath}

The tuple \( (c_1, c_2) \) is called the Jones vector, and compactly encodes the geometry of the pattern that the electric field traces out in the transverse plane.

\subsection{Circular polarization basis.}
\index{circular polarization}
\index{left circular polarization}
\index{right circular polarization}

The time domain field when written out explicitly in terms of the Jones vector components is

\begin{dmath}\label{eqn:polarization:160}
F(\Bx, t) = (1 + \Be_3) \lr{
\lr{ \alpha_1 \Be_1 + \alpha_2 \Be_2 } \cos\lr{ \omega t - \beta z }
-\lr{ \beta_1 \Be_1 + \beta_2 \Be_2 } \sin\lr{ \omega t - \beta z }
}.
\end{dmath}

Linear, circular, and elliptical polarization patterns can be obtained using specific values for \( \alpha_1, \alpha_2, \beta_1, \beta_2 \).
In particular,
a field for which the
change in phase \( \phi = \omega t - \beta z \) results in the electric field tracing out a (counterclockwise,clockwise) circle

\begin{dmath}\label{eqn:polarization:180}
\begin{aligned}
\BE_\txtR &= \Abs{\BE} \lr{ \Be_1 \cos\phi + \Be_2 \sin\phi } = \Abs{\BE} \Be_1 \exp\lr{  \Be_{12} \phi } \\
\BE_\txtL &= \Abs{\BE} \lr{ \Be_1 \cos\phi - \Be_2 \sin\phi } = \Abs{\BE} \Be_1 \exp\lr{ -\Be_{12} \phi },
\end{aligned}
\end{dmath}

is referred to as having
(left,right) circular polarization.
There are different conventions for the polarization orientation, and here the IEEE antenna convention discussed in \citep{balanis1989advanced} are used.

Fixme: check that I have this orientation right.  Noticed after the fact that the figures in Balanis use an orientation with x-axis up and y-axis right!

The bivector exponential representation of the circularly polarized electric fields in \cref{eqn:polarization:180} indicates that it is possible to represent arbitrary field polarization in a GA form that does not require any real part operation, as follows

\begin{dmath}\label{eqn:polarization:200}
F = \lr{ 1 + \Be_3 } \Be_1 \lr{ \alpha_\txtR e^{i\phi} + \alpha_\txtL e^{-i\phi} },
\end{dmath}

where the constants \( \alpha_\txtR, \alpha_\txtL \) are both complex with respect to the unit bivector imaginary \( i = \Be_{12} \) representing the plane transverse to the propagation direction

\begin{dmath}\label{eqn:polarization:220}
\begin{aligned}
\alpha_\txtR &= \alpha_{\txtR 1} + i \alpha_{\txtR 2} \\
\alpha_\txtL &= \alpha_{\txtL 1} + i \alpha_{\txtL 2}.
\end{aligned}
\end{dmath}

If a transformation from scalar to bivector imaginary \( j \rightarrow \Be_{12} = i \) is made in the Jones vector component representation of \cref{eqn:polarization:140},
then
the coefficients \cref{eqn:polarization:220} of the circular polarization states are related to the Jones vector by (\cref{problem:polarization:1})

\begin{dmath}\label{eqn:polarization:260}
\begin{aligned}
\alpha_\txtR &= \inv{2}\lr{ c_1 - i c_2 } \\
\alpha_\txtL &= \inv{2}\lr{ c_1 + i c_2 }^\dagger.
\end{aligned}
\end{dmath}

\subsection{Linear polarization.}

Linear polarization is described by

\begin{dmath}\label{eqn:polarization:280}
\begin{aligned}
\alpha_\txtR &= \inv{2}\Abs{\BE} \Be_1 e^{i(\psi + \theta)} \\
\alpha_\txtL &= \inv{2}\Abs{\BE} \Be_1 e^{i(\psi - \theta)},
\end{aligned}
\end{dmath}

or
\begin{dmath}\label{eqn:polarization:300}
F = \lr{ 1 + \Be_3 } \Abs{\BE} \Be_1 e^{i\psi} \cos( \omega t - \beta z + \theta ).
\end{dmath}

The electric field \( \BE \) traces out all the points along the line spanning the points between \( \pm \Be_1 e^{i\psi} \Abs{\BE} \), whereas the magnetic field \( \BH \) traces
out all the points along \( \pm \Be_2 e^{i\psi} \Abs{\BE}/\eta \) as illustrated (with \( \eta = 1 \)) in
\cref{fig:linearPolarization:linearPolarizationFig1}.

\imageFigure{../figures/GAelectrodynamics/linearPolarizationFig1}{Linear polarization.}{fig:linearPolarization:linearPolarizationFig1}{0.3}

\subsection{Elliptical parameterization.}

An ellipical polarized electric field can be parameterized as
\begin{dmath}\label{eqn:ellipticalWaves:340}
\BE
=
E_a \Be_1 \cos\theta + E_b \Be_2 \sin\theta,
\end{dmath}

which corresponds to a Jones vector \( (E_a, -i E_b) \), or circular polarization coefficients with values

\begin{dmath}\label{eqn:polarization:400}
\begin{aligned}
\alpha_\txtR &= \inv{2}\lr{ E_a - E_b } \\
\alpha_\txtL &= \inv{2}\lr{ E_a + E_b }.
\end{aligned}
\end{dmath}

Therefore an elliptically polarized field can be represented as

\begin{dmath}\label{eqn:polarization:420}
F = \inv{2} (1 + \Be_3) \Be_1 \lr{ (E_a + E_b) e^{i\phi} + (E_a - E_b) e^{-i\phi} }.
\end{dmath}

An interesting variation of the elliptical polarization uses a hyperbolic parameterization.
If \( a, b \) are the semi-major/minor axes of the ellipse (i.e. \( a > b \)),
and \( \Ba = a \Be_1 e^{i\psi} \) is the vectoral representation of the semimajor axis (not necessarily placed along \( \Be_1 \)),
and \( e = \sqrt{1 - (b/a)^2} \) is the eccentricity of the ellipse,
then it can be shown (\citep{hestenes1999nfc})
that an elliptic parameterization can be written
in the compact form

\begin{dmath}\label{eqn:ellipticalWaves:360}
\Br(\phi)
=
e \Ba \cosh( \Atanh(b/a) + i \phi).
\end{dmath}

Using the bivector
This is also real and has only vector grades, so the electromagnetic field for a general elliptic wave has the form

\begin{dmath}\label{eqn:ellipticalWaves:380}
\begin{aligned}
F &= e E_a \lr{ 1 + \Be_3 } \Be_1 e^{ i \psi } \cosh\lr{ \mu + i \phi} \\
\mu &= \Atanh\lr{ E_b/E_a } \\
e &= \sqrt{1 - {(E_b/E_a)}^2 },
\end{aligned}
\end{dmath}

where \( E_a(E_b) \) are the magnitudes of the electric field components lying along the semi-major(minor) axes, and the propagation direction \( \Be_3 \) is normal to both the major and minor axis directions, as illustrated in \cref{fig:ellipticalPolarization:ellipticalPolarizationFig1}.
Observe that setting \( E_b = 0 \) results in the linearly polarized field of \cref{eqn:polarization:300}.
\imageFigure{../figures/GAelectrodynamics/ellipticalPolarizationFig1}{Electric field with elliptical polarization.}{fig:ellipticalPolarization:ellipticalPolarizationFig1}{0.3}

\subsection{Pseudoscalar imaginary.}

...

\subsection{Problems.}

\makeproblem{Circular polarization coefficients relationship to the Jones vector.}{problem:polarization:1}{
By substituting \cref{eqn:polarization:220} into \cref{eqn:polarization:200}, and comparing to \cref{eqn:polarization:160},
show that the circular state coefficients have the following relationship to the Jones vector coordinates
\begin{equation*}
\begin{aligned}
\alpha_\txtR &= \lr{ \alpha_1 + \beta_2 }/2 + i \lr{ -\alpha_2 + \beta_1 }/2 \\
\alpha_\txtL &= \lr{ \alpha_1 - \beta_2 }/2 + i \lr{ -\alpha_2 - \beta_1 }/2,
\end{aligned}
\end{equation*}
and use this to prove \cref{eqn:polarization:260}.
} % problem

%%
% Copyright © 2017 Peeter Joot.  All Rights Reserved.
% Licenced as described in the file LICENSE under the root directory of this GIT repository.
%
%{
Geometric algebra takes a vector space and adds two additional operations, a vector multiplication operation, and a generalized addition operation that extends vector addition to include addition of scalars and products of vectors.
Multiplication of vectors is indicated by juxtaposition, for example, if \( \Bx, \By, \Be_1, \Be_2, \Be_3, \cdots \) are vectors, then some vector products are
\begin{dmath}\label{eqn:multivector:20}
\begin{aligned}
&\Bx \By, \Bx \By \Bx, \Bx \By \Bx \By, \\
&\Be_1 \Be_2, \Be_2 \Be_1, \Be_2 \Be_3, \Be_3 \Be_2, \Be_3 \Be_1, \Be_1 \Be_3, \\
&\Be_1 \Be_2 \Be_3, \Be_3 \Be_1 \Be_2, \Be_2 \Be_3 \Be_1, \Be_3 \Be_2 \Be_1, \Be_2 \Be_1 \Be_3, \Be_1 \Be_3 \Be_2, \\
&\Be_1 \Be_2 \Be_3 \Be_1, \Be_1 \Be_2 \Be_3 \Be_1 \Be_3 \Be_2, \cdots
\end{aligned}
\end{dmath}

Vector multiplication is constrained by a rule, called the contraction axiom, that gives a meaning to the square of a vector
\boxedEquation{eqn:multivector:120}{
\Bx \Bx \equiv \Bx \cdot \Bx.
}

The square of a vector, by this definition, is the squared length of the vector, and is a scalar.
This may not appear to be a useful way to assign meaning to the simplest of vector products, since the product and the vector live in separate spaces.
If we want a closed algebraic system that includes both vectors and their products, we have to allow for the addition of scalars, vectors, or any products of vectors.  Such a sum is called a multivector, an example of which is
\begin{dmath}\label{eqn:multivector:40}
1 + 2 \Be_1 + 3 \Be_1 \Be_2 + 4 \Be_1 \Be_2 \Be_3.
\end{dmath}
In this example, we have added a
scalar (or 0-vector) \( 1 \), to a
vector (or 1-vector) \( 2 \Be_1 \), to a
bivector (or 2-vector) \( 3 \Be_1 \Be_2 \), to a
trivector (or 3-vector) \( 4 \Be_1 \Be_2 \Be_3 \).
Geometric algebra uses vector multiplication to build up a hierarchy of geometrical objects, representing points, lines, planes, volumes and hypervolumes (in higher dimensional spaces.)

\index{scalar}
\index{0-vector}
\paragraph{Scalar.}
A scalar, which we will also call a 0-vector, is a zero-dimensional object with sign, and a magnitude.
We may geometrically interpret a scalar as a (signed) point in space.
%The sign of a scalar can be represented graphically as an arrow with a head and a tail pointing into the paper (or chalkboard),
%as illustrated in
%\cref{fig:scalarOrientation:scalarOrientationFig1} where a crossed circle represents the tail, and a solid dot represents the head.
%%\footnote{We don't usually try to represent the magnitude of a scalar graphically, but could do so by scaling the size of the cross or dot.}
%\imageFigure{../figures/GAelectrodynamics/scalarOrientationFig1}{Scalar illustration.}{fig:scalarOrientation:scalarOrientationFig1}{0.05}

\index{vector}
\index{1-vector}
\paragraph{Vector.}
A vector, which we will also call a 1-vector, is a one-dimensional object with a sign, a magnitude, and a rotational attitude within the space it is embedded.
XX
\index{bivector}
\index{2-vector}
\paragraph{Bivector.}

We now wish to define a bivector, or 2-vector, as a 2 dimensional object representing a signed plane segment with magnitude and orientation.  Formally,
assuming a vector product, the algebraic description of a bivector is

\makedefinition{Bivector.}{dfn:multivector:60}{
A bivector, or 2-vector, is a sum of products of pairs of orthogonal vectors.
Given an \( N \) dimensional vector space \( V \) with an orthonormal basis \( \setlr{ \Be_1, \Be_2, \cdots, \Be_N } \),
a general bivector can be expressed as
\begin{equation*}
\sum_{1 \le i < j \le N} B_{ij} \Be_i \Be_j,
\end{equation*}
where \( B_{ij} \) is a scalar.
The vector basis \( V \) is said to be a generator of a bivector space.
} % definition

The bivector provides a structure that can encode plane oriented quantities such as torque, angular momentum, or a general plane of rotation.
A quantity like angular momentum can be represented as a magnitude times a quantity that represents the orientation of the plane of rotation.
In conventional vector algebra we use the normal of the plane to describe this orientation, but that is problematic in higher dimensional spaces where there is no unique normal.
Use of the normal to represent a plane is also logically problematic in two dimensional spaces, which have to be extended to three dimensions to use normal centric constructs like the cross product.
A bivector representation of a plane can eliminate the requirement to utilize a third (normal) dimension, which may not be relevant in the problem, and can allow some concepts (like the cross product) to be generalized to dimensions other than three when desirable.

One of the implications of the contraction axiom \cref{eqn:multivector:120}, to be discussed in more detail a bit later, is a linear dependence between bivectors formed from orthogonal products.  For example, given any pair of unit bivectors, where \( i \ne j \) we have
\begin{dmath}\label{eqn:multivector:140}
\Be_i \Be_j + \Be_j \Be_i = 0,
\end{dmath}
This is why the sum in \cref{dfn:multivector:60} was over only half the possible pairs of \( i \ne j \) indexes.
The reader can check that the set of all bivectors is a vector space per
\cref{def:prerequisites:vectorspace}, so we will call the set of all bivectors a bivector space.
In \R{2} a basis for the bivector space is \( \setlr{ \Be_1 \Be_2 } \), whereas in \R{3} a basis for the bivector space is
\( \setlr{ \Be_1 \Be_2, \Be_2 \Be_3, \Be_3 \Be_1 } \).  The unit bivectors for two possible \R{3} bivector space bases are illustrated in
\cref{fig:unitBivectors:unitBivectorsFig}.
\imageTwoFigures
{../figures/GAelectrodynamics/unitBivectorsFig1}
{../figures/GAelectrodynamics/unitBivectorsFig2}
{Unit bivectors for \R{3}}
{fig:unitBivectors:unitBivectorsFig}
{scale=0.35}

We interpret the sign of a vector as an indication of the sense of the vector's ``head'' vs ``tail''.
For a bivector, we can interpret the sign as a representation of a
a ``top'' vs. ``bottom'', or equivalently a left or right ``handedness'', as illustrated using arrows around a plane segment in
\cref{fig:circularBivectorsIn3D:circularBivectorsIn3DFig1}.
\imageFigure{../figures/GAelectrodynamics/circularBivectorsIn3DFig1}{Circular representation of two bivectors.}{fig:circularBivectorsIn3D:circularBivectorsIn3DFig1}{0.3}
For a product like \( \Be_1 \Be_2 \), the sense of the handedness follows the path \( 0 \rightarrow \Be_1 \rightarrow \Be_1 + \Be_2 \rightarrow \Be_2 \rightarrow 0 \) around the unit square in the x-y plane.
This is illustrated for all the unit bivectors in \cref{fig:unitBivectors:unitBivectorsFig}.
In \R{3} we can use the right hand rule to visualize such a handedness.  You could say that we are using the direction of the fingers around the normal to indicate the sign of the bivector, but without actually drawing that normal.

Similar to the interpretation of the magnitude of a vector as the length of that vector, we interpret the magnitude of a bivector (to be defined more exactly later), as the area of the bivector.
Other than having a boundary that surrounds a given area, a graphical bivector representation as a plane segment need not have any specific geometry, which is illustrated in
\cref{fig:bivectorRepresentationsInPlane:bivectorRepresentationsInPlaneFig1} for a set of bivectors all representing \( \Be_1 \Be_2 \).
\imageFigure{../figures/GAelectrodynamics/bivectorRepresentationsInPlaneFig1}{Graphical representations of \( \Be_1 \Be_2 \).}{fig:bivectorRepresentationsInPlane:bivectorRepresentationsInPlaneFig1}{0.3}

An oriented plane segment can always be represented as a bivector in any number of dimensions, however, when the generating vector space has dimension \( N \ge 4 \) not all bivectors defined by \cref{dfn:multivector:60} necessarily represent oriented plane segments.
The restrictions required for a bivector to have an associated oriented plane segment interpretation in higher dimensional spaces will be defined later.

Vector addition can be performed graphically by connecting vectors head to tail, and joining the first tail to the last head.  A similar procedure can be used for bivector addition as well, but gets complicated if the bivectors lie in different planes.  Here is a simple bivector sum
\begin{dmath}\label{eqn:multivector:160}
3 \Be_1 \Be_2 - 2 \Be_1 \Be_2 + 5 \Be_1 \Be_2 = 6 \Be_1 \Be_2,
\end{dmath}
which can be interpreted as taking a 3 unit area, subtracting a 2 unit area, and adding a 5 unit area.  This sum is illustrated in
\cref{fig:bivectorAdditionInPlane:bivectorAdditionInPlaneFig1}.
An visualization of arbitrarily oriented bivector addition can be found in
\cref{fig:AdditionOfBivectors:AdditionOfBivectorsFig2}, where \( \text{red} + \text{blue} = \text{green} \).  This visualization shows that the
moral of the story is that we will almost exclusively be adding bivectors algebraically, but can interpret the sum geometrically after the fact.
\imageFigure{../figures/GAelectrodynamics/bivectorAdditionInPlaneFig1}{Graphical representation of bivector addition in plane.}{fig:bivectorAdditionInPlane:bivectorAdditionInPlaneFig1}{0.2}
\imageFigure{../figures/GAelectrodynamics/AdditionOfBivectorsFig2}{Bivector addition.}{fig:AdditionOfBivectors:AdditionOfBivectorsFig2}{0.3}
%The same can be done with bivectors, where the bivectors are also connected with compatible orientation to construct a sum.
%This is illustrated graphically in \cref{fig:AdditionOfBivectors:AdditionOfBivectorsFig1}, where a blue bivector with a right handed orientation is added to a red bivector with right handed orientation, to form a green bivector also with right handed orientation, where all orientations are with respect to the exterior of the bounding surface formed by the three bivectors.
%\imageFigure{../figures/GAelectrodynamics/AdditionOfBivectorsFig1}{Bivector addition.}{fig:AdditionOfBivectors:AdditionOfBivectorsFig1}{0.3}

\index{trivector}
\index{3-vector}
\paragraph{Trivector.}

Again, assuming a vector product

\makedefinition{Trivector.}{dfn:multivector:80}{
A trivector, or 3-vector, is a sum of products of triplets of mutually orthogonal vectors.
Given an \( N \) dimensional vector space \( V \) with an orthonormal basis \( \setlr{ \Be_1, \Be_2, \cdots, \Be_N } \), a trivector is any value
\begin{equation*}
\sum_{1 \le i < j < k \le N} T_{ijk} \Be_i \Be_j \Be_k,
\end{equation*}
where \( \T_{ijk} \) is a scalar.
The vector space \( V \) is said to generate a trivector space.
} % definition

In \R{3} all trivectors are scalar multiples of \( \Be_1 \Be_2 \Be_3 \).
Like scalars, there is no direction to such a quantity, but like scalars trivectors may be signed.  The magnitude of a trivector may be interpreted as a volume.
We will defer interpreting the sign of a trivector geometrically until we tackle integration theory.
%%%, which requires some interpretation.
%%%We can interpret the magnitude of a trivector as a volume, but what is a signed volume?
%%%One answer to this question is that we can interpret the sign of the volume as the exterior or the interior of the surface on the boundry of the volume.
%%%We will see another answer when we study integration theory, since geometric integration theory uses signed volume elements, and
%%%swapping the order of two adjacent products in the volume element toggles the sign.
%%%\footnote{In conventional integration theory,
%%%this sign change occurs when swapping rows or columns in the Jacobian, but this is masked by taking the absolute value of the Jacobian after coordinate transformation.}
%%%One possible interpretation of this sign is the interior or the exterior of the bounding surface of a volume.
%%%%This orientation can be visualized with a normal pointing into or out of the volume, or like bivectors, with a cyclic direction on the surface of the volume as in illustrated with the spherical volume of \cref{fig:orientedVolume:orientedVolumeFig1}.
%%%%\imageFigure{../figures/GAelectrodynamics/orientedVolumeFig1}{Oriented Volume}{fig:orientedVolume:orientedVolumeFig1}{0.3}
%%%%In greater than three dimensions, a trivector can have a ``direction'' in the higher dimensional space, as well as a sidedness.
%%%%As was the case with the bivector, because not all the products \( \Be_i \Be_j \Be_k \) for any set of indexes \( i, j, k \) are independent, it is possible to form a trivector as a sum over a more restricted set, such as \( \sum_{1 \le i < j < k \le N} T_{ijk} \Be_i \Be_j \Be_k \).
%%%%In particular, in three dimensions, all trivectors can be expressed as scalar multiples of \( \Be_1 \Be_2 \Be_3 \).
%%%%
\index{k-vector}
\index{grade}
\paragraph{K-vector.}
\makedefinition{K-vector and grade.}{dfn:multivector:100}{
A k-vector is a sum of products of \( k \) mutually orthogonal vectors.
Given an \( N \) dimensional vector space with an orthonormal basis \( \setlr{ \Be_1, \Be_2, \cdots, \Be_N } \),
a general k-vector can be expressed as
\begin{equation*}
\sum_{1 \le i < j \cdots < m \le N} K_{i j \cdots m} \Be_{i} \Be_{j} \cdots \Be_{m},
\end{equation*}
where \( K_{i j \cdots m} \) is a scalar, indexed by \( k \) indexes \( i, j, \cdots, m \).

The number \( k \) of orthogonal vectors that generate a k-vector is called the grade.

A 0-vector is a scalar.

The vector space \( V \) is said to generate the k-vector space.
} % definition

Illustrating by example, \( 1 \) is a 0-vector with grade 0, \( \Be_1 \) is a 1-vector with grade 1, \( \Be_1 \Be_2, \Be_2 \Be_3 \), and \( \Be_3 \Be_1 \) are 2-vectors with grade 2, and \( \Be_1 \Be_2 \Be_3 \) is a 3-vector with grade 3.

We will see that the highest grade for a k-vector in an N dimensional vector space is \( N \).

\index{multivector}
\index{multivector space}
\paragraph{Multivector space.}
\makedefinition{Multivector space.}{def:multiplication:multivectorspace}{
   Given an N dimensional (generating) vector space \( V \) 
and a vector multiplication operation represented by juxtaposition,
a multivector is a sum of k-vectors, \( k \in [ 1, N ] \).

The multivector space generated by \( V \) is a set \( M = \setlr{ x, y, z, \cdots } \) of multivectors, where the following axioms are satisfied

\begin{tablebox}[tabularx={X|Y}]{Multivector space axioms.}
    Contraction & \( \Bx^2 = \Bx \cdot \Bx, \,\forall \Bx \in V \) \\ \hline
    \( M \) is closed under addition & \( x + y \in M \) \\ \hline
    \( M \) is closed under multiplication & \( x y \in M \) \\ \hline
    Addition is associative & \( (x + y) + z = x + (y + z) \) \\ \hline
    Addition is commutative & \( y + x = x + y \) \\ \hline
    There exists a zero element \( 0 \in M \)  & \( x + 0 = x \) \\ \hline
    For all \( x \in M \) there exists a negative additive inverse \( -x \in M \) & \( x + (-x) = 0 \) \\ \hline
    Multiplication is distributive  & \( x( y + z ) = x y + x z \), \( (x + y)z = x z + y z \) \\ \hline
    Multiplication is associative & \( (x y) z = x ( y z ) \) \\ \hline
    There exists a multiplicative identity \( 1 \in M \) & \( 1 x = x \) \\ \hline
\end{tablebox}
}

(CUT)

with an orthonormal basis \( \setlr{ \Be_1, \Be_2, \cdots, \Be_N } \),
%a basis \( \setlr{ \Bx_1, \Bx_2, \cdots } \),
, such as
   \( a_0 + \sum_i a_i \Be_i + \sum_{i \ne j} a_{ij} \Be_i \Be_j + \sum_{i \ne j \ne k} a_{ijk} \Be_i \Be_j \Be_k + \cdots \), where \( a_0, a_i, a_{ij}, \cdots \) are scalars.

Compared to the vector space, \cref{def:prerequisites:vectorspace}, the multivector space

\begin{itemize}
\item specifies a rule providing the meaning of a squared vector (the contraction axiom).
\item presumes a vector multiplication operation, which is not assumed to be commutative (order matters),
\item generalizes vector addition to multivector addition,
\item generalizes scalar multiplication to multivector multiplication (of which scalar multiplication and vector multiplication are special cases),
\end{itemize}

The contraction axiom is arguably the most important of the multivector space axioms, as it allows for multiplicative closure without an infinite dimensional multivector space.
The remaining set of non-contraction axioms of a multivector space are almost that of a field, however,
%\footnote{A mathematician would call a multivector space a non-commutative ring with identity \citep{van1943modern}, and could state the multivector space definition much more compactly without listing all the properties of a ring explicitly as done above.}
%(as encountered in the study of complex inner products),
%as they describe most of the properties one
%would expect of a ``well behaved'' set of number-like quantities.
a field also requires a multiplicative inverse element for all elements of the space, which exists for some multivector subspaces, but not in general.

%These axioms may seem simple enough, especially since they are not that different from the familiar axioms of the vector space,
%but it will take considerable work to extract all their consequences.
%The subject of Geometric Algebra can be viewed as the study of the impliciations of the axioms
%of the multivector space.

%}

%%
% Copyright © 2017 Peeter Joot.  All Rights Reserved.
% Licenced as described in the file LICENSE under the root directory of this GIT repository.
%

Ampere's law relating the line integral of the magnetic field around a loop can be used to compute the azimuthal magnetic field around a current source.
Let's compute the magnetic field using superposition in a region of space that lies between two z-axis currents of magnitude \( I_1, I_2 \) situated respectively at \( \Bp_1, \Bp_2 \), as illustrated in
\cref{fig:amperesLawBetweenTwoCurrents:amperesLawBetweenTwoCurrentsFig1}.

\imageFigure{../figures/GAelectrodynamics/amperesLawBetweenTwoCurrentsFig1}{Magnetic field between two current sources.}{fig:amperesLawBetweenTwoCurrents:amperesLawBetweenTwoCurrentsFig1}{0.3}

We will need to add azimuthal field components along the \( \phicap_1, \phicap_2 \) directions, a task that is more tractable with GA than in vector algebra using coordinates.

First consider the field surrounding a single current source, say \( I_1 \) at point \( \Bp_1 \).
Combining the two parameter Stokes' theorem integral from 
%\cref{eqn:twoparameter:280}
\cref{thm:surfaceintegral:500}
, and Maxwell's magnetostatics equations for \( \BB \) in \cref{eqn:magnetostatics:380} we have

\begin{dmath}\label{eqn:amperesLawMagnetostaticsExample:20}
\int_A d^2 \Bx \cdot (\spacegrad \wedge \BB) = \ointclockwise d\Bx \cdot \BB = \mu \int d^2 \Bx \cdot (I \BJ).
\end{dmath}

Flipping the orientation of the Stokes' integral so that we are integrating along the \( \phicap \) direction, we have

\begin{dmath}\label{eqn:amperesLawMagnetostaticsExample:40}
\ointctrclockwise d\Bx \cdot \BB
= -\mu I \int d^2 \Bx \wedge \BJ
= \mu I_\txte,
\end{dmath}
where \( I_\txte \) is the enclosed charge (in this case \( I_1 \)).
Because of symmetry, the magnetic field due to this one line charge is

\begin{dmath}\label{eqn:amperesLawMagnetostaticsExample:60}
\BB
= \frac{\mu I_\txte \phicap}{2 \pi R},
\end{dmath}
where \( R \) is the radius of the circle centred one the current, out to the point \( \Br \) where the field is observed.
Each of the current sources at points \( \Bp_k \) contributes a magnetic field

\begin{dmath}\label{eqn:amperesLawMagnetostaticsExample:80}
\BB_k(\Br)
= \frac{\mu I_k \rcap_k \Be_{12} }{2 \pi \Norm{ \Br - \Bp_k} }
= \frac{\mu I_k \lr{ \Br - \Bp_k} \Be_{12} }{2 \pi \lr{ \Br - \Bp_k}^2 }
= \frac{\mu I_k}{2\pi} \inv{ \Br - \Bp_k} \Be_{12},
\end{dmath}
where the azimuthal angle has been determined by rotating the radial unit vector counterclockwise by 90 degrees using \( \phicap_k = \rcap_k\, \Be_{12} \).
The total magnetic field bivector between the charges is

\begin{equation}\label{eqn:amperesLawMagnetostaticsExample:100}
I \BB(\Br)
= \frac{\mu}{2\pi} \sum_{k = 1}^2 I_k \inv{\Bp_k - \Br } \Be_3.
\end{equation}

%The radius of the k-th circles are \( \Norm{ \Bp_k - \Br } \).
A product including the (inverse) vector difference \( \Bp_k - \Br \) encodes both the magnitude of the field contribution at the point \( \Br \) as well as the orientation of the magnetic field bivector components in the x-z and y-z planes.
There was no need to fall back to coordinates, which could easily become cumbersome and error prone.

%
% Copyright © 2017 Peeter Joot.  All Rights Reserved.
% Licenced as described in the file LICENSE under the root directory of this GIT repository.
%
%original ideas from gabookII/electrodynamics/transverseField.tex:
We now wish to consider more general solutions to the source free Maxwell's equation than the plane wave solutions derived in \cref{chap:planewavesMultivector}.
One way of tackling this problem is to assume the solution exists, but ask how the field components that lie strictly along the propagation direction are related to the transverse components of the field.
Without loss of generality, it can be assumed that the propagation direction is along the z-axis.

\maketheorem{Transverse and propagation field components.}{thm:transverseField:288}{
If \( \Be_3 \) is the
propagation direction, the components of a field \( F \) in the propagation direction and in the transverse plane are respectively
\begin{equation*}
\begin{aligned}
F_z &= \inv{2} \lr{ F + \Be_3 F \Be_3 } \\
F_t &= \inv{2} \lr{ F - \Be_3 F \Be_3 },
\end{aligned}
\end{equation*}
where \( F = F_z + F_t \).
} % theorem

To determine the components of the field that lie in the propagation direction and transverse planes, we state the field in the propagation direction, building it from the electric and magnetic field projections along the z-axis
\begin{dmath}\label{eqn:transverseField:108}
F_z
=
\lr{ \BE \cdot \Be_3 }
 \Be_3
+ I \eta \lr{ \BH \cdot \Be_3 } \Be_3
=
\inv{2}
\lr{ \BE \Be_3 + \Be_3 \BE }
 \Be_3
+ \inv{2} I \eta \lr{ \BH \Be_3 + \Be_3 \BH } \Be_3
=
\inv{2}
\lr{ \BE + \Be_3 \BE \Be_3 }
+ \inv{2} I \eta \lr{ \BH + \Be_3 \BH \Be_3 }
=
\inv{2} \lr{ F + \Be_3 F \Be_3 }.
\end{dmath}
The difference \( F - F_z \) is the transverse component
\begin{dmath}\label{eqn:transverseField:308}
F_t
= F - F_z
=
F -
\inv{2} \lr{ F + \Be_3 F \Be_3 }
=
\inv{2} \lr{ F - \Be_3 F \Be_3 },
\end{dmath}
as claimed.

We wish to split the gradient into transverse and propagation direction components.

\makedefinition{Transverse and propagation direction gradients.}{dfn:transverseField:328}{
Define the \textit{propagation direction gradient} as \( \Be_3 \partial_z \), and
\textit{transverse gradient} by
\begin{equation*}
\spacegrad_t = \spacegrad - \Be_3 \partial_z.
\end{equation*}
} % definition

Given this definition, we seek to show that

\maketheorem{Transverse and propagation field solutions.}{thm:transverseField:348}{
Given a field propagating along the z-axis (either forward or backwards), with angular frequency \( \omega \), represented by the real part of
\begin{equation*}
F(x, y, z, t) = F(x, y) e^{j \omega t \mp j k z},
\end{equation*}
the field components that solve the source free Maxwell's equation are related by
\begin{equation*}
\begin{aligned}
F_t &= j \inv{ \frac{\omega}{c} \mp k \Be_3 } \spacegrad_t F_z \\
F_z &= j \inv{ \frac{\omega}{c} \mp k \Be_3 } \spacegrad_t F_t.
\end{aligned}
\end{equation*}
Written out explicitly, the transverse field component expands as
\begin{equation*}
\begin{aligned}
\BE_t &=
\frac{j}{{\frac{\omega}{c}}^2 - k^2}
\lr{
   \pm k \spacegrad_t E_z
   + \frac{\omega \eta}{c} \Be_3 \cross \spacegrad_t H_z
}
\\
\eta \BH_t &=
\frac{j}{{\frac{\omega}{c}}^2 - k^2}
\lr{
   \pm k \eta \spacegrad_t H_z
   -
   \frac{\omega}{c}
   \Be_3 \cross \spacegrad_t E_z
}.
\end{aligned}
\end{equation*}
} % theorem

To prove we first insert the assumed phasor representation into Maxwell's equation, which gives
\begin{equation}\label{eqn:transverseField:summaryMax2}
\lr{\spacegrad_t + j \lr{ \frac{\omega}{c} \mp k \Be_3 } } F(x,y) = 0.
\end{equation}

Dropping the \( x, y \) dependence for now (i.e.  \( F(x, y) \rightarrow F \), we find a relation between the transverse gradient of \( F \) and the propagation direction gradient of \( F \)

\begin{dmath}\label{eqn:transverseField:148}
\spacegrad_t F = - j \lr{ \frac{\omega}{c} \mp k \Be_3 } F.
\end{dmath}
From this we now seek to determine the relationships between \( F_t \) and \( F_z \).

Since \( \spacegrad_t \) has no \( \xcap, \ycap \) components, \( \Be_3 \) anticommutes with the transverse gradient
\begin{dmath}\label{eqn:transverseField:168}
\Be_3 \spacegrad_t = - \spacegrad_t \Be_3,
\end{dmath}
but commutes with \( 1 \mp \Be_3 \).
%In \cref{eqn:transverseField:168} it is implied that the action of \( \spacegrad_t \) is on everything to its right.
This means that
\begin{dmath}\label{eqn:transverseField:188}
\inv{2} \lr{ \spacegrad_t F \pm \Be_3 \lr{ \spacegrad_t F } \Be_3 }
=
\inv{2} \lr{ \spacegrad_t F \mp \spacegrad_t \Be_3 F \Be_3 }
=
\spacegrad_t
\inv{2} \lr{ F \mp \Be_3 F \Be_3 },
\end{dmath}
or
\begin{dmath}\label{eqn:transverseField:208}
\begin{aligned}
\inv{2} \lr{ \spacegrad_t F + \Be_3 \lr{ \spacegrad_t F } \Be_3 } &= \spacegrad_t F_t \\
\inv{2} \lr{ \spacegrad_t F - \Be_3 \lr{ \spacegrad_t F } \Be_3 } &= \spacegrad_t F_z,
\end{aligned}
\end{dmath}
so Maxwell's equation \cref{eqn:transverseField:148} becomes
\begin{dmath}\label{eqn:transverseField:228}
\begin{aligned}
\spacegrad_t F_t &= - j \lr{ \frac{\omega}{c} \mp k \Be_3 } F_z \\
\spacegrad_t F_z &= - j \lr{ \frac{\omega}{c} \mp k \Be_3 } F_t.
\end{aligned}
\end{dmath}

Provided \( \omega^2 \ne (k c)^2 \), these can be inverted.
Such an inversion allows an application of the transverse gradient to whichever one
of \( F_z, F_t \) is known, to compute the other, as stated in
\cref{thm:transverseField:348}.

The relation for \( F_t \) in
\cref{thm:transverseField:348}
is usually stated in terms of the electric and magnetic fields.
To perform that expansion, we must first evaluate the multivector inverse explicitly
\begin{dmath}\label{eqn:transverseField:348}
\begin{aligned}
F_z &= j \frac{ \frac{\omega}{c} \pm k \Be_3 }{ \lr{\frac{\omega}{c}}^2 - k^2 } \spacegrad_t F_t \\
F_t &= j \frac{ \frac{\omega}{c} \pm k \Be_3 }{ \lr{\frac{\omega}{c}}^2 - k^2 } \spacegrad_t F_z.
\end{aligned}
\end{dmath}
so that we are in position to expand most of the terms in the numerator
\begin{dmath}\label{eqn:transverseField:268}
\lr{ \frac{\omega}{c} \pm k \Be_3 } \spacegrad_t F_z
=
-\lr{ \Be_3 \frac{\omega}{c} \pm k } \spacegrad_t \Be_3 F_z
=
\lr{ \pm k - \Be_3 \frac{\omega}{c} } \spacegrad_t \lr{ E_z + I \eta H_z }
=
\lr{
   \pm k \spacegrad_t E_z
   + \frac{\omega \eta}{c} \Be_3 \cross \spacegrad_t H_z
}
+ I \lr{
   \pm k \eta \spacegrad_t H_z
   -
   \frac{\omega}{c}
   \Be_3 \cross \spacegrad_t E_z
},
\end{dmath}
from which the transverse electric and magnetic fields stated in
\cref{thm:transverseField:348} can be read off.
A similar expansion for \( \BE_z, \BH_z \) in terms of \( \BE_t, \BH_t \) is also possible.

%There is considerably more complexity required to express the transverse field in terms of separate electric and magnetic components
%compared to the equivalent total transverse field expression of...

\makeproblem{Transverse electric and magnetic field components.}{problem:transverseField:1}{
Fill in the missing details in the steps of \cref{eqn:transverseField:268}.
} % problem

\makeproblem{Propagation direction components.}{problem:transverseField:2}{
Perform an expansion like \cref{eqn:transverseField:268} to find
\( \BE_z, \BH_z \) in terms of \( \BE_t, \BH_t \).
} % problem


%}
\EndArticle
%\EndNoBibArticle
