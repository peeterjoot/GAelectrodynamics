%
% Copyright � 2018 Peeter Joot.  All Rights Reserved.
% Licenced as described in the file LICENSE under the root directory of this GIT repository.
%
\maketheorem{Lorentz force and power.}{thm:lorentzForce:300}{
The respective power and force experienced by particles with electric (and/or magnetic) charges is described by
\cref{dfn:lorentzForce:280} is
\begin{equation*}
\inv{c} \frac{dT}{dt} = \gpgrade{ F Q^\dagger }{0,1} = \inv{2} \lr{ F^\dagger Q + F Q^\dagger }.
\end{equation*}
where \( \gpgradezero{dT/dt} = \ifrac{d\calE}{dt} \) is the power and \( \gpgradeone{dT/dt} = c \ifrac{d\Bp}{dt} \) is the force on the particle, and
\( Q^\dagger \) is the electric or magnetic charge/velocity multivector of \cref{dfn:lorentzForce:280}.
The conventional representation of the Lorentz force/power equations
\begin{equation*}
\begin{aligned}
\gpgradeone{ F Q^\dagger } &= \ddt{\Bp} = q \lr{ \BE + \Bv \cross \BB } \\
c \gpgradezero{ F Q^\dagger } &= \ddt{\calE} = q \BE \cdot \Bv.
\end{aligned}
\end{equation*}
%given by \cref{eqn:freespace:180}
may be recovered by grade selection operations.
For magnetic particles, such a grade selection gives
\begin{equation*}
\begin{aligned}
\gpgradeone{ F Q^\dagger } &= \frac{d\Bp}{dt} = q_\txtm \lr{ c \BB - \inv{c} \Bv_\txtm \cross \BE } \\
c \gpgradezero{ F Q^\dagger } &= \frac{d\calE}{dt} = \inv{\eta} q_\txtm \BB \cdot \frac{\Bv_\txtm}{c}.
\end{aligned}
\end{equation*}
} % theorem
