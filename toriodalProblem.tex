%
% Copyright © 2023 Peeter Joot.  All Rights Reserved.
% Licenced as described in the file LICENSE under the root directory of this GIT repository.
%
%{
\makeproblem{Curvilinear basis for toroidal parameterization.}{problem:toriodalProblem:1}{
Prove \cref{eqn:torusCenterOfMassParameterization:3}.
} % problem

\makeanswer{problem:toriodalProblem:1}{
We'll only compute \( \Bx_\theta \) here explicitly, as the other two vectors can be computed by inspection.

We start with a plain old chain rule expansion, with the cavaet that we must b e careful not to commute \( j \) with anything but the \( e^{\pm j \theta/2} \) terms.
\begin{equation}\label{eqn:toriodal:21}
\begin{aligned}
\Bx_\theta &= \PD{\theta}{\Bx} \\
&=
-\frac{j}{2}
e^{-j\theta/2} \left( \rho \Be_1 e^{ i \phi } + R \Be_3 \right) e^{j \theta/2}
+
e^{-j\theta/2} \left( \rho \Be_1 e^{ i \phi } + R \Be_3 \right) e^{j \theta/2}
\frac{j}{2}
\end{aligned}
\end{equation}
Note that the bivector \( j \) commutes with \( \Be_1 \), and then proceed to compute the \( \rho \) dependent part of \( \Bx_\theta \)
\begin{equation}\label{eqn:toriodal:41}
\begin{aligned}
& \frac{\rho}{2} e^{-j \theta/2} \Be_1 \lr{ -j e^{i \phi}  + e^{i \phi} j } e^{ j\theta/2 } \\
&=\frac{\rho}{2} e^{-j \theta/2} \Be_1 \lr{ - \Be_{32} \lr{ \cos\phi + \Be_{13} \sin\phi} + \lr{ \cos\phi + \Be_{13} \sin\phi} \Be_{32} } e^{ j\theta/2 } \\
&=\frac{\rho}{2} e^{-j \theta/2} \Be_1 \lr{ - \Be_{3213}  \sin\phi + \Be_{1332} \sin\phi } e^{ j\theta/2 } \\
&=\frac{\rho}{2} e^{-j \theta/2} \Be_1 \lr{ - \Be_{21}  \sin\phi + \Be_{12} \sin\phi } e^{ j\theta/2 } \\
&=      \rho e^{-j \theta/2} \Be_{112} \sin\phi e^{ j\theta/2 } \\
&=      \rho e^{-j \theta/2} \Be_{2} \sin\phi e^{ j\theta/2 }.
%&=      \rho \sin\phi \Be_2 e^{j \theta/2} e^{ j\theta/2 } \\
%&=      \rho \sin\phi \Be_2 e^{j \theta}.
\end{aligned}
\end{equation}
Similarly, the \( R \) dependent contribution is
\begin{equation}\label{eqn:toriodal:61}
\begin{aligned}
& \frac{R}{2} e^{-j \theta/2} \lr{ -j \Be_3  + \Be_3 j } e^{ j\theta/2 } \\
& \frac{R}{2} e^{-j \theta/2} \lr{ -\Be_{323}  + \Be_{332} } e^{ j\theta/2 } \\
& \frac{R}{2} e^{-j \theta/2} \lr{ \Be_{2}  + \Be_{2} } e^{ j\theta/2 } \\
&       R     e^{-j \theta/2} \Be_{2}  e^{ j\theta/2 }.
\end{aligned}
\end{equation}
Putting the pieces together, we have
\begin{equation}\label{eqn:toriodal:81}
\Bx_\theta = e^{-j \theta/2} \lr{ R + \rho \sin\phi } \Be_{2}  e^{ j\theta/2 },
\end{equation}
as claimed.
} % answer
%}
