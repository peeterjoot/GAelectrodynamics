Let's now look at how
\cref{eqn:SimpleProducts2:1440},
the dot and wedge product decomposition of the vector product, can be applied to compute vector projection and rejection, which are defined as

\makedefinition{Projection and rejection}{dfn:projectionAndRejection:projectionandrejection}{
Given a vector \( \Bx \) and vector \( \Bu \) the projection of \( \Bx \) onto the direction of \( \Bu \) is defined as

\begin{equation*}
\Proj_\Bu(\Bx) = (\Bx \cdot \ucap) \ucap,
\end{equation*}

where \( \ucap = \Bu/\Norm{\Bu} \).  The rejection of \( \Bu \) from \( \Bx \) is defined as the component of \( \Bx \) that is
normal to \( \Bu \)

\begin{equation*}
\Rej{\Bu}{\Bx} = \Bx - \Proj_\Bu(\Bx).
\end{equation*}
} % definition

It is possible to factor unity as \( 1 = \ucap \ucap \) to derive a useful GA representation of the projective and rejective components of a vector

\begin{dmath}\label{eqn:SimpleProducts2:680}
\Bx =
\Bx \ucap \ucap
=
\lr{ \Bx \ucap } \ucap
=
\Biglr{ \Bx \cdot \ucap + \Bx \wedge \ucap } \ucap
=
\lr{ \Bx \cdot \ucap } \ucap + \lr{ \Bx \wedge \ucap } \ucap.
\end{dmath}

The vector \( \Bx \) is split nicely into its projection and rejective components in a complementary fashion

\begin{dmath}\label{eqn:projectionAndRejection:900}
\begin{aligned}
\Proj_\Bu(\Bx) &= \lr{ \Bx \cdot \ucap } \ucap \\
\Rej{\Bu}{\Bx} &= \lr{ \Bx \wedge \ucap } \ucap.
\end{aligned}
\end{dmath}

By construction,
\( \lr{ \Bx \wedge \ucap } \ucap \) must be a vector, despite any appearance of a multivector nature.

The utility of this multivector rejection formula is not for hand (or computer algebra) calculations, where it will generally be faster and simpler to compute \( \Bx - (\Bx \cdot \ucap) \ucap \).  Instead this will come in handy as a new abstract algebraic tool.

When it is desirable to perform this calculation explicitly, it can be done more effieciently using a no-op grade selection operation.
In particular, a vector can be written as it's own grade-1 selection

\begin{dmath}\label{eqn:projectionAndRejection:920}
\Bx = \gpgradeone{ \Bx },
\end{dmath}

so the rejection can be reexpressed as a generalized bivector-vector dot product

\begin{dmath}\label{eqn:projectionAndRejection:940}
\Rej{\Bu}{\Bx}
= \gpgradeone{ \lr{ \Bx \wedge \ucap } \ucap }
= \lr{ \Bx \wedge \ucap } \cdot \ucap,
\end{dmath}

using
\cref{dfn:gradeselection:100}.

Let's consider some example rejection calculations to verify that this produces the expected results, starting with a simple two dimensional example

\makeexample{\R{2} rejection}{example:projectionAndRejection:1}{

Let \( \Bx = a \Be_1 + b \Be_2 \) and \( \Bu = \Be_1 \) for which the wedge is \( \Bx \wedge \ucap = b \Be_2 \Be_1 ).

The rejection of \( \Bu \) from \( \Bx \) is
\begin{dmath}\label{eqn:projectionAndRejection:1000}
\Rej{\Bu}{\Bx}
=
\lr{ \Bx \wedge \ucap } \ucap
=
(b \Be_2 \Be_1 )\Be_1
=
b \Be_2 (\Be_1 \Be_1)
=
b \Be_2,
\end{dmath}

as expected.
} % example

Geometrically, the wedge operation produces a pseudoscalar for the plane spanned by \( \setlr{\Bx, \Bu} \), and multiplication of that pseudoscalar by \( \ucap \) rotates \( \ucap \) by \( \pi/2 \) radians in that plane, yeilding the normal component of the vector \( \Bx \).

Here's a slightly less trivial \R{3} example

\makeexample{\R{3} rejection.}{example:projectionAndRejection:r3rejection}{
Let \( \Bx = a \Be_2 + b \Be_3 \) and \( \ucap = ( \Be_1 + \Be_2 )/\sqrt{2} \) for which the 
wedge product is

\begin{dmath}\label{eqn:projectionAndRejection:1040}
\Bx \wedge \ucap = \inv{\sqrt{2}}
\begin{vmatrix}
\Be_{23} & \Be_{31} & \Be_{12} \\
0 & a & b \\
1 & 1 & 0
\end{vmatrix}
=
\inv{\sqrt{2}}
\lr{
\Be_{23}(-b) - \Be_{31}(-b) + \Be_{12} (-a)
}
=
\inv{\sqrt{2}}
\lr{
b (\Be_{32} + \Be_{31} ) + a \Be_{21}
}.
\end{dmath}

Using \cref{eqn:projectionAndRejection:940} the rejection of \( \Bu \) from \( \Bx \) is

\begin{dmath}\label{eqn:projectionAndRejection:1060}
(\Bx \wedge \ucap) \cdot \ucap
=
\inv{2}
\lr{ b (\Be_{32} + \Be_{31} ) + a \Be_{21} } \cdot ( \Be_1 + \Be_2 )
=
\inv{2}
\lr{ b \Be_3 + a \Be_2 + b \Be_3 - a \Be_1 }
=
b \Be_3 + \frac{a}{2}( \Be_2 - \Be_1 ).
\end{dmath}

The reader should confirm that this equals \( \Bx - (\Bx \cdot \ucap) \ucap \).
} % example

The normal nature of the projection and rejection in this form can be demonstrated by calculating the dot product using the scalar selection operator

\begin{dmath}\label{eqn:SimpleProducts2:720}
\Rej{\Bu}{\Bx} \cdot \Proj_\Bu(\Bx)
=
\gpgradezero{ \Rej{\Bu}{\Bx} \Proj_\Bu(\Bx) }
=
\gpgradezero{ \lr{ \Bx \wedge \ucap } \ucap \lr{ \Bx \cdot \ucap } \ucap }
=
\lr{ \Bx \cdot \ucap } \gpgradezero{ \lr{ \Bx \wedge \ucap } \ucap^2 }
=
\lr{ \Bx \cdot \ucap } \gpgradezero{ \Bx \wedge \ucap },
\end{dmath}

however, selecting the scalar grade of the wedge product, a (grade-2) bivector, is zero.

(cut)
The pythagorean property of these two vector components can also be checked.
Computing the squared length using \( \Norm{\By}^2 = \By \cdot \By = \By^2 \), the squared length of the projective component is

\begin{dmath}\label{eqn:SimpleProducts2:740}
\lr{ \lr{\Bx \cdot \ucap } \ucap }^2
=
\lr{\Bx \cdot \ucap }^2
=
(x_1 u_1 + x_2 u_2)^2
=
x_1^2 u_1^2 + x_2^2 u_2^2 + 2 x_1 x_2 u_1 u_2.
\end{dmath}

The squared length of the rejective component is
\begin{dmath}\label{eqn:SimpleProducts2:760}
\lr{ \lr{\Bx \wedge \ucap } \ucap }^2
=
-(\Bx \wedge \ucap) \ucap^2 (\Bx \wedge \ucap)
=
-
\lr{\begin{vmatrix}
   x_1 & x_2 \\
   u_1 & u_2
\end{vmatrix}}^2
(\Be_1 \Be_2)^2
=
x_1^2 u_2^2 + x_2^2 u_1^2 - 2 x_1 x_2 u_1 u_2.
\end{dmath}

Adding these together gives

\begin{dmath}\label{eqn:SimpleProducts2:780}
\lr{ \lr{\Bx \cdot \ucap } \ucap }^2 + \lr{ \lr{\Bx \wedge \ucap } \ucap }^2
=
x_1^2 u_1^2 + x_2^2 u_2^2
+x_1^2 u_2^2 + x_2^2 u_1^2
=
x_1^2 ( u_1^2 + u_2^2 )
+
x_2^2 ( u_1^2 + u_2^2 )
=
\Bx^2,
\end{dmath}

recovering the squared length of the vector as expected.
It is generally true in higher dimensions that the projection and rejection can be written as

\begin{dmath}\label{eqn:SimpleProducts2:800}
\begin{aligned}
\Proj_\ucap(\Bx) &= (\Bx \cdot \ucap) \ucap \\
\RejName_\ucap(\Bx) &= (\Bx \wedge \ucap) \ucap.
\end{aligned}
\end{dmath}

The Pythagorean aspect of this statement in higher degree spaces
will be demonstrated later in a coordinate free fashion after some additional identities have been derived.

The unit vector restriction defining the direction of projection and rejection can be relaxed in a compact fashion by introducing the vector \boldTextAndIndex{inverse}, which is always well defined and unique in a Euclidean space

\boxedEquation{eqn:SimpleProducts2:860}{
\inv{\Bu} \equiv \frac{\Bu}{\Bu^2}.
}

Now the projection and rejection onto the direction of \( \Bu \) are

\boxedEquation{eqn:SimpleProducts2:880}{
\begin{aligned}
\Proj_\Bu(\Bx) &= (\Bx \cdot \Bu) \inv{\Bu} \\
\RejName_\Bu(\Bx) &= (\Bx \wedge \Bu) \inv{\Bu}.
\end{aligned}
}
\index{projection}
\index{rejection}

An illustrative example of both projection and rejection is plotted in \cref{fig:projectionAndRejection:projectionAndRejectionFig1}.

\imageFigure{../figures/GAelectrodynamics/projectionAndRejectionFig1}{Projection and rejection illustrated.}{fig:projectionAndRejection:projectionAndRejectionFig1}{0.3}

\makedigression{Wedge and cross product relationships}{
Given that the wedge product has the ``cross product like'' properties \( \Bx \wedge \Bx = 0 \), and \( \Bx \wedge \By = -\By \wedge \Bx \),
and because the \R{3} expression of the rejection is

\begin{equation}\label{eqn:SimpleProducts2:820}
\RejName_\ucap(\Bx) = \Bx - (\Bx \cdot \ucap) \ucap = \ucap \cross (\Bx \cross \ucap),
\end{equation}

which is clearly similar to that of \cref{eqn:SimpleProducts2:800}, the observant reader may see from this expression that the wedge product in \R{3} seems to be related to the cross product in some fashion.
The precise nature of that relationship will be detailed later.
} % digression

\makeproblem{Rejection normality}{problem:projectionAndRejection:rejectionnormality}{
Prove, without any use of GA, that \( \Bx - \Proj_\Bu(\Bx) \) is normal to \( \Bu \), as claimed in
\cref{dfn:projectionAndRejection:projectionandrejection}.
} % problem


