%
% Copyright © 2017 Peeter Joot.  All Rights Reserved.
% Licenced as described in the file LICENSE under the root directory of this GIT repository.
%
Geometric algebra (GA for short), generalizes and extends vector algebra.
The following section contains a lighting review of some 
foundational concepts, including
vector space, basis, orthonormality, and metric.
If you are inclined to skip this, please at least examine the
stated dot product definition, since the conventional positive definite property is not assumed.

\subsection{Vector.}

(PASTED: ... fixup per notes on paper copy).
%, and possible orientation, as illustrated in
%\cref{fig:VectorsWithOppositeOrientation:VectorsWithOppositeOrientationFig1}, where the length of the line segment represents the magnitude, and
%the sign of the vector can be represented graphically using the relative placement of the head vs. tail of the vector.
When vectors are negated the relative placement of the ``head'' vs. the ``tail'' are toggled as illustrated in \cref{fig:VectorsWithOppositeOrientation:VectorsWithOppositeOrientationFig1}.
\imageFigure{../figures/GAelectrodynamics/VectorsWithOppositeOrientationFig1}{Vector illustration.}{fig:VectorsWithOppositeOrientation:VectorsWithOppositeOrientationFig1}{0.3}
%One vector in isolation is a one dimensional object, however, when embedded in a higher order space, such as a plane or a volume, it also has an orientation in that space.

Vectors can be added graphically by connecting them head to tail in sequence, and joining the first tail point to the final head, as
illustrated in
\cref{fig:vectorAddition:vectorAdditionFig1}.
\imageFigure{../figures/GAelectrodynamics/vectorAdditionFig1}{Graphical vector addition.}{fig:vectorAddition:vectorAdditionFig1}{0.3}

