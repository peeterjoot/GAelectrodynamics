%
% Copyright © 2017 Peeter Joot.  All Rights Reserved.
% Licenced as described in the file LICENSE under the root directory of this GIT repository.
%
\index{Green's function}
\index{convolution}
From \cref{thm:gradientGreensFunctionEuclidean:720}
%From \cref{thm:gradientGreensFunctionEuclidean:1}, the
\R{3} Green's function for the gradient (on an infinite spherical bounding surface) is

\begin{dmath}\label{eqn:electrostatics_invertingGradient:260}
G(\Bx, \Bx') = \inv{4 \pi} \frac{\Bx - \Bx'}{\Norm{\Bx - \Bx'}^3},
\end{dmath}
so the convolution that inverts the electric field gradient equation is
\begin{dmath}\label{eqn:electrostatics_invertingGradient:621}
\BE(\Bx)
= \int_V dV' G(\Bx, \Bx') \spacegrad' \BE(\Bx')
= \int_V dV' G(\Bx, \Bx') \lr{ \inv{\epsilon}\rho(\Bx') }
= \inv{4\pi} \int_V dV' \frac{\Bx - \Bx'}{ \Abs{\Bx - \Bx'}^3 } \lr{ \inv{\epsilon}\rho(\Bx') },
\end{dmath}
or
%\begin{dmath}\label{eqn:electrostatics_invertingGradient:340}
\boxedEquation{eqn:electrostatics_invertingGradient:340}{
\BE(\Bx) =
\inv{4 \pi \epsilon} \int dV' \rho(\Bx') \frac{\Bx - \Bx'}{\Norm{\Bx - \Bx'}^3},
}
%\end{dmath}
which is Coulomb's law.

The convolution for the magnetic field is trivial
\begin{dmath}\label{eqn:electrostatics_invertingGradient:641}
\BB(\Bx)
= \int_V dV' G(\Bx, \Bx') \spacegrad' \BB(\Bx')
= \int_V dV' G(\Bx, \Bx') (0),
\end{dmath}
so the magnetic field is zero everywhere
\begin{dmath}\label{eqn:electrostatics_invertingGradient:320}
\BB(\Bx) = 0.
\end{dmath}

%Question: would a non-zero magnetic field solution be possible if a Green's function for a finite bounded surface were to be used instead?

