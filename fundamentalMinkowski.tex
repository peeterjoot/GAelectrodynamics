%
% Copyright � 2018 Peeter Joot.  All Rights Reserved.
% Licenced as described in the file LICENSE under the root directory of this GIT repository.
%
%{
\input{../latex/blogpost.tex}
\renewcommand{\basename}{fundamentalMinkowski}
%\renewcommand{\dirname}{notes/phy1520/}
\renewcommand{\dirname}{notes/ece1228-electromagnetic-theory/}
%\newcommand{\dateintitle}{}
%\newcommand{\keywords}{}

\input{../latex/peeter_prologue_print2.tex}

\usepackage{peeters_layout_exercise}
\usepackage{peeters_braket}
\usepackage{peeters_figures}
\usepackage{siunitx}
%\usepackage{mhchem} % \ce{}
%\usepackage{macros_bm} % \bcM
%\usepackage{macros_qed} % \qedmarker
%\usepackage{txfonts} % \ointclockwise

\beginArtNoToc

% https://math.stackexchange.com/questions/2819680/the-fundamental-theorem-of-geometric-calculus-in-a-lorentzian-manifold
\generatetitle{Fundamental theorem of calclus in a Minkowski metric}
%\chapter{Fundamental theorem of calclus in a Minkowski metric}
%\label{chap:fundamentalMinkowski}

The fundamental theorem itself is independent of metric.
Illustrating by example, given a two parameter surface
\begin{dmath}\label{eqn:fundamentalMinkowski:20}
x = x(a,b) \quad (=\gamma_\mu x^\mu(a, b)),
\end{dmath}
for which the area element is
\begin{dmath}\label{eqn:fundamentalMinkowski:40}
\begin{aligned}
d^2 x
&= \lr{ \PD{a}{x} \wedge \PD{b}{x} } da db \\
&= (x_a \wedge x_b) da db \\
&= dx_a \wedge dx_b,
\end{aligned}
\end{dmath}
and for which the vector derivative is (no summation convention here)
\begin{dmath}\label{eqn:fundamentalMinkowski:60}
\partial = x^a \partial_a + x^b \partial_b,
\end{dmath}
the integral expands as
\begin{dmath}\label{eqn:fundamentalMinkowski:80}
\begin{aligned}
\int d^2 x \partial F
&=
\int da db (x_a \wedge x_b) \lr{ x^a \partial_a + x^b \partial_b } F \\
&=
\int da db \lr{ -x_b \partial_a F + x_a \partial_b F } \\
&=
\int da db \lr{ -\PD{a}{} \lr{ x_b F} + \PD{b}{}\lr{ x_a F} }
-\int da db \lr{ -\PD{a}{} \lr{ x_b } + \PD{b}{}\lr{ x_a } } F
\\
&=
\int -dx_b \evalbar{F}{\Delta a} +dx_a \evalbar{F}{\Delta b},
\end{aligned}
\end{dmath}
where the second integral was killed by equality of mixed partials (\( -\PD{a}{} \PD{b}{x } + \PD{b}{} \PD{a }{x} = 0 \).)

The curvilinear coordinates \( x_a, x_b \) and the reciprocal frame vectors \( x^a, x^b \), do the heavy lifting in this expansion, and don't require any explicit mention of the metric.

The metric can be brought into the mix here explicitly if desired, since the
(Minkowski four-) gradient relates the curvilinear coordinates associated with the parameterization:
\begin{dmath}\label{eqn:fundamentalMinkowski:100}
\begin{aligned}
x^a &= \grad a = \gamma^\mu \partial_\mu a \\
x^b &= \grad b = \gamma^\mu \partial_\mu b.
\end{aligned}
\end{dmath}
(summation convention here.)

To prove the fundamental theorem for \( m > 2 \), which determines the form of \( d^{m-1} x \), a simular procedure is required.
The results of interest for Minkowski space are (assuming parameters \( a, b, c, d \))

\begin{dmath}\label{eqn:fundamentalMinkowski:120}
\begin{aligned}
\int d^1 x \partial F &= \evalbar{F}{\Delta a} \\
\int d^2 x \partial F &=
-\int dx_b \evalbar{F}{\Delta a}
+\int dx_a \evalbar{F}{\Delta b} \\
\int d^3 x \partial F &=
\int dx_a \wedge dx_b \evalbar{F}{\Delta c}
+\int dx_b \wedge dx_c \evalbar{F}{\Delta a}
+\int dx_c \wedge dx_a \evalbar{F}{\Delta b} \\
\int d^4 x \partial F &=
\int dx_a \wedge dx_b \wedge dx_c \evalbar{F}{\Delta d}
-\int dx_d \wedge dx_a \wedge dx_b \evalbar{F}{\Delta c}
+\int dx_c \wedge dx_d \wedge dx_a \evalbar{F}{\Delta b}
-\int dx_b \wedge dx_c \wedge dx_d \evalbar{F}{\Delta a}
\end{aligned}
\end{dmath}

%}
\EndArticle
%\EndNoBibArticle
