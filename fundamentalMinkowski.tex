%
% Copyright � 2018 Peeter Joot.  All Rights Reserved.
% Licenced as described in the file LICENSE under the root directory of this GIT repository.
%
%{
\input{../latex/blogpost.tex}
\renewcommand{\basename}{fundamentalMinkowski}
%\renewcommand{\dirname}{notes/phy1520/}
\renewcommand{\dirname}{notes/ece1228-electromagnetic-theory/}
%\newcommand{\dateintitle}{}
%\newcommand{\keywords}{}

\input{../latex/peeter_prologue_print2.tex}

\usepackage{peeters_layout_exercise}
\usepackage{peeters_braket}
\usepackage{peeters_figures}
\usepackage{siunitx}
%\usepackage{mhchem} % \ce{}
%\usepackage{macros_bm} % \bcM
%\usepackage{macros_qed} % \qedmarker
%\usepackage{txfonts} % \ointclockwise

\beginArtNoToc

% https://math.stackexchange.com/questions/2819680/the-fundamental-theorem-of-geometric-calculus-in-a-lorentzian-manifold
\generatetitle{Fundamental theorem of calclus in a Minkowski metric}
%\chapter{Fundamental theorem of calclus in a Minkowski metric}
%\label{chap:fundamentalMinkowski}

The fundamental theorem itself is independent of metric.
Illustrating by example, given a two parameter surface
\begin{dmath}\label{eqn:fundamentalMinkowski:20}
\Bx = \Bx(a,b) \quad (=\gamma_\mu x^\mu(a, b)),
\end{dmath}
for which the area element is
\begin{dmath}\label{eqn:fundamentalMinkowski:40}
\begin{aligned}
d^2 \Bx
&= \lr{ \PD{a}{\Bx} \wedge \PD{b}{\Bx} } da db \\
&= (\Bx_a \wedge \Bx_b) da db \\
&= d\Bx_a \wedge d\Bx_b,
\end{aligned}
\end{dmath}
and for which the vector derivative is (no summation convention here)
\begin{dmath}\label{eqn:fundamentalMinkowski:60}
\partial = \Bx^a \partial_a + \Bx^b \partial_b,
\end{dmath}
the integral expands as
\begin{dmath}\label{eqn:fundamentalMinkowski:80}
\begin{aligned}
\int d^2 \Bx \partial F
&=
\int da db (\Bx_a \wedge \Bx_b) \lr{ \Bx^a \partial_a + \Bx^b \partial_b } F \\
&=
\int da db \lr{ -\Bx_b \partial_a F + \Bx_a \partial_b F } \\
&=
\int da db \lr{ -\PD{a}{} \lr{ \Bx_b F} + \PD{b}{}\lr{ \Bx_a F} }
-\int da db \lr{ -\PD{a}{} \lr{ \Bx_b } + \PD{b}{}\lr{ \Bx_a } } F
\\
&=
\int -d\Bx_b \evalbar{F}{\Delta a} +d\Bx_a \evalbar{F}{\Delta b},
\end{aligned}
\end{dmath}
where the second integral was killed by equality of mixed partials (\( -\PD{a}{} \PD{b}{\Bx } + \PD{b}{} \PD{a }{\Bx} = 0 \).)

The curvilinear coordinates \( \Bx_a, \Bx_b \) and the reciprocal frame vectors \( \Bx^a, \Bx^b \), do the heavy lifting in this expansion, and don't require any explicit mention of the metric.

The metric can be brought into the mix here explicitly if desired, since the
(Minkowski four-) gradient relates the curvilinear coordinates associated with the parameterization:
\begin{dmath}\label{eqn:fundamentalMinkowski:100}
\begin{aligned}
\Bx^a &= \grad a = \gamma^\mu \partial_\mu a \\
\Bx^b &= \grad b = \gamma^\mu \partial_\mu b.
\end{aligned}
\end{dmath}
(summation convention here.)

To prove the fundamental theorem for \( m > 2 \), which determines the form of \( d^{m-1} \Bx \), a simular procedure is required.
The results of interest for Minkowski space are (assuming parameters \( a, b, c, d \))
\begin{dmath}\label{eqn:fundamentalMinkowski:120}
\begin{aligned}
\int d^1 \Bx \partial F &= \evalbar{F}{\Delta a} \\
\int d^2 \Bx \partial F &=
-\int d\Bx_b \evalbar{F}{\Delta a}
+\int d\Bx_a \evalbar{F}{\Delta b} \\
\int d^3 \Bx \partial F &=
\int d\Bx_a \wedge d\Bx_b \evalbar{F}{\Delta c}
+\int d\Bx_b \wedge d\Bx_c \evalbar{F}{\Delta a}
+\int d\Bx_c \wedge d\Bx_a \evalbar{F}{\Delta b} \\
\int d^4 \Bx \partial F &=
\int d\Bx_a \wedge d\Bx_b \wedge d\Bx_c \evalbar{F}{\Delta d}
-\int d\Bx_d \wedge d\Bx_a \wedge d\Bx_b \evalbar{F}{\Delta c}
+\int d\Bx_c \wedge d\Bx_d \wedge d\Bx_a \evalbar{F}{\Delta b}
-\int d\Bx_b \wedge d\Bx_c \wedge d\Bx_d \evalbar{F}{\Delta a}
\end{aligned}
\end{dmath}

To make this more concrete, consider the following specific parameterization of a spacetime surface, with a linear timelike path component, and a component that is a boost along the x-direction
\begin{dmath}\label{eqn:fundamentalMinkowski:140}
\Bx(a, b) = \gamma_0 a + \gamma_1 \exp\lr{ \gamma_1 \gamma_0 b }.
\end{dmath}
Here \( a, b \) are any two of the four possible coordinates \( ct, x, y, z \).

The curvilinear coordinates with respect to parameters \( a, b \) are
\begin{dmath}\label{eqn:fundamentalMinkowski:160}
\begin{aligned}
\Bx_a &= \gamma_0 \\
\Bx_b &= - \gamma_0 \exp\lr{ \gamma_1 \gamma_0 b },
\end{aligned}
\end{dmath}
so the area element is
\begin{dmath}\label{eqn:fundamentalMinkowski:180}
\begin{aligned}
d^2 \Bx
&=
\lr{ \Bx_a \wedge \Bx_b } da db \\
&=
- \gpgradetwo{ \exp\lr{ \gamma_1 \gamma_0 b } } da db \\
&=
- \gamma_1 \gamma_0 \sinh( b ) da db.
\end{aligned}
\end{dmath}
The reciprocal frame vectors are
\begin{dmath}\label{eqn:fundamentalMinkowski:200}
\begin{aligned}
\Bx^a &= \Bx_b \cdot \inv{ \Bx_a \wedge \Bx_b } = - \frac{\gamma_1  \exp\lr{ \gamma_1 \gamma_0 b }}{\sinh(b)} \\
\Bx^b &= \Bx_a \cdot \inv{ \Bx_b \wedge \Bx_a } = - \frac{\gamma_1}{\sinh(b)},
\end{aligned}
\end{dmath}
which satisfy \( \Bx^a \cdot \Bx_a = \Bx^b \cdot \Bx_b = 1 \), and \( \Bx^a \cdot \Bx_b = \Bx^b \cdot \Bx_a =  0 \), which can be shown explicitly with relative ease using scalar selection operations.
The vector derivative is
\begin{dmath}\label{eqn:fundamentalMinkowski:220}
\partial = - \frac{\gamma_1}{\sinh(b)} \lr{ \PD{b}{} + \exp\lr{ \gamma_1 \gamma_0 b } \PD{a}{} }.
\end{dmath}

This is enough to explicitly express the left and right hand sides of the fundamental theorem identity
\begin{dmath}\label{eqn:fundamentalMinkowski:240}
\int d^2 \Bx \partial F
=
\gamma_0 \int da db \lr{ \PD{b}{} + \exp\lr{ \gamma_1 \gamma_0 b } \PD{a}{} } F,
\end{dmath}
and
\begin{dmath}\label{eqn:fundamentalMinkowski:260}
-\int d\Bx_b \evalbar{F}{\Delta a}
+\int d\Bx_a \evalbar{F}{\Delta b}
=
\gamma_0 \int db \exp\lr{ \gamma_1 \gamma_0 b } \evalbar{F}{\Delta a}
+ \gamma_0 \int da \evalbar{F}{\Delta b}.
\end{dmath}
As expected by the theorem, these are clearly identical.

%}
\EndArticle
%\EndNoBibArticle
