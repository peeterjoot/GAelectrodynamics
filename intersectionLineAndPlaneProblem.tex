%
% Copyright © 2017 Peeter Joot.  All Rights Reserved.
% Licenced as described in the file LICENSE under the root directory of this GIT repository.
%
%{
\makeproblem{Intersection of a line and plane.}{problem:solutionOfLinearSystem:1}{
Let a line be parameterized by
\begin{equation*}
\Br(a) = \Bp + \alpha \Ba,
\end{equation*}
and a plane be parameterized by
\begin{equation*}
\Br(b,c) = \Bq + \beta \Bb + \gamma \Bc.
\end{equation*}
\makesubproblem{}{problem:solutionOfLinearSystem:1:a}
For the intersection of the two, state the vector equation to be solved, and its solution for \( a \) in terms of a ratio of wedge products.
\makesubproblem{}{problem:solutionOfLinearSystem:1:b}
State the conditions for which the solution exist in \R{3} and \R{N}.
\makesubproblem{}{problem:solutionOfLinearSystem:1:c}
In terms of coordinates in \R{3} write out the ratio of wedge products as determinants and compare to the Cramer's rule solution.
} % problem
\makeanswer{problem:solutionOfLinearSystem:1}{
\makesubanswer{}{problem:solutionOfLinearSystem:1:a}
We are looking for solutions \( \alpha, \beta, \gamma \) such that the equality
\begin{equation}\label{eqn:solutionOfLinearSystem:1320}
\Bp + \alpha \Ba = \Bq + \beta \Bb + \gamma \Bc,
\end{equation}
is satisfied.
We have only to wedge with \( \Bb \wedge \Bc \), to find
\begin{equation}\label{eqn:solutionOfLinearSystem:1340}
\Bp \wedge \Bb \wedge \Bc + \alpha \lr{ \Ba \wedge \Bb \wedge \Bc }  = \Bq \wedge \Bb \wedge \Bc,
\end{equation}
or
\begin{equation}\label{eqn:solutionOfLinearSystem:1360}
\alpha = \frac{\lr{ \Bq - \Bp } \wedge \Bb \wedge \Bc }{ \Ba \wedge \Bb \wedge \Bc }.
\end{equation}
\makesubanswer{}{problem:solutionOfLinearSystem:1:b}
For \R{3}, a solution exists provided \( \Ba \wedge \Bb \wedge \Bc \ne 0 \), but for \R{N} a solution also requires
\begin{equation}\label{eqn:solutionOfLinearSystem:1380}
\lr{ \Bq - \Bp } \wedge \Bb \wedge \Bc \propto \Ba \wedge \Bb \wedge \Bc.
\end{equation}
For instance, there is no solution if \( \lr{ \Bq - \Bp } \wedge \Bb \wedge \Bc = \Be_{124} \), but \( \Ba \wedge \Bb \wedge \Bc = \Be_{234} \).
\makesubanswer{}{problem:solutionOfLinearSystem:1:c}
To solve this equation using coordinates, we seek solutions to
\begin{equation}\label{eqn:solutionOfLinearSystem:1400}
\Bp - \Bq = -\alpha \Ba + \beta \Bb + \gamma \Bc,
\end{equation}
or
\begin{equation}\label{eqn:solutionOfLinearSystem:1420}
\lr{ \Bp - \Bq } \cdot \Be_k = \lr{ -\alpha \Ba + \beta \Bb + \gamma \Bc } \cdot \Be_k,
\end{equation}
\( \forall k \in [1, N] \).  In matrix form, this is
\begin{equation}\label{eqn:solutionOfLinearSystem:1440}
\begin{bmatrix}
p_1 - q_1 \\
p_2 - q_2 \\
\vdots \\
p_N - q_N \\
\end{bmatrix}
=
\begin{bmatrix}
-a_1 & b_1 & c_1 \\
-a_2 & b_2 & c_2 \\
     & \vdots & \\
-a_N & b_N & c_N \\
\end{bmatrix}
\begin{bmatrix}
\alpha \\
\beta \\
\gamma
\end{bmatrix}.
\end{equation}
The Cramer's rule solution only applies to the \R{3} system, and has the form
\begin{equation}\label{eqn:solutionOfLinearSystem:1460}
\begin{bmatrix}
\alpha \\
\beta \\
\gamma
\end{bmatrix}
=
\frac{
\begin{vmatrix}
p_1 - q_1 & b_1 & c_1 \\
p_2 - q_2 & b_2 & c_2 \\
p_3 - q_3 & b_3 & c_3 \\
\end{vmatrix}
}
{
\begin{vmatrix}
-a_1 & b_1 & c_1 \\
-a_2 & b_2 & c_2 \\
-a_3 & b_3 & c_3 \\
\end{vmatrix}
}
=
\frac{
\begin{vmatrix}
q_1 - p_1 & b_1 & c_1 \\
q_2 - p_2 & b_2 & c_2 \\
q_3 - p_3 & b_3 & c_3 \\
\end{vmatrix}
}
{
\begin{vmatrix}
a_1 & b_1 & c_1 \\
a_2 & b_2 & c_2 \\
a_3 & b_3 & c_3 \\
\end{vmatrix}
}.
\end{equation}
This is obviously equivalent to the GA solution, where the ratio of determinants is found immediately from the coordinate representation of a triple wedge product.  We can't solve this system of equations using Cramer's rule for \R{N} when \( N > 3 \) since the system is overspecified in that case.  That overspecification is why we require the additional
\( \lr{ \Bq - \Bp } \wedge \Bb \wedge \Bc \propto \Ba \wedge \Bb \wedge \Bc \)
constraint for the GA solution using wedge products.  Note that this wedge product solution method is unlikely to be numerically stable for \( N > 3 \), and we are probably better off solving with SVD, so that we have some estimation of the numerical errors that either rule out or validate the solution.
} % answer
%}
