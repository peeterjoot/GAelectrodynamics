%
% Copyright © 2018 Peeter Joot.  All Rights Reserved.
% Licenced as described in the file LICENSE under the root directory of this GIT repository.
%
%{
\mychapter{Multivector calculus.}
   \section{Reciprocal frames.}
      \subsection{Motivation and definition.}
         \input{reciprocal.tex}
         \input{curvilinearDefinedCoordinates.tex}
      \subsection{\R{2} reciprocal frame.}
         \input{reciprocal_R2.tex}
      \subsection{\R{3} reciprocal frame.}
         \input{reciprocal_R3.tex}
      \subsection{Problems.}
         \input{reciprocal_orthogonal_problem.tex}
         \input{2dreciprocalMatrixCalculation.tex}
         \input{2subspaceR3reciprocalExample.tex}
   \section{Curvilinear bases.}
      \subsection{Two parameters.}
         \input{curvilinearDefined.tex}
      \subsection{Three (or more) parameters.}
         \input{curvilinearThree.tex}
      \subsection{Gradient.}
         \input{gradient.tex}
      \subsection{Vector derivative.}
         \input{vectorDerivative.tex}
      \subsection{Examples.}
         \input{curvilinearExamples.tex}
      \subsection{Problems.}
         %{
\makeproblem{Spherical polar exponential deriatives.}{problem:dexpquestion:1}{
Some steps were skipped in the computation of \cref{eqn:curvilinearspherical:140}, and there is a subtle aspect to that computation that was not discussed.

We have a spherical polar exponential representation: \( i = \mathbf{e}_{12}, \, j = \mathbf{e}_{31} e^{i\phi}, \, \mathbf{x} = r \mathbf{e}_3 e^{j\theta}\), and want to compute
\begin{equation*}
\frac{\partial \mathbf{x}}{\partial \phi} = r \mathbf{e}_3 \frac{\partial e^{j\theta}}{\partial \phi}.
\end{equation*}
We need to understand how to correctly compute that exponential derivative, where all the \( \phi \) dependence has been squirrelled away in the bivector \( j \).

The niave way to do that is to use the ``identity'' $(e^a)' = a' e^a$, and get:
\begin{equation*}
\frac{\partial e^{j\theta}}{\partial \phi} = \theta \frac{\partial j}{\partial \phi} e^{j\theta},
\end{equation*}
but this is wrong.  The correct way to do this is:
\begin{equation*}
\frac{\partial e^{j\theta}}{\partial \phi} = \frac{\partial }{\partial \phi} (\cos\theta + j \sin\theta) = \frac{\partial j }{\partial \phi} \sin\theta.
\end{equation*}

Demonstrate why the second approach works, but the first does not.

Hints: First show that $j j' \ne j' j$.  Then consider the derivative using the Taylor series representation of the exponential.  This has to be treated very carefully, as the binomial combinatorics for the derivatives of \( j^k \) in the Taylor expansion of \( e^{j\theta} \), do not behave in the usual (scalar) way.  That is \(  (j^k)' \ne k j' j^{k-1} \).  The take away is, just as with matrix algebra, we can't assume that $(e^a)' = a' e^a$, unless $a', a$ commute.
} % problem
\makeanswer{problem:dexpquestion:1}{
From the power series representation of the exponential, we compute
\begin{equation}\label{eqn:dexpquestion:60}
\begin{aligned}
\PD{\phi}{e^{j\theta}}
&= \sum_{k = 0}^\infty \PD{\phi}{} \frac{ (j \theta)^k }{k!} \\
&= \sum_{k = 1}^\infty \PD{\phi}{j^k} \frac{ \theta^k }{k!}.
\end{aligned}
\end{equation}
If you treat \( j \) as a complex number, this then reduces to
\begin{equation}\label{eqn:dexpquestion:80}
\begin{aligned}
\PD{\phi}{e^{j\theta}}
&= \sum_{k = 1}^\infty k \PD{\phi}{j} j^{k-1} \frac{ \theta^k }{k!} \\
&=
\theta \PD{\phi}{j} \sum_{k = 1}^\infty \frac{ (j\theta)^{k-1} }{(k-1)!} \\
&=
\theta \PD{\phi}{j} e^{j\theta}.
\end{aligned}
\end{equation}
But, this is wrong.  The reason that this is wrong is because \( \PDi{\phi}{j} \) does not commute with \( j \), as we can
see by computation.  Let \( j' = \PDi{\phi}{j} \), so that we have
\begin{equation}\label{eqn:dexpquestion:180}
\begin{aligned}
j &= \Be_{31} e^{i\phi} \\
j' &= \Be_{32} e^{i\phi}.
\end{aligned}
\end{equation}
Each of the products is a bivector
\begin{equation}\label{eqn:dexpquestion:200}
\begin{aligned}
j j'
&= \Be_{31} e^{i\phi} \Be_{32} e^{i\phi} \\
&= \Be_{3132} e^{-i\phi} e^{i\phi} \\
&= -\Be_{12},
\end{aligned}
\end{equation}
and
\begin{equation}\label{eqn:dexpquestion:220}
\begin{aligned}
j' j
&= \Be_{32} e^{i\phi} \Be_{31} e^{i\phi} \\
&= \Be_{3231} e^{-i\phi} e^{i\phi} \\
&= \Be_{12},
\end{aligned}
\end{equation}
but find that \( j \) and \( j' \), in this case, anticommute.  That is
\begin{equation}\label{eqn:dexpquestion:240}
j j' = -j' j.
\end{equation}

Because \( j j' \ne j' j \), we cannot write
\begin{equation}\label{eqn:dexpquestion:n}
\PD{\phi}{j^k} = k \lr{ \PD{\phi}{j} } j^{k-1},
\end{equation}
but instead have
\begin{equation}\label{eqn:dexpquestion:100}
\PD{\phi}{j^k} = \PD{\phi}{j} j^{k-1} + j \PD{\phi}{j} j^{k-2} + \cdots
\end{equation}
We can use the commuation relations above to compute
\begin{equation}\label{eqn:dexpquestion:260}
\begin{aligned}
\PD{\phi}{j^k}
&= j' j^{k-1} + j j' j^{k-2} + j^2 j' j^{k-3} \cdots \\
&= j' j^{k-1} - j' j^{k-1} + (-1)^2 j' j^{k-1} \cdots
\end{aligned}
\end{equation}
This is zero for any even \( k \) and \( j' j^{k-1} \) for odd \( k \).

Plugging this back into our Taylor series for the derivative (before we messed it up), we find
\begin{equation}\label{eqn:dexpquestion:280}
\begin{aligned}
\PD{\phi}{e^{j\theta}}
&= \sum_{k = 1, k \in \mathrm{odd}}^\infty j' j^{k-1} \frac{ \theta^k }{k!} \\
&= j' \inv{j}
\sum_{k = 1,\, k \in \mathrm{odd}}^\infty \frac{ (j\theta)^k }{k!} \\
&= j' \inv{j} \sinh( j \theta ) \\
&= j' \inv{j} j \sin( \theta ) \\
&= j' \sin( \theta ).
\end{aligned}
\end{equation}
This is exactly the result that we had when we expanded \( e^{j\theta} \) in it's cis form, and then took derivatives, so we have now reconciled the two different approaches.
%Observe that, as a side effect of this exploration, we know also know how to compute the derivative of \( e^{j\theta} \) for the special case where \( j j' = -j' j \).
} % answer
%}

         \input{sphericaldot.tex}
         %
% Copyright � CCYY Peeter Joot.  All Rights Reserved.
% Licenced as described in the file LICENSE under the root directory of this GIT repository.
%
\makeproblem{Spherical volume element pseudoscalar.}{problem:volumeselection:1}{
Using geometric algebra, perform the reduction of the grade three selection made in the final step of \cref{eqn:curvilinearspherical:300}.
} % problem
\makeanswer{problem:volumeselection:1}{
\begin{equation}\label{eqn:volumeselection:17}
\begin{aligned}
\Be_{31} e^{i\phi} \Be_2 e{i\phi}
&=
\Be_{31} e^{i\phi} e^{-i\phi} \Be_2 \\
&=
\Be_{312} \\
&= I.
\end{aligned}
\end{equation}
} % answer

         %
% Copyright © 2018 Peeter Joot.  All Rights Reserved.
% Licenced as described in the file LICENSE under the root directory of this GIT repository.
%
%{
\makeproblem{Spherical volume Jacobian.}{problem:curvilinearspherical:1}{
Without software, expand and simplify the determinant of \cref{eqn:curvilinearspherical:320}.
} % problem
\makeanswer{problem:curvilinearspherical:1}{
A bit of shorthand is useful.  We can write our Jacobian as
\begin{equation}\label{eqn:curvilinearsphericalProblem:20}
J
=
\begin{vmatrix}
S_\theta C_\phi & S_\theta S_\phi & C_\theta \\
r C_\theta C_\phi & r C_\theta S_\phi & - r S_\theta \\
- r S_\theta S_\phi & r S_\theta C_\phi & 0
\end{vmatrix}
=
r^2 \begin{vmatrix}
S_\theta C_\phi & S_\theta S_\phi & C_\theta \\
 C_\theta C_\phi &  C_\theta S_\phi & -  S_\theta \\
-  S_\theta S_\phi &  S_\theta C_\phi & 0
\end{vmatrix},
\end{equation}
where the common factor of the two last rows has been factored out.  Expanding the cofactors along the bottom row we have
\begin{equation}\label{eqn:curvilinearsphericalProblem:40}
\begin{aligned}
J
&=
-  r^2 S_\theta S_\phi
\begin{vmatrix}
S_\theta S_\phi & C_\theta \\
C_\theta S_\phi & -  S_\theta \\
\end{vmatrix}
-  r^2 S_\theta C_\phi
\begin{vmatrix}
S_\theta C_\phi & C_\theta \\
 C_\theta C_\phi &  -  S_\theta \\
\end{vmatrix} \\
&=
- r^2 S_\theta S_\phi^2
\begin{vmatrix}
S_\theta & C_\theta \\
C_\theta & -  S_\theta \\
\end{vmatrix}
-  r^2 S_\theta C_\phi^2
\begin{vmatrix}
S_\theta C_\theta \\
 C_\theta &  -  S_\theta \\
\end{vmatrix} \\
&=
- r^2 S_\theta S_\phi^2 \lr{ - S_\theta^2 - C_\theta^2 }
-  r^2 S_\theta C_\phi^2 \lr{ -S_\theta^2 - C_\theta^2 } \\
&=
r^2 S_\theta \lr{ S_\phi^2 + C_\phi^2 } \\
&=
r^2 S_\theta.
\end{aligned}
\end{equation}
Had we done a traditional first row or column determinant expansion, things would have been considerably messier.
} % problem
\makeproblem{Spherical polar coordinates.}{problem:curvilinearspherical:2}{
Starting with
\begin{equation*}
\Bx = r \lr{ \Be_1 \sin\theta \cos\phi + \Be_2 \sin\theta \sin\phi + \Be_3 \cos\theta },
\end{equation*}
first factor out \( \Be_1 \) from the \( \sin\theta \) terms, and group the remaining factors into complex exponential form.  Then factor our \( \Be_3 \) from both remaining terms to factor out a \( \phi \) dependent unit bivector, and put the entire expression into complex exponential form.
} % problem
\makeanswer{problem:curvilinearspherical:2}{
\begin{equation}\label{eqn:curvilinearspherical:20}
\begin{aligned}
\Bx
&= r \lr{ \Be_1 \sin\theta \cos\phi + \Be_2 \sin\theta \sin\phi + \Be_3 \cos\theta } \\
&= r \lr{ \sin\theta \Be_1 (\cos\phi + \Be_{12} \sin\phi ) + \Be_3 \cos\theta } \\
&= r \lr{ \sin\theta \Be_1 e^{\Be_{12} \phi } + \Be_3 \cos\theta } \\
&= r \Be_3 \lr{ \cos\theta + \sin\theta \Be_3 \Be_1 e^{\Be_{12} \phi } }.
\end{aligned}
\end{equation}
Writing \( j = \Be_3 \Be_1 e^{\Be_{12} \phi} \), this is
\begin{equation}\label{eqn:curvilinearspherical:21}
\Bx = r \Be_3 e^{j\theta}.
\end{equation}
} % answer
%}

         %
% Copyright © 2023 Peeter Joot.  All Rights Reserved.
% Licenced as described in the file LICENSE under the root directory of this GIT repository.
%
%{
\makeproblem{Curvilinear basis for toroidal parameterization.}{problem:toriodalProblem:1}{
Prove \cref{eqn:torusCenterOfMassParameterization:3}.
} % problem

\makeanswer{problem:toriodalProblem:1}{
We'll only compute \( \Bx_\theta \) here explicitly, as the other two vectors can be computed by inspection.

We start with a plain old chain rule expansion, with the cavaet that we must b e careful not to commute \( j \) with anything but the \( e^{\pm j \theta/2} \) terms.
\begin{equation}\label{eqn:toriodal:21}
\begin{aligned}
\Bx_\theta &= \PD{\theta}{\Bx} \\
&=
-\frac{j}{2}
e^{-j\theta/2} \left( \rho \Be_1 e^{ i \phi } + R \Be_3 \right) e^{j \theta/2}
+
e^{-j\theta/2} \left( \rho \Be_1 e^{ i \phi } + R \Be_3 \right) e^{j \theta/2}
\frac{j}{2}
\end{aligned}
\end{equation}
Note that the bivector \( j \) commutes with \( \Be_1 \), and then proceed to compute the \( \rho \) dependent part of \( \Bx_\theta \)
\begin{equation}\label{eqn:toriodal:41}
\begin{aligned}
& \frac{\rho}{2} e^{-j \theta/2} \Be_1 \lr{ -j e^{i \phi}  + e^{i \phi} j } e^{ j\theta/2 } \\
&=\frac{\rho}{2} e^{-j \theta/2} \Be_1 \lr{ - \Be_{32} \lr{ \cos\phi + \Be_{13} \sin\phi} + \lr{ \cos\phi + \Be_{13} \sin\phi} \Be_{32} } e^{ j\theta/2 } \\
&=\frac{\rho}{2} e^{-j \theta/2} \Be_1 \lr{ - \Be_{3213}  \sin\phi + \Be_{1332} \sin\phi } e^{ j\theta/2 } \\
&=\frac{\rho}{2} e^{-j \theta/2} \Be_1 \lr{ - \Be_{21}  \sin\phi + \Be_{12} \sin\phi } e^{ j\theta/2 } \\
&=      \rho e^{-j \theta/2} \Be_{112} \sin\phi e^{ j\theta/2 } \\
&=      \rho e^{-j \theta/2} \Be_{2} \sin\phi e^{ j\theta/2 }.
%&=      \rho \sin\phi \Be_2 e^{j \theta/2} e^{ j\theta/2 } \\
%&=      \rho \sin\phi \Be_2 e^{j \theta}.
\end{aligned}
\end{equation}
Similarly, the \( R \) dependent contribution is
\begin{equation}\label{eqn:toriodal:61}
\begin{aligned}
& \frac{R}{2} e^{-j \theta/2} \lr{ -j \Be_3  + \Be_3 j } e^{ j\theta/2 } \\
& \frac{R}{2} e^{-j \theta/2} \lr{ -\Be_{323}  + \Be_{332} } e^{ j\theta/2 } \\
& \frac{R}{2} e^{-j \theta/2} \lr{ \Be_{2}  + \Be_{2} } e^{ j\theta/2 } \\
&       R     e^{-j \theta/2} \Be_{2}  e^{ j\theta/2 }.
\end{aligned}
\end{equation}
Putting the pieces together, we have
\begin{equation}\label{eqn:toriodal:81}
\Bx_\theta = e^{-j \theta/2} \lr{ R + \rho \sin\phi } \Be_{2}  e^{ j\theta/2 },
\end{equation}
as claimed.
} % answer
%}

   \section{Integration theory.}
      \subsection{Line integral.}
         \input{lineintegraldef.tex}
      \subsection{Surface integral.}
         \input{surfaceintegraldef.tex}
      \subsection{Volume integral.}
         \input{volumeintegraldef.tex}
      \subsection{Bidirectional derivative operators.}
         \input{fundamentalTheoremOfCalculus.tex}
      \subsection{Fundamental theorem.}
         \input{fundamentalTheorem.tex}
      \subsection{Stokes' theorem.}
         \input{stokesTheoremTheStatement.tex}
      \subsection{Fundamental theorem for Line integral.}
         \input{lineintegral.tex}
      \subsection{Fundamental theorem for Surface integral.}
         \input{surfaceintegral.tex}
      \subsection{Fundamental theorem for Volume integral.}
         \input{volumeintegral.tex}
   \section{Multivector Fourier transform and phasors.}
      \input{fourier.tex}
   \section{Green's functions.}
      \input{greensFunctionOverview.tex}
      \subsection{Green's function solutions.}
         \input{greensFunctionSolutions.tex}
      \subsection{Helmholtz equation.}
         \input{greensFunctionHelmholtz.tex}
      \subsection{First order Helmholtz equation.}
         \input{greensFunctionFirstOrderHelmholtz.tex}
      \subsection{Spacetime gradient.}
         \input{greensFunctionSpacetimeGradient.tex}
   \section{Helmholtz theorem.}
      \input{helmholtzDerviationMultivectorStatement.tex}
      \input{helmholtzDerviationMultivectorSolution.tex}
   \section{Problem solutions.}
      \shipoutAnswer
%}
