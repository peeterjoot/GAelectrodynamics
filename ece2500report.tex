%
% Copyright � 2018 Peeter Joot.  All Rights Reserved.
% Licenced as described in the file LICENSE under the root directory of this GIT repository.
%
%{
\input{../latex/blogpost.tex}
\renewcommand{\basename}{ece2500report}
%\renewcommand{\dirname}{notes/phy1520/}
\renewcommand{\dirname}{notes/ece1228-electromagnetic-theory/}
%\newcommand{\dateintitle}{}
%\newcommand{\keywords}{}

\input{../latex/peeter_prologue_print2.tex}

\usepackage{peeters_layout_exercise}
\usepackage{peeters_braket}
\usepackage{peeters_figures}
%\usepackage{siunitx}
%\usepackage{mhchem} % \ce{}
%\usepackage{macros_bm} % \bcM
%\usepackage{macros_qed} % \qedmarker
\usepackage{txfonts} % \ointclockwise

\beginArtNoToc

\generatetitle{Project report ECE2500.  Geometric Algebra for Electrical Engineers}
\section{Motivation.}
This is the report for an ECE2500 M.Eng project course.

\subsubsection{Goals.}
This project had a few goals
\begin{enumerate}
\item Perform a literature review of applications of geometric algebra\footnote{To be defined.} to the study of electromagnetism.
\item Identify the subset of the literature that had direct relevance to electrical engineering.
\item Create a complete, and as compact as possible, introduction of the prerequisites required
for a graduate or advanced undergraduate electrical engineering student to be able to apply
geometric algebra to problems in electromagnetism.
\end{enumerate}

\subsubsection{Why geometric algebra?}

\paragraph{Vector formalism.}
Euclidean vector algebra, in particular the vector algebra and calculus of \R{3},
is the de-facto language of electrical engineering for electromagnetism.
Maxwell's equations in the Heaviside-Gibbs vector formalism are
\begin{dmath}\label{eqn:ece2500report:20}
\begin{aligned}
%\spacegrad \cross \BE &= - \BM - \PD{t}{\BB} \\
\spacegrad \cross \BE &= - \PD{t}{\BB} \\
\spacegrad \cross \BH &= \BJ + \PD{t}{\BD} \\
\spacegrad \cdot \BD &= \rho \\
%\spacegrad \cdot \BB &= \rho_\txtm.
\spacegrad \cdot \BB &= 0.
\end{aligned}
\end{dmath}
We are all intimately familiar with these equations, with the dot and the cross products, and with gradient, divergence and curl operations that are used to express them.
Given how comfortable we are with this mathematical formalism, there has to be a really good reason to switch to something else.

\paragraph{Tensor formalism.}
For high energy physics applications the use of relativistic tensor formalism can be more natural.
In the tensor formalism, Maxwell's equations are reduced to a set of two tensor relationships \citep{griffiths1999introduction}, \citep{jackson1975cew}
\begin{dmath}\label{eqn:ece2500report:40}
\begin{aligned}
\partial_\mu F^{\mu \nu} &= \mu_0 J^\nu \\
\epsilon^{\alpha \beta \mu \nu} \partial_\beta F_{\mu \nu} &= 0,
\end{aligned}
\end{dmath}
where \( F^{\mu\nu} \) is a rank-2 antisymmetric tensor that contains all six electric and magnetic field components, and \( J^\nu \) is a four-vector current containing both charge density and current density components.
\Cref{eqn:ece2500report:40} provides a unified and simpler theoretical framework for electromagnetism.
One of the costs of this formalism is that we loose the clear separation of the electric and magnetic fields that we are so comfortable with.
Another cost is that we loose the
distinction between space and time, both concepts near and dear to any down to earth electrical engineer.

\paragraph{Differential forms.}
It has been argued that a differential forms treatment of
electromagnetism provides some of the same theoretical advantages as the tensor formalism, without the disadvantages
of introducing a hellish mess of index manipulation into the mix.
With differential forms it is also possible to express Maxwell's equations as two equations.
The free space differentials forms equivalent to \cref{eqn:ece2500report:40} is
\begin{dmath}\label{eqn:ece2500report:60}
\begin{aligned}
d \alpha &= 0 \\
d *\alpha &= 0,
\end{aligned}
\end{dmath}
where \( \alpha = \lr{ E_1 dx^1 + E_2 dx^2 + E_3 dx^3 }(c dt) + \lr{ H_1 dx^2 dx^3 + H_2 dx^3 dx^1 + H_3 dx^1 dx^2 } \) \citep{flanders1989dfa}.
One of the advantages of this representation is that it is valid even for curvilinear coordinate representations, which are handled naturally in differential forms.
However, this formalism also comes with a number of costs.
One cost (or benefit), like that of the tensor formalism, is that this is implicitly a relativistic
approach subject to non-Euclidean orthonormality conditions \( (dx^i, dx^j) = \delta^{ij}, (dx^i, c dt) = 0, (c dt, c dt) = -1 \).
Most grievous of the costs is the requirement to use differentials \( dx^1, dx^2, dx^3, c dt \), instead of coordinates, even when using a non-curvilinear basis, which is easily viewed as unnatural.

\paragraph{Space time algebra.}
An elegant and powerful alternative to electrodynamics using tensor and differential forms, is STA, the \textit{Space Time Algebra}.
This is a relativistic geometric algebra that allows Maxwell's equations \cref{eqn:ece2500report:40} to be combined into one equation \citep{doran2003gap}
\begin{dmath}\label{eqn:ece2500report:80}
\grad F = J,
\end{dmath}
where \( F = \BE + I c \BB \) is a bivector field containing both the electric and magnetic field ``vectors'', \( \grad = \gamma^\mu \partial_\mu \) is the spacetime derivative, \( J \) is a four vector containing electric charge and current components, and \( I = \gamma_0 \gamma_1 \gamma_2 \gamma_3 \) is the spacetime pseudoscalar, the ordered product of the basis vectors \( \setlr{ \gamma_\mu } \).
In this formalism ``spatial'' vectors \( \Bx = \sum_{k>0} \gamma_k \gamma_0 x^k \) are represented as spacetime bivectors, requiring a small slight of hand when switching between STA notation and conventional vector representation.
The STA representation is explicitly relativistic with a non-Euclidean relationships between the basis vectors \( \gamma_0 \cdot \gamma_0 = 1 = -\gamma_k \cdot \gamma_k, \forall k > 0 \).
Having a single PDE for all of Maxwell's equations allows for direct Green's function solution of the field, and has a number of other advantages.
There is extensive literature exploring applications of the STA formalism to electrodynamics.
Many powerful and elegant theoretical results have been derived using this formalism that require significantly more complex approaches using conventional vector or tensor analysis.
Unfortunately, much of the STA literature is inaccessible to the engineering student, practising engineers, or engineering instructors.
To even start reading the literature, one must learn geometric algebra, aspects of special relativity and non-Euclidean geometry, generalized integration theory, and even some tensor analysis.

In the geometric algebra literature, there are a few authors who have endorsed the use of Euclidean geometric algebras for relativistic applications \citep{baylis2004electrodynamics}, \citep{chappell2014geometric}.
These authors use an Euclidean basis ``vector'' \( \Be_0 = 1 \) for the timelike direction, and use a
hybrid scalar plus vector representation of four vectors (called paravectors).
Lorentz transformation and manipulation of paravectors requires a
variety of conjugation, real and imaginary operators.
They present arguments that it provides an effective pedagodical bridge from Euclidean geometry to the Minkowski geometry of special relativity.
It is the opinion of this author that for relativisitic operations, STA is a much more natural and less confusing choice.

\paragraph{Euclidean geometric algebra.}
This project explored non-relativistic applications of geometric algebra to electromagnetism.
The aim was to cut a few of the mathematical prerequisites our of the picture and attempt to construct a geometric algebra treatment that includes many of the results from STA without imposing all of the costs required to learn that formalism.
A self contained introduction to geometric algebra and generalized integration theory was written, with a focus on geometric algebras constructed from \R{2} and \R{3} spaces.
Following all the prerequisite material is an exploration of geometric algebra applications to electromagnetism, covering many of the fundamental results that can be derived from Maxwell's equations in a more streamlined and compact fashion.

Some researchers may find it distasteful that STA and relativity have been avoided completely.
Maxwell's equations are inherently relativistic, and
STA expresses the relativistic aspects of electromagnetism in an exceptional and beautiful fashion.
It is the opinion of the author that an effective pedagogical path has been constructed in this work, one that bridges the gap between the conventional vector formalism to one using geometric algebra.
Most of the results in this work can be generalized or translated to an STA approach with ease, which makes the STA literature much more accessible.

\subsubsection{Results.}

The end product of this project was a small book covering geometric algebra, geometric calculus, with applications to electromagnetism.
This report summarizes these results, omitting most derivations, and attempts to provide an overview of the work that can be used as a road map to the book for further exploration.

\subsubsection{Road map.}
\section{Geometric Algebra.}
\section{Geometric Calculus.}
\section{Electromagnetism.}

%}
\EndArticle
