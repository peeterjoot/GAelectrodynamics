%
% Copyright � 2018 Peeter Joot.  All Rights Reserved.
% Licenced as described in the file LICENSE under the root directory of this GIT repository.
%
%{
\input{../latex/blogpost.tex}
\renewcommand{\basename}{ece2500report}
%\renewcommand{\dirname}{notes/phy1520/}
\renewcommand{\dirname}{notes/ece1228-electromagnetic-theory/}
%\newcommand{\dateintitle}{}
%\newcommand{\keywords}{}

\input{../latex/peeter_prologue_print2.tex}

\usepackage{peeters_layout_exercise}
\usepackage{peeters_braket}
\usepackage{peeters_figures}
%\usepackage{siunitx}
%\usepackage{mhchem} % \ce{}
\usepackage{macros_bm} % \bcM
%\usepackage{macros_qed} % \qedmarker
\usepackage{txfonts} % \ointclockwise

\beginArtNoToc

\generatetitle{Project report ECE2500.  Geometric Algebra for Electrical Engineers}
\section{Motivation.}
This is the report for an ECE2500 M.Eng project course.

\subsubsection{Goals.}
This project had a few goals
\begin{enumerate}
\item Perform a literature review of applications of geometric algebra to the study of electromagnetism.
\item Identify the subset of the literature that had direct relevance to electrical engineering.
\item Create a complete, and as compact as possible, introduction of the prerequisites required
for a graduate or advanced undergraduate electrical engineering student to be able to apply
geometric algebra to problems in electromagnetism.
\end{enumerate}

\subsubsection{Why geometric algebra?}

Euclidean vector algebra, in particular the vector algebra and calculus of \R{3}, 
is the defacto language of electromagnetism within electrical engineering.
We are all intimately familiar with dot and cross products, and with gradient, divergence and curl operations.
There has to be a really good reason to introduce an alternate mathematical formalism.

For high energy physics applications the use of relativistic tensor formalism can be more natural.  
In the tensor formalism, Maxwell's equations are reduced to a set of two tensor relationships
\begin{dmath}\label{eqn:ece2500report:n}
\begin{aligned}
\partial_\mu F^{\mu \nu} &= J^\nu \\
\epsilon_{\alpha \beta \gamma \kappa} F^{\alpha \beta} &= 0.
\end{aligned}
\end{dmath}
This provides a more unified and simpler theoretical framework for electromagnetism.  
The tensor formalism is built around the concept of the four-vector, and comes with a few costs.  
One cost is that we loose the vector electric and magnetic fields we are so comfortable with, and have to work with rank-2 antisymmetric tensors \( F^{\mu\nu} \) that contain all six electric and magnetic field components.  There is also 
the cost of loosing the distinction between space and time, both concepts near and dear to any down to earth electrical engineer.

It has been argued that a differential forms treatment of 
electromagnetism provides some of the same theoretical advantages as the tensor formalism, without the disadvanges 
of introducing a hellish mess of index manipulation into the mix.
With differential forms it is also possible to express Maxwell's equations as two equations
FIXME: state.
A number of calculations are more natural in this context.  
However, this formalism also comes with a number of costs.  
Most grievious of the costs is the requirement to use differentials based on the one-forms \( dx, dy, dz, dt \) as the basis for space and time.  
There are not likely many engineers that would find the use of differentials for electric and magnetic fields and sources natural.

\subsubsection{Results.}

\section{Geometric Algebra.}
\section{Geometric Calculus.}
\section{Electromagnetism.}

%}
%\EndArticle
\EndNoBibArticle
