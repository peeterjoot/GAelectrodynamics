%
% Copyright � 2018 Peeter Joot.  All Rights Reserved.
% Licenced as described in the file LICENSE under the root directory of this GIT repository.
%
%{
\input{../latex/blogpost.tex}
\renewcommand{\basename}{ece2500report}
%\renewcommand{\dirname}{notes/phy1520/}
\renewcommand{\dirname}{notes/ece1228-electromagnetic-theory/}
%\newcommand{\dateintitle}{}
%\newcommand{\keywords}{}

% too many alphabets fix:
% https://tex.stackexchange.com/a/243541/15
\newcommand\hmmax{0}
\newcommand\bmmax{0}

\input{../latex/peeter_prologue_print2.tex}

\usepackage{peeters_layout_exercise}
\usepackage{peeters_braket}
\usepackage{peeters_figures}

%\newcommand{\dLambertian}[0]{\square}
\newcommand{\dLambertian}[0]{\Box}

\newcommand{\dispdot}[2][.2ex]{\dot{\raisebox{0pt}[\dimexpr\height+#1][\depth]{$#2$}}}% \dispdot[<displace>]{<stuff>}
\newcommand{\dotBJ}[0]{\dispdot{\mathbf{J}}}

\newcommand{\stgrad}[0]{\lr{ \spacegrad + \inv{c} \PD{t}{}}}
\newcommand{\conjstgrad}[0]{\lr{ \spacegrad - \inv{c} \PD{t}{}}}

\usepackage{siunitx}
%\usepackage{mhchem} % \ce{}
\usepackage{macros_bm} % \bcM
\usepackage{macros_cal}
%\usepackage{macros_qed} % \qedmarker
\usepackage{txfonts} % \ointclockwise
\usepackage{enumerate}

\beginArtNoToc

\generatetitle{Project report ECE2500.  Geometric Algebra for Electrical Engineers}

\section{Motivation.}
This is the report for an ECE2500 M.Eng project course.

\subsubsection{Goals.}
This project had a few goals
\begin{enumerate}
\item Perform a literature review of applications of geometric algebra
%\footnote{Geometric algebra will be defined precisely later, along with bivector, trivector, multivector and other geometric algebra generalizations of the vector.}
to the study of electromagnetism.
\item Identify the subset of the literature that had direct relevance to electrical engineering.
\item Create a complete, and as compact as possible, introduction of the prerequisites required
for a graduate or advanced undergraduate electrical engineering student to be able to apply
geometric algebra to problems in electromagnetism.
\end{enumerate}

Geometric algebra
generalizes vectors, providing algebraic representations of not just directed line segments, but also points, plane segments, volumes, and higher degree geometric objects (hypervolumes.)
The geometric algebra representation of planes, volumes and hypervolumes requires a vector dot product, a vector multiplication operation, and a generalized addition operation.
The dot product
provides the length of a vector and a test for whether or not any two vectors are perpendicular.
The vector multiplication operation is used to construct
directed plane segments (bivectors),
and directed volumes (trivectors), which are built from the respective products of two or three mutually perpendicular vectors.
The addition operation allows for sums of scalars, vectors, or any products of vectors.  Such a sum is called a multivector.

The power to add scalars, vectors, and products of vectors can be exploited to simplify much of electromagnetism.
In particular, Maxwell's equations for isotropic media can be merged into a single multivector equation
\begin{dmath}\label{eqn:quaternion2maxwellWithGA:20}
\stgrad \lr{ \BE + I c \BB } = \eta\lr{ c \rho - \BJ },
\end{dmath}
where \( \spacegrad \) is the gradient, \( I = \Be_1 \Be_2 \Be_3 \) is the ordered product of the three \R{3} basis vectors, \( c = 1/\sqrt{\mu\epsilon}\) is the group velocity of the medium, \( \eta = \sqrt{\mu/\epsilon} \), \( \BE, \BB \) are the electric and magnetic fields, and
\( \rho \) and \( \BJ \) are the charge and current densities.
We will write this as
\begin{dmath}\label{eqn:ece2500report:40}
\stgrad F = J,
\end{dmath}
where \( F = \BE + I c \BB \) is the combined (multivector) electromagnetic field, and \( J = \eta\lr{ c \rho - \BJ } \) is the multivector current.
As a single PDE, the complete Green's function toolbox may be thrown at
\cref{eqn:ece2500report:40}, inverting Maxwell's equation for the electromagnetic field, given any charge and current density distribution
\begin{dmath}\label{eqn:ece2500report:60}
F(\Bx, t)
= \int dt' dV' G(\Bx, \Bx' ; t, t') J(\Bx', t').
\end{dmath}
Green's functions may also be applied to static and frequency domain field configurations.
Solving for, or working with, the combined field \( F \) shows the
hidden structure behind a number of seemingly
disparate ideas in electromagnetism.
This project explored a number of ideas along these lines.
For example, a
Green's function solutions for the static field configurations simultaneously yields Coulomb's and the Biot-Savart law.
Plane, circular and elliptical waves may be expressed compactly in a multivector form, naturally expressing the mutual perpendicularity of the electric field, magnetic field and the propagation directions, as well
as the relationships between the electric and magnetic field components.
The field energy density and Poynting vectors have a simple multivector form expressed in terms of \( F \) alone.
Calculations of radiation pressure can be performed using only the normal component of what is known as the energy momentum tensor in the conventional representation, which has a particularly compact multivector
representation.

Much of the geometric algebra literature for electrodynamics is presented with a relativistic bias, or assumes high levels of mathematical or physics sophistication.
The aim of this work was an attempt to make the study of electromagnetism using geometric algebra more accessible, especially to an electrical engineering audience.
In particular, this project explored non-relativistic applications of geometric algebra to electromagnetism.
The end product of this project was a fairly small self contained book, titled ``Geometric Algebra for Electrical Engineers''.
This book includes an introduction to Euclidean geometric algebra focused on \R{2} and \R{3} (64 pages), an introduction to geometric calculus and multivector Green's functions (64 pages), applications to electromagnetism (82 pages), and some appendices.
This report summarizes some results from this book, omitting most derivations, and attempts to provide an overview that may be used as a road map for the book for further exploration.
Many of the fundamental results of electromagnetism are derived directly from
\cref{eqn:ece2500report:40}, the multivector Maxwell's equation, in a streamlined and compact fashion.
This includes some new results, and many of the existing non-relativistic results from the geometric algebra literature.
As a
conceptual bridge, the book includes many examples of how to extract
familiar conventional results from simpler multivector representations.
Also included are some sample calculations exploiting unique capabilities that geometric algebra provides.  In particular, vectors in a plane may be manipulated much like complex numbers, which has a number of advantages over working with coordinates explicitly.

It is my belief that systematically working through all of the introductory and advanced topics in electromagnetism using geometric algebra
would provide significant insight, as well as a new set of powerful tools and procedures of practical value to the electrical engineer or physics practitioner.
The book produced in this project provides the prerequisite material for such exploration, and some first steps along the path of such an expedition.

\section{Geometric Algebra.}
\section{Geometric Calculus.}
\section{Electromagnetism.}
\subsection{Conventional Maxwell's equations.}
In the book it is presumed that the reader is familiar with Maxwell's equations, and that no attempt to motivate them is required.  We start with Maxwell's equations with the antenna theory extensions (fictious magnetic sources)
\begin{subequations}
\label{eqn:freespace:3399}
\begin{dmath}\label{eqn:freespace:3100}
%\spacegrad \cross \BE = - \PD{t}{\BB}
\spacegrad \cross \BE = - \BM - \PD{t}{\BB}
\end{dmath}
\begin{dmath}\label{eqn:freespace:3120}
%\spacegrad \cross \BB = \mu_0 \lr{ \BJ + \epsilon_0 \PD{t}{\BE} }
\spacegrad \cross \BH = \BJ + \PD{t}{\BD}
\end{dmath}
\begin{dmath}\label{eqn:freespace:3140}
%\spacegrad \cdot \BE = \frac{\rho}{\epsilon_0}
\spacegrad \cdot \BD = \rho
\end{dmath}
\begin{dmath}\label{eqn:freespace:3160}
%\spacegrad \cdot \BB = 0.
\spacegrad \cdot \BB = \rho_\txtm.
\end{dmath}
\end{subequations}
with fields
\begin{itemize}
	\item \( \BE(\Bx, t) \) : Electric field intensity [\si{V/m}] (Volts/meter)
	\item \( \BH(\Bx, t) \) : Magnetic field intensity [\si{A/m}] (Amperes/meter),
	\item \( \BD(\Bx, t) \) : Electric flux density (or displacement vector) [\si{C/m}] (Coulombs/meter)
	\item \( \BB(\Bx, t) \) : Magnetic flux density [\si{W/m^2}] (Webers/square meter),
\end{itemize}
charge densities
\begin{itemize}
	\item \( \rho(\Bx, t) \) : Electric charge density [\si{C/m^3}] (Coulombs/cubic meter)
   \item \( \rho_\txtm(\Bx, t) \) : Magnetic charge density [\si{W/m^3}] (Webers/cubic meter),
\end{itemize}
and current densities
\begin{itemize}
	\item \( \BJ(\Bx, t) \) : Electric current density [\si{A/m^2}] (Amperes/square meter),
   \item \( \BM(\Bx, t) \) : Magnetic current density [\si{V/m^2}] (Volts/square meter).
\end{itemize}
Much of the book presumes isotropic constitutive relationships between the electric and magnetic fields
\begin{subequations}
\label{eqn:freespace:300}
\begin{dmath}\label{eqn:freespace:320}
\BB = \mu \BH
\end{dmath}
\begin{dmath}\label{eqn:freespace:340}
\BD = \epsilon \BE,
\end{dmath}
\end{subequations}
where \( \epsilon = \epsilon_r \epsilon_0 \) is the permittivity of the medium [\si{F/m}] (Farads/meter), and \( \mu = \mu_r \mu_0 \) is the permeability of the medium [\si{H/m}] (Henries/meter).
\subsection{Maxwell's equation.}
For isotropic media and constitutive relationships \cref{eqn:freespace:300} we may define
a multivector field that includes both electric and magnetic components
\makedefinition{Electromagnetic field strength.}{dfn:isotropicMaxwells:640}{
The \textit{electromagnetic field strength} ([\si{V/m}] (Volts/meter)) is defined as
\begin{equation*}
F = \BE + I \eta \BH \quad(= \BE + I c \BB),
\end{equation*}
where
\begin{itemize}
\item \( \eta = \sqrt{\mu/\epsilon} \) (\( [\Omega] \) Ohms), is the impedance of the media.
\item \( c = 1/\sqrt{\epsilon\mu} \) ([\si{m/s}] meters/second), is the group velocity of a wave in the media.  When \( \epsilon = \epsilon_0, \mu = \mu_0 \), \( c \) is the speed of light.
\end{itemize}
\( F \) is called the \textit{F}araday by some authors.
} % definition

The factors of \( \eta \) (or \( c \)) that multiply the magnetic fields are for dimensional consistency, since \( [\sqrt{\epsilon} \BE] = [\sqrt{\mu} \BH] = [\BB/\sqrt{\mu}]\).
The justification for imposing a dual (or complex) structure on the electromagnetic field strength can be found in the historical development of
Maxwell's equations, but we will also see such a structure arise naturally in short order.

No information is lost by imposing the complex structure of
\cref{dfn:isotropicMaxwells:640}, since we can always obtain the
electric field vector \( \BE \) and the magnetic field bivector \( I \BH \) by grade selection
from the electromagnetic field strength when desired
\begin{dmath}\label{eqn:isotropicMaxwells:620}
\begin{aligned}
\BE &= \gpgradeone{ F } \\
I \BH &= \inv{\eta} \gpgradetwo{ F }.
\end{aligned}
\end{dmath}

We will also
define a multivector current containing all charge densites and current densities
\makedefinition{Multivector current.}{dfn:isotropicMaxwells:660}{
The \textit{current} ([\si{A/m^2}] (Amperes/square meter)) is defined as
\begin{equation*}
J = \eta \lr{ c \rho - \BJ } + I \lr{ c \rho_\txtm - \BM }.
\end{equation*}
} % definition
When the fictitious magnetic source terms \((\rho_\txtm, \BM)\) are included, the current has one grade for each possible source (scalar, vector, bivector, trivector).  With only conventional electric sources, the current is still a multivector, but contains only scalar and vector grades.

Given the multivector field and current, it is now possible to state Maxwell's equation (singular) in its geometric algebra form
\maketheorem{Maxwell's equation.}{dfn:isotropicMaxwells:680}{
Maxwell's equation is a multivector equation relating the change in the electromagnetic field strength to charge and current densities and is written as
\begin{equation*}
\stgrad F = J.
\end{equation*}
} % theorem
Maxwell's equation in this form will be the starting place for all the subsequent analysis in this book.
As mentioned in \cref{chap:GreensFunctions}, the operator \( \spacegrad + (1/c) \partial_t \) will be called the \textit{spacetime gradient}\footnote{This form of spacetime gradient is given a special symbol by a number of authors, but there is no general agreement on what to use.
Instead of entering the fight, it will be written it out in full in this book.}.

See the book for a proof of \cref{dfn:isotropicMaxwells:680}.  The workhorse of the proof is the identity \( \spacegrad \Bb = \spacegrad \cdot \Bb + I \spacegrad \cross \Bb \) which allows Maxwell's equations into two gradient equations, one for each of \( \spacegrad \BE \), and \( \spacegrad \BH \), which can then be further grouped to complete the proof.
There is a lot of information packed into this single equation.  Expanded in full it is
\begin{dmath}\label{eqn:isotropicMaxwells:580}
\stgrad \lr{ \BE + I \eta \BH } = \eta\lr{ c \rho - \BJ } + I \lr{ c \rho_\txtm - \BM }.
\end{dmath}

\subsection{Wave equation and continuity.}
Some would argue that the conventional form \cref{eqn:freespace:3100} of Maxwell's equations have built in redundancy since continuity equations on the charge and current densities couple some of these equations.
We will take an opposing view, and show that such continuity equations are neccessary consequences of Maxwell's equation in its wave equation form, and derive those conditions.
This amounts to a statement that the multivector current \( J \) is not completely unconstrained.

\maketheorem{Electromagnetic wave equation and continuity conditions.}{thm:continuity:600}{
The electromagnetic field is a solution to the non-homogeneous wave equation
\begin{equation*}
%\lr{ \spacegrad^2 - \inv{c^2} \PDSq{t}{} }
\dLambertian
F =
\conjstgrad J.
\end{equation*}
In source free conditions, this reduces to a homogeneous wave equation, with group velocity \( c \), the speed of the wave in the media.
When expanded explicitly in terms of electric and magnetic fields, and charge and current densities, this single equation resolves to a
non-homogeneous wave equation for each of the electric and magnetic fields
\begin{equation*}
\begin{aligned}
%\lr{ \spacegrad^2 - \inv{c^2} \PDSq{t}{} }
\dLambertian
\BE
%&= \gpgrade{\conjstgrad J}{1}
&= \inv{\epsilon} \spacegrad \rho + \mu \PD{t}{\BJ} + \spacegrad \cross \BM \\
%\lr{ \spacegrad^2 - \inv{c^2} \PDSq{t}{} }
\dLambertian
\BH
%&= \inv{I \eta} \gpgrade{\conjstgrad J}{2}
&= \inv{\mu} \spacegrad \rho_\txtm + \epsilon \PD{t}{\BM} - \spacegrad \cross \BJ,
\end{aligned}
\end{equation*}
as well as a pair of continuity equations coupling the respective charge and current densities
\begin{equation*}
\begin{aligned}
\spacegrad \cdot \BJ + \PD{t}{\rho} &= 0 \\
\spacegrad \cdot \BM + \PD{t}{\rho_\txtm} &= 0.
\end{aligned}
\end{equation*}
} % theorem

The proof is in the book, but basically just requires operating on Maxwell's equation with \( \conjstgrad \), which yields two equations
equations, one for grades 1,2 and one for grades 0,3
\begin{dmath}\label{eqn:continuity:130}
\begin{aligned}
%\lr{ \spacegrad^2 - \inv{c^2} \PDSq{t}{} }
\dLambertian
F &= \gpgrade{ \conjstgrad J }{1,2} \\
                                           0 &= \gpgrade{ \conjstgrad J }{0,3}.
\end{aligned}
\end{dmath}
The grade 0,3 selection of \cref{eqn:continuity:130} provides the continuity equations.
\subsection{Plane waves.}
With all sources zero,
the free space Maxwell's equation as given by \cref{dfn:isotropicMaxwells:680} for the
electromagnetic field strength reduces to just
\begin{dmath}\label{eqn:planewavesMultivector:300}
\stgrad F(\Bx, t) = 0.
\end{dmath}

Utilizing a phasor representation of the form \cref{dfn:greensFunctionOverview:300},
we will define the
phasor representation of the field as
\makedefinition{Plane wave.}{dfn:planewavesMultivector:680}{
We represent the
electromagnetic field strength
plane wave solution of Maxwell's equation in phasor form as
\begin{equation*}
F(\Bx, t) = \Real \lr{ F(\Bk) e^{ j \omega t }  },
\end{equation*}
where the complex valued multivector \( F(\Bk) \) also has a presumed exponential dependence
\begin{equation*}
F(\Bk)
=
\tilde{F}
e^{ -j \Bk \cdot \Bx }.
\end{equation*}
} % definition

In the book, we show that solutions of the electromagnetic field wave equation have the form
\maketheorem{Plane wave solutions to Maxwell's equation.}{thm:planewavesMultivector:620}{
Single frequency plane wave solutions of Maxwell's equation have the form
\begin{equation*}
F(\Bx, t)
=
\Real \lr{
\lr{ 1 + \kcap }
\kcap \wedge \BE\,
e^{-j \Bk \cdot \Bx + j \omega t}
}
,
\end{equation*}
where \( \Norm{\Bk} = \omega/c \), \( \kcap = \Bk/\Norm{\Bk} \) is the unit vector pointing along the propagation direction, and \( \BE \) is any complex-valued vector variable.  When a \( \BE \cdot \Bk = 0 \) constraint is imposed on the vector variable \( \BE \), that variable can be interpreted as the electric field, and the solution reduces to
\begin{equation*}
F(\Bx, t)
=
\Real \lr{
\lr{ 1 + \kcap }
\BE\,
e^{-j \Bk \cdot \Bx + j \omega t}
}
,
\end{equation*}
showing that the field phasor \( F(\Bk) = \BE(\Bk) + I \eta \BH(\Bk) \) splits naturally into electric and magnetic components
\begin{equation*}
\begin{aligned}
\BE(\Bk) &= \BE e^{-j \Bk \cdot \Bx} \\
\eta \BH(\Bk) &= \kcap \cross \BE \, e^{-j \Bk \cdot \Bx},
\end{aligned}
\end{equation*}
where the directions \( \kcap, \BE, \BH \) form a right handed triple.
} % theorem

A full proof and discussion is in the book.  The key step is that after insertion of the presumed phasor relationship, we find that
\begin{dmath}\label{eqn:planewavesMultivector:60}
0
=
-j \lr{ \Bk - \frac{\omega}{c} } F(\Bk),
\end{dmath}
which can be satisfied by insisting that \( F \) has a \( \Bk + \omega/c \) factor and that \( \Norm{\Bk} = \omega/c \).  The observation that
\( \kcap, \BE, \BH \) form a right handed triple, is expressed geometrically by \( I = \kcap \Ecap \Hcap \), from which we can also find \( \kcap = \Ecap \cross \Hcap \).

\subsection{Statics solution.}
If we restrict attention to time invariant fields (\( \partial_t F = 0\)) and time invariant sources (\(\partial_t J = 0\)),
Maxwell's equation is reduced to an invertible first order gradient equation
\begin{dmath}\label{eqn:statics:20}
\spacegrad F(\Bx) = J(\Bx),
\end{dmath}

\maketheorem{Maxwell's statics solution.}{thm:statics:100}{
The solution to the Maxwell statics equation is given by
\begin{equation*}
F(\Bx)
= \inv{4\pi} \int_V dV' \frac{\gpgrade{(\Bx - \Bx') J(\Bx')}{1,2}}{\Norm{\Bx - \Bx'}^3} + F_0,
\end{equation*}
where \( F_0 \) is any function for which \( \spacegrad F_0 = 0 \).
The explicit expansion in electric and magnetic fields and charge and current densities is given by
\begin{equation*}
\begin{aligned}
\BE(\Bx)
&=
\inv{4\pi} \int_V dV' \inv{\Norm{\Bx - \Bx'}^3}
\lr{
{\color{DarkOliveGreen}
   \inv{\epsilon}(\Bx - \Bx') \rho(\Bx')
}
   +
   (\Bx - \Bx') \cross \BM(\Bx')
} \\
\BH(\Bx)
&=
\inv{4\pi} \int_V dV' \inv{\Norm{\Bx - \Bx'}^3}
\lr{
{\color{Maroon}
  \BJ(\Bx') \cross (\Bx - \Bx')
}
+ \inv{\mu} (\Bx - \Bx') \rho_m(\Bx')
}.
\end{aligned}
\end{equation*}
} % theorem

We see that the solution incorporates both a {\color{DarkOliveGreen}Coulomb's law} contribution and a {\color{Maroon}Biot-Savart law} contribution, as well as their magnetic source analogues if applicable.

The proof is essentially a convolution with the (vector valued) Green's function for the (first order) gradient \cref{eqn:greensFunctionFirstOrderHelmholtz:900}.
\subsection{Statics: Enclosed charge.}
In conventional electrostatics we obtain a relation between the normal electric field component and the enclosed charge by integrating the electric field divergence.
The geometric algebra generalization of this relates the product of the normal and the electromagnetic field strength related to the enclosed multivector current
\maketheorem{Enclosed multivector current.}{thm:enclosedCurrent:60}{
The total multivector current in the volume is related to the surface integral of \( \ncap F \) over the boundary of the volume by
\begin{equation*}
\int_{\partial V} dA \ncap F = \int_V dV J.
\end{equation*}
This is a multivector equation, carrying inforamation for each grade in the multivector current, and after explicit expansion is equivalent to
\begin{equation*}
\begin{aligned}
\int_{\partial V} dA\, \ncap \cdot \BE        &=  \inv{\epsilon} \int_V dV\, \rho \\
\int_{\partial V} dA\, \ncap \cross \BH       &=                 \int_V dV\, \BJ \\
\int_{\partial V} dA\, \ncap \cross \BE       &=               - \int_V dV\, \BM \\
\int_{\partial V} dA\, \ncap \cdot \BH        &=  \inv{\mu} \int_V dV\, \rho_\txtm.
\end{aligned}
\end{equation*}
} % theorem

To prove \cref{thm:enclosedCurrent:60} one evaluate the volume integral of the gradient of the field using \cref{thm:volumeintegral:100}.  The rest of the proof follows by grade selection.  For full details on both see the book.
\subsection{Statics: Enclosed current.}
Ampere's law may be generalized to line integrals of the total electromagnetic field strength.
\maketheorem{Line integral of the field.}{thm:amperes:280}{
The line integral of the electromagnetic field strength is
\begin{equation*}
\ointclockwise_{\partial A} d\Bx\, F
=
I \int_A dA \lr{ \ncap J - \PD{n}{F} },
\end{equation*}
where \( \PDi{n}{F} = \lr{ \ncap \cdot \spacegrad } F \).
Expressed in terms of the conventional consistent fields and sources, this multivector relationship expands to four equations, one for each grade
\begin{equation*}
\begin{aligned}
\ointclockwise_{\partial A} d\Bx \cdot \BE &=  \int_A dA\, \ncap \cdot \BM \\
\ointclockwise_{\partial A} d\Bx \cross \BH
&=
\int_A dA
\lr{
   - \ncap \cross \BJ
   + \frac{ \ncap \rho_\txtm }{\mu}
   - \PD{n}{\BH}
} \\
\ointclockwise_{\partial A} d\Bx \cross \BE &=
\int_A dA
\lr{
     \ncap \cross \BM
   + \frac{\ncap \rho}{\epsilon}
   - \PD{n}{\BE}
} \\
\ointclockwise_{\partial A} d\Bx \cdot \BH &= -\int_A dA\, \ncap \cdot \BJ.
\end{aligned}
\end{equation*}
} % theorem
The last of the scalar equations in
\cref{thm:amperes:280}
is Ampere's law
\begin{equation}\label{eqn:amperes:20}
\ointctrclockwise_{\partial A} d\Bx \cdot \BH = \int_A \ncap \cdot \BJ = I_{\textrm{enc}},
\end{equation}
and the first is the dual of Ampere's law for (fictious) magnetic current density\footnote{Even without the fictitious magnetic sources, neither the name nor applications of the two cross product line integrals with the normal derivatives are familiar to the author.}.
In \cref{eqn:amperes:20} the flux of the electric current density equals the enclosed current flowing through an open surface.  This enclosed current equals the line integral of the magnetic field around the boundary of that surface.

The proof and additional details can be found in the book.

It is worth pointing out that for pure magnetostatics problems where \( J = \eta \BJ, F = I \eta \BH \), that Ampere's law can be written in a trivector form
\begin{equation}\label{eqn:amperes:260}
\ointclockwise_{\partial A} d\Bx \wedge F = I \int_A dA\, \ncap \cdot J = I \eta \int_A dA\, \ncap \cdot \BJ.
\end{equation}
This encodes the fact that the magnetic field component of the total electromagnetic field strength is most naturally expressed in
geometric algebra as a bivector.

\subsection{Statics: Example field calculations.}
A number of worked examples were calculated to illustrate geometric algebra techniques.
\begin{itemize}
\item A finite line charge with line charge density \( \lambda \).  With \( i = \Be_1 \Be_3 \), the (electric) field was found to be \(
F = \ifrac{\lambda \Be_1}{(4 \pi \epsilon r)} \int_{a/r}^{b/r} du \lr{ e^{i\theta} - u }\lr{ 1 + u^2 - 2 u \cos\theta }^{-3/2} \).  The scalar portion of the integral is strictly a scale factor for the component of the field that lies along the x-axis, whereas the ``complex exponential'' factor of the integrand represents a rotational term along the direction \( \Be_1 e^{i\theta} = \Be_1 \cos\theta + \Be_3 \sin\theta \).  In problems like this, with only two degrees of freedom, there will often be a complex like representation possible using geometric algebra.
%This is illustrated in \cref{fig:linecharge:linechargeFig1}.
%\imageFigure{../figures/GAelectrodynamics/linechargeFig1}{Line charge density.}{fig:linecharge:linechargeFig1}{0.3}
\item The field for infinite static charge and current densities lying along the z-axis \( \rho(\Bx) = \lambda \delta(x) \delta(y), \BJ(\Bx) = \Bv \rho(\Bx) \) was found to be \( F = \ifrac{\lambda}{(2\pi \epsilon R)} \rhocap \lr{ 1 - \Bv/c}\).  The field splits naturally into electric (vector) and magnetic (bivector) grades as \( F = \BE \lr{ 1 - \Bv/c } = \BE + I \lr{ \ifrac{\Bv}{c} \cross \BE } \).
\item As a problem, the reader was asked to compute the field for
the magnetic charge density \( \rho_m = \lambda_m \delta(x) \delta(y) \), and current density \( \BM = v \Be_3 \rho_m = \Bv \rho_m \).  That field is
\( F = \ifrac{\lambda_m c}{(4 \pi R)} I \rhocap \lr{ 1 - \ifrac{\Bv}{c} } \), which may be split into electric and magnetic components as
\( F = \BB \cross \Bv + c I \BB \), where \( \BB = \lambda_m \rhocap/(4 \pi R) \).
\item The field for a uniform infinite planar charge density \( \rho(\Bx) = \sigma \delta(z) \) and associated current density \( \BJ(\Bx) = \Bv \rho(\Bx) \), where \( \Bv = v \Be_1 e^{i\theta}, \quad i = \Be_{12} \) was found to be
\( F = \ifrac{\sigma \sgn(z)}{(4 \pi \epsilon)} \Be_3 \lr{ 1 - \frac{\Bv}{c}} \).  As should be expected by superposition, the field splits neatly into electric field (vector) and magnetic field (bivector) components associated with the respective pure electrostatic and magnetostatics problems.
\item As a problem the reader is asked to show that the field for an infinite planar 
magnetic charge density \( \rho_m = \sigma_m \delta(z) \), and current density \( \BM = \Bv \rho_m, \Bv = v \Be_1 e^{i\theta}, i = \Be_{12}\) is \( F = \ifrac{\sigma_m c \sgn(z)}{(4 \pi)} i \lr{ 1 - \ifrac{\Bv}{c} } \).
\item The field for a line charge density \( \lambda \) along a circular arc segment \( \phi' \in [a,b] \), of radius \( r \) in the x-y plane was found to be \( F = \ifrac{\lambda r}{(4 \pi \epsilon_0 R^2)} \int_{a-\phi}^{b-\phi} d\alpha \lr{ \rcap + \phicap u i e^{i \alpha } } \lr{ 1 + u^2 - 2 u \sin\theta \cos \alpha }^{-3/2} \), where \( i = \Be_{12} \).  This problem is often given as an example or problem in electrostatics, but usually for circular charge distribution, and an observation point on the z-axis where symmetries kill off all but the z-axis component of the field.  The freedom to represent rotational terms as complex exponentials in geometric algebra allows the more general problem to be calculated without much additional difficulty.  The resulting integrals can be evaluated easily with any existing numerical integration software as the vector factors \( \rcap, \phicap \) may be pulled out of the integrals, leaving strict scalar or complex valued integrands.
%\imageFigure{../figures/GAelectrodynamics/lineChargeArcFig1}{Circular line charge.}{fig:circularlinecharge:circularlinechargeFig1}{0.5}
\item To illustrate the algebraic flexibility available, the circular ring charge problem is tackled in cylindrical coordinates instead of spherical (as previous).  For a static charge line density \( \lambda \) on a ring at \( z = 0 \), and an azimuthal current density \( \BJ = \Bv \rho \), we find a closed form solution for the field is found \( F = \ifrac{\lambda}{(4 \pi \epsilon R)}
\lr{
\lr{ \tilde{z} \Be_3 + \tilde{r} \rhocap -\ifrac{v i}{c} } A
- \lr{
\rhocap + \ifrac{v}{c} \lr{ \tilde{z} \Be_{3} \rhocap + \tilde{r} i } } B
} \), where \( \tilde{z} = z/R, \tilde{r} = r/z \).
The symmetry of the ring configuration allows for a closed form solution (numerical integration not required) of the field, but comes with the cost of requiring 
elliptic integrals \( A, B \), which are detailed in the book along with the derivation and plots of the resulting fields.
%\imageFigure{../figures/GAelectrodynamics/chargeAndCurrentOnRingFig1}{Field due to a circular distribution.}{fig:chargeAndCurrentOnRing:chargeAndCurrentOnRingFig1}{0.3}
%}
\item The final worked statics problem in the book is a use of Ampere's law, to compute the magnetic field for a
pair of z-axis oriented electric currents of magnitude \( I_1, I_2 \) flowing through the \( z = 0 \) points \( \Bp_1, \Bp_2 \) on the x-y plane.  The geometry and derivation is detailed in the book, but we use the multivector line integral form of Ampere's law \( \ointctrclockwise_{\partial A} d\Bx F = -I \int_A dA \Be_3 (-\eta \BJ) = I \eta I_\txte \), and superposition to compute the field
\( F = \sum_{k = 1,2} \ifrac{\eta I_k}{(2 \pi)} \ifrac{1}{(\Be_3 \wedge \lr{ \Br - \Bp_k})} \).
The bivector (magnetic) nature of a field with only electric current density sources is naturally represented by the wedge product \( \Be_3 \wedge \lr{ \Br - \Bp_k} \) which is a vector product of \( \Be_3 \) and the projection of \( \Br - \Bp_k \) onto the x-y plane.
%\cref{fig:amperesLawBetweenTwoCurrents:amperesLawBetweenTwoCurrentsFig1}.
%\imageFigure{../figures/GAelectrodynamics/amperesLawBetweenTwoCurrentsFig1}{Magnetic field between two current sources.}{fig:amperesLawBetweenTwoCurrents:amperesLawBetweenTwoCurrentsFig1}{0.3}
\end{itemize}
\subsection{Dynamics.}
Maxwell's equation (\cref{dfn:isotropicMaxwells:680}) is invertable, with solution.
\maketheorem{Jefimenkos solution.}{thm:jefimenkosEquations:120}{
The solution of Maxwell's equation is given by
\begin{equation*}
F(\Bx, t)
=
F_0(\Bx, t)
+
\inv{4 \pi}
\int dV'
\lr{
   \frac{\rcap}{r^2} J(\Bx', t_r)
   +
   \inv{c r} \lr{ 1 + \rcap } \dispdot{J}(\Bx', t_r)
},
\end{equation*}
where \( F_0(\Bx, t) \) is any specific solution of the homogenoous equation \( \lr{ \spacegrad + (1/c) \partial_t } F_0 = 0 \),
time derivatives are denoted by overdots, and all times are evaluated at the retarded time \( t_r = t - r/c \).
When expanded in terms of the electric and magnetic fields (ignoring magnetic sources), the non-homoegenous portion of this solution is known as the
Jefimenkos' equations
\begin{dmath}\label{eqn:jefimenkosEquations:100}
\begin{aligned}
\BE &=
\inv{4 \pi}
\int dV'
\lr{
\frac{\rcap}{\epsilon r} \lr{
\frac{\rho(\Bx', t_r)}{r} + \frac{\dispdot{\rho}(\Bx', t_r) }{c} }
   - \frac{\eta }{ c r } \dotBJ(\Bx', t_r)
} \\
\BH &=
\inv{4 \pi}
\int dV'
\lr{
   \frac{1}{c r} \dotBJ(\Bx', t_r)
+
   \frac{1}{r^2} \BJ(\Bx', t_r)
} \cross \rcap,
\end{aligned}
\end{dmath}
%which checks against Griffiths.
} % theorem
This is found fairly easily using the Green's function for the spacetime gradient \cref{thm:greensFunctionSpacetimeGradient:120}, and the details can be found in the book.  Unlike the conventional approach (\citep{griffiths1999introduction}), we are able to find the field directly without first having to determine the retarded time potentials, nor having to take their derivatives.

\subsection{Energy and momentum.}
In the energy and momentum section of the book, the geometric algebra formulation of the field energy density, Poynting vector, Maxwell stress tensor, and more generally, the energy momentum tensor are fully detailed.
\makedefinition{(Conventional) Energy and momentum density and Poynting vector.}{dfn:poyntingF:1220}{
The quantities \( \calE \) and \( \bcP \) defined as
%\label{eqn:poyntingF:20}
\begin{equation*}
\begin{aligned}
\calE &
%=
%\inv{2} \lr{ \BD \cdot \BE + \BB \cdot \BH }
= \inv{2} \lr{ \epsilon \BE^2 + \mu \BH^2 } \\
\bcP c &= \inv{c} \BE \cross \BH,
\end{aligned}
\end{equation*}
are known respectively as the field energy and momentum densities.
\( \BS = c^2 \bcP = \BE \cross \BH \) is called the Poynting vector.
} % definition
%\makedefinition{(Conventional) energy momentum and Maxwell stress tensors.}{dfn:poyntingF:1180}{
%The rank-2 symmetric tensor \( \Theta^{\mu\nu} \), with components
%%\label{eqn:poyntingF:840}
%\begin{equation*}
%\begin{aligned}
%\Theta^{00} &= \frac{\epsilon}{2} \lr{ \BE^2 + \eta^2 \BH^2 } \\
%\Theta^{0i} &= \inv{c} \lr{ \BE \cross \BH } \cdot \Be_i \\
%\Theta^{ij} &= -\epsilon \lr{ E_i E_j + \eta^2 H_i H_j - \inv{2} \delta_{ij} \lr{ \BE^2 + \eta^2 \BH^2 } },
%\end{aligned}
%\end{equation*}
%is called the energy momentum tensor.
%The spatial index subset of this tensor is known as the Maxwell stress tensor, and is often
%represented in dyadic notation
%\begin{equation*}
%\lr{ \Ba \cdot \lrT } \cdot \Bb
%=
%\sum_{i,j} a_i T_{ij} b_j,
%\end{equation*}
%or
%\begin{equation*}
%\Ba \cdot \lrT \equiv \sum_{i,j} a_i T_{ij} \Be_j
%\end{equation*}
%where \( T_{ij} = -\Theta^{ij} \).
%} % definition
%Here we use the usual convention of Greek indices such as \( \mu,\nu \) for ranging over both time (0) and spatial \( \setlr{1,2,3} \) indexes, and
%and Latin letters such as \( i, j \)
%for the ``spatial'' indexes
%\( \setlr{1,2,3} \).
%The names and notation for the tensors vary considerably\footnote{\( \Theta^{\mu\nu} \) in
%\cref{dfn:poyntingF:1180}
%is called the symmetric
%stress tensor by some authors \citep{jackson1975cew},
%and the energy momentum tensor by others, and is sometimes written \( T^{\mu\nu} \) (\citep{landau1980classical}, \citep{doran2003gap}).
%The sign conventions and notation for the spatial components \( \Theta^{ij} \) vary as well, but all authors appear to call this subset the Maxwell stress tensor.
%The Maxwell stress tensor may be written as \( \sigma_{ij} (=-\Theta^{ij}) \) \citep{landau1980classical}, or as
%\( T_{ij} (=-\Theta^{ij}) \)
%(\citep{griffiths1999introduction}, \citep{jackson1975cew}.)
%}.
In geometric algebra the energy momentum tensor, and the Maxwell stress tensor may be represented as linear grade 0,1 multivector valued functions of a grade 0,1 multivector.
\makedefinition{Energy momentum and Maxwell stress tensors.}{dfn:poyntingF:1200}{
We define the \textit{energy momentum tensor} as
\begin{equation*}
T(a) = \inv{2} \epsilon F a F^\dagger,
\end{equation*}
where \( a \) is a 0,1 multivector parameter.
We introduce a shorthand notation for grade one selection with vector valued parameters
\begin{equation*}
\BT(\Ba) = \gpgradeone{T(\Ba)},
\end{equation*}
and call this the \textit{Maxwell stress tensor}.
} % definition

\maketheorem{Expansion of the energy momentum tensor.}{thm:poyntingF:1240}{
Given a scalar parameter \( \alpha \), and a vector parameter \( \Ba = \sum_k a_k \Be_k \), the energy momentum tensor of
\cref{dfn:poyntingF:1200} is a grade 0,1 multivector, and may be expanded in terms of \( \calE, \BS \) and \( \BT(\Ba) \) as
%\label{eqn:poyntingF:1120}
\begin{equation*}
T(\alpha + \Ba)
=
\alpha \lr{
   \calE + \frac{\BS}{c}
}
-
\Ba \cdot \frac{\BS}{c}
+ \BT(\Ba),
\end{equation*}
where \( \BT(\Be_i) \cdot \Be_j = -\Theta^{ij} \), or \( \BT(\Ba) = \Ba \cdot \lrT \).
} % theorem

\Cref{thm:poyntingF:1240} relates the geometric algebra definition of the energy momentum tensor to the quantities found in the conventional
electromagnetism literature.  In the book, the conventional indexed representation is detailed more completely for comparision purposes.

Associated with the energy momentum tensor are a number of conservation relationships, which are most compactly stated utilizing the adjoint of the energy momenutum tensor.
\makedefinition{Adjoint.}{dfn:poyntingTheorem:1120}{
The adjoint \( \overbar{A}(x) \) of a linear operator \( A(x) \) is defined implicitly by the scalar selection
\begin{equation*}
\gpgradezero{ y \overbar{A}(x) } =
\gpgradezero{ x A(y) }.
\end{equation*}
} % definition
\maketheorem{Poynting's theorem (differential form.)}{thm:poyntingTheorem:1180}{
The adjoint energy momentum tensor of the spacetime gradient satisfies the following multivector equation
\begin{equation*}
\overbar{T}(\spacegrad + (1/c)\partial_t) = \frac{\epsilon}{2} \lr{ F^\dagger J + J^\dagger F }.
\end{equation*}
Note that the multivector \( F^\dagger J + J^\dagger F \) can only have scalar and vector grades, since it equals its reverse.
This equation can be put into a form that is more obviously a conservation law by stating it as a set of
scalar grade identities
\begin{equation*}
\spacegrad \cdot \gpgradeone{ T(a) } + \inv{c} \PD{t}{} \gpgradezero{ T(a) }
=
\frac{\epsilon}{2} \gpgradezero{ a( F^\dagger J + J \dagger F) }.
\end{equation*}
These may be written as respective
scalar and vector grades equations
%%which expands to the multivector equation
%\begin{equation*}
%\inv{c} \PD{t}{} \lr{ \calE - \frac{\BS}{c} }
%+ \spacegrad \cdot \frac{\BS}{c}
%+ \BT(\spacegrad)
%=
%-\inv{c} \lr{ \BE \cdot \BJ + \BH \cdot \BM }
%+
%\rho \BE + \epsilon \BE \cross \BM
%+
%\rho_\txtm \BH + \mu \BJ \cross \BH,
%\end{equation*}
%or as separate scalar and vector equations
\begin{equation*}
\begin{aligned}
\inv{c} \PD{t}{\calE} + \spacegrad \cdot \frac{\BS}{c} &= -\inv{c} \lr{ \BE \cdot \BJ + \BH \cdot \BM } \\
-\inv{c^2} \PD{t}{\BS} + \BT(\spacegrad) &= \rho \BE + \epsilon \BE \cross \BM + \rho_\txtm \BH + \mu \BJ \cross \BH.
\end{aligned}
\end{equation*}
Conventionally, only the scalar grade relating the time rate of change of the energy density to the flux of the Poynting vector, is called Poynting's theorem.
} % theorem
or in an integral form
\maketheorem{Poynting's theorem (integral form.)}{thm:poyntingTheoremRewrite:1420}{
\begin{dmath}\label{eqn:poyntingTheoremRewrite:1400}
\begin{aligned}
&\PD{t}{}
\int_V \calE dV
=
-\int_{\partial V} dA \ncap \cdot \BS
-
\int_V dV \lr{
   \BJ \cdot \BE
   +
   \BM \cdot \BH
} \\
&
\int_V dV \lr{ \rho \BE + \BJ \cross \BB }
+ \int_V dV \lr{ \rho_\txtm \BH - \epsilon \BM \cross \BE }
=
-
\PD{t}{ }
\int_V dV \bcP
+
\int_{\partial V} dA \BT(\ncap).
\end{aligned}
\end{dmath}
} % theorem

As the field in the volume is carrying the (electromagnetic) momentum \( \Bp_{\textrm{em}} = \int_V dV \bcP \), we can identify the sum of the Maxwell stress tensor's normal component over the bounding integral as time rate of change of the mechanical and electromagnetic momentum
\begin{equation}\label{eqn:ece2500report:n}
\frac{d\Bp_{\textrm{mech}}}{dt} + \frac{d\Bp_{\textrm{em}}}{dt} = \int_{\partial V} dA \BT(\ncap).
\end{equation}
The rate of change of mechanical momentum density \( \ifrac{d\Bp_{\textrm{mech}}}{dt} \) is, in fact, the continuous equivalent of the Lorentz force, which is found to be a direct consequence of conservation relationships associated with Maxwell's equation.

There were many details left out above.  Please refer to the text for more information.

\subsubsection{Example energy momentum calculations.}
To illustrate the ideas above, the energy momentum tensor components for all of the static fields computed previously are determined.  For brievity, these are omitted from this report, but it should be noted that we see that geometric algebra allows for a particularly compact coordinate free representation of the energy momentum tensor components.
\subsubsection{Complex power.}
The geometric algebra forms of the \( T(1) \) (field energy density and Poynting vector) are found to be
\maketheorem{Complex power representation.}{thm:poyntingFComplexPower:300}{
Given a time domain representation of a phasor based field \( F = F(\omega) \)
\begin{equation*}
F(t)
= \Real\lr{ F e^{j \omega t} },
\end{equation*}
the energy momentum tensor multivector \( T(1) \) has the representation
\begin{equation*}
T(1) = \calE + \frac{\BS}{c}
=
\frac{\epsilon}{4} \Real \lr{ F^\conj F^\dagger + F F^\dagger e^{2 j \omega t} }.
\end{equation*}
With the usual definition of the complex Poynting vector
%\label{eqn:poyntingFComplexPower:240}
\begin{equation*}
\calS = \inv{2} \BE \cross \BH^\conj = \inv{2} \lr{ I \BH^\conj } \cdot \BE,
\end{equation*}
the energy and momentum components of \( T(1) \), for real \( \mu, \epsilon \) are
%\label{eqn:poyntingFComplexPower:260}
\begin{equation*}
\begin{aligned}
\calE &=
\inv{4} \lr{
\epsilon \Abs{\BE}^2 + \mu \Abs{\BH}^2 }
+
\inv{4} \Real
\lr{
   \lr{ \epsilon \BE^2 + \mu \BH^2}
   e^{2 j \omega t }
} \\
\BS &= \Real \calS
+
\inv{2} \Real
\lr{
\lr{ \BE \cross \BH }
   e^{2 j \omega t }
}.
\end{aligned}
\end{equation*}
} % theorem

\subsection{Lorentz force.}
The Lorentz force equation can be stated in terms of the total electromagnetic field strength and chacurrent density
\maketheorem{Lorentz force and power.}{thm:lorentzForce:300}{
Given an energy momentum multivector \( T = \calE + c \Bp \), and a charge associated with a small bounded multivector current density \( Q = \int_V J dV \),
the respective power and force experienced by particles with electric (and/or magnetic) charges is described by
\cref{dfn:lorentzForce:280} is
\begin{equation*}
\inv{c} \frac{dT}{dt} = \gpgrade{ F Q^\dagger }{0,1} = \inv{2} \lr{ F^\dagger Q + F Q^\dagger }.
\end{equation*}
where \( \gpgradezero{dT/dt} = \ifrac{d\calE}{dt} \) is the power and \( \gpgradeone{dT/dt} = c \ifrac{d\Bp}{dt} \) is the force on the particle, and
\( Q^\dagger \) is the electric or magnetic charge/velocity multivector of \cref{dfn:lorentzForce:280}.
The conventional representation of the Lorentz force/power equations
\begin{equation*}
\begin{aligned}
\gpgradeone{ F Q^\dagger } &= \ddt{\Bp} = q \lr{ \BE + \Bv \cross \BB } \\
c \gpgradezero{ F Q^\dagger } &= \ddt{\calE} = q \BE \cdot \Bv.
\end{aligned}
\end{equation*}
%given by \cref{eqn:freespace:180}
may be recovered by grade selection operations.
For magnetic particles, such a grade selection gives
\begin{equation*}
\begin{aligned}
\gpgradeone{ F Q^\dagger } &= \frac{d\Bp}{dt} = q_\txtm \lr{ c \BB - \inv{c} \Bv \cross \BE } \\
c \gpgradezero{ F Q^\dagger } &= \frac{d\calE}{dt} = \inv{\eta} q_\txtm \BB \cdot \frac{\Bv}{c}.
\end{aligned}
\end{equation*}
} % theorem

\subsubsection{Constant magnetic field.}
As another example of geometric algebra in action, the Lorentz force equation for a constant external magnetic field bivector \( F = I c \BB \)
\begin{dmath}\label{eqn:lorentzForce_constantMagnetic:60}
m \frac{d\Bv}{dt} = q F \cdot \frac{\Bv}{c},
\end{dmath}
is solved in a fashion unique to this algebra.  With 
\( \Omega = -\ifrac{q F}{m c} \), the Lorentz force equation is reduced to \( \ifrac{d\Bv}{dt} = \Bv \cdot \Omega \), which may be solved
using a multivector integration factor.  The solution is shown to be
\begin{dmath}\label{eqn:lorentzForce_constantMagnetic:200}
\Bv(t) = e^{-\Omega t/2} \Bv(0) e^{\Omega t/2}.
\end{dmath}
Any component of the initial velocity \( \Bv(0)_\perp \) perpendicular to the \( \Omega \) plane is untouched by this rotation operation, whereas components of the initial velocity \( \Bv(0)_\parallel \) that lie in the \( \Omega \) plane will trace out a circular path, so the velocity of the charged particle traces out a helical path.

A multivector integration factor method for solving the Lorentz force equation in constant mixed external electric and magnetic fields can be found in \citep{hestenes1999nfc}.  More general examples are considered in the literature cited in the book.

\subsection{Polarization}
Following the usual convention, the geometric algebra treatment of polarization in the book 
aligns the propagation direction along the z-axis.
The field is
\begin{dmath}\label{eqn:polarization:20}
\begin{aligned}
F(\Bx, \omega) &= (1 + \Be_3) \BE e^{-j \beta z} \\
F(\Bx, t) &= \Real\lr{ F(\Bx, \omega) e^{j \omega t} },
\end{aligned}
\end{dmath}
where \( \BE \cdot \Be_3 = 0 \).
Here the imaginary \( j \) has no intrinsic geometrical interpretation, but we are able to dispense with it and use geometric imaginaries instead.  This is done by first assuming the electric field is given by \( \BE = \lr{ \alpha_1 + j \beta_1 } \Be_1 + \lr{ \alpha_2 + j \beta_2 } \Be_2 \), so that the 
time domain representation of the field is given by
\begin{dmath}\label{eqn:polarization_circular:160}
F(\Bx, t) = (1 + \Be_3) \lr{
\lr{ \alpha_1 \Be_1 + \alpha_2 \Be_2 } \cos\lr{ \omega t - \beta z }
-\lr{ \beta_1 \Be_1 + \beta_2 \Be_2 } \sin\lr{ \omega t - \beta z }
}.
\end{dmath}

Two geometric representations are possible.  The first uses the pseudoscalar for the transverse plane \( \Be_{12} \), denoted \( i \) here, and the other uses the \R{3} pseudoscalar as the imaginary.
\subsubsection{Transverse plane imaginary.}
\maketheorem{Circular polarization coefficients.}{thm:polarizationRewrite:700}{
The time domain representation of the field in \cref{eqn:polarization_circular:160} can be stated in terms of the total phase as
\begin{equation*}
F = \lr{ 1 + \Be_3 } \Be_1 \lr{ \alpha_\txtR e^{i\phi} + \alpha_\txtL e^{-i\phi} },
\end{equation*}
where
\begin{equation*}
\begin{aligned}
\alpha_\txtR &= \inv{2}\lr{ c_1 + i c_2 } \\
\alpha_\txtL &= \inv{2}\lr{ c_1 - i c_2 }^\dagger,
\end{aligned}
\end{equation*}
where \( c_1, c_2 \) are the 0,2 grade multivector representations of the Jones vector coordinates
\begin{equation*}
\begin{aligned}
c_1 &= \alpha_1 + i \beta_1 \\
c_2 &= \alpha_2 + i \beta_2,
\end{aligned}
\end{equation*}
and \( \phi(z,t) = \omega t - \beta z \) is the phase angle.
} % definition

\begin{itemize}
\item
Linear polarization at an angle \( \psi\) from the x-axis in the transverse plane is given by
\( \alpha_\txtR = \inv{2}\Norm{\BE} e^{i(\psi + \theta)},
\alpha_\txtL = \inv{2}\Norm{\BE} e^{i(\psi - \theta)} \), for which the field is \( F = \lr{ 1 + \Be_3 } \Norm{\BE} \Be_1 e^{i \psi} \cos(\phi + \theta) \),
where \( \theta \) is an initial phase angle.
\item Following the IEEE antenna convention discussed in \citep{balanis1989advanced}, we define right(left) circular polarization as the 
a change in phase that 
results in the electric field tracing out a (clockwise,counterclockwise) circle
\begin{dmath}\label{eqn:polarization_circular:180}
\begin{aligned}
\BE_\txtR &= \Norm{\BE} \lr{ \Be_1 \cos\phi + \Be_2 \sin\phi } = \Norm{\BE} \Be_1 \exp\lr{  \Be_{12} \phi } \\
\BE_\txtL &= \Norm{\BE} \lr{ \Be_1 \cos\phi - \Be_2 \sin\phi } = \Norm{\BE} \Be_1 \exp\lr{ -\Be_{12} \phi }.
\end{aligned}
\end{dmath}
Right and left circular polarization in this representation are given by
\(\alpha_\txtR = \Norm{\BE}, \alpha_\txtL = 0 \) and \(\alpha_\txtL = \Norm{\BE}, \alpha_\txtR = 0 \) respectively.  The right(left) polarized fields are just 
\( F = (1 + \Be_3) \Norm{\BE} \Be_1 e^{\pm i(\omega t - k z)} \).
\item An ellipically polzarized field is given by
\( \alpha_\txtR = \inv{2}\lr{ E_a - E_b },
\alpha_\txtL = \inv{2}\lr{ E_a + E_b } \), or 
\( F = \inv{2} (1 + \Be_3) \Be_1 \lr{ (E_a + E_b) e^{i\phi} + (E_a - E_b) e^{-i\phi} } \).  A hyperbolic hyperbolic parameterization of the elliptically polarized wave is also discussed in the book.  In particular, 
\begin{dmath}\label{eqn:polarization_elliptical:380}
\begin{aligned}
F &= e E_a \lr{ 1 + \Be_3 } \Be_1 e^{ i \psi } \cosh\lr{ m + i \phi} \\
m &= \tanh^{-1}\lr{ E_b/E_a } \\
e &= \sqrt{1 - {(E_b/E_a)}^2 },
\end{aligned}
\end{dmath}
where \( E_a(E_b) \) are the magnitudes of the electric field components lying along the semi-major axis directed along \( 
\begin{bmatrix}
\Be_1 \\
\Be_2 
\end{bmatrix}
e^{i\psi} \) respectively.
Additional discussion and diagrams can be found in the book.
%\imageFigure{../figures/GAelectrodynamics/ellipticalPolarizationFig1}{Electric field with elliptical polarization.}{fig:ellipticalPolarization:ellipticalPolarizationFig1}{0.3}
\end{itemize}

Each of the polarizations considered above (linear, circular, elliptical) have the same general form
\begin{dmath}\label{eqn:polarizationRewrite:760}
F = \lr{ 1 + \Be_3 } \Be_1 e^{i\psi} f(\phi),
\end{dmath}
where \( f(\phi) \) is a complex valued function (i.e. grade 0,2).  The structure of \cref{eqn:polarizationRewrite:760} could be more general than considered so far.  For example, a Gaussian modulation could be added into the mix with \( f(\phi) = e^{i \phi - (\phi/\sigma)^2/2 } \).  The simple complex structure that encodes all the phase dependence allows the 
energy, momentum and Maxwell stress tensor to be computed easily.  
\maketheorem{Energy momentum tensor components for a plane wave.}{thm:polarizationRewrite:780}{
The energy momentum tensor components for the plane wave given by \cref{eqn:polarizationRewrite:760} are
\begin{equation*}
\begin{aligned}
T(1) &= -T(\Be_3) = \epsilon \lr{ 1 + \Be_3 } f f^\dagger = \calE + \frac{\BS}{c} \\
T(\Be_1) &= T(\Be_2) = 0.
\end{aligned}
\end{equation*}
} % theorem

Only the propagation direction of a plane wave, regardless of its polarization (or even whether or not there are Gaussian or other damping factors), carries any energy or momentum, and only the propagation direction component of the Maxwell stress tensor \( \BT(\Ba) \) is non-zero.

Using \cref{thm:polarizationRewrite:780} the energy momentum vector may be computed for each of the polarizations considered above.
\begin{itemize}
\item
For the linearly polarized wave of \cref{eqn:polarization_linearPolarization:300} we have \( T(1) = \frac{\epsilon}{2} \lr{ 1 + \Be_3 } \Norm{\BE}^2 \cos^2( \phi + \theta ) \).
\item For the circularly polarization \cref{eqn:polarizationRewrite:720} \( T(1) = \frac{\epsilon}{2} (1 + \Be_3) \Norm{\BE}^2 \).
A circularly polarized wave carries maximum energy and momentum, whereas the energy and momentum of a linearly polarized wave
oscillates with the phase angle.
\item For the elliptical polarization \( 
T(1)
= \frac{\epsilon}{2} \lr{ 1 + \Be_3 } e^2 E_a^2 \lr{ \frac{E_b^2}{E_a^2} + 2 \lr{ 1 - \frac{E_b^2}{E_a^2} } \cos^2 \phi } \).
As expected, the phase dependent portion of the energy momentum tensor vanishes as the wave function approaches circular polarization.
\end{itemize}

\subsubsection{Pseudoscalar imaginary.}
Alternatively, it is possible to encode the sines and cosines in the time domain representation of the field in terms of the \R{3} pseudoscalar.

\maketheorem{Circular polarization coefficients.}{thm:polarizationRewrite:940}{
The time domain representation of the field in \cref{eqn:polarization_circular:160} can be stated in terms of the total phase as
\begin{equation*}
F = \lr{ 1 + \Be_3 } \Be_1 \lr{ \alpha_\txtR e^{-I\phi} + \alpha_\txtL e^{I\phi} },
\end{equation*}
where
\begin{equation*}
\begin{aligned}
\alpha_\txtR &= \inv{2}\lr{ c_1 + I c_2 }^\dagger \\
\alpha_\txtL &= \inv{2}\lr{ c_1 - I c_2 },
\end{aligned}
\end{equation*}
where \( c_1, c_2 \) are the 0,2 grade multivector representation of the Jones vector coordinates
\begin{equation*}
\begin{aligned}
c_1 &= \alpha_1 + I \beta_1 \\
c_2 &= \alpha_2 + I \beta_2,
\end{aligned}
\end{equation*}
defined here as 0,3 complex numbers, using \( I \) as the imaginary.
} % definition

There appear to be some advantages to pseudoscalar description of polarization, especially for computing energy momentum tensor components since \( I \) commutes with all grades.  For example, we can see practically by inspection that
\begin{equation}\label{eqn:polarization_pseudoscalarImaginary:620}
T(1) = \calE + \frac{\BS}{v} =
\epsilon \lr{ 1 + \Be_3 } \lr{ \Abs{\alpha_\txtR}^2 + \Abs{\alpha_\txtL}^2 },
\end{equation}
where the absolute value is computed using the reverse as the conjugation operation \( \Abs{z}^2 = z z^\dagger \).

\chapter{REWRITE}
\subsection{Transverse fields in a waveguide.}
%
% Copyright © 2017 Peeter Joot.  All Rights Reserved.
% Licenced as described in the file LICENSE under the root directory of this GIT repository.
%
%original ideas from gabookII/electrodynamics/transverseField.tex:
We now wish to consider more general solutions to the source free Maxwell's equation than the plane wave solutions derived in \cref{chap:planewavesMultivector}.
One way of tackling this problem is to assume the solution exists, but ask how the field components that lie strictly along the propagation direction are related to the transverse components of the field.
Without loss of generality, it can be assumed that the propagation direction is along the z-axis.

\maketheorem{Transverse and propagation field components.}{thm:transverseField:288}{
If \( \Be_3 \) is the
propagation direction, the components of a field \( F \) in the propagation direction and in the transverse plane are respectively
\begin{equation*}
\begin{aligned}
F_z &= \inv{2} \lr{ F + \Be_3 F \Be_3 } \\
F_t &= \inv{2} \lr{ F - \Be_3 F \Be_3 },
\end{aligned}
\end{equation*}
where \( F = F_z + F_t \).
} % theorem

To determine the components of the field that lie in the propagation direction and transverse planes, we state the field in the propagation direction, building it from the electric and magnetic field projections along the z-axis
\begin{dmath}\label{eqn:transverseField:108}
F_z
=
\lr{ \BE \cdot \Be_3 }
 \Be_3
+ I \eta \lr{ \BH \cdot \Be_3 } \Be_3
=
\inv{2}
\lr{ \BE \Be_3 + \Be_3 \BE }
 \Be_3
+ \inv{2} I \eta \lr{ \BH \Be_3 + \Be_3 \BH } \Be_3
=
\inv{2}
\lr{ \BE + \Be_3 \BE \Be_3 }
+ \inv{2} I \eta \lr{ \BH + \Be_3 \BH \Be_3 }
=
\inv{2} \lr{ F + \Be_3 F \Be_3 }.
\end{dmath}
The difference \( F - F_z \) is the transverse component
\begin{dmath}\label{eqn:transverseField:308}
F_t
= F - F_z
=
F -
\inv{2} \lr{ F + \Be_3 F \Be_3 }
=
\inv{2} \lr{ F - \Be_3 F \Be_3 },
\end{dmath}
as claimed.

We wish to split the gradient into transverse and propagation direction components.

\makedefinition{Transverse and propagation direction gradients.}{dfn:transverseField:328}{
Define the \textit{propagation direction gradient} as \( \Be_3 \partial_z \), and
\textit{transverse gradient} by
\begin{equation*}
\spacegrad_t = \spacegrad - \Be_3 \partial_z.
\end{equation*}
} % definition

Given this definition, we seek to show that

\maketheorem{Transverse and propagation field solutions.}{thm:transverseField:348}{
Given a field propagating along the z-axis (either forward or backwards), with angular frequency \( \omega \), represented by the real part of
\begin{equation*}
F(x, y, z, t) = F(x, y) e^{j \omega t \mp j k z},
\end{equation*}
the field components that solve the source free Maxwell's equation are related by
\begin{equation*}
\begin{aligned}
F_t &= j \inv{ \frac{\omega}{c} \mp k \Be_3 } \spacegrad_t F_z \\
F_z &= j \inv{ \frac{\omega}{c} \mp k \Be_3 } \spacegrad_t F_t.
\end{aligned}
\end{equation*}
Written out explicitly, the transverse field component expands as
\begin{equation*}
\begin{aligned}
\BE_t &=
\frac{j}{{\frac{\omega}{c}}^2 - k^2}
\lr{
   \pm k \spacegrad_t E_z
   + \frac{\omega \eta}{c} \Be_3 \cross \spacegrad_t H_z
}
\\
\eta \BH_t &=
\frac{j}{{\frac{\omega}{c}}^2 - k^2}
\lr{
   \pm k \eta \spacegrad_t H_z
   -
   \frac{\omega}{c}
   \Be_3 \cross \spacegrad_t E_z
}.
\end{aligned}
\end{equation*}
} % theorem

To prove we first insert the assumed phasor representation into Maxwell's equation, which gives
\begin{equation}\label{eqn:transverseField:summaryMax2}
\lr{\spacegrad_t + j \lr{ \frac{\omega}{c} \mp k \Be_3 } } F(x,y) = 0.
\end{equation}

Dropping the \( x, y \) dependence for now (i.e.  \( F(x, y) \rightarrow F \), we find a relation between the transverse gradient of \( F \) and the propagation direction gradient of \( F \)

\begin{dmath}\label{eqn:transverseField:148}
\spacegrad_t F = - j \lr{ \frac{\omega}{c} \mp k \Be_3 } F.
\end{dmath}
From this we now seek to determine the relationships between \( F_t \) and \( F_z \).

Since \( \spacegrad_t \) has no \( \xcap, \ycap \) components, \( \Be_3 \) anticommutes with the transverse gradient
\begin{dmath}\label{eqn:transverseField:168}
\Be_3 \spacegrad_t = - \spacegrad_t \Be_3,
\end{dmath}
but commutes with \( 1 \mp \Be_3 \).
%In \cref{eqn:transverseField:168} it is implied that the action of \( \spacegrad_t \) is on everything to its right.
This means that
\begin{dmath}\label{eqn:transverseField:188}
\inv{2} \lr{ \spacegrad_t F \pm \Be_3 \lr{ \spacegrad_t F } \Be_3 }
=
\inv{2} \lr{ \spacegrad_t F \mp \spacegrad_t \Be_3 F \Be_3 }
=
\spacegrad_t
\inv{2} \lr{ F \mp \Be_3 F \Be_3 },
\end{dmath}
or
\begin{dmath}\label{eqn:transverseField:208}
\begin{aligned}
\inv{2} \lr{ \spacegrad_t F + \Be_3 \lr{ \spacegrad_t F } \Be_3 } &= \spacegrad_t F_t \\
\inv{2} \lr{ \spacegrad_t F - \Be_3 \lr{ \spacegrad_t F } \Be_3 } &= \spacegrad_t F_z,
\end{aligned}
\end{dmath}
so Maxwell's equation \cref{eqn:transverseField:148} becomes
\begin{dmath}\label{eqn:transverseField:228}
\begin{aligned}
\spacegrad_t F_t &= - j \lr{ \frac{\omega}{c} \mp k \Be_3 } F_z \\
\spacegrad_t F_z &= - j \lr{ \frac{\omega}{c} \mp k \Be_3 } F_t.
\end{aligned}
\end{dmath}

Provided \( \omega^2 \ne (k c)^2 \), these can be inverted.
Such an inversion allows an application of the transverse gradient to whichever one
of \( F_z, F_t \) is known, to compute the other, as stated in
\cref{thm:transverseField:348}.

The relation for \( F_t \) in
\cref{thm:transverseField:348}
is usually stated in terms of the electric and magnetic fields.
To perform that expansion, we must first evaluate the multivector inverse explicitly
\begin{dmath}\label{eqn:transverseField:348}
\begin{aligned}
F_z &= j \frac{ \frac{\omega}{c} \pm k \Be_3 }{ \lr{\frac{\omega}{c}}^2 - k^2 } \spacegrad_t F_t \\
F_t &= j \frac{ \frac{\omega}{c} \pm k \Be_3 }{ \lr{\frac{\omega}{c}}^2 - k^2 } \spacegrad_t F_z.
\end{aligned}
\end{dmath}
so that we are in position to expand most of the terms in the numerator
\begin{dmath}\label{eqn:transverseField:268}
\lr{ \frac{\omega}{c} \pm k \Be_3 } \spacegrad_t F_z
=
-\lr{ \Be_3 \frac{\omega}{c} \pm k } \spacegrad_t \Be_3 F_z
=
\lr{ \pm k - \Be_3 \frac{\omega}{c} } \spacegrad_t \lr{ E_z + I \eta H_z }
=
\lr{
   \pm k \spacegrad_t E_z
   + \frac{\omega \eta}{c} \Be_3 \cross \spacegrad_t H_z
}
+ I \lr{
   \pm k \eta \spacegrad_t H_z
   -
   \frac{\omega}{c}
   \Be_3 \cross \spacegrad_t E_z
},
\end{dmath}
from which the transverse electric and magnetic fields stated in
\cref{thm:transverseField:348} can be read off.
A similar expansion for \( \BE_z, \BH_z \) in terms of \( \BE_t, \BH_t \) is also possible.

%There is considerably more complexity required to express the transverse field in terms of separate electric and magnetic components
%compared to the equivalent total transverse field expression of...

\makeproblem{Transverse electric and magnetic field components.}{problem:transverseField:1}{
Fill in the missing details in the steps of \cref{eqn:transverseField:268}.
} % problem

\makeproblem{Propagation direction components.}{problem:transverseField:2}{
Perform an expansion like \cref{eqn:transverseField:268} to find
\( \BE_z, \BH_z \) in terms of \( \BE_t, \BH_t \).
} % problem
      \section{Multivector potential.}
         \subsection{Definition.}
%
% Copyright � 2018 Peeter Joot.  All Rights Reserved.
% Licenced as described in the file LICENSE under the root directory of this GIT repository.
%
%{
%\input{../latex/blogpost.tex}
%\renewcommand{\basename}{gaugeTransformation}
%%\renewcommand{\dirname}{notes/phy1520/}
%\renewcommand{\dirname}{notes/ece1228-electromagnetic-theory/}
%%\newcommand{\dateintitle}{}
%%\newcommand{\keywords}{}
%
%\input{../latex/peeter_prologue_print2.tex}
%
%\usepackage{peeters_layout_exercise}
%\usepackage{peeters_braket}
%\usepackage{peeters_figures}
%\usepackage{siunitx}
%%\usepackage{mhchem} % \ce{}
%%\usepackage{macros_bm} % \bcM
%\usepackage{macros_qed} % \qedmarker
%%\usepackage{txfonts} % \ointclockwise
%
%%\newcommand{\dLambertian}[0]{\Box}
%\newcommand{\dLambertian}[0]{\square}
%
%\newcommand{\stgrad}[0]{\lr{ \spacegrad + \inv{c} \PD{t}{}}}
%\newcommand{\conjstgrad}[0]{\lr{ \spacegrad - \inv{c} \PD{t}{}}}
%
%\beginArtNoToc
%
%\generatetitle{Multivector potentials.}
%%\chapter{Multivector potentials.}
\label{chap:gaugeTransformation}

Conventional electromagnetism utilizes scalar and vector potentials, so it is reasonable to expect that
the desired multivector representation of the potential is a grade 0,1 multivector.
A potential representation with grades 2,3 works for (fictitous) magnetic sources, so we may generally
allow a multivector potential to have any grades.  Such a potential is related to the field as follows.

\makedefinition{Multivector potential.}{thm:generalPotential:80}{
The electromagnetic field strength \( F \) for a \textit{multivector potential} \( A \) is
\begin{equation*}
F = \gpgrade{\conjstgrad A}{1,2}.
\end{equation*}
} % definition

Before unpacking \( \conjstgrad A \), we want to label the
different grades of the multivector potential, and do so in a way that is consisent with the conventional
potential representation of the electric and magnetic fields.
\makedefinition{Multivector potential representation.}{dfn:unpackStaticPotential:80}{
Let
%\label{eqn:gaugeTransformation:1111}
\begin{equation*}
A =
      - \phi
      + c \BA
      + \eta I \lr{ -\phi_m + c \BF },
\end{equation*}
where
\begin{enumerate}
\item \( \phi \) is the scalar potential \si{V} (Volts).
\item \( \BA \) is the vector potential \si{W/m} (Webers/meter).
\item \( \phi_m \) is the scalar potential for (fictitious) magnetic sources \si{A} (Amperes).
\item \( \BF \) is the vector potential for (fictitious) magnetic sources \si{C} (Coulombs).
\end{enumerate}
} % definition
This specific breakdown of \( A \) into scalar and vector potentials, and dual (pseudoscalar and bivector) potentials has been chosen to match SI conventions, specifically those of \citep{balanis2005antenna} (which includes fictitious magnetic sources.)

We can now express the fields in terms of the potentials.

\maketheorem{Fields and the potential wave equations.}{thm:generalPotential:40}{
In terms of the potential components, the electric field vector and the magnetic field bivector are
\begin{equation*}
\begin{aligned}
\BE &=
\gpgrade{\conjstgrad A}{1}
=
   - \spacegrad \phi
   - \PD{t}{\BA}
   - \inv{\epsilon} \spacegrad \cross \BF \\
I \eta \BH &=
\gpgrade{\conjstgrad A}{2}
=
   I \eta
   \lr{
      - \spacegrad \phi_\txtm
      - \PD{t}{\BF}
      + \inv{\mu} \spacegrad \cross \BA
   }
.
\end{aligned}
\end{equation*}
The potentials are related to the sources by
\begin{equation*}
\begin{aligned}
\dLambertian
\phi &= -\frac{\rho}{\epsilon} - \PD{t}{} \lr{ \spacegrad \cdot \BA + \inv{c^2} \PD{t}{\phi} } \\
\dLambertian
\BA &= -\mu \BJ + \spacegrad \lr{ \spacegrad \cdot \BA + \inv{c^2} \PD{t}{\phi} } \\
\dLambertian
\BF &= - \epsilon \BM + \spacegrad \lr{ \spacegrad \cdot \BF + \inv{c^2} \PD{t}{\phi_\txtm} } \\
\dLambertian
\phi_\txtm &= -\frac{\rho_\txtm}{\mu} - \PD{t}{} \lr{ \spacegrad \cdot \BF + \inv{c^2} \PD{t}{\phi_\txtm} }
\end{aligned}
\end{equation*}
} % theorem

To prove \cref{thm:generalPotential:40} we start by expanding \( (\spacegrad - (1/c)\partial_t) A \) using
\cref{dfn:unpackStaticPotential:80} and then group by grade to find
\begin{dmath}\label{eqn:gaugeTransformation:1111}
\begin{aligned}
\conjstgrad A
&=
\conjstgrad \lr{  - \phi
      + c \BA
      + \eta I \lr{ -\phi_m + c \BF } } \\
&=
- \spacegrad \phi + c \spacegrad \cdot \BA + c \spacegrad \wedge \BA + \inv{c} \PD{t}{\phi} - \PD{t}{\BA} \\
&\quad + I \eta
\lr{
- \spacegrad \phi_\txtm + c \spacegrad \cdot \BF + c \spacegrad \wedge \BF + \inv{c} \PD{t}{\phi_\txtm} - \PD{t}{\BF}
} \\
&=
c \spacegrad \cdot \BA
+ \inv{c} \PD{t}{\phi}
\\
&
+
\mathLabelBox[ labelstyle={below of=m\themathLableNode, below of=m\themathLableNode} ]
{
   - \spacegrad \phi
   - \PD{t}{\BA}
   - \inv{\epsilon} \spacegrad \cross \BF
}
{
\(\BE\)
}
+
\mathLabelBox[ labelstyle={below of=m\themathLableNode, below of=m\themathLableNode} ]
{
   I \eta
   \lr{
      - \spacegrad \phi_\txtm
      - \PD{t}{\BF}
      + \inv{\mu} \spacegrad \cross \BA
   }
}
{\(I \eta \BH\)
} \\
&
+ I \eta\lr{
  c \spacegrad \cdot \BF
+ \inv{c} \PD{t}{\phi_\txtm}
},
\end{aligned}
\end{dmath}
which shows the claimed field split.

In terms of the potentials Maxwell's equation \( \stgrad F = J \) is
\begin{dmath}\label{eqn:gaugeTransformation:20}
\stgrad \gpgrade{\conjstgrad A}{1,2} = J,
\end{dmath}
or
\begin{dmath}\label{eqn:gaugeTransformation:40}
\dLambertian A = J + \stgrad \gpgrade{\conjstgrad A}{0,3}.
\end{dmath}
This is almost a wave equation.
Inserting \cref{eqn:gaugeTransformation:1111} into \cref{eqn:gaugeTransformation:40} and selecting each grade gives four almost-wave equations
\begin{equation*}
\begin{aligned}
-
\dLambertian
\phi &= \frac{\rho}{\epsilon} + \inv{c} \PD{t}{} \lr{ c \spacegrad \cdot \BA + \inv{c} \PD{t}{\phi} } \\
c
\dLambertian
\BA &= -\eta \BJ + \spacegrad \lr{ c \spacegrad \cdot \BA + \inv{c} \PD{t}{\phi} } \\
\eta c I
\dLambertian
\BF &= - I \BM + \spacegrad \cdot \lr{ I \eta\lr{ c \spacegrad \cdot \BF + \inv{c} \PD{t}{\phi_\txtm} } } \\
-I \eta
\dLambertian
\phi_\txtm &= I c \rho_\txtm + \inv{c} \PD{t}{} I \eta\lr{ c \spacegrad \cdot \BF + \inv{c} \PD{t}{\phi_\txtm} }
\end{aligned}
\end{equation*}
Using \( \eta = \mu c, \eta c \epsilon = 1 \), and
\( \spacegrad \cdot (I \psi) = I \spacegrad \psi \) for scalar \(\psi\), a bit
of rearrangement completes the proof.

\subsection{Gauge transformations.}
Clearly it is desirable if potentials can be found for which \( \spacegrad \cdot \BA + (1/c^2) \partial_t \phi = \spacegrad \cdot \BF + (1/c^2) \partial_t \phi_\txtm = 0 \).
Finding such potentials relies on the fact that the potential representation is not unique.
In particular,
we have the freedom to add any spacetime gradient of any scalar or pseudoscalar potential without changing the field.
\maketheorem{Gauge invariance.}{thm:gaugeTransformation:60}{
The spacetime gradient of a grade 0,3 multivector \( \Psi \) may be added to a multivector potential
\begin{equation*}
A' = A + \stgrad \Psi,
\end{equation*}
without changing the field.
That is
\begin{equation*}
F
= \gpgrade{\conjstgrad A}{1,2}
= \gpgrade{\conjstgrad A'}{1,2}.
\end{equation*}
} % theorem

To prove \cref{thm:gaugeTransformation:60} let
\begin{dmath}\label{eqn:gaugeTransformation:100}
A' = A + \stgrad (\psi + I \phi),
\end{dmath}
where \( \psi \) and \( \phi \) are scalar functions.
The field for potential \( A' \) is
\begin{dmath}\label{eqn:gaugeTransformation:120}
F'
= \gpgrade{\conjstgrad A'}{1,2}
= \gpgrade{\conjstgrad \lr{A + \stgrad (\psi + I \phi)} }{1,2}
= \gpgrade{\conjstgrad A}{1,2} + \gpgrade{ \conjstgrad \stgrad (\psi + I \phi)} {1,2}
= F + \gpgrade{ \dLambertian (\psi + I \phi)} {1,2},
\end{dmath}
which is just \( F \) since
since the d'Lambertian operator \( \dLambertian \) is a scalar operator and \( \psi + I \phi \) has no vector nor bivector grades.

We say that we are working in the Lorenz gauge, if the 0,3 grades of \( \conjstgrad A \) are zero, or a transformation that kills those grades is made.
\maketheorem{Lorentz gauge transformation.}{thm:gaugeTransformation:140}{
Given any multivector potential \( A \) solution of Maxwell's equation, the transformation
\begin{equation*}
A' = A - \stgrad \Psi,
\end{equation*}
where
\begin{equation*}
\dLambertian \Psi = \gpgrade{ \conjstgrad A }{0,3},
\end{equation*}
allows Maxwell's equation to be written in wave equation form
\begin{equation*}
\dLambertian A' = J.
\end{equation*}
} % theorem

To prove \cref{thm:gaugeTransformation:140}, let
\begin{dmath}\label{eqn:gaugeTransformation:200}
A = A' + \stgrad \Psi,
\end{dmath}
so Maxwell's equation becomes
\begin{dmath}\label{eqn:gaugeTransformation:220}
J
= \stgrad \gpgrade{ \conjstgrad A }{1,2}
= \dLambertian A - \stgrad \gpgrade{ \conjstgrad A }{0,3}
= \dLambertian A' + \dLambertian \stgrad \Psi - \stgrad \gpgrade{ \conjstgrad A }{0,3}
= \dLambertian A' + \stgrad \lr{ \dLambertian \Psi - \gpgrade{ \conjstgrad A }{0,3} }.
\end{dmath}
Requiring
\begin{dmath}\label{eqn:gaugeTransformation:240}
\dLambertian \Psi = \gpgrade{ \conjstgrad A }{0,3},
\end{dmath}
completes the proof.
Observe that \( \Psi \) has only grades 0,3 as required of a gauge function.

Such a transformation completely decouples Maxwell's equation, providing one scalar wave equation for each grade of \( \dLambertian A' = J \), relating each grade of the potential \(A'\) to exactly one grade of the source multivector current \( J \).
We are free to immediately solve for \( A' \) using the (causal) Green's function for the d'Lambertian
\begin{dmath}\label{eqn:gaugeTransformation:160}
A'(\Bx, t)
= -\int dV' dt' \frac{\delta(\Abs{\Bx - \Bx'} - c(t - t')}{4 \pi \Norm{\Bx - \Bx'} } J(\Bx', t')
= -\inv{4\pi} \int dV' \frac{J(\Bx', t - \inv{c} \Norm{\Bx - \Bx'})}{\Norm{\Bx - \Bx'}},
\end{dmath}
which is the sum of all the current contributions relative to the point \( \Bx \) at the retarded time \( t_\txtr = t - (1/c) \Norm{\Bx - \Bx'}\).
The field follows immediately by differentiation and grade selection
\begin{dmath}\label{eqn:gaugeTransformation:180}
F = \gpgrade{ \conjstgrad A' }{1,2}.
\end{dmath}

Again, using the Green's function for the d'Lambertian, the explicit form of the gauge function \( \Psi \) is
\begin{dmath}\label{eqn:gaugeTransformation:260}
\Psi = -\inv{4\pi} \int dV' \frac{\gpgrade{ \conjstgrad A(\Bx', t_\txtr) }{0,3}}{\Norm{\Bx - \Bx'}},
\end{dmath}
however, we don't actually need to compute this.
Instead, we only have to know we are free to construct a field from any solution \( A' \) of \( \dLambertian A' = J \) using \cref{eqn:gaugeTransformation:180}.

%}
%\EndArticle
         \subsection{Far field.}
%
% Copyright © 2018 Peeter Joot.  All Rights Reserved.
% Licenced as described in the file LICENSE under the root directory of this GIT repository.
%
%{

\maketheorem{Far field magnetic vector potential.}{thm:potentialSection_farfield:1}{
Given a vector potential with a radial spherical wave representation
%\label{eqn:potentialSection_farfield:2400}
\begin{equation*}
\BA = \frac{e^{-j k r}}{r} \bcA( \theta, \phi ),
\end{equation*}
the far field (\(r \gg 1 \)) electromagnetic field is
%\label{eqn:potentialSection_farfield:2520}{
\begin{equation*}
F = -j \omega \lr{ 1 + \rcap } \lr{ \rcap \wedge \BA}.
\end{equation*}
If \( \BA_\perp = \rcap \lr{ \rcap \wedge \BA} \) represents the
non-radial component of the potential, the respective electric and magnetic field components are
%\begin{dmath}\label{eqn:potentialSection_farfield:2560}
\begin{equation*}
\begin{aligned}
\BE &= -j \omega \BA_\perp \\
\BH &= \inv{\eta} \rcap \cross \BE.
\end{aligned}
\end{equation*}
%\end{dmath}
} % theorem

To prove \cref{thm:potentialSection_farfield:1}, we will utilize a
spherical representation of the gradient
\begin{dmath}\label{eqn:potentialSection_farfield:2420}
\begin{aligned}
\spacegrad &= \rcap \partial_r + \spacegrad_\perp \\
\spacegrad_\perp &= \frac{\thetacap}{r} \partial_\theta + \frac{\phicap}{r\sin\theta} \partial_\phi.
\end{aligned}
\end{dmath}

The gradient of the vector potential is
\begin{dmath}\label{eqn:potentialSection_farfield:2440}
\spacegrad \BA
=
\biglr{ \rcap \partial_r + \spacegrad_\perp } \frac{e^{-j k r}}{r} \bcA
=
\rcap \lr{ -j k - \inv{r} } \frac{e^{-j k r}}{r} \bcA
+
\frac{e^{-j k r}}{r}
\spacegrad_\perp
\bcA
= - \lr{ j k + \inv{r} } \rcap \BA + O(1/r^2)
\approx
- j k \rcap \BA.
\end{dmath}

Here, all the \( O(1/r^2) \) terms, including the action of the non-radial component of the gradient on the \( 1/r \) potential, have been neglected.
From \cref{eqn:potentialSection_farfield:2440} the far field divergence and the (bivector) curl of \( \BA \) are
\begin{dmath}\label{eqn:potentialSection_farfield:2460}
\begin{aligned}
\spacegrad \cdot \BA &= - j k \rcap \cdot \BA \\
\spacegrad \wedge \BA &= - j k \rcap \wedge \BA.
\end{aligned}
\end{dmath}

Finally, the far field gradient of the divergence of \( \BA \) is
\begin{dmath}\label{eqn:potentialSection_farfield:2480}
\spacegrad \lr{ \spacegrad \cdot \BA }
=
\biglr{ \rcap \partial_r + \spacegrad_\perp } \lr{ - j k \rcap \cdot \BA }
\approx
-j k \rcap \partial_r \lr{ \rcap \cdot \BA }
=
-j k \rcap \lr{ -j k - \inv{r} } \lr{ \rcap \cdot \BA }
\approx
-k^2 \rcap \lr{ \rcap \cdot \BA },
\end{dmath}
again neglecting any \( O(1/r^2) \) terms.  The field is
\begin{dmath}\label{eqn:potentialSection_farfield:2500}
F
=
- j \omega \BA  -j \frac{c^2}{\omega} \spacegrad \lr{ \spacegrad \cdot \BA } + c \spacegrad \wedge \BA
=
- j \omega \BA  +j \omega \rcap \lr{ \rcap \cdot \BA } - j k c \rcap \wedge \BA
=
- j \omega \lr{ \BA - \rcap \lr{ \rcap \cdot \BA }} - j \omega \rcap \wedge \BA
=
-j \omega \rcap \lr{ \rcap \wedge \BA} - j \omega \rcap \wedge \BA
=
-j \omega \lr{ \rcap + 1 } \lr{ \rcap \wedge \BA},
\end{dmath}
which completes the first part of the proof.  Extraction of the electric and magnetic fields can be done by inspection and is left to the reader to prove.

One interpretation of this is that the (bivector) magnetic field is represented by the plane perpendicular to the direction of propagation, and the electric field by a vector in that plane.

\maketheorem{Far field electric vector potential.}{thm:potentialSection_farfield:2}{
Given a vector potential with a radial spherical wave representation
%\label{eqn:potentialSection_farfield:2400}
\begin{equation*}
\BF = \frac{e^{-j k r}}{r} \bcF( \theta, \phi ),
\end{equation*}
the far field (\(r \gg 1 \)) electromagnetic field is
%\label{eqn:potentialSection_farfield:2520}
\begin{equation*}
F = -j \omega \eta I \lr{ \rcap + 1 } \lr{ \rcap \wedge \BF }.
\end{equation*}
If \( \BF_\perp = \rcap \lr{ \rcap \wedge \BF} \) represents the
non-radial component of the potential, the respective electric and magnetic field components are
%\begin{dmath}\label{eqn:potentialSection_farfield:2560}
\begin{equation*}
\begin{aligned}
\BE &= j \omega \eta \rcap \cross \BF \\
\BH &= -j \omega \BF_\perp.
\end{aligned}
\end{equation*}
%\end{dmath}
} % theorem

The proof of \cref{thm:potentialSection_farfield:2} is left to the reader.

\makeexample{Vertical dipole potential.}{example:potentialSection:1}{
We will calculate the far field along the propagation direction vector \( \kcap \) in the z-y plane
\begin{dmath}\label{eqn:potentialSection_farfield:2620}
\begin{aligned}
\kcap &= \Be_3 e^{i \theta} \\
i &= \Be_{32},
\end{aligned}
\end{dmath}
for the infinitesimal dipole potential
\begin{dmath}\label{eqn:potentialSection_farfield:2640}
\BA = \frac{\Be_3 \mu I_0 l}{4 \pi r} e^{-j k r},
\end{dmath}
as illustrated in \cref{fig:vectorPotential:vectorPotentialFig1}.

\imageFigure{../figures/GAelectrodynamics/vectorPotentialFig1}{Vertical infinitesimal dipole and selected propagation direction.}{fig:vectorPotential:vectorPotentialFig1}{0.3}

The wedge of \( \kcap \) with \( \BA \) is proportional to
\begin{dmath}\label{eqn:potentialSection_farfield:2660}
\kcap \wedge \Be_3
=
\gpgradetwo{
\kcap \Be_3
}
=
\gpgradetwo{
\Be_3 e^{i \theta}
\Be_3
}
=
\gpgradetwo{
\Be_3^2 e^{-i \theta}
}
=
-i \sin\theta,
\end{dmath}
so from \cref{eqn:potentialSection_farfield:2520} the field is
\begin{dmath}\label{eqn:potentialSection_farfield:2680}
F = j \omega \lr{ 1 + \Be_3 e^{i\theta} } i \sin\theta \frac{\mu I_0 l}{4 \pi r} e^{-j k r}.
\end{dmath}

The electric and magnetic fields can be found from the respective vector and bivector grades of \cref{eqn:potentialSection_farfield:2680}
\begin{dmath}\label{eqn:potentialSection_farfield:2700}
\BE
=
\frac{j \omega \mu I_0 l}{4 \pi r} e^{-j k r} \Be_3 e^{i\theta} i \sin\theta
=
\frac{j \omega \mu I_0 l}{4 \pi r} e^{-j k r} \Be_2 e^{i\theta} \sin\theta
=
\frac{j k \eta I_0 l \sin\theta}{4 \pi r} e^{-j k r} \lr{ \Be_2 \cos\theta - \Be_3 \sin\theta },
\end{dmath}
and
\begin{dmath}\label{eqn:potentialSection_farfield:2720}
\BH
=
\inv{I \eta}
j \omega i \sin\theta_0 \frac{\mu I_0 l}{4 \pi r} e^{-j k r}
=
\inv{\eta} \Be_{321} \Be_{32}
j \omega \sin\theta_0 \frac{\mu I_0 l}{4 \pi r} e^{-j k r}
=
-\Be_1 \frac{ j k \sin\theta_0 I_0 l}{4 \pi r} e^{-j k r}.
\end{dmath}

The multivector electrodynamic field expression
\cref{eqn:potentialSection_farfield:2680} for
\( F \) is more algebraically compact than the separate electric and magnetic field expressions, but this comes with the complexity of dealing with different types of imaginaries.
There are two explicit unit imaginaries in \cref{eqn:potentialSection_farfield:2680}, the scalar imaginary \( j \) used to encode the time harmonic nature of the field, and \( i = \Be_{32} \) used to represent the plane that the far field propagation direction vector lay in.
Additionally, when the magnetic field component was extracted, the pseudoscalar \( I = \Be_{123} \) entered into the mix.
Care is required to keep these all separate, especially since \( I, j \) commute with all grades, but \( i \) does not.
} % example

%}
      \section{Dielectric and magnetic media.}
%
% Copyright � 2018 Peeter Joot.  All Rights Reserved.
% Licenced as described in the file LICENSE under the root directory of this GIT repository.
%
%{
%\input{../latex/blogpost.tex}
%\renewcommand{\basename}{dielectric}
%%\renewcommand{\dirname}{notes/phy1520/}
%\renewcommand{\dirname}{notes/ece1228-electromagnetic-theory/}
%%\newcommand{\dateintitle}{}
%%\newcommand{\keywords}{}
%
%\input{../latex/peeter_prologue_print2.tex}
%
%\usepackage{peeters_layout_exercise}
%\usepackage{peeters_braket}
%\usepackage{peeters_figures}
%\usepackage{siunitx}
%%\usepackage{mhchem} % \ce{}
%%\usepackage{macros_bm} % \bcM
%\usepackage{macros_qed} % \qedmarker
%%\usepackage{txfonts} % \ointclockwise
%
%\newcommand{\dLambertian}[0]{\square}
%
%\newcommand{\stgrad}[0]{\lr{ \spacegrad + \inv{c} \PD{t}{}}}
%\newcommand{\conjstgrad}[0]{\lr{ \spacegrad - \inv{c} \PD{t}{}}}
%
%\beginArtNoToc
%
%\generatetitle{Maxwell's equation in media}
%\chapter{Maxwell's equation in media}
\label{chap:dielectric}

\subsection{Statement.}

Without imposing the constitutive relationships \cref{eqn:freespace:300} the geometric algebra form of Maxwell's equations requires
a pair of equations, multivector fields, and multivector sources, instead of one of each.
\maketheorem{Maxwell's equations in media.}{thm:dielectric:20}{
Maxwell's equations in media are
\begin{equation*}
\begin{aligned}
\gpgrade{ \stgrad F }{0,1} &= J_\txte \\
\gpgrade{ \stgrad G }{2,3} &= I J_\txtm,
\end{aligned}
\end{equation*}
where \( c \) is the group velocity of \( F, G \) in the medium,
the fields are grade 1,2 multivectors
\begin{equation*}
\begin{aligned}
F &= \BD + \frac{I}{c}\BH \\
G &= \BE + I c \BB,
\end{aligned}
\end{equation*}
and the sources are grade 0,1 multivectors
\begin{equation*}
\begin{aligned}
J_\txte &= \rho - \inv{c}\BJ \\
J_\txtm &= c \rho_\txtm - \BM.
\end{aligned}
\end{equation*}
} % theorem

To prove \cref{thm:dielectric:20} we may simply expand the spacetime gradients and grade selection operations,
and compare to
\cref{eqn:freespace:3399}, the conventional representation of Maxwell's equations.  For \( F \) we have
\begin{dmath}\label{eqn:dielectric:80}
\rho - \frac{\BJ}{c}
=
\gpgrade{
\stgrad F
}{0,1}
=
\gpgrade{
\stgrad \lr{ \BD + \frac{I}{c}\BH }
}{0,1}
=
\gpgrade{
\spacegrad \cdot \BD
+
\spacegrad \wedge \BD
+
\frac{I}{c} \spacegrad \cdot \BH
+
\frac{I}{c} \spacegrad \wedge \BH
+
\inv{c} \PD{t}{\BD}
+ I
\inv{c^2} \PD{t}{\BH}
}{0,1}
=
\spacegrad \cdot \BD
+
\inv{c} \PD{t}{\BD}
-\frac{1}{c} \spacegrad \cross \BH,
\end{dmath}
and for \( G \)
\begin{dmath}\label{eqn:dielectric:100}
I\lr{ c \rho_\txtm - \BM }
=
\gpgrade{ \stgrad G }{2,3}
=
\gpgrade{
   \stgrad \lr{\BE + I c \BB }
}{2,3}
=
\gpgrade{
   \spacegrad \cdot \BE
   +\spacegrad \wedge \BE
   + I c \spacegrad \cdot \BB
   + I c \spacegrad \wedge \BB
   + \inv{c} \PD{t}{\BE}
   + I \PD{t}{\BB}
}{2,3}
=
    \spacegrad \wedge \BE
   + I c \spacegrad \cdot \BB
   + I \PD{t}{\BB}
=
I \lr{
    \spacegrad \cross \BE
   + c \spacegrad \cdot \BB
   + \PD{t}{\BB}
}.
\end{dmath}
Applying further grade selection operations, rescaling (cancelling all factors of \( c \) and \( I \)), and a bit of rearranging, gives
\begin{dmath}\label{eqn:dielectric:120}
\begin{aligned}
\spacegrad \cdot \BD &= \rho \\
\spacegrad \cross \BH &= \BJ + \PD{t}{\BD} \\
\spacegrad \cdot \BB &= \rho_\txtm \\
\spacegrad \cross \BE &= -\BM -\PD{t}{\BB},
\end{aligned}
\end{dmath}
which are Maxwell's equations, completing the proof.

\makeproblem{Maxwell's equations in media.}{problem:dielectric:140}{
The proof above is somewhat unfriendly, as it works backwards from the answer.  No
motivation was given for why the particular multivector fields were chosen, nor
why grade selection operations were required.
To obtain some insight on why this works, prove
\cref{thm:dielectric:20}
from
\cref{eqn:freespace:300} directly as follows:
\begin{enumerate}
\item Eliminate cross products using \( \spacegrad \cross \Bf = I (\spacegrad \wedge \Bf) \).
\item Introduce a scalar constant \( c \) with dimensions of velocity and redimensionalize any time derivatives \( \PDi{t}{\Bf} = (1/c) \PDi{t}{(c \Bf)} \), so that \( [(1/c)\PDi{t}{}] = [\spacegrad] \).
\item If required, multiply each of Maxwell's equations by a factor of \( I \), to obtain a scalar and vector equation for \( \BD, \BH \), and a bivector and pseudoscalar equation for \( \BE, \BB \).
\item Sum the pairs of equations to form a multivector equation for each of \( \BD, \BH \) and \( \BE, \BB \).
\item Factor the terms in each equation into a product of the spacetime gradient and the respective fields \( F, G \), and show the result may be simplified by grade selection.
\end{enumerate}
} % problem

%\makeanswer{problem:dielectric:160}{
%To prove \cref{thm:dielectric:20} we start by eliminating the cross products from
%and grouping the fields
%\begin{dmath}\label{eqn:dielectric:20}
%\begin{aligned}
%\PD{t}{\BD} +I \spacegrad \wedge \BH &= -\BJ  \\
%\spacegrad \cdot \BD &= \rho \\
%I \PD{t}{\BB} + \spacegrad \wedge \BE &= -I \BM  \\
%I \spacegrad \cdot \BB &= I \rho_\txtm.
%\end{aligned}
%\end{dmath}
%There are time and divergence operations that we may assume should be grouped into a single spacetime gradient.  Let \( c \) be a constant with dimensions of velocity (to be determined), and then redimensionalize
%\begin{dmath}\label{eqn:dielectric:40}
%\begin{aligned}
%\inv{c} \PD{t}{\BD} +I \spacegrad \wedge \frac{\BH}{c} &= -\frac{\BJ}{c} \\
%\spacegrad \cdot \BD &= \rho \\
%\inv{c} \PD{t}{c I \BB} + \spacegrad \wedge \BE &= -I \BM  \\
%I \spacegrad \cdot (c \BB) &= I c \rho_\txtm.
%\end{aligned}
%\end{dmath}
%The dimensions of the pair of equations in \( (\BD,\BH) \) and \( \BE, \BB \) are now consistent.  The grades of these equations are respectively vector, scalar, bivector, and pseudoscalar, so the scalar-vector and bivector-pseudoscalar equations may be added without any loss of information.
%\begin{dmath}\label{eqn:dielectric:60}
%\begin{aligned}
%\spacegrad \cdot \BD + \inv{c} \PD{t}{\BD} + I \spacegrad \wedge \frac{\BH}{c} &= \rho -\frac{\BJ}{c} \\
%I \spacegrad \cdot (c \BB) + \inv{c} \PD{t}{c I \BB} + \spacegrad \wedge \BE &= I\lr{ c \rho_\txtm - \BM }.
%\end{aligned}
%\end{dmath}
%The right hand sides are the \( J_\txte, I J_\txtm \) sources defined above.  Let's expand the grade selections of
%} % answer

\subsection{Alternative form.}

\maketheorem{Grade selection free equations.}{thm:dielectric:200}{
Given multivector solutions \( F', G' \) to
\begin{equation*}
\begin{aligned}
J_\txte &= \stgrad F' \\
I J_\txtm &= \stgrad G',
\end{aligned}
\end{equation*}
these can be related to solutions \( F, G \) of Maxwell's equations given by \cref{thm:dielectric:20} by
\begin{equation*}
\begin{aligned}
F &= \gpgrade{F'}{1,2} \\
G &= \gpgrade{G'}{1,2},
\end{aligned}
\end{equation*}
if
\begin{equation*}
\begin{aligned}
\gpgrade{\stgrad \gpgradezero{F'}}{0,1} &= 0 \\
\gpgrade{\stgrad \gpgradethree{G'}}{2,3} &= 0.
\end{aligned}
\end{equation*}
} % theorem

To prove we select the grade 0,1 and grade 2,3 components from space time gradient equations of
\cref{thm:dielectric:200}.  For the electric sources, this gives
\begin{dmath}\label{eqn:dielectric:260}
J_\txte
= \gpgrade{\stgrad F'}{0,1}
= \gpgrade{\stgrad \gpgrade{F'}{1,2}}{0,1}
+ \gpgrade{\stgrad \gpgradezero{F'}}{0,1}
+ \gpgrade{\stgrad \gpgradethree{F'}}{0,1},
\end{dmath}
however \( \stgrad \gpgrade{F'}{3} \) has only grade 2,3 components, leaving just
\begin{dmath}\label{eqn:dielectric:280}
J_\txte
= \gpgrade{\stgrad \gpgrade{F'}{1,2}}{0,1}
+ \gpgrade{\stgrad \gpgradezero{F'}}{0,1},
\end{dmath}
as claimed.  For the magnetic sources, we have
\begin{dmath}\label{eqn:dielectric:300}
I J_\txtm
= \gpgrade{\stgrad G'}{2,3}
= \gpgrade{\stgrad \gpgrade{G'}{1,2}}{2,3}
+ \gpgrade{\stgrad \gpgradezero{G'}}{2,3}
+ \gpgrade{\stgrad \gpgradethree{G'}}{2,3},
\end{dmath}
however \( \stgrad \gpgrade{G'}{0} \) has only grade 0,1 components, leaving just
\begin{dmath}\label{eqn:dielectric:320}
I J_\txtm
= \gpgrade{\stgrad \gpgrade{G'}{1,2}}{2,3}
+ \gpgrade{\stgrad \gpgradezero{G'}}{2,3},
\end{dmath}
completing the proof.

\Cref{thm:dielectric:200} is probably a more effect geometric algebra form for solution of Maxwell's equations in matter, as the grade selection free spacetime gradients can be solved for \( F', G' \) directly using Green's function convolution.  However, we have an open question of how to impose a zero scalar grade constraint on \( F' \) and a zero pseudoscalar grade constraint on \( G' \).

\paragraph{Question:} Is the solution as simple as grade selection of the convolution?

\begin{dmath}\label{eqn:dielectric:200}
\begin{aligned}
F &= \int dt' dV' \gpgrade{G(\Bx - \Bx', t - t') J_\txte}{1,2} \\
G &= \int dt' dV' \gpgrade{G(\Bx - \Bx', t - t') I J_\txtm}{1,2},
\end{aligned}
\end{dmath}
where \( G(\Bx - \Bx', t - t') \),
is the Green's function for the space time gradient \cref{thm:greensFunctionSpacetimeGradient:120},
not to be confused with \( G = \BE + I c \BB \),

\subsection{Gauge like transformations.}

Because of the grade selection operations in \cref{thm:dielectric:20}, we cannot simply solve for \( F, G \) using the Green's function
for the spacetime gradient.  However, we may make a gauge-like transformation of the fields.  Additional exploration is required to determine if such transformations can be utilized to solve \cref{thm:dielectric:20}.
\maketheorem{Multivector transformation of the fields.}{thm:dielectric:180}{
If \( F, G \) are solutions to \cref{thm:dielectric:20}, then so are
\begin{equation*}
\begin{aligned}
F' &= F + \gpgrade{ \conjstgrad \Psi_{2,3} }{1,2} \\
G' &= G + \gpgrade{ \conjstgrad \Psi_{0,1} }{1,2},
\end{aligned}
\end{equation*}
where \( \Psi_{2,3} \) is any multivector with grades 2,3 and \( \Psi_{0,1} \) is any multivector with grades 0,1.
} % theorem

To prove \cref{thm:dielectric:180} we need to show that
\begin{dmath}\label{eqn:dielectric:160}
\begin{aligned}
\gpgrade{\stgrad F'}{0,1} &= \gpgrade{\stgrad F}{0,1} \\
\gpgrade{\stgrad G'}{2,3} &= \gpgrade{\stgrad G}{2,3}.
\end{aligned}
\end{dmath}
Let's start with \( F \)
\begin{dmath}\label{eqn:dielectric:140}
\begin{aligned}
\gpgrade{\stgrad F'}{0,1}
&=
\gpgrade{\stgrad F}{0,1}
+
\gpgrade{\stgrad \gpgrade{ \conjstgrad \Psi_{2,3} }{1,2} }{0,1} \\
&=
\gpgrade{\stgrad F}{0,1}
+
\gpgrade{ \dLambertian \Psi_{2,3} }{0,1} \\
&\quad -
\gpgrade{ \stgrad \gpgrade{ \conjstgrad \Psi_{2,3} }{0,3} }{0,1}.
\end{aligned}
\end{dmath}
The second term is killed since \( \Psi_{2,3} \) has no grade 0,1 components by definition, so neither does \( \dLambertian \Psi_{2,3} \).
To see that the last term is zero, note that \( \conjstgrad \Psi_{2,3} \) can have only grades 1,2,3, so \( \gpgrade{ \conjstgrad \Psi_{2,3} }{0,3} \) is a trivector.  This means that \( \stgrad \gpgrade{ \conjstgrad \Psi_{2,3} }{0,3} \) has only grades 2,3, which are obliterated by the final grade 0,1 selection operation, leaving just \( \gpgrade{\stgrad F}{0,1} \).
For \( G \) we have
\begin{dmath}\label{eqn:dielectric:180}
\begin{aligned}
\gpgrade{\stgrad G'}{2,3}
&=
\gpgrade{\stgrad G}{2,3}
+
\gpgrade{\stgrad \gpgrade{ \conjstgrad \Psi_{0,1} }{1,2} }{2,3} \\
&=
\gpgrade{\stgrad G}{2,3}
+
\gpgrade{ \dLambertian \Psi_{0,1} }{2,3} \\
&\quad -
\gpgrade{ \stgrad \gpgrade{ \conjstgrad \Psi_{0,1} }{0,3} }{2,3}.
\end{aligned}
\end{dmath}
As before the d'Lambertian term is killed as it has no grades 2,3.
To see that the last term is zero, note that \( \conjstgrad \Psi_{0,1} \) can have only grades 0,1,2, so \( \gpgrade{ \conjstgrad \Psi_{0,1} }{0,3} \) is a scalar.  This means that \( \stgrad \gpgrade{ \conjstgrad \Psi_{0,1} }{0,3} \) has only grades 0,1, which are obliterated by the final grade 2,3 selection operation, leaving \( \gpgrade{\stgrad G}{2,3} \), completing the proof.

An additional variation of \cref{thm:dielectric:180} is also possible.

\maketheorem{Multivector transformation of the fields.}{thm:dielectric:220}{
If \( F, G \) are solutions to \cref{thm:dielectric:20}, then so are
\begin{equation*}
\begin{aligned}
F' &= F + \conjstgrad \Psi_{2,3} \\
G' &= G + \conjstgrad \Psi_{0,1}
\end{aligned}
\end{equation*}
where \( \Psi_{2,3} \) is any multivector with grades 2,3 and \( \Psi_{0,1} \) is any multivector with grades 0,1.
} % theorem

\Cref{thm:dielectric:220} can be proven by direct substitution.  For \( F \)
\begin{dmath}\label{eqn:dielectric:220}
\gpgrade{ \stgrad \lr{ F + \conjstgrad \Psi_{2,3} } }{0,1}
=
\gpgrade{ \stgrad F + \dLambertian \Psi_{2,3} }{0,1}
=
\gpgrade{ \stgrad F },
\end{dmath}
and for \( G\)
\begin{dmath}\label{eqn:dielectric:240}
\gpgrade{ \stgrad \lr{ G + \conjstgrad \Psi_{0,1} } }{2,3}
=
\gpgrade{ \stgrad G + \dLambertian \Psi_{0,1} }{2,3}
=
\gpgrade{ \stgrad G },
\end{dmath}
which completes the proof.

%}
%\EndNoBibArticle
         \subsection{Boundary value conditions.}
%
% Copyright © 2017 Peeter Joot.  All Rights Reserved.
% Licenced as described in the file LICENSE under the root directory of this GIT repository.
%
%{
\maketheorem{Boundary value relations.}{thm:boundarySurfaceSources:480}{
The difference in the normal and tangential components of the electromagnetic field spanning a surface on which there are
a surface current or surface charge or current densities \( J_\txte = J_{\textrm{es}} \delta(n), J_\txtm = J_{\textrm{ms}} \delta(n) \)
can be related to those surface sources as follows
%\label{eqn:boundarySurfaceSources:420}
\begin{equation*}
\begin{aligned}
\gpgrade{\ncap (F_2 - F_1) }{0,1} &= J_{\textrm{es}} \\
\gpgrade{\ncap (G_2 - G_1) }{2,3} &= I J_{\textrm{ms}},
\end{aligned}
\end{equation*}
where \( F_k = \BD_k + I \BH_k/c, G_k = \BE_k + I c \BB_k, k = 1,2 \) are the fields in the
where \( \ncap = \ncap_2 = -\ncap_1 \) is the outwards facing normal in the second medium.
In terms of the conventional constituent fields, these may be written
%\label{eqn:boundarySurfaceSources:460}
\begin{equation*}
\begin{aligned}
\ncap \cdot \lr{ \BD_2 - \BD_1 } &= \rho_\txts \\
\ncap \cross \lr{ \BH_2 - \BH_1 } &= \BJ_\txts \\
\ncap \cdot \lr{ \BB_2 - \BB_1 } &= \rho_{\textrm{ms}} \\
\ncap \cross \lr{ \BE_2 - \BE_1 } &= -\BM_\txts.
\end{aligned}
\end{equation*}
} % theorem

\Cref{fig:ps3Problem1Pillbox:ps3Problem1PillboxFig1} illustrates a surface where we seek to find the fields above the surface (region 2), and below the surface (region 1).
These fields will be determined by integrating Maxwell's equation over the pillbox configuration, allowing the height \( n \) of that pillbox above or below the surface to tend to zero,
and the area of the pillbox top to also tend to zero.

%\imageFigure{../figures/ece1228-electromagnetic-theory/ps3Problem1PillboxFig1}{Pillbox integration volume.}{fig:ps3Problem1Pillbox:ps3Problem1PillboxFig1}{0.2}
\imageFigure{../figures/GAelectrodynamics/pillboxIntegrationVolumeFig1}{Pillbox integration volume.}{fig:ps3Problem1Pillbox:ps3Problem1PillboxFig1}{0.3}

We will work with \cref{thm:dielectric:20}, Maxwell's equations in media, in their frequency domain form
\begin{dmath}\label{eqn:boundarySurfaceSources:480}
\begin{aligned}
\gpgrade{ \spacegrad F }{0,1} + j k \BD &= J_{\textrm{es}} \delta(n) \\
\gpgrade{ \spacegrad G }{2,3} + j k I c \BB &= I J_{\textrm{ms}} \delta(n),
\end{aligned}
\end{dmath}
and integrate these over the pillbox volume in the figure.  That is
\begin{dmath}\label{eqn:boundarySurfaceSources:500}
\begin{aligned}
\int dV \gpgrade{ \spacegrad F }{0,1} + j k \int dV \BD &= \int dn dA J_{\textrm{es}} \delta(n) \\
\int dV \gpgrade{ \spacegrad G }{2,3} + j k I c \int dV \BB &= I \int dn dA J_{\textrm{ms}} \delta(n).
\end{aligned}
\end{dmath}
The gradient integrals can be evaluated with \cref{thm:volumeintegral:100}.  Evaluating the delta functions picks leaves an area integral on the surface.  Additionally, we assume that we are making the pillbox volume small enough that we can employ the mean value theorem for the \( \BD, \BB \) integrals
\begin{dmath}\label{eqn:boundarySurfaceSources:520}
\begin{aligned}
\int_{\partial V} dA \gpgrade{ \ncap F }{0,1} + j k \Delta A \lr{ n_1 \tilde{\BD}_1 + n_2 \tilde{\BD}_2 } &= \Delta A J_{\textrm{es}} \\
\int_{\partial V} dA \gpgrade{ \ncap G }{2,3} + j k I c \Delta A \lr{ n_1 \tilde{\BB}_1 + n_2 \tilde{\BB}_2} &= I \Delta A J_{\textrm{ms}}.
\end{aligned}
\end{dmath}
We now let \( n_1, n_2 \) tend to zero, which kills off the \( \BD, \BB \) contributions, and also kills off the side wall contributions in the first pillbox surface integral.  This leaves
\begin{dmath}\label{eqn:boundarySurfaceSources:540}
\begin{aligned}
\gpgrade{ \ncap_2 F_2 }{0,1} + \gpgrade{ \ncap_1 F_1 }{0,1} &= J_{\textrm{es}} \\
\gpgrade{ \ncap_2 G_2 }{2,3} + \gpgrade{ \ncap_1 G_1 }{2,3} &= J_{\textrm{ms}}.
\end{aligned}
\end{dmath}
Inserting \( \ncap = \ncap_2 = -\ncap_1 \) completes the first part of the proof.

Expanding the grade selection operations, we find
\begin{dmath}\label{eqn:boundarySurfaceSources:440}
\begin{aligned}
\ncap \cdot (\BD_2 - \BD_1) &= \rho_s \\
I \ncap \wedge \lr{ \BH_2/c - \BH_1/c } &= - \BJ_s/c \\
\ncap \wedge (\BE_2 - \BE_1) &= -I \BM_s \\
I c \ncap \cdot (\BB_2 - \BB_1) &= I c \rho_{ms},
\end{aligned}
\end{dmath}
and expansion of the wedge's as cross's using \cref{eqn:SimpleProducts2:1620} completes the proof.
%It is somewhat remarkable that the
%crazy jumble of dot products, cross products and field components in the conventional statement of the boundary conditions, follows directly from the evaluation of the product of the normal with the multivector fields.

In the special case where there are surface charge and current densities along the interface surface, but the media is uniform (\(\epsilon_1 = \epsilon_2, \mu_1 = \mu_2\)), then the field and current relationship has a particularily simple form \citep{chappell2014geometric}
\begin{dmath}\label{eqn:boundarySurfaceSources:421}
\ncap (F_2 - F_1) = J_s.
\end{dmath}

\makeproblem{Uniform media with currents and densities.}{problem:boundarySurfaceSources:1}{
Prove that \cref{eqn:boundarySurfaceSources:421} holds when \( \epsilon_1 = \epsilon_2, \mu_1 = \mu_2 \).
} % problem
%}
%   \section{Radiation and scattering}
%TODO.
   %   \subsection{Problem solutions}
   %      \shipoutAnswer

%}
\EndArticle
