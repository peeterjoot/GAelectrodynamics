%
% Copyright � 2016 Peeter Joot.  All Rights Reserved.
% Licenced as described in the file LICENSE under the root directory of this GIT repository.
%
%----------------------------------------------------------------------------------------
\part{Geometric Algebra}
   \chapter{Basics}
      \section{Did you ever ask your teacher how to multiply vectors?}
         \input{GAmotivation.tex}
         \subsection{Problems}
            \input{ComplexInnerProductVsDotAndCrossProduct.tex}
      \section{Vector multiplication}
         %
% Copyright � 2016 Peeter Joot.  All Rights Reserved.
% Licenced as described in the file LICENSE under the root directory of this GIT repository.
%
%{

%\footnote{Similar to Feynman on gravitation \citep{feynman1963flp} ``... have shall said everything required, for a sufficiently talented mathematician could then deduce all the consequences of these principles.  However, since you are not assumed to be sufficiently talented yet, we shall discuss the consequences in more detail''.}.

The contraction axiom is arguably the most important of the multivector space axioms, but may be unintuitive.

For one justification of this rule, consider a one dimensional vector space spanned by a single unit vector \( \setlr{ \Be } \).  That span, for real \( x \) is all the values

\begin{dmath}\label{eqn:multiplication:20}
\Bx = x \Be.
\end{dmath}

A one dimensional vector space is isomorphic with a number line \( \setlr{x}, x \in \bbR \), so can we use the rules for numeric multiplication to justify a rule for vector multiplication in a one dimensional vector space?

Consider, for example, two vectors \( -3\Be \) and \( 7 \Be \), in \R{1} and in the number line space as plotted in
\cref{fig:1Darrows:1DarrowsFig2}.
\imageTwoFigures
{../figures/GAelectrodynamics/1DarrowsFig2}
{../figures/GAelectrodynamics/1DnumberlineFig1}
{Equivalent vectors in \R{1} and on a number line.}{fig:1Darrows:1DarrowsFig2}{scale=0.5}
%\imageFigure{../figures/GAelectrodynamics/1DarrowsFig2}{Vectors in 1D space.}{fig:1Darrows:1DarrowsFig2}{0.03}
%\imageFigure{../figures/GAelectrodynamics/1DnumberlineFig1}{Points on a number line.}{fig:1Dnumberline:1DnumberlineFig1}{0.045}
%\cref{fig:1Dnumberline:1DnumberlineFig1}.

The multiplication rules for real numbers require that for any point \( x \) distant from the origin, we have

\begin{equation}\label{eqn:multiplication:60}
(\pm x)^2 = \Abs{x}^2 = x^2.
\end{equation}

Requiring this of the equivalent one dimensional vector space requires

\begin{equation}\label{eqn:multiplication:40}
(\pm \Bx)^2 = \Abs{x}^2 = x^2,
\end{equation}

which is a statement of the contraction axiom for a one dimensional Euclidean vector space.
In this sense the contraction axiom is just taking the rule for real multiplication, and applying it to vector spaces.

%%}
%%%\EndArticle
%%\EndNoBibArticle

         \subsection{Problems}
            \input{2dvectorsquare.tex}
            \input{normalAnticommutation.tex}
      \section{Definitions}
         %
% Copyright � 2016 Peeter Joot.  All Rights Reserved.
% Licenced as described in the file LICENSE under the root directory of this GIT repository.
%
A few new GA terms have been introduced in an ad-hoc fashion as required.  Here is a systematic exposition of some of the key definitions used to refer to the types of the geometric objects that will be encountered.

The grade of a scalar, vector, bivector, and trivector are 0, 1, 2, and 3 respectively.
(cut)

         \subsection{Problems}
            %
% Copyright © 2016 Peeter Joot.  All Rights Reserved.
% Licenced as described in the file LICENSE under the root directory of this GIT repository.
%

As with the use of the symbol \( i \) for the \R{2} pseudoscalar, it is not a coincidence that \( I \) was used 
for the 
\R{3} pseudoscalar.  It is also true that 
\( I = \Be_1 \Be_2 \Be_3 \) behaves like a complex imaginary with \( I^2 = -1 \).  This follows 
directly from repeated anticommutation

\begin{dmath}\label{eqn:projectionAndRejection:1140}
I^2
=
(\Be_1 \Be_2 \Be_3)(\Be_1 \Be_2 \Be_3)
=
\Be_1 \Be_2 (\Be_3 \Be_1) \Be_2 \Be_3
=
\Be_1 \Be_2 (-\Be_1 \Be_3) \Be_2 \Be_3
=
-\Be_1 \Be_2 \Be_1 (\Be_3 \Be_2) \Be_3
=
-\Be_1 \Be_2 \Be_1 (-\Be_2 \Be_3) \Be_3
=
+\Be_1 \Be_2 \Be_1 \Be_2 (\Be_3 \Be_3)
=
(\Be_1 \Be_2)^2
=
-1.
\end{dmath}

%\makeproblem{\R{3} pseudoscalar square}{problem:gradeselection:R3PseudoscalarSquare}{
%With the \R{3} pseudoscalar of \cref{eqn:definitions:340} show that \( I^2 = -1 \).
%} % problem
%
%\makeanswer{problem:gradeselection:R3PseudoscalarSquare}{
%
%\begin{dmath}\label{eqn:gaTutorial:160}
%I^2
%=
%...
%=
%-1,
%\end{dmath}
%
%as expected, showing that this quantity also has characteristics of an imaginary number.
%} % answer

      \section{Grade selection, dot and wedge product operators}
         %
% Copyright © 2017 Peeter Joot.  All Rights Reserved.
% Licenced as described in the file LICENSE under the root directory of this GIT repository.
%
The workhorse operator of geometric algebra is called grade selection, defined as
\index{grade selection}
\index{\(\gpgrade{M}{k}\)}
\index{\(\gpgradezero{M}\)}
\makedefinition{Grade selection operator}{dfn:gradeselection:gradeselection}{
Given a set of k-vectors \( M_k, k \in [0,N] \), and any multivector of their sum
\begin{equation*}
M = \sum_{i = 0}^N M_i,
\end{equation*}
the grade selection operator is defined as
\begin{equation*}\label{eqn:multivector_nomenclature:40}
\gpgrade{M}{k} = M_k.
\end{equation*}
Due to its importance, selection of the (scalar) zero grade is given the shorthand
\begin{equation*}
\gpgradezero{M} = \gpgrade{M}{0} = M_0.
\end{equation*}
}

To illustrate grade selection by example, given a multivector \( M = 3 - \Be_3 + 2 \Be_1 \Be_2 + 7 \Be_1 \Be_2 \Be_4 \), then
\begin{equation}\label{eqn:multivector_nomenclature:80}
\begin{aligned}
\gpgrade{M}{0} &= \gpgradezero{M} = 3 \\
\gpgrade{M}{1} &= - \Be_3 \\
\gpgrade{M}{2} &= 2 \Be_1 \Be_2 \\
\gpgrade{M}{3} &= 7 \Be_1 \Be_2 \Be_4.
\end{aligned}
\end{equation}

Grade selection is the fundamental operation of geometric algebra.  Grade selection will be used directly as a tool, and will
also be used to define a number of other auxillary operators,
including a 
generalized multivector dot product, and the wedge product which is related to the \R{3} cross product by a multivector constant, and shares some properties of the cross product in other dimensions.

         \subsection{Problems}
            \input{RnDotProduct.tex}
            \input{cyclicpermutationtwo.tex}
            \input{dotprodSymmetricSum.tex}
            \input{planeRotationsExponentials.tex}
            \input{complexNumbers.tex}
            \input{R3PseudoscalarCommutation.tex}
            \input{gradeselVectorWedge.tex}
            \input{WedgeRelationshipToCrossProduct.tex}
            \input{vectorBivectorDot.tex}
            %
% Copyright © 2016 Peeter Joot.  All Rights Reserved.
% Licenced as described in the file LICENSE under the root directory of this GIT repository.
%

%\makeproblem{Vector trivector dot product}{problem:gradeselection:vectorTrivectorDot}{

\maketheorem{Vector-trivector dot product.}{thm:vectorTrivectorDot:vectorTrivectorDot}{
Given a vector \( \Ba \) and a blade \( \Bb \wedge \Bc \wedge \Bd \) formed by wedging three vectors, the dot product of the two can be expanded as bivectors like

\begin{dmath}\label{eqn:vectorTrivectorDot:20}
\Ba \cdot \lr{ \Bb \wedge \Bc \wedge \Bd}
=
\lr{ \Bb \wedge \Bc \wedge \Bd} \cdot \Ba
=
( \Ba \cdot \Bb ) (\Bc \wedge \Bd)
-( \Ba \cdot \Bc ) (\Bb \wedge \Bd)
+( \Ba \cdot \Bd ) (\Bb \wedge \Bc).
\end{dmath}
} % theorem

%\makeanswer{problem:gradeselection:vectorTrivectorDot}{
The proof follows by expansion in coordinates

\begin{dmath}\label{eqn:vectorTrivectorDot:40}
\Ba \cdot \lr{ \Bb \wedge \Bc \wedge \Bd}
= \sum_{j \ne k \ne l} a_i b_j c_k d_l
\gpgradetwo{ \Be_i \Be_j \Be_k \Be_l }.
\end{dmath}

The products within the grade two selection operator can be of either grade two or grade four, so only the terms where one of
\( i = j \), \( i = k \), or \( i = l \) contributes.  Repeated anticommutation of the normal unit vectors can put each such pair adjacent, where they square to unity.  Those are respectively

\begin{dmath}\label{eqn:vectorTrivectorDot:60}
\begin{aligned}
\gpgradetwo{ \Be_i \Be_i \Be_k \Be_l } &= \Be_k \Be_l  \\
\gpgradetwo{ \Be_i \Be_j \Be_i \Be_l } &= -\gpgradetwo{ \Be_j \Be_i \Be_i \Be_l } = - \Be_j \Be_l \\
\gpgradetwo{ \Be_i \Be_j \Be_k \Be_i } &= -\gpgradetwo{ \Be_j \Be_i \Be_k \Be_i } = +\gpgradetwo{ \Be_j \Be_k \Be_i \Be_i } = \Be_j \Be_k
\end{aligned}
\end{dmath}

Substitution back into \cref{eqn:gradeselectionProblems:681} gives

\begin{dmath}\label{eqn:vectorTrivectorDot:80}
\Ba \cdot \lr{ \Bb \wedge \Bc \wedge \Bd}
= \sum_{j \ne k \ne l} a_i b_j c_k d_l
\lr{
\Be_i \cdot \Be_j (\Be_k \Be_l)
-
\Be_i \cdot \Be_k (\Be_j \Be_l)
+
\Be_i \cdot \Be_l (\Be_j \Be_k)
}
=
( \Ba \cdot \Bb ) (\Bc \wedge \Bd)
-( \Ba \cdot \Bc ) (\Bb \wedge \Bd)
+( \Ba \cdot \Bd ) (\Bb \wedge \Bc).
\end{dmath}

%Repeating this from the other direction gives the same result.
%} % answer

\Cref{thm:vectorTrivectorDot:vectorTrivectorDot} is 
a specific case of the more general identity

\maketheorem{Vector blade dot product distribution.}{thm:vectorTrivectorDot:dotblade}{
A vector dotted with a \( n-blade \) distributes as

%\begin{dmath}\label{eqn:vectorTrivectorDot:100}
%\boxedEquation{eqn:vectorTrivectorDot:100}{
\begin{equation*}
\Bx \cdot \lr{ \By_1 \wedge \By_2 \wedge \cdots \wedge \By_n }
=
\sum_{i = 1}^n (-1)^i (\Bx \cdot \By_i) \lr{ \By_1 \wedge \cdots \wedge \By_{i-1} \wedge \By_{i+1} \wedge \cdots \wedge \By_n }.
\end{equation*}
%}
%\end{dmath}

This dot product is symmetric(antisymmetric) when the grade of the blade the vector is dotted with is odd(even).
} % theorem

For a proof of \cref{thm:vectorTrivectorDot:dotblade} (valid for all metrics) see
\citep{doran2003gap}.
%} % problem


            %
% Copyright © 2016 Peeter Joot.  All Rights Reserved.
% Licenced as described in the file LICENSE under the root directory of this GIT repository.
%
\makeproblem{Dot product expansion of two bivectors.}{problem:multiplication:bivectorDot}{
Show that

\boxedEquation{eqn:bivectorDot:20}{
(\Ba \wedge \Bb) \cdot (\Bc \wedge \Bd)
=
\lr{ (\Ba \wedge \Bb) \cdot \Bc} \cdot \Bd,
}

and hence
\boxedEquation{eqn:bivectorDot:40}{
(\Ba \wedge \Bb) \cdot (\Bc \wedge \Bd)
=
(\Bb \cdot \Bc) (\Ba \cdot \Bd)
-(\Ba \cdot \Bc)( \Bb \cdot \Bd).
}
} % problem

\makeanswer{problem:multiplication:bivectorDot}{
FIXME: this proof relies on triple wedge products, which haven't been introduced yet.  Can rephrase using grade-3 selection.
Also rework as non-problem.
\begin{dmath}\label{eqn:bivectorDot:60}
(\Ba \wedge \Bb) \cdot (\Bc \wedge \Bd)
=
\gpgradezero{
(\Ba \wedge \Bb) (\Bc \wedge \Bd)
}
=
\gpgradezero{
(\Ba \wedge \Bb) (\Bc \Bd - \cancel{\Bc \cdot \Bd})
}
=
\gpgradezero{
\lr{
(\Ba \wedge \Bb) \cdot \Bc
+ \cancel{(\Ba \wedge \Bb) \wedge \Bc }
}
\Bd
}
=
\gpgradezero{
((\Ba \wedge \Bb) \cdot \Bc ) \cdot \Bd
+
\cancel{((\Ba \wedge \Bb) \cdot \Bc ) \wedge \Bd}
}
=
((\Ba \wedge \Bb) \cdot \Bc ) \cdot \Bd.
\end{dmath}

Above, any product that could not possibly contribute a scalar grade has been cancelled.  The remains are now straightforward to expand

\begin{dmath}\label{eqn:bivectorDot:80}
((\Ba \wedge \Bb) \cdot \Bc ) \cdot \Bd
=
(
\Ba (\Bb \cdot \Bc)
-
\Bb (\Ba \cdot \Bc)
)
\cdot \Bd
=
(\Ba \cdot \Bd) (\Bb \cdot \Bc)
-
(\Bb \cdot \Bd) (\Ba \cdot \Bc).
\end{dmath}

} % answer

            %
% Copyright © 2016 Peeter Joot.  All Rights Reserved.
% Licenced as described in the file LICENSE under the root directory of this GIT repository.
%

\makeproblem{\R{4} wedge of a non-blade with itself.}{problem:gradeselection:r4nonzerobivectorwedgewithself}{
While the wedge product of a blade with itself is always zero, this is not generally true of the wedge products of arbitrary k-vectors in higher dimensional spaces.
To demonstrate this, show that the wedge of the bivector
\( B = \Be_1 \Be_2 + \Be_3 \Be_4 \) with itself is non-zero.
Why is this bivector not a blade?
%, show that \( B \wedge B \ne 0 \).
} % problem

            %
% Copyright � CCYY Peeter Joot.  All Rights Reserved.
% Licenced as described in the file LICENSE under the root directory of this GIT repository.
%
\makeproblem{Cyclic permutation within scalar selection.}{problem:scalarPermutation:1}{
Show that, for multivectors \( A \), \( B \) a cyclic permutation of the multivectors within a grade zero selection is possible

\begin{equation}\label{eqn:scalarPermutation:1}
\gpgradezero{A B C}
=
\gpgradezero{B C A}.
\end{equation}
} % problem

\makeanswer{problem:scalarPermutation:1}{

It is sufficient to show that

\begin{equation}\label{eqn:scalarPermutation:21}
\gpgradezero{A B}
=
\gpgradezero{B A}.
\end{equation}

If the maximum grade of \( A \) and \( B \) are \( a \) and \( b \) respectively, then

\begin{equation}\label{eqn:scalarPermutation:41}
\gpgradezero{A B}
=
\sum_{r = 0}^a \sum_{s = 0}^b \gpgradezero{A_r B_s}
=
\sum_{k = 0}^{\min(a,b)} \gpgradezero{A_k B_k}
\end{equation}

Because \( \gpgradezero{M}^\dagger = \gpgradezero{M^\dagger} = \gpgradezero{M} \), reversing all the factors in each of these grade zero selections leaves the result unchanged.  That is

\begin{equation}\label{eqn:scalarPermutation:61}
\gpgradezero{A B}
=
\sum_{k = 0}^{\min(a,b)} \gpgradezero{B_k^\dagger A_k^\dagger}.
\end{equation}

Using \cref{eqn:scalarPermutation:101}, this is

\begin{dmath}\label{eqn:scalarPermutation:121}
\gpgradezero{A B}
=
\sum_{k = 0}^{\min(a,b)} \lr{(-1)^{k(k-1)/2} }^2 \gpgradezero{B_k A_k}
=
\sum_{k = 0}^{\min(a,b)} \gpgradezero{B_k A_k}
=
\gpgradezero{ B A }. \qedmarker
\end{dmath}

} % answer

      \section{Product of two vectors}
         %
% Copyright � 2016 Peeter Joot.  All Rights Reserved.
% Licenced as described in the file LICENSE under the root directory of this GIT repository.
%
%{
%\input{../blogpost.tex}
%\renewcommand{\basename}{vectorproduct}
%%\renewcommand{\dirname}{notes/phy1520/}
%\renewcommand{\dirname}{notes/ece1228-electromagnetic-theory/}
%%\newcommand{\dateintitle}{}
%%\newcommand{\keywords}{}
%
%\input{../peeter_prologue_print2.tex}
%
%\usepackage{peeters_layout_exercise}
%\usepackage{peeters_braket}
%\usepackage{peeters_figures}
%\usepackage{siunitx}
%%\usepackage{mhchem} % \ce{}
%%\usepackage{macros_bm} % \bcM
%%\usepackage{macros_qed} % \qedmarker
%%\usepackage{txfonts} % \ointclockwise
%
%\beginArtNoToc
%
%\generatetitle{XXX}
%%\chapter{XXX}
%%\label{chap:vectorproduct}
%
Given two vectors \( \Bx, \By \) the scalar grade of the vector product \( \Bx \By \) was shown (\cref{problem:gradeselection:RnDotProduct}) to be
\begin{equation}\label{eqn:vectorproduct:20}
\gpgradezero{ \Bx \By }
=
\sum_{i = 1}^N x_i y_i
=
\Bx \cdot \By.
\end{equation}

The grade two selection of this product was found (\cref{problem:gradeselection:vectorwedge}) to be

\begin{equation}\label{eqn:vectorproduct:40}
\gpgradetwo{ \Bx \By }
=
\sum_{i < j}
%(x_i y_j - x_j y_i)
\begin{vmatrix}
x_i & x_j \\
y_i & y_j
\end{vmatrix}
\Be_i \Be_j
=
\Bx \wedge \By
=
-\By \wedge \Bx.
\end{equation}

The reader should convince themself that the vector product \( \Bx \By \) has only even grades (0,2), and can therefore be expanded as

\begin{dmath}\label{eqn:vectorproduct:60}
\Bx \By
=
\gpgradezero{ \Bx \By }
+
\gpgradetwo{ \Bx \By },
\end{dmath}

or
\boxedEquation{eqn:vectorproduct:80}{
\Bx \By
=
\Bx \cdot \By
+
\Bx \wedge \By.
}

This is a fundamental and very useful relationship.  In these notes this is a consequence of the axioms and the generalized definitions of the dot and wedge products.  Some authors will use this to define the geometric product of two vectors.

Using \cref{problem:gradeselection:dotprod} and \cref{eqn:vectorproduct:80} it can be shown that the wedge product is an explicit antisymmetrized sum of vector products, just as the dot product is the symmetrized vector product sum

\boxedEquation{eqn:vectorproduct:300}{
\begin{aligned}
\Bx \cdot \By &= \inv{2} \lr{ \Bx \By + \By \Bx } \\
\Bx \wedge \By &= \inv{2} \lr{ \Bx \By - \By \Bx }
\end{aligned}
}

Some authors will use these as the respective definitions of the dot and wedge products.

The non-commutative nature of the vector product was one of the first observed consequences of the axioms.  The vector product is also not generally anticommutative, as was the case for normal vectors.  Rearranging \cref{eqn:vectorproduct:300} provides the general commutation identity for two vectors

%\begin{dmath}\label{eqn:vectorproduct:320}
\boxedEquation{eqn:vectorproduct:320}{
\By \Bx = 2 \Bx \cdot \By - \Bx \By.
}
%\end{dmath}

Observe that when the vectors are perpendicular, the strict anticommutation result follows.
This can be a handy tool for abstract multivector expression manipulation.

An additional, and incredibly useful, relationship follows from \cref{eqn:vectorproduct:80} for \R{3} (\cref{problem:gradeselection:WedgeRelationshipToCrossProduct})

\boxedEquation{eqn:vectorproduct:100}{
\Bx \By
=
\Bx \cdot \By
+
I
(\Bx \cross \By).
}

This is the GA equivalent of the Pauli relationship \cref{eqn:GAmotivation:120} that will be familiar to a student of quantum spin states.

The ability to combine dot and cross product relationships into a single multivector equation is not just a theoretical nicety.  This is also one of the primary reasons that GA is so applicable to the study of electromagnetism.   To illustrate this, and provide a hint of things to come, consider the GA formulation of the electrostatic and magnetostatic Maxwell equations.

\makeexample{Electrostatic and magnetostatics.}{example:vectorproduct:electrostatics}{

With no magnetic current, no magnetic sources, and no time derivatives, Maxwell's equations in simple media take the form

\begin{dmath}\label{eqn:vectorproduct:120}
\begin{aligned}
\spacegrad \cdot \BB &= 0 \\
\spacegrad \cross \BB &= \mu \BJ \\
\spacegrad \cross \BE &= 0 \\
\spacegrad \cdot \BE &= \frac{\rho}{\epsilon}.
\end{aligned}
\end{dmath}

For electrostatic conditions \( \BJ = 0 \), so using \cref{eqn:vectorproduct:100} the first and last equations can be combined into a single first order homogeneous multivector gradient equation

\begin{equation}\label{eqn:vectorproduct:140}
\spacegrad \BB
=
\spacegrad \cdot \BB +I (\spacegrad \cross \BB )
=
0.
\end{equation}

The electric gradient equation is

\begin{equation}\label{eqn:vectorproduct:160}
\spacegrad \BE
=
\spacegrad \cdot \BE +I (\spacegrad \cross \BE )
=
\frac{\rho}{\epsilon}.
\end{equation}

Maxwell's equations are reduced to two multivector equations with this transformation
\begin{dmath}\label{eqn:vectorproduct:180}
\begin{aligned}
\spacegrad \BE &= \frac{\rho}{\epsilon} \\
\spacegrad \BB &= 0.
\end{aligned}
\end{dmath}

For magnetostatics \( \rho = 0 \), and the same assembly of Maxwell's equations gives

\begin{dmath}\label{eqn:vectorproduct:220}
\begin{aligned}
\spacegrad \BB &= I \mu \BJ \\
\spacegrad \BE &= 0.
\end{aligned}
\end{dmath}

It will be seen later that it is actually more natural to express magnetic fields as a bivector \( I \BB \).  Using \( I^2 = -1 \) (\cref{problem:gradeselection:R3PseudoscalarSquare}) the magnetostatic equation takes the form

\begin{dmath}\label{eqn:vectorproduct:240}
\spacegrad (I \BB) = - \mu \BJ.
\end{dmath}

Both the electrostatic and magnetostatic equations can be solved directly using the Green's function for the gradient, producing the Coulomb integral for the electric field and Biot-Savart's law for the magnetic field.
Before demonstrating this, the concepts required to attack multivector integrals must be formulated.
} % example

The dot plus wedge product components of the vector product have a geometrical interpretation.  To understand this, consider the components of a vector \( \Bb \) onto the direction of \( \Ba \) and the perpendicular.  The projection component is

\begin{dmath}\label{eqn:vectorproduct:420}
\Proj_\Ba \Bb = \acap \lr{ \acap \cdot \Bb },
\end{dmath}

and the rejection (the component of \( \Bb \) perpendicular to \( \Ba \)), is
\begin{dmath}\label{eqn:vectorproduct:440}
\RejName_\Ba \Bb
=
\Bb - \acap \lr{ \acap \cdot \Bb }
=
\Abs{\Bb} \lr{ \bcap - \acap \lr{ \acap \cdot \Bb } }
=
\Abs{\Bb} \acap \lr{ \acap \bcap - \acap \cdot \Bb }
=
\Abs{\Bb} \acap \lr{ \acap \wedge \bcap }
=
\acap \lr{ \acap \wedge \Bb }.
\end{dmath}

An example is plotted in \cref{fig:projectionAndRejection:projectionAndRejectionFig1}.

\imageFigure{../figures/projectionAndRejectionFig1}{Projection and rejection illustrated.}{fig:projectionAndRejection:projectionAndRejectionFig1}{0.4}

The magnitudes of \( \acap \lr{ \acap \cdot \bcap } \), and \( \acap \lr{ \acap \wedge \bcap } \) are neccessarily the cosine and sines of the angle between \( \Ba \) and \( \Bb \), regardless of the dimension of the underlying vector space.  Those respective magnitudes are

\begin{dmath}\label{eqn:vectorproduct:460}
\begin{aligned}
\Abs{ \acap \lr{ \acap \cdot \bcap } }^2 &= \lr{ \acap \cdot \bcap }^2 \\
\Abs{ \acap \lr{ \acap \wedge \bcap } }^2 &= -\lr{ \acap \wedge \bcap }^2,
\end{aligned}
\end{dmath}

which allows an identification
\begin{dmath}\label{eqn:vectorproduct:500}
\begin{aligned}
\cos\theta &= \acap \cdot \bcap \\
\sin\theta &= \Abs{\acap \wedge \bcap},
\end{aligned}
\end{dmath}

where \( \Abs{\acap \wedge \bcap} = \sqrt{ -\lr{\acap \wedge \bcap}^2 } \).

It is now possible to express the product of vectors in a trigonometric or exponential form

\begin{dmath}\label{eqn:vectorproduct:480}
\acap \bcap
= \acap \cdot \bcap
+ \acap \wedge \bcap
=
\acap \cdot \bcap
+ \frac{\acap \wedge \bcap}{\Abs{\acap \wedge \bcap}} \Abs{\acap \wedge \bcap}
=
\cos\theta
+ \frac{\acap \wedge \bcap}{\Abs{\acap \wedge \bcap}} \sin\theta,
\end{dmath}

or

\boxedEquation{eqn:vectorproduct:520}{
\Ba \Bb
=
\Abs{\Ba}
\Abs{\Bb}
\exp\lr{ \frac{\acap \wedge \bcap}{\Abs{\acap \wedge \bcap}} \theta }.
}

The interpretation of this is that the product of two vectors produces a rotation operator that acts in the plane spanned by these vectors, but also scales any such rotated vector from this plane by the product of the magnitudes of the vector product factors.  When those vectors are unit vectors, the vector product is a non-scaling rotation operator

\begin{dmath}\label{eqn:vectorproduct:560}
\acap \bcap
=
\exp\lr{ \frac{\acap \wedge \bcap}{\Abs{\acap \wedge \bcap}} \theta },
\end{dmath}

that (when applied from the right) rotates any vector in \( \Span{\acap, \bcap} \) by \( \theta \) radians in the direction of shortest rotation from \( \acap \) to \( \bcap \), and when applied from the left rotates by \( -\theta \).

In particular, if the unit vectors are perpendicular, the rotation operator is
\begin{dmath}\label{eqn:vectorproduct:580}
R(\theta)
=
\exp\lr{ \acap \bcap \theta }.
\end{dmath}

For \R{3} the wedge product in \cref{eqn:vectorproduct:520} can be expressed as a cross product

\begin{equation}\label{eqn:vectorproduct:540}
\frac{\acap \wedge \bcap}{\Abs{\acap \wedge \bcap}}
=
I \frac{\acap \cross \bcap}{\Abs{\acap \cross \bcap}}
=
I \ncap,
\end{equation}

This allows the \R{3} vector product to be written as

\begin{equation}\label{eqn:vectorproduct:n}
\Bx \By
=
\Abs{\Bx}
\Abs{\By}
\exp\lr{ I \ncap \theta }.
\end{equation}

In this form it is particularly easy to verify that the factor \( I \ncap \),
the dual of the normal representing the plane of rotation from \( \Bx \) to \( \By \), acts as an imaginary

\begin{dmath}\label{eqn:vectorproduct:400}
(I \ncap)^2
=
(I \ncap) (I \ncap)
=
I^2 \ncap^2
=
(-1)(1)
=
-1.
\end{dmath}

Observe the similarity between this and the complex inner product \( z w^\conj = r \rho e^{i(\theta-\alpha)} \) for the complex numbers of \cref{eqn:GAmotivation:200}.  The primary difference is that the GA imaginary factor also has a spatial orientation that the complex imaginary does not.

%}
%\EndNoBibArticle

         \subsection{Miscellanious theorems}
            %
% Copyright © 2016 Peeter Joot.  All Rights Reserved.
% Licenced as described in the file LICENSE under the root directory of this GIT repository.
%

\maketheorem{K-vector dot and wedge product relations.}{thm:bladeDotWedgeSymmetryIdentities:180}{
Given a k-vector \( B \) and a vector \( \Ba \), the dot and wedge products have the following commutation relationships
\boxedEquation{eqn:bladeDotWedgeSymmetryIdentities:200}{
\begin{aligned}
B \cdot \Ba  &= (-1)^{k-1} \Ba \cdot B \\
B \wedge \Ba &= (-1)^k \Ba \wedge B,
\end{aligned}
}
and can be expressed as symmetric and antisymmetric sums depending on the grade of the blade
\boxedEquation{eqn:bladeDotWedgeSymmetryIdentities:220}{
\begin{aligned}
\Ba \wedge B &= \inv{2}\lr{ \Ba B + (-1)^k B \Ba } \\
\Ba \cdot B &= \inv{2}\lr{ \Ba B - (-1)^k B \Ba }.
\end{aligned}
}
} % theorem

For example, if \( B \) and \( \Ba \) are both vectors, we recover \cref{thm:symmetricAndAntiSymmetricVectorSums:symmetricAndAnti}.  As an other example,
if \( B \) is a 2-vector, then
\begin{equation}\label{eqn:bladeDotWedgeSymmetryIdentitiesTheorem:480}
\begin{aligned}
2 ( \Ba \wedge B ) &= \Ba B + B \Ba  \\
2 ( \Ba \cdot B ) &= \Ba B - B \Ba.
\end{aligned}
\end{equation}
Observe that the dot(wedge) of two vectors is a (anti)symmetric sum of products, whereas the wedge(dot) of a vector and a bivector is an (anti)symmetric sum.

\begin{proof}
To prove \cref{thm:bladeDotWedgeSymmetryIdentities:180}, split the blade into components that intersect with and are disjoint from \( \Ba \) as follows
\begin{dmath}\label{eqn:bladeDotWedgeSymmetryIdentitiesTheorem:240}
B
=
\inv{\Ba} \Bn_1 \Bn_2 \cdots \Bn_{k-1} + \Bm_1 \Bm_2 \cdots \Bm_k,
\end{dmath}
where \( \Bn_i \) orthogonal to \( \Ba \) and each other, and where \( \Bm_i \) are all orthogonal.  The products of \( B \) with \( \Ba \) are
\begin{dmath}\label{eqn:bladeDotWedgeSymmetryIdentitiesTheorem:340}
\Ba B
=
\Ba \inv{\Ba} \Bn_1 \Bn_2 \cdots \Bn_{k-1} + \Ba \Bm_1 \Bm_2 \cdots \Bm_k
=
\Bn_1 \Bn_2 \cdots \Bn_{k-1} + \Ba \Bm_1 \Bm_2 \cdots \Bm_k,
\end{dmath}
and
\begin{dmath}\label{eqn:bladeDotWedgeSymmetryIdentitiesTheorem:360}
B \Ba
=
\inv{\Ba} \Bn_1 \Bn_2 \cdots \Bn_{k-1} \Ba + \Bm_1 \Bm_2 \cdots \Bm_k \Ba
=
(-1)^{k-1} \Bn_1 \Bn_2 \cdots \Bn_{k-1} + (-1)^k \Ba \Bm_1 \Bm_2 \cdots \Bm_k
=
(-1)^k \lr{ - \Bn_1 \Bn_2 \cdots \Bn_{k-1} + \Ba \Bm_1 \Bm_2 \cdots \Bm_k },
\end{dmath}
or
\begin{dmath}\label{eqn:bladeDotWedgeSymmetryIdentitiesTheorem:380}
(-1)^k B \Ba
=
- \Bn_1 \Bn_2 \cdots \Bn_{k-1} + \Ba \Bm_1 \Bm_2 \cdots \Bm_k.
\end{dmath}

Respective addition and subtraction of \cref{eqn:bladeDotWedgeSymmetryIdentitiesTheorem:340} and \cref{eqn:bladeDotWedgeSymmetryIdentitiesTheorem:380} gives
\begin{dmath}\label{eqn:bladeDotWedgeSymmetryIdentitiesTheorem:400}
\Ba B + (-1)^k B \Ba
= 2 \Ba \Bm_1 \Bm_2 \cdots \Bm_k
= 2 \Ba \wedge B,
\end{dmath}
and
\begin{dmath}\label{eqn:bladeDotWedgeSymmetryIdentitiesTheorem:420}
\Ba B - (-1)^k B \Ba
=
2
\Bn_1 \Bn_2 \cdots \Bn_{k-1}
= 2 \Ba \cdot B,
\end{dmath}
proving \cref{eqn:bladeDotWedgeSymmetryIdentities:220}.  Grade selection from \cref{eqn:bladeDotWedgeSymmetryIdentitiesTheorem:380} gives
\begin{dmath}\label{eqn:bladeDotWedgeSymmetryIdentitiesTheorem:440}
(-1)^k B \cdot \Ba
=
- \Bn_1 \Bn_2 \cdots \Bn_{k-1}
= - \Ba \cdot B,
\end{dmath}
and
\begin{dmath}\label{eqn:bladeDotWedgeSymmetryIdentitiesTheorem:460}
(-1)^k B \wedge \Ba
=
\Ba \Bm_1 \Bm_2 \cdots \Bm_k
= \Ba \wedge B,
\end{dmath}
which proves \cref{eqn:bladeDotWedgeSymmetryIdentities:200}.
\end{proof}


            %
% Copyright © 2016 Peeter Joot.  All Rights Reserved.
% Licenced as described in the file LICENSE under the root directory of this GIT repository.
%
\maketheorem{Distribution of inner products}{thm:stokesTheoremGeometricAlgebra:1420}{
Given two blades \(A_s, B_r\) with grades subject to \(s > r > 0\), and a vector \(\Bb\), the inner product distributes according to
\begin{equation*}
A_s \cdot \lr{ \Bb \wedge B_r } = \lr{ A_s \cdot \Bb } \cdot B_r.
\end{equation*}
}

         \subsection{Problems}
            \input{vectorproductCyclicPermutation.tex}
            \input{wedgeantisym.tex}
            \input{gradethreeselectionWedge.tex}
            \input{../stokesTheorem/bladeDotWedgeSymmetryIdentities.tex}
            \input{../gabookI/appendix/wedgeDistributionIdentityProblems.tex}
      \section{Problem solutions}
         \shipoutAnswer
   \chapter{Geometry}
      \section{Bivectors}
gabookI: 3.9
      \section{Problems}
      \section{Trivectors}
      \subsection{Problems}
      %\section{Projection and rejection}
      \subsection{Problems}
      \section{Rotations}
gabookI: 2.5 rotations. esp fig 2.1, fig 2.2.
            \input{../gabookI/basics/gaQuickIntroRotations.tex}
gabookI: 10.4.3 bivector generator of rotations.
gabookI: 29.1
      \subsection{Problems}
      \section{Equivalent identities}
gabookI: 4.1+
      \section{Cramer's rule}
gabookI: 5.  Generalize examples to higher dimensions.
         \subsection{Problems}
      \section{Problem solutions}
         \shipoutAnswer
   \chapter{Vector calculus}
      \section{Reciprocal frames}
         %
% Copyright � 2016 Peeter Joot.  All Rights Reserved.
% Licenced as described in the file LICENSE under the root directory of this GIT repository.
%
%{
%\input{../blogpost.tex}
%\renewcommand{\basename}{reciprocal}
%%\renewcommand{\dirname}{notes/phy1520/}
%\renewcommand{\dirname}{notes/ece1228-electromagnetic-theory/}
%%\newcommand{\dateintitle}{}
%%\newcommand{\keywords}{}
%
%\input{../peeter_prologue_print2.tex}
%
%\usepackage{peeters_layout_exercise}
%\usepackage{peeters_braket}
%\usepackage{peeters_figures}
%\usepackage{siunitx}
%%\usepackage{mhchem} % \ce{}
%%\usepackage{macros_bm} % \bcM
%%\usepackage{macros_qed} % \qedmarker
%%\usepackage{txfonts} % \ointclockwise
%
%\beginArtNoToc
%
%\generatetitle{Reciprocal frame vectors}
%%\chapter{reciprocal frame vectors}
%%\label{chap:reciprocal}
%
The end goal of this chapter is to be able to integrate multivector functions along curves and surfaces, known collectively as manifolds.
For our purposes, a manifold is defined by a parameterization, such as the vector valued function \( \Bx(a,b) \) where \( a, b\) are scalar parameters.  With one parameter the vector traces out a curve, with two a surface, three a volume, and so forth.
The respective partial derivatives of such a parameterized vector define a local basis for the surface at the point at which the partials are evaluated.
The span of such a basis is called the tangent space, and the partials that constitute it are not necessarily orthonormal, or even normal.

Unfortunately, in order to work with the curvilinear non-orthonormal bases that will be encountered in general integration theory, some
additional tools are required.

\begin{itemize}
\item
We introduce a reciprocal frame which partially generalizes the notion of normal to non-orthonormal bases.
\item
We will borrow the upper and lower index (tensor) notation from relativistic physics that is useful for the intrinsically non-orthonormal spaces encountered in that study, as this notation works well to define the reciprocal frame.
\end{itemize}

\index{reciprocal frame}
\makedefinition{Reciprocal frame}{dfn:reciprocal:frame}{
Given a basis for a subspace \( \setlr{ \Bx_1, \Bx_2, \cdots \Bx_n } \), where the vectors \( \Bx_i \) are not necessarily orthonormal, the reciprocal frame is defined as the set of vectors \( \setlr{ \Bx^1, \Bx^2, \cdots \Bx^n } \) satisfying

\begin{dmath*}
\Bx_i \cdot \Bx^j = {\delta_i}^j,
\end{dmath*}

where the vector \( \Bx^j \) is not the j-th power of \( \Bx \), but is a superscript index, the conventional way of denoting a reciprocal frame vector, and \( {\delta_i}^j \) is the Kronecker delta.
} % definition

This definition introduces mixed index variables for the first time in this text, which may be unfamiliar.  These are most often used in tensor algebra, where any expression that has pairs of upper and lower indexes implies a sum, and is called the summation (or Einstein) convention.  For example:

\begin{dmath}\label{eqn:reciprocal:400}
\begin{aligned}
a_i b^i &\equiv \sum_i a_i b^i \\
{A^{i}}_j B_i C^j &\equiv \sum_{i,j} {A^{i}}_j B_i C^j.
\end{aligned}
\end{dmath}

Summation convention will not be used explicitly in this text, as it deviates from normal practises in electrical engineering\footnote{Generally, when summation convention is used, explicit summation is only used explicitly when upper and lower indexes are not perfectly matched, but summation is still implied.  Readers of texts that use summation convention can check for proper matching of upper and lower indexes to ensure that the expressions make sense.  Such matching is the reason a mixed index Kronecker delta has been used in the definition of the reciprocal frame.}.

The most important application of a reciprocal frame is for the computation of the coordinates of a vector with respect to a non-orthonormal frame.
Let a vector \( \Ba \) have coordinates \( a^i \) with respect to a basis \( \setlr{ \Bx_i } \)

\begin{dmath}\label{eqn:reciprocal:20}
\Ba = \sum_j a^j \Bx_j,
\end{dmath}

where \( j \) is an index not a power\footnote{In tensor algebra, any index that is found in matched upper and lower index pairs, is known as a dummy summation index, whereas an index that is unmatched is known as a free index.  For example, in \( a^j b_{ij} \) (summation implied) \( j \) is a summation index, and \( i \) is a free index.  We are free to make a change of variables of any summation index, so for the same example we can write
\( a^k b_{ik} \).  These index tracking conventions are obvious when summation symbols are included explicitly, as we will do.}.

Dotting with the reciprocal frame vectors \( \Bx^i \) provides these coordinates \( a^i \)

\begin{dmath}\label{eqn:reciprocal:40}
\Ba \cdot \Bx^i
= \lr{\sum_j a^j \Bx_j} \cdot \Bx^i
= \sum_j a^j {\delta_j}^i
= a^i.
\end{dmath}

The vector can also be expressed with coordinates taken with respect to the reciprocal frame.  Let those coordinates be \( a_i \), so that

\begin{dmath}\label{eqn:reciprocal:60}
\Ba = \sum_i a_i \Bx^i.
\end{dmath}

Dotting with the basis vectors \( \Bx_i \) provides the reciprocal frame relative coordinates \( a_i \)

\begin{dmath}\label{eqn:reciprocal:80}
\Ba \cdot \Bx_i
= \lr{\sum_j a_j \Bx^j} \cdot \Bx_i
= \sum_j a_j {\delta^j}_i
= a_i.
\end{dmath}

We can summarize \cref{eqn:reciprocal:40} and \cref{eqn:reciprocal:80} by stating that a vector can be expressed in terms of coordinates relative to either the original or reciprocal basis as follows

\begin{equation}\label{eqn:reciprocal:420}
\Ba
= \sum_j \lr{ \Ba \cdot \Bx^j } \Bx_j
= \sum_j \lr{ \Ba \cdot \Bx_j } \Bx^j.
\end{equation}

In tensor algebra the basis is generally implied\footnote{
In tensor algebra, a vector, identified by the coordinates \( a^i \) is called a contravariant vector.
When that vector is identified by the coordinates \( a_i \) it is called a covariant vector.  These labels relate to how the coordinates transform with respect to norm preserving transformations.
We have no need of this nomenclature, since we never transform coordinates in isolation, but will always transform the coordinates along with their associated basis vectors.}.

%When doing tensor algebra manipulations, you'll generally have the freedom to swap any pairs of upper and lower indexes as done above.

An example of a 2D oblique Euclidean basis and a corresponding reciprocal basis is plotted in \cref{fig:obliqueReciprocal:obliqueReciprocalFig2}.
Also plotted are the superposition of the projections required to arrive at a given point \( (4,2) \)) along the \( \Be_1, \Be_2 \) directions or the \( \Be^1, \Be^2 \) directions.
In this plot, neither of the reciprocal frame vectors \( \Be^i \) are normal to the corresponding basis vectors \( \Be_i \).
When one of \( \Be_i \) is increased(decreased) in magnitude, there will be a corresponding decrease(increase) in the magnitude of \( \Be^i \), but if the orientation is remained fixed, the corresponding direction of the reciprocal frame vector stays the same.

\imageFigure{../figures/GAelectrodynamics/obliqueReciprocalFig2}{Oblique and reciprocal bases.}{fig:obliqueReciprocal:obliqueReciprocalFig2}{0.5}

How are the reciprocal frame vectors computed?  While these vectors have a natural GA representation, this is not intrinsically a GA problem, and can be solved with standard linear algebra, using a matrix inversion.
For example, given a 2D basis \( \setlr{ \Bx_1, \Bx_2 } \), the reciprocal basis can be assumed to have a coordinate representation in the original basis

\begin{dmath}\label{eqn:reciprocal:100}
\begin{aligned}
\Bx^1 &= a \Bx_1 + b \Bx_2 \\
\Bx^2 &= c \Bx_1 + d \Bx_2.
\end{aligned}
\end{dmath}

Imposing the constraints of \cref{dfn:reciprocal:frame} leads to a pair of 2x2 linear systems that are easily solved to find
\begin{dmath}\label{eqn:reciprocal:120}
\begin{aligned}
\Bx^1 &= \inv{ (\Bx_1)^2 (\Bx_2)^2 - \lr{ \Bx_1 \cdot \Bx_2}^2 } \lr{ (\Bx_2)^2 \Bx_1 - \lr{ \Bx_1 \cdot \Bx_2 } \Bx_2 } \\
\Bx^2 &= \inv{ (\Bx_1)^2 (\Bx_2)^2 - \lr{ \Bx_1 \cdot \Bx_2}^2 } \lr{ (\Bx_1)^2 \Bx_2 - \lr{ \Bx_1 \cdot \Bx_2 } \Bx_1 } \\
\end{aligned}
\end{dmath}

The reader may notice that for \R{3} the denominator is related to the norm of the cross product \( \Bx_1 \cross \Bx_2 \).
More generally, this can be expressed as the square of the bivector \( \Bx_1 \wedge \Bx_2 \)

\begin{dmath}\label{eqn:reciprocal:140}
-\lr{\Bx_1 \wedge \Bx_2 }^2
=
-\lr{\Bx_1 \wedge \Bx_2 } \cdot \lr{\Bx_1 \wedge \Bx_2 }
=
-\lr{ \lr{\Bx_1 \wedge \Bx_2 } \cdot \Bx_1 } \cdot \Bx_2
=
(\Bx_1)^2 (\Bx_2)^2 - \lr{\Bx_1 \cdot \Bx_2}^2.
\end{dmath}

Additionally, the numerators are each dot products of \( \Bx_1, \Bx_2 \) with that same bivector

\begin{dmath}\label{eqn:reciprocal:160}
\begin{aligned}
\Bx^1 &= \frac{\Bx_2 \cdot \lr{ \Bx_1 \wedge \Bx_2 } }{ \lr{ \Bx_1 \wedge \Bx_2}^2 } \\
\Bx^2 &= \frac{\Bx_1 \cdot \lr{ \Bx_2 \wedge \Bx_1 } }{ \lr{ \Bx_1 \wedge \Bx_2}^2 },
\end{aligned}
\end{dmath}

or

%\begin{dmath}\label{eqn:reciprocal:180}
\boxedEquation{eqn:reciprocal:180}{
\begin{aligned}
\Bx^1 &= \Bx_2 \cdot \inv{ \Bx_1 \wedge \Bx_2 } \\
\Bx^2 &= \Bx_1 \cdot \inv{ \Bx_2 \wedge \Bx_1 }.
\end{aligned}
}
%\end{dmath}

Geometrically, dotting with the bivector of the plane is a hybrid rotation and scaling operation.
For example, for \R{2} with \( \Bx_1 = a_1 \Be_1 + a_2 \Be_2, \Bx_2 = b_1 \Be_1 + b_2 \Be_2 \), that pseudoscalar for this basis is

\begin{dmath}\label{eqn:reciprocal:260}
\Bx_1 \wedge \Bx_2
=
\lr{ a_1 \Be_1 + a_2 \Be_2 } \wedge \lr{ b_1 \Be_1 + b_2 \Be_2 }
=
\lr{ a_1 b_2 - a_2 b_1 } \Be_{12}.
\end{dmath}

This has inverse
\begin{dmath}\label{eqn:reciprocal:280}
\inv{\Bx_1 \wedge \Bx_2 }
=
\inv{ a_1 b_2 - a_2 b_1 } \Be_{21}.
\end{dmath}

So for the \R{2} the reciprocal frame is just

\begin{dmath}\label{eqn:reciprocal:300}
\begin{aligned}
\Bx^1 &= \Bx_2 \frac{\Be_{21}}{ a_1 b_2 - a_2 b_1 } \\
\Bx^2 &= \Bx_1 \frac{\Be_{12}}{ a_1 b_2 - a_2 b_1 }
\end{aligned}
\end{dmath}

The vector \( \Bx^1 \) is obtained by rotating \( \Bx_2 \) by \( -\pi/2 \), and rescaling it.
The vector \( \Bx^2 \) is similarly obtained by a scaling and a rotation of \( \Bx_1 \) by \( \pi/2 \).

Generalizing \cref{eqn:reciprocal:180} is almost possible by inspection.
Given
a subspace spanned by a three vector basis \( \setlr{ \Bx_1, \Bx_2, \Bx_3 } \) the reciprocal frame vectors can be written as dot products

\begin{dmath}\label{eqn:reciprocal:320}
\begin{aligned}
\Bx^1 &= \lr{ \Bx_2 \wedge \Bx_3 } \cdot \lr{ \Bx^3 \wedge \Bx^2 \wedge \Bx^1 } \\
\Bx^2 &= \lr{ \Bx_3 \wedge \Bx_1 } \cdot \lr{ \Bx^1 \wedge \Bx^3 \wedge \Bx^2 } \\
\Bx^3 &= \lr{ \Bx_1 \wedge \Bx_2 } \cdot \lr{ \Bx^2 \wedge \Bx^1 \wedge \Bx^3 } \\
\end{aligned}
\end{dmath}

Each of those trivector terms equals \( - \Bx^1 \wedge \Bx^2 \wedge \Bx^3 \) and can be related to the (known) pseudoscalar \( \Bx_1 \wedge \Bx_2 \wedge \Bx_3 \) by observing that

\begin{dmath}\label{eqn:reciprocal:340}
\lr{ \Bx^1 \wedge \Bx^2 \wedge \Bx^3 } \cdot \lr{ \Bx_3 \wedge \Bx_2 \wedge \Bx_1 }
=
\Bx^1 \cdot \lr{ \Bx^2 \cdot \lr{ \Bx^3 \cdot \lr{ \Bx_3 \wedge \Bx_2 \wedge \Bx_1 } }}
=
\Bx^1 \cdot \lr{ \Bx^2 \cdot \lr{ \Bx_2 \wedge \Bx_1 } }
=
\Bx^1 \cdot \Bx_1
=
1,
\end{dmath}

which means that

\begin{dmath}\label{eqn:reciprocal:360}
-\Bx^1 \wedge \Bx^2 \wedge \Bx^3
= -\inv{ \Bx_3 \wedge \Bx_2 \wedge \Bx_1 }
= \inv{ \Bx_1 \wedge \Bx_2 \wedge \Bx_3 },
\end{dmath}

and

\boxedEquation{eqn:reciprocal:380}{
\begin{aligned}
\Bx^1 &= \lr{ \Bx_2 \wedge \Bx_3 } \cdot \inv{ \Bx_1 \wedge \Bx_2 \wedge \Bx_3 } \\
\Bx^2 &= \lr{ \Bx_3 \wedge \Bx_1 } \cdot \inv{ \Bx_1 \wedge \Bx_2 \wedge \Bx_3 } \\
\Bx^3 &= \lr{ \Bx_1 \wedge \Bx_2 } \cdot \inv{ \Bx_1 \wedge \Bx_2 \wedge \Bx_3 }
\end{aligned}
}

Geometrically, this trivector division is a duality transformation within the subspace spanned by the three vectors \( \Bx_1, \Bx_2, \Bx_3 \), also scaling the result so that the \( \Bx_i \cdot \Bx^j = {\delta_i}^j \) condition is satisfied.

It should be clear how to generalize the reciprocal basis calculation formulas of
\cref{eqn:reciprocal:180} and \cref{eqn:reciprocal:380} to higher dimensions if desired.
%}
%\EndNoBibArticle

         \subsection{Problems}
            \input{2dreciprocalMatrixCalculation.tex}
            \input{2subspaceR3reciprocalExample.tex}
      \section{Curvilinear coordinates}
         %
% Copyright © 2017 Peeter Joot.  All Rights Reserved.
% Licenced as described in the file LICENSE under the root directory of this GIT repository.
%
\index{curvilinear coordinates}
Curvilinear coordinates can be defined for any subspace spanned by a parameterized vector into that space.
%Consider a continuous subspace parameterized by a two parameter vector function \( \Bx = \Bx(u_1, u_2) \) that is differentiable with respect to either parameter
As an example, consider a two parameter planar subspace of parameterized by the following continuous vector function

\begin{dmath}\label{eqn:curvilinearDefined:480}
\Bx(u_1, u_2) = u_1 \Be_1 \frac{\sqrt{3}}{2} \cosh\lr{ \Atanh(1/2) + \Be_{12} u_2 },
\end{dmath}

where \( u_1 \in [0,1] \) and \( u_2 \in [0, \pi/2] \).
This parameterization spans the first quadrant of the ellipse with semi-major axis length 1, and semi-minor axis length \( 1/2 \)
\footnote{
A parameterization of an elliptic area may or may not not be of much use in electrodynamics.  It does, however, provide a fairly simple but non-trivial example of a non-orthonormal parameterization.}
Contours for this parameterization are plotted in \cref{fig:ellipticalContours:ellipticalContoursFig1}.
The radial contours are for fixed values of \( u_2 \) and the elliptical contours fix the value of \( u_1 \), and depict a set of ellipic curves
with a semi-major/major axis ratio of \( 1/2 \).

\imageFigure{../figures/GAelectrodynamics/ellipticalContoursFig1}{Contours for an elliptical region.}{fig:ellipticalContours:ellipticalContoursFig1}{0.3}

We define a curvilinear basis associated with each point in the region by the partials

\begin{dmath}\label{eqn:curvilinearDefined:80}
\begin{aligned}
\Bx_{1} &= \PD{u_1}{\Bx} \\
\Bx_{2} &= \PD{u_2}{\Bx}.
\end{aligned}
\end{dmath}

For our the function \cref{eqn:curvilinearDefined:480} our curvilinear basis elements are

\begin{dmath}\label{eqn:curvilinearDefined:520}
\begin{aligned}
\Bx_{1} &= \Be_1 \frac{\sqrt{3}}{2} \cosh\lr{ \Atanh(1/2) + \Be_{12} u_2 } \\
\Bx_{2} &= u_1 \Be_2 \frac{\sqrt{3}}{2} \sinh\lr{ \Atanh(1/2) + \Be_{12} u_2 }.
\end{aligned}
\end{dmath}

We form vector valued differentials for each parameter

\begin{dmath}\label{eqn:curvilinearDefined:500}
\begin{aligned}
d\Bx_{1} &= \Bx_1 du_1 \\
d\Bx_{2} &= \Bx_2 du_2.
\end{aligned}
\end{dmath}

For \cref{eqn:curvilinearDefined:480},
the values of these differentials \( d\Bx_1, d\Bx_2 \) with \( du_1 = du_2 = 0.1 \) are plotted
in
\cref{fig:ellipticalContours:ellipticalContoursFig2}
for the points
\( (u_1, u_2) = (0.7, 5 \pi/20), (0.9, 3 \pi/20), (1.0, 5 \pi/20) \)
in
(dark-thick) red, blue and purple respectively.

\imageFigure{../figures/GAelectrodynamics/ellipticalContoursFig2}{Differentials for an elliptical parameterization.}{fig:ellipticalContours:ellipticalContoursFig2}{0.3}

In this case and in general there is no reason to presume that there is any orthonormality constraint on the basis \( \setlr{ \Bx_{1}, \Bx_{2} } \) for a given two parameter subspace.

Should we wish to calculate the reciprocal frame
for \cref{eqn:curvilinearDefined:480}
, we would find
(\cref{problem:curvilinearDefined:560}) that

\begin{dmath}\label{eqn:curvilinearDefined:540}
\begin{aligned}
\Bx^{1} &= \Be_1 \sqrt{3} \sinh\lr{ \Atanh(1/2) + \Be_{12} u_2 } \\
\Bx^{2} &= \frac{\Be_2}{u_1} \sqrt{3} \cosh\lr{ \Atanh(1/2) + \Be_{12} u_2 }.
\end{aligned}
\end{dmath}

These are plotted (scaled by \( da = 0.1 \) so they fit in the image nicely) in \cref{fig:ellipticalContours:ellipticalContoursFig2} using thin light arrows.

When evaluating surface integrals, we will form
oriented (bivector) area elements from the wedge product of the differentials

\begin{dmath}\label{eqn:curvilinearDefined:60}
d^2 \Bx \equiv d\Bx_{1} \wedge d\Bx_{2}.
\end{dmath}

This absolute value of this area element \( \sqrt{-(d^2 \Bx)^2} \) is the area of the parallelogram spanned by \( d\Bx_1, d\Bx_2 \).
In this example, all such area elements lie in the \( x-y \) plane, but that need not be the case.

Also note that we will only perform integrals for those parametrizations for which the area element \( d^2 \Bx \) is non-zero.

%If the spacing between the contours is made small enough, the boundaries of each partition will define a planar region at the point of evaluation.
%All points in the interior will be accessible by a combination of the vectors formed from the partials of \( \Bx \) at that point.

\makeproblem{Elliptic curvilinear and reciprocal basis.}{problem:curvilinearDefined:560}{
From \cref{eqn:curvilinearDefined:480}, compute the
curvilinear coordinates \cref{eqn:curvilinearDefined:520}, and the reciprocal frame vectors \cref{eqn:curvilinearDefined:540}.
Check using scalar grade selection that \( \Bx^i \cdot \Bx_j = {\delta^i}_j \).
Hints: Given \( \mu = \Atanh(1/2) \),
\begin{itemize}
\item \( \cosh( \mu + i \theta ) \Be_2 = \Be_2 \cosh( \mu - i \theta ) \).
\item \( \Real\lr{ \cosh( \mu - i \theta ) \sinh( \mu + i \theta ) } = 2/3 \).
\end{itemize}
} % problem

\paragraph{fixme:}
don't introduce the idea of tangent space until a 3D example.
Remove the \R{3} reference above, and keep this first example planar.

At the point of evaluation, the span of these differentials is called the tangent space.
In this particular case the tangent space at all points in the region is the entire x-y plane.
These partials locally span the tangent space at a given point on the surface.

\subsubsection{Curved two parameter surfaces.}

Continuing to illustrate by example, let's now consider a non-planar two parameter surface

\begin{dmath}\label{eqn:curvilinearDefined:560}
\Bx(u_1, u_2) =
(u_1-u_2)^2
\Be_1
+ (1-(u_2)^2 ) \Be_2
+ u_1 u_2 \Be_3.
\end{dmath}

The curvilinear basis elements are
\begin{dmath}\label{eqn:curvilinearDefined:580}
\begin{aligned}
\Bx_1 &= 2 (u_1 - u_2) \Be_1 + u_2 \Be_3 \\
\Bx_2 &= 2 (u_2 - u_1) \Be_1 - 2 u_2 \Be_2 + u_1 \Be_3.
\end{aligned}
\end{dmath}

These vectors and two examples of the oriented plane (rescaled to fit) formed by \( \Bx_1 \wedge \Bx_2 \) is plotted in
\cref{fig:2dmanifold:2dmanifoldFig1}.
This plane is called the tangent space at the point in question, and has been evaluated at \( (u_1, u_2) = (0.5,0.5), (0.35, 0.75) \).

\imageFigure{../figures/GAelectrodynamics/2dmanifoldFig1}{Two parameter manifold.}{fig:2dmanifold:2dmanifoldFig1}{0.3}

%\imageFigure{../figures/GAelectrodynamics/twoParameterDifferentialFieldFig1}{Curvilinear coordinates along a two parameter surface.}{fig:twoParameterDifferentialField:twoParameterDifferentialFieldFig1}{0.3}



Many of the concepts are illuminated nicely by considering some examples.

         \subsection{2D Cylindrical coordinates.}
            %
% Copyright © 2017 Peeter Joot.  All Rights Reserved.
% Licenced as described in the file LICENSE under the root directory of this GIT repository.
%
%\index{cylindrical coordinates}
\index{polar coordinates}
\index{curvilinear coordinates}
One of the simplest curvilinear coordinate systems are polar coordinates (cylindrical coordinates in a plane.)

FIXME: Wolfgang: ``picture.''

The parameterization associated with such a space is

\begin{dmath}\label{eqn:2Dcylindrical:100}
\Bx(\rho, \phi) = \rho \Be_1 \exp\lr{ \Be_{12} \phi }.
\end{dmath}

The curvilinear coordinate basis is therefore

\begin{subequations}
\label{eqn:2Dcylindrical:120}
\begin{dmath}\label{eqn:2Dcylindrical:140}
\Bx_\rho
= \PD{\rho}{} \lr{ \rho \Be_1 \exp\lr{ \Be_{12} \phi } }
= \Be_1 \exp\lr{ \Be_{12} \phi }
\end{dmath}
\begin{dmath}\label{eqn:2Dcylindrical:160}
\Bx_\phi
= \PD{\phi}{} \lr{ \rho \Be_1 \exp\lr{ \Be_{12} \phi } }
= \rho
\Be_1 \Be_{12} \exp\lr{ \Be_{12} \phi }
= \rho
\Be_2 \exp\lr{ \Be_{12} \phi }.
\end{dmath}
\end{subequations}

\index{reciprocal basis}
Noting that this is a normal set of vectors, the reciprocal basis can be found by inspection

\begin{dmath}\label{eqn:2Dcylindrical:180}
\begin{aligned}
\Bx^\rho &= \Be_1 \exp\lr{ \Be_{12} \phi } \\
\Bx^\phi &= \inv{\rho} \Be_2 \exp\lr{ \Be_{12} \phi }.
\end{aligned}
\end{dmath}

\index{gradient}
For completeness, it's worth verifying that the gradient representation of the reciprocal frame provides this same result.
The \( x, y \) variables are related to \( \rho, \phi \) through

\begin{dmath}\label{eqn:2Dcylindrical:620}
\begin{aligned}
x &= r \cos\phi \\
y &= r \sin\phi.
\end{aligned}
\end{dmath}

Rearranging slightly to facilitate evaluation of the \( x, y \) partials

\begin{dmath}\label{eqn:2Dcylindrical:500}
\begin{aligned}
\rho^2 &= x^2 + y^2 \\
\tan\phi &= y/x,
\end{aligned}
\end{dmath}

we can evaluate the components of the gradients by implicit differentiation

\begin{dmath}\label{eqn:2Dcylindrical:520}
\begin{aligned}
2 \rho \PD{x}{\rho} &= 2 x \\
2 \rho \PD{y}{\rho} &= 2 y \\
\inv{\cos^2\phi} \PD{x}{\phi} &= -\frac{y}{x^2} \\
\inv{\cos^2\phi} \PD{y}{\phi} &= \inv{x},
\end{aligned}
\end{dmath}

The gradients are
\begin{subequations}
\label{eqn:2Dcylindrical:540}
\begin{dmath}\label{eqn:2Dcylindrical:560}
\spacegrad \rho
= \inv{\rho} (\cos\phi, \sin\phi)
= \Be_1 e^{\Be_{12} \phi}
= \Bx^\rho
\end{dmath}
\begin{dmath}\label{eqn:2Dcylindrical:580}
\spacegrad \phi
=
\cos^2 \phi \lr{ -\frac{y}{x^2}, \inv{x} }
=
\inv{\rho} ( -\sin\phi, \cos\phi )
=
\frac{\Be_2}{\rho} ( \cos\phi + \Be_{12} \sin\phi )
=
\frac{\Be_2}{\rho} e^{ \Be_{12} \phi }
=
\Bx^\phi,
\end{dmath}
\end{subequations}

which is consistent with the result found by inspection as desired.

In this particular parameterization, it is convenient to define a locally orthonormal coordinate basis \( \setlr{ \rhocap, \phicap } \)

\begin{dmath}\label{eqn:2Dcylindrical:200}
\begin{aligned}
\rhocap &= \Bx_\rho = \Be_1 \exp\lr{ \Be_{12} \phi } \\
\phicap &= \inv{\rho} \Bx_\phi = \Be_2 \exp\lr{ \Be_{12} \phi },
\end{aligned}
\end{dmath}

so that \( \Bx^\rho = \Bx_\rho = \rhocap \), \( \Bx_\phi = \rho \rhocap \), and \( \Bx^\phi = \rhocap/\rho \), and the gradient is

\begin{dmath}\label{eqn:2Dcylindrical:600}
\spacegrad
=
\Bx^\rho \PD{\rho}{}
+ \Bx^\phi \PD{\phi}{}
=
\rhocap \PD{\rho}{}
+\inv{\rho} \phicap \PD{\phi}{}.
\end{dmath}

The volume element for this subspace is
\begin{dmath}\label{eqn:2Dcylindrical:220}
d\Bx_\rho \wedge d\Bx_\phi
=
d\rho d\phi
\Bx_\rho \wedge \Bx_\phi
=
d\rho d\phi
\gpgradetwo{
\Bx_\rho \Bx_\phi
}
=
d\rho d\phi
\gpgradetwo{
\Be_1 \exp\lr{ \Be_{12} \phi } \rho
\Be_2 \exp\lr{ \Be_{12} \phi }
}
\end{dmath}

To evaluate this we use \cref{thm:SimpleProducts2:1780}, property (c), and change the order of a pair of vector and complex exponentials, performing the required conjugation of that exponential

\begin{dmath}\label{eqn:2Dcylindrical:640}
d\Bx_\rho \wedge d\Bx_\phi
=
\rho d\rho d\phi
\gpgradetwo{
\Be_1 \Be_2 \exp\lr{ -\Be_{12} \phi }
\exp\lr{ \Be_{12} \phi }
}
=
\rho d\rho d\phi \Be_{12}.
\end{dmath}

Observe that the (oriented) volume of a circular region of radius \( r \) in this space has the expected result

\begin{dmath}\label{eqn:2Dcylindrical:360}
\int d\Bx_\rho \wedge d\Bx_\phi
=
\int_0^r \rho d\rho \int_0^{2\pi} d\phi \Be_{12}
= \pi r^2 \Be_{12}.
\end{dmath}

Given a vector \( \Bv = \Be_1 f(\rho, \phi) + \Be_2 g(\rho, \phi) \), the cylindrical representation \( \Bv = \Bv_\rho \rhocap + \Bv_\phi \phicap \) can be found by computing the dot products

\begin{subequations}
\label{eqn:2Dcylindrical:420}
\begin{dmath}\label{eqn:2Dcylindrical:440}
\Bv \cdot \rhocap
=
\gpgradezero{ (\Be_1 f + \Be_2 g) \Be_1 e^{\Be_{12} \phi} }
=
f \cos\phi + g \sin\phi
\end{dmath}
\begin{dmath}\label{eqn:2Dcylindrical:460}
\Bv \cdot \phicap
=
\gpgradezero{ (\Be_1 f + \Be_2 g) \Be_2 e^{\Be_{12} \phi} }
=
g \cos\phi - f \sin\phi,
\end{dmath}
\end{subequations}

so
\begin{dmath}\label{eqn:2Dcylindrical:480}
\Bv = \lr{ f \cos\phi + g \sin\phi } \rhocap + \lr{ g \cos\phi - f \sin\phi } \phicap.
\end{dmath}


%gabook: 31.1
%Also: Stokes chapter.  Lots of examples there that should really be separated out from the stokes core content
%(now included here).
      \section{Green's theorem}
         %
% Copyright © 2013 Peeter Joot.  All Rights Reserved.
% Licenced as described in the file LICENSE under the root directory of this GIT repository.
%
Given a two parameter (\(u,v\)) surface parameterization, the curvilinear coordinate representation of a vector \(\Bf\) has the form

\begin{dmath}\label{eqn:stokesTheoremGeometricAlgebra:1640}
\Bf = f_u \Bx^u + f_v \Bx^v + f_\perp \Bx^\perp.
\end{dmath}

We assume that the vector space is of dimension two or greater but otherwise unrestricted, and need not have an Euclidean basis.  Here \(f_\perp \Bx^\perp\) denotes the rejection of \(\Bf\) from the tangent space at the point of evaluation.  Green's theorem relates the integral around a closed curve to an ``area'' integral on that surface

\maketheorem{Green's Theorem}{thm:stokesTheoremGeometricAlgebra:1660}{
\index{Green's theorem}
\begin{equation*}
\ointctrclockwise \Bf \cdot d\Bl
=
\iint \lr{
-\PD{v}{f_u}
+\PD{u}{f_v}
}
du dv
\end{equation*}
}

Following the arguments used in \citep{schwartz1987pe} for Stokes theorem in three dimensions, we first evaluate the loop integral along the differential element of the surface at the point \(\Bx(u_0, v_0)\) evaluated over the range \((du, dv)\), as shown in the infinitesimal loop of \cref{fig:loopIntegralInfinitesimal:loopIntegralInfinitesimalFig1}.

\imageFigure{../figures/gabook/loopIntegralInfinitesimalFig1}{Infinitesimal loop integral}{fig:loopIntegralInfinitesimal:loopIntegralInfinitesimalFig1}{0.35}

Over the infinitesimal area, the loop integral decomposes into

\begin{dmath}\label{eqn:stokesTheoremGeometricAlgebra:1700}
\ointctrclockwise \Bf \cdot d\Bl
=
\int \Bf \cdot d\Bx_1
+\int \Bf \cdot d\Bx_2
+\int \Bf \cdot d\Bx_3
+\int \Bf \cdot d\Bx_4,
\end{dmath}

where the differentials along the curve are

\begin{dmath}\label{eqn:stokesTheoremGeometricAlgebra:1600}
\begin{aligned}
d\Bx_1 &= \evalbar{ \PD{u}{\Bx} }{v = v_0} du \\
d\Bx_2 &= \evalbar{ \PD{v}{\Bx} }{u = u_0 + du} dv \\
d\Bx_3 &= -\evalbar{ \PD{u}{\Bx} }{v = v_0 + dv} du \\
d\Bx_4 &= -\evalbar{ \PD{v}{\Bx} }{u = u_0} dv.
\end{aligned}
\end{dmath}

It is assumed that the parameterization change \((du, dv)\) is small enough that this loop integral can be considered planar (regardless of the dimension of the vector space).  Making use of the fact that \(\Bx^\perp \cdot \Bx_\alpha = 0\) for \(\alpha \in \setlr{u,v}\), the loop integral is

\begin{dmath}\label{eqn:stokesTheoremGeometricAlgebra:1620}
\ointctrclockwise \Bf \cdot d\Bl
=
\int
\lr{
f_u \Bx^u + f_v \Bx^v + f_\perp \Bx^\perp
}
\cdot
\Bigl(
\Bx_u(u, v_0) du - \Bx_u(u, v_0 + dv) du
+\Bx_v(u_0 + du, v) dv - \Bx_v(u_0, v) dv
\Bigr)
=
\int
f_u(u, v_0) du - f_u(u, v_0 + dv) du
+
f_v(u_0 + du, v) dv - f_v(u_0, v) dv
\end{dmath}

With the distances being infinitesimal, these differences can be rewritten as partial differentials

\begin{dmath}\label{eqn:stokesTheoremGeometricAlgebra:1860}
\ointctrclockwise \Bf \cdot d\Bl
=
\iint \lr{
-\PD{v}{f_u}
+\PD{u}{f_v}
}
du dv.
\end{dmath}

We can now sum over a larger area as in \cref{fig:loopIntegralInfinitesimalSum:loopIntegralInfinitesimalSumFig2}

\imageFigure{../figures/gabook/loopIntegralInfinitesimalSumFig2}{Sum of infinitesimal loops}{fig:loopIntegralInfinitesimalSum:loopIntegralInfinitesimalSumFig2}{0.35}

All the opposing oriented loop elements cancel, so the integral around the complete boundary of the surface \(\Bx(u, v)\) is given by the \(u,v\) area integral of the partials difference.

We will see that Green's theorem is a special case of the Stokes theorem.  This observation will also provide a geometric interpretation of the right hand side area integral of \cref{thm:stokesTheoremGeometricAlgebra:1660}, and allow for a coordinate free representation.

\paragraph{Special case:}

An important special case of Green's theorem is for a Euclidean two dimensional space where the vector function is

\begin{dmath}\label{eqn:stokesTheoremGeometricAlgebra:1720}
\Bf = P \Be_1 + Q \Be_2.
\end{dmath}

Here Green's theorem takes the form

\boxedEquation{eqn:stokesTheoremGeometricAlgebra:1710}{
\ointctrclockwise P dx + Q dy
=
\iint \lr{
\PD{x}{Q}
-\PD{y}{P}
}
dx dy.
}

         %\subsection{Problems}
      \section{Stokes' theorem}
         \subsection{Statement}
            %
% Copyright © 2016 Peeter Joot.  All Rights Reserved.
% Licenced as described in the file LICENSE under the root directory of this GIT repository.
%

Stokes' theorem is fairly easy to state, but takes a fair amount of work to understand and unpack its implications.

%
% Copyright © 2013 Peeter Joot.  All Rights Reserved.
% Licenced as described in the file LICENSE under the root directory of this GIT repository.
%
An important consequence of the fundamental theorem of geometric calculus is the
geometric algebra generalization of Stokes' theorem.  This form of Stokes' theorem is equivalent to the same from the theory of differential forms.
Stokes' theorem in differential forms and geometric algebra is more general and powerful than Stokes' theorem from conventional vector calculus which only relates
surface integrals to the line integral around the bounding surface.

\maketheorem{Stokes' Theorem}{thm:stokesTheoremGeometricAlgebra:1740}{
Given a \(k\) volume element \(d^k \Bx \) and an s-blade \( F, s < k \)
\begin{equation*}%\label{eqn:stokesTheoremTheStatement:120}
\int_V d^k \Bx \cdot (\boldpartial \wedge F) = \int_{\partial V} d^{k-1} \Bx \cdot F.
\end{equation*}
%Here the volume integral is over a \(k\) dimensional hypervolumesurface (manifold).  The derivative operator \(\boldpartial\) is called the vector derviative and is the projection of the gradient onto the tangent space of the manifold.  Integration over the boundary of \(V\) is indicated by \( \partial V \).
}

We will see that most of the well known scalar and vector integral theorems are consequences of \cref{thm:stokesTheoremGeometricAlgebra:1740}.

To prove the theorem, set \( F = 1 \) in \cref{thm:fundamentalTheoremOfCalculus:1}, and require that \( G \) is an s-blade, with grade \( s < k \).  We select the \( k-(s+1) \) grade, the lowest grade of \( d^k \Bx (\boldpartial \wedge G) \) from
both sides of
\cref{thm:fundamentalTheoremOfCalculus:1}.

For the grade selection of the hypervolume integral we have
\begin{dmath}\label{eqn:stokesTheoremTheStatement:100}
\gpgrade{ \int_V d^k \Bx \boldpartial G }{k-(s+1)}
=
\gpgrade{
\int_V d^k \Bx (\boldpartial \cdot G )
+
\int_V d^k \Bx (\boldpartial \wedge G )
}{k-(s-1)},
\end{dmath}
however, the lowest grade of \( d^k \Bx (\boldpartial \cdot G ) \) is \( k -(s-1) = k - s + 1 > k - (s+1) \), so the divergence integral is zero.  This leaves
\begin{dmath}\label{eqn:stokesTheoremTheStatement:110}
\int_V d^k \Bx \cdot (\boldpartial \wedge G )
= \int_{\partial V} \gpgrade{d^{k-1} \Bx G}{k-(s+1)}
= \int_{\partial V} d^{k-1} \Bx \cdot G,
\end{dmath}
proving the theorem.

%%%\paragraph{FIXME: (rewrite) old proof using gagc.}
%%%The vector derivative is defined by
%%%\begin{equation}\label{eqn:stokesTheoremTheStatement:1400}
%%%\boldpartial = \Bx^i \partial_i = \sum_i \Bx_i \PD{u^i}{}.
%%%\end{equation}
%%%
%%%where \( \Bx^i \) are reciprocal frame vectors dual to the tangent vector basis \( \Bx_i \) associated with the parameters \( u^1, u^2, \cdots \).
%%%%These will be defined in more detail in the next section.
%%%Once the volume element, vector product and the other concepts are defined, the proof of
%%%Stokes theorem is really just a statement that
%%%\boxedEquation{eqn:stokesTheoremGeometricAlgebra:2840}{
%%%\int_V d^k \Bx \cdot (\Bx^i \partial_i \wedge F) =
%%%\int_V \lr{ d^k \Bx \cdot \Bx^i } \cdot \partial_i F.
%%%}
%%%
%%%This dot product expansion applies to any degree blade and volume element provided the degree of the blade is less than that of the volume element (i.e. \(s < k\)).  That magic follows directly from \cref{thm:stokesTheoremGeometricAlgebra:1420}.


This dot product defines the oriented surface ``area'' elements associated with the ``volume'' element \( d^k \Bx \).  That area element can be obtained from the mnemonic

\begin{dmath}\label{eqn:stokesTheoremCore:1561}
\sum_i d^k \Bx \cdot \Bx^i,
\end{dmath}

with each of the ith differentials evaluated.  This will be made clear by example.


         \subsection{One parameter specialization of Stokes' theorem.}
            %
% Copyright © 2016 Peeter Joot.  All Rights Reserved.
% Licenced as described in the file LICENSE under the root directory of this GIT repository.
%
\index{differential form}
An example parameterization with one parameter, and the corresponding differential with respect to that parameter, is plotted in
\cref{fig:oneParameterDifferential:oneParameterDifferentialFig1}, for a parameterization over \( [a, b] \in [0,1]\otimes[0,1] \).

\imageFigure{../figures/GAelectrodynamics/oneParameterDifferentialFig1}{One parameter manifold.}{fig:oneParameterDifferential:oneParameterDifferentialFig1}{0.3}

The differential with respect to the parameter \( a \) is

\begin{equation}\label{eqn:stokesTheoremCore:20}
d\Bx_a = \PD{a}{\Bx} da = \Bx_a da.
\end{equation}

On this curve the projection of the gradient has just one component

\begin{dmath}\label{eqn:stokesTheoremCore:40}
\boldpartial
=
\sum_\mu \Bx^\mu (\Bx_\mu \cdot \spacegrad)
=
\Bx^a \PD{a}{}
\equiv
\Bx^a \partial_a.
\end{dmath}

Please see \citep{aMacdonaldVAGC} for a full justification of the curvilinear coordinate representation of the vector derivative (or the gradient).
That text also discusses pertinent issues with the connectivity of the manifold ignored here.

Stokes' theorem for a one parameter manifold can only be expressed for scalar fields.
That is

\begin{dmath}\label{eqn:stokesTheoremCore:60}
\int d\Bx \cdot (\boldpartial \wedge \psi)
=
\int d\Bx \cdot \boldpartial \psi
=
\int da \PD{a}{ \psi }
= \evalbar{\psi}{\Delta a}.
\end{dmath}

Observe that the vector derivative can be replaced by the gradient since \( d\Bx \cdot \boldpartial = d\Bx \cdot \spacegrad \).
This is the case since dotting the
gradient with a differential element \( d\Bx \) on this curve, no component of the gradient that isn't colinear to the curve makes no contribution.

\index{Stokes' theorem}
That means that Stokes' theorem for a one parameter curve is exactly the fundamental theorem of calculus for line integrals

%\begin{dmath}\label{eqn:stokesTheoremCore:80}
\boxedEquation{eqn:stokesTheoremCore:80}{
\int_{\Ba}^{\Bb} d\Bx \cdot \spacegrad \psi = \psi(\Bb) - \psi(\Ba).
}
%\end{dmath}

         \subsection{Two parameter specialization of Stokes' theorem.}
            %
% Copyright © 2016 Peeter Joot.  All Rights Reserved.
% Licenced as described in the file LICENSE under the root directory of this GIT repository.
%

An example parameterization with two parameters, and the corresponding differentials with respect to those parameters, is plotted in
\cref{fig:twoParameterDifferential:twoParameterDifferentialFig1}.

\imageFigure{../figures/GAelectrodynamics/twoParameterDifferentialFig1}{Two parameter manifold differentials.}{fig:twoParameterDifferential:twoParameterDifferentialFig1}{0.4}

Given parameters \( a, b \), the differentials along each of the parameterization directions are

\begin{dmath}\label{eqn:stokesTheoremCore:100}
\begin{aligned}
d\Bx_a &= \PD{a}{\Bx} da = \Bx_a da \\
d\Bx_b &= \PD{b}{\Bx} db = \Bx_b db.
\end{aligned}
\end{dmath}

The ``volume'' element for this parameterization (a surface area element) is

\begin{equation}\label{eqn:stokesTheoremCore:120}
d^2 \Bx
=
d\Bx_a \wedge
d\Bx_b
=
da db (\Bx_a \wedge \Bx_b).
\end{equation}

The vector derivative, the projection of the gradient onto the surface at the point of integration (also called the tangent space), now has two components

\begin{dmath}\label{eqn:stokesTheoremCore:200}
\boldpartial
=
\sum_\mu \Bx^\mu (\Bx_\mu \cdot \spacegrad)
=
\Bx^a \PD{a}{}
+
\Bx^b \PD{b}{}
\equiv
\Bx^a \partial_a
+
\Bx^b \partial_b.
\end{dmath}

The Stokes integral can be evaluated over this volume element for either scalar fields \( \psi \) or vector fields \( \Bf \), and takes the form

\begin{subequations}
\label{eqn:stokesTheoremCore:140}
\begin{equation}\label{eqn:stokesTheoremCore:160}
\int_A d^2 \Bx \cdot (\boldpartial \wedge \psi) =
\int_A (d^2 \Bx \cdot \boldpartial) \psi
=
\int_{\partial A} d^1 \Bx \psi
\end{equation}
\begin{equation}\label{eqn:stokesTheoremCore:180}
\int_A d^2 \Bx \cdot (\boldpartial \wedge \Bf) =
\int_A (d^2 \Bx \cdot \boldpartial) \cdot \Bf
=
\int_{\partial A} d^1 \Bx \cdot \Bf.
\end{equation}
\end{subequations}

To extract the full meaning of this the boundary differential \( d^1 \Bx \) must be computed.  This has the same structure for a vector or scalar field

\begin{dmath}\label{eqn:stokesTheoremCore:220}
\begin{aligned}
\int_A d^2 \Bx \cdot (\boldpartial \wedge \Bf)
&=
\int_A (d^2 \Bx \cdot \boldpartial) \cdot \Bf \\
&=
\sum_\mu \int_A (d^2 \Bx \cdot \Bx^\mu) \cdot \partial_\mu \Bf \\
&=
\sum_\mu \int_A da db  \lr{ \Bx_a \wedge \Bx_b ) \cdot \Bx^\mu } \cdot \partial_\mu \Bf \\
&=
\sum_\mu \int_A da db  \lr{ \Bx_a {\delta_b}^\mu - \Bx_b {\delta_a}^\mu } \cdot \partial_\mu \Bf \\
&=
\int_A da db  \lr{ \Bx_a \cdot \PD{b}{ \Bf} - \Bx_b \cdot \PD{a}{\Bf} }
\end{aligned}
\end{dmath}

While \( \Bx_a, \Bx_b \) both depend on both parameters \( a, b \), the differential form immediately above is still a perfect integral in the variables of the partials since \( \Bx_a \) is computed with \( b \) held fixed, and \( \Bx_b \) is computed with \( a \) held fixed.  Proceeding with the integrals that match the respective partials, this gives

\begin{dmath}\label{eqn:stokesTheoremCore:240}
\int_A d^2 \Bx \cdot (\boldpartial \wedge \Bf)
=
\int
da \Bx_a \cdot \evalbar{\Bf}{\Delta b}
-\int
db \Bx_b \cdot \evalbar{\Bf}{\Delta a}
=
\int
d\Bx_a \cdot \evalbar{\Bf}{\Delta b}
-\int
d\Bx_b \cdot \evalbar{\Bf}{\Delta a}.
\end{dmath}

This shows that the boundary differential \( d^1 \Bx \) in \cref{eqn:stokesTheoremCore:140} is given by

\begin{dmath}\label{eqn:stokesTheoremCore:260}
d^1 \Bx = d\Bx_a - d\Bx_b,
\end{dmath}

where it is implied that the field in question is evaluated at the boundaries of the parameter that has been eliminated by this first integration.  This boundary integral can be interpretted as the integral around a contour, as indicated in
\cref{fig:twoParameterDifferentialBoundary:twoParameterDifferentialBoundaryFig2}.

\imageFigure{../figures/GAelectrodynamics/twoParameterDifferentialBoundaryFig2}{Contour for two parameter surface boundary.}{fig:twoParameterDifferentialBoundary:twoParameterDifferentialBoundaryFig2}{0.4}

Additionally, as with the single parameter case, a substitution of the gradient does not change the result, since any component of the gradient that lies outside of the tangent space on the surface has a zero dot product with the surface volume element \( d^2 \Bx \).
This allows the two parameter Stokes integrals to be written as

%\begin{dmath}\label{eqn:stokesTheoremCore:280}
\boxedEquation{eqn:stokesTheoremCore:280}{
\begin{aligned}
\int_A d^2 \Bx \cdot \spacegrad \psi &= \ointclockwise d\Bx \psi \\
\int_A d^2 \Bx \cdot (\spacegrad \wedge \Bf) &= \ointclockwise d\Bx \cdot \Bf.
\end{aligned}
}
%\end{dmath}

It can be shown that this two parameter Stokes integral is equivalent to Green's theorem.

         \subsection{Three parameter specialization of Stokes' theorem.}
            \input{../stokesTheorem/threeparameter.tex}
         \subsection{Using scalar volume elements}
            %
% Copyright © 2016 Peeter Joot.  All Rights Reserved.
% Licenced as described in the file LICENSE under the root directory of this GIT repository.
%

In \R{3} the area and volume elements of \cref{eqn:twoparameter:140}, and \cref{eqn:threeparameter:1481} can be reexpressed as scalars, recovering a number of the integral calculus identities that are more familiar than the wedge product variants above.

The pseudoscalar volume element can be written

\begin{dmath}\label{eqn:scalarVolumeElement:1621}
d^3 \Bx = I dV,
\end{dmath}

and the (oriented) area elements can be written as

\begin{dmath}\label{eqn:scalarVolumeElement:1641}
d^2 \Bx \ncap = I dA,
\end{dmath}

or
\begin{dmath}\label{eqn:scalarVolumeElement:1661}
d^2 \Bx = I \ncap dA.
\end{dmath}

For \( \psi \in \bigwedge^0, \Bf \in \bigwedge^1, B \in \bigwedge^2 \), this gives

\begin{subequations}
\label{eqn:scalarVolumeElement:1681}
\begin{equation}\label{eqn:scalarVolumeElement:1701}
I \int_A dA \ncap \wedge \spacegrad \psi = \ointclockwise d\Bx \psi
\end{equation}
\begin{equation}\label{eqn:scalarVolumeElement:1721}
I \int_A dA \ncap \wedge \spacegrad \wedge \Bf = \ointclockwise d\Bx \cdot \Bf
\end{equation}
\begin{equation}\label{eqn:scalarVolumeElement:1741}
\int_V dV \spacegrad \psi = \int_{\partial V} dA \ncap \psi
\end{equation}
\begin{equation}\label{eqn:scalarVolumeElement:1761}
\int_V dV \spacegrad \wedge \Bf = \int_{\partial V} dA \ncap \wedge \Bf
\end{equation}
\begin{equation}\label{eqn:scalarVolumeElement:1781}
\int dV \spacegrad \wedge B = \int_{\partial V} dA \ncap \wedge B
\end{equation}
\end{subequations}

It is straightforward to re-express all the wedge products above in their dual forms.
With \( B = I \Bf \), that is

\begin{subequations}
\label{eqn:scalarVolumeElement:1801}
\begin{equation}\label{eqn:scalarVolumeElement:1821}
\int_A dA \ncap \cross \spacegrad \psi = \ointctrclockwise d\Bx \psi
\end{equation}
\begin{equation}\label{eqn:scalarVolumeElement:1841}
\int_A dA \ncap \cdot (\spacegrad \cross \Bf) = \ointctrclockwise d\Bx \cdot \Bf
\end{equation}
\begin{equation}\label{eqn:scalarVolumeElement:1861}
\int_V dV \spacegrad \psi = \int_{\partial V} dA \ncap \psi
\end{equation}
\begin{equation}\label{eqn:scalarVolumeElement:1881}
\int_V dV \spacegrad \cross \Bf = \int_{\partial V} dA \ncap \cross \Bf
\end{equation}
\begin{equation}\label{eqn:scalarVolumeElement:1901}
\int dV \spacegrad \cdot \Bf = \int_{\partial V} dA \ncap \cdot \Bf.
\end{equation}
\end{subequations}

Each of the cross product terms above can also be put into dual forms, giving

\begin{subequations}
\label{eqn:scalarVolumeElement:1801c}
\begin{equation}\label{eqn:scalarVolumeElement:1821c}
\int_A dA \ncap \cdot \lr{ I \spacegrad \psi } = \ointclockwise d\Bx \psi
\end{equation}
\begin{equation}\label{eqn:scalarVolumeElement:1841c}
\int_A dA \ncap \cdot (\spacegrad \cdot B) = \ointctrclockwise d\Bx \cdot (I B)
\end{equation}
\begin{equation}\label{eqn:scalarVolumeElement:1881c}
\int_V dV \spacegrad \cdot B = \int_{\partial V} dA \ncap \cdot B.
\end{equation}
\end{subequations}

Note that all of
\cref{eqn:scalarVolumeElement:1861}, \cref{eqn:scalarVolumeElement:1901}, and \cref{eqn:scalarVolumeElement:1881c} all have the same form

%\begin{equation}\label{eqn:scalarVolumeElement:1881d}
\boxedEquation{eqn:scalarVolumeElement:1881d}{
\int_V dV \spacegrad \cdot A = \int_{\partial V} dA \ncap \cdot A.
}
%\end{equation}
\index{divergence theorem}

This is also true for pseudoscalar grades, which can be demonstrated by multiplying both sides of \cref{eqn:scalarVolumeElement:1741} with \( I \).
This implies that \cref{eqn:scalarVolumeElement:1881d} is valid for any \R{3} multivector, generalizing the conventional divergence theorem over a 3D volume to all spatial grades.

         \subsection{Problems}
            %
% Copyright � CCYY Peeter Joot.  All Rights Reserved.
% Licenced as described in the file LICENSE under the root directory of this GIT repository.
%
\makeproblem{Stokes' theorem relation to Green's theorem}{problem:stokesAndGreens:1}{
Show that Stokes' theorem, in its two parameter form, applied to a vector field recovers Green's theorem.
\index{Green's theorem}
\index{Stokes' theorem}
} % problem

\makeanswer{problem:stokesAndGreens:1}{

To demonstrate this, expand the LHS of the Stokes identity

\begin{dmath}\label{eqn:stokesAndGreens:20}
\int_A d^2 \Bx \cdot (\boldpartial \wedge \Bf) = \ointclockwise d\Bx \cdot \Bf.
\end{dmath}

Assuming \( u, v\) parameterization

\begin{dmath}\label{eqn:stokesAndGreens:40}
\int_A d^2 \Bx \cdot (\boldpartial \wedge \Bf)
=
\int_A (d\Bx_u \wedge d\Bx_v) \cdot (\boldpartial \wedge \Bf)
=
\int_A ((d\Bx_u \wedge d\Bx_v) \cdot \Bx^u) \cdot \partial_u \Bf
+
\int_A ((d\Bx_u \wedge d\Bx_v) \cdot \Bx^v) \cdot \partial_v \Bf
=
-\int_A du dv \Bx_v \cdot \partial_u \Bf
+
\int_A du dv \Bx_u \cdot \partial_v \Bf
=
-\int_A du dv \Bx_v \cdot \partial_u \Bf
+
\int_A du dv \lr{
-\Bx_v \cdot \partial_u \Bf
+
\Bx_u \cdot \partial_v \Bf
}.
\end{dmath}

The coordinate expansion of \( \Bf \) with respect to the tangent space coordinates is

\begin{dmath}\label{eqn:stokesAndGreens:60}
\Bf = \Bx^u f_u + \Bx^v f_v + \Bf_\perp
\end{dmath}

where \( \Bf_\perp \) lies in normal to the tangent space at the point in question.
Because \( \Bx_v \) is computed with \( u \) held fixed and \( \Bx_u \) computed with \( v \) held fixed, the area integrand can be written

\begin{dmath}\label{eqn:stokesAndGreens:80}
-\Bx_v \cdot \partial_u \Bf
+
\Bx_u \cdot \partial_v \Bf
=
-\PD{u}{}\lr{ \Bx_v \cdot \Bf }
+\PD{v}{}\lr{ \Bx_u \cdot \Bf }
=
-\PD{u}{f_v}
+\PD{v}{f_u},
\end{dmath}

which gives
\begin{dmath}\label{eqn:stokesAndGreens:100}
\int_A du dv \lr{ -\PD{u}{f_v}
+\PD{v}{f_u}
}
=
\ointclockwise d\Bx \cdot \Bf,
\end{dmath}

which recovers \cref{thm:stokesTheoremGeometricAlgebra:1660} as desired.
} % answer

            %
% Copyright © 2016 Peeter Joot.  All Rights Reserved.
% Licenced as described in the file LICENSE under the root directory of this GIT repository.
%

\makeproblem{\R{3} dual forms of Stokes' theorem.}{problem:stokesTheoremCoreProblems:1}{
Prove
\makesubproblem{}{problem:stokesTheoremCoreProblems:1:a}
\cref{eqn:scalarVolumeElement:1681},
\makesubproblem{}{problem:stokesTheoremCoreProblems:1:b}
\cref{eqn:scalarVolumeElement:1801},
\makesubproblem{}{problem:stokesTheoremCoreProblems:1:c}
and \cref{eqn:scalarVolumeElement:1801c}.
} % problem

\makeanswer{problem:stokesTheoremCoreProblems:1}{

The volume elements are
\makeSubAnswer{}{problem:stokesTheoremCoreProblems:1:a}
\begin{subequations}
\label{eqn:stokesTheoremCoreProblems:20}
\begin{dmath}\label{eqn:stokesTheoremCoreProblems:40}
d^2 \Bx \cdot \spacegrad
=
dA \gpgradeone{ I \ncap \spacegrad }
=
dA I \ncap \wedge \spacegrad
\end{dmath}
\begin{dmath}\label{eqn:stokesTheoremCoreProblems:60}
d^2 \Bx \cdot (\spacegrad \wedge \BA)
=
dA \gpgradezero{ I \ncap \spacegrad \BA }
=
dA I \ncap \wedge \spacegrad \wedge \BA
\end{dmath}
\begin{dmath}\label{eqn:stokesTheoremCoreProblems:80}
d^3 \Bx \cdot \spacegrad \phi
=
dV \gpgradetwo{ I \spacegrad \phi }
=
dV I \spacegrad \phi
\end{dmath}
\begin{dmath}\label{eqn:stokesTheoremCoreProblems:100}
d^3 \Bx \cdot (\spacegrad \wedge \BA)
=
dV \gpgradeone{ I (\spacegrad \wedge \BA) }
=
dV I \spacegrad \wedge \BA
\end{dmath}
\begin{dmath}\label{eqn:stokesTheoremCoreProblems:120}
d^3 \Bx \cdot (\spacegrad \wedge B)
=
dV \gpgradezero{ I (\spacegrad \wedge B) }
=
dV I \spacegrad \wedge B.
\end{dmath}
\end{subequations}

The corresponding boundary forms are
\begin{subequations}
\label{eqn:stokesTheoremCoreProblems:140}
\begin{equation}\label{eqn:stokesTheoremCoreProblems:160}
d\Bx \psi
\end{equation}
\begin{dmath}\label{eqn:stokesTheoremCoreProblems:180}
d\Bx \cdot \BA
\end{dmath}
\begin{dmath}\label{eqn:stokesTheoremCoreProblems:200}
d^2 \Bx \psi
=
dA I \ncap \psi
\end{dmath}
\begin{dmath}\label{eqn:stokesTheoremCoreProblems:220}
d^2 \Bx \cdot \BA
=
dA \gpgradeone{ I \ncap \BA }
=
dA I \ncap \wedge \BA
\end{dmath}
\begin{dmath}\label{eqn:stokesTheoremCoreProblems:240}
d^2 \Bx \cdot B
=
dA \gpgradezero{ I \ncap B }
=
dA I \ncap \wedge B.
\end{dmath}
\end{subequations}

Assembling these pieces back into the integrals proves the relationships.

\makeSubAnswer{}{problem:stokesTheoremCoreProblems:1:b}

To show \cref{eqn:scalarVolumeElement:1841} note that

\begin{dmath}\label{eqn:stokesTheoremCoreProblems:260}
I (\Ba \wedge \Bb \wedge \Bc)
=
\gpgradezero{ I \Ba \wedge \Bb \wedge \Bc }
=
\gpgradezero{ I \Ba (\Bb \wedge \Bc) -
I \Ba \cdot (\Bb \wedge \Bc)
}
=
\gpgradezero{ I \Ba I(\Bb \cross \Bc) }
=
- \Ba \cdot (\Bb \cross \Bc).
\end{dmath}

To show \cref{eqn:scalarVolumeElement:1901} note that

\begin{dmath}\label{eqn:stokesTheoremCoreProblems:280}
\Ba \wedge (I \BA)
=
\Ba \wedge (I \BA)
=
\gpgradethree{ \Ba I \BA }
=
\gpgradethree{ I \Ba \cdot \BA }
=
I (\Ba \cdot \BA).
\end{dmath}

\makeSubAnswer{}{problem:stokesTheoremCoreProblems:1:c}

For vector \( \Ba \), these transformations all follow from

\begin{dmath}\label{eqn:stokesTheoremCoreProblems:300}
\Ba \cross \Bf
=
\gpgradeone{ -I \Ba \wedge \Bf}
=
\gpgradeone{ -I \Ba \Bf}
=
-\gpgradeone{ \Ba I \Bf}
=
-\Ba \cdot (I \Bf)
=
\Ba \cdot B.
\end{dmath}

} % answer


      \section{Fundamental theorem of geometric calculus}
         \subsection{Fundamental Theorem of Geometric Calculus}
            %
% Copyright � 2016 Peeter Joot.  All Rights Reserved.
% Licenced as described in the file LICENSE under the root directory of this GIT repository.
%
%{
%\input{../blogpost.tex}
%\renewcommand{\basename}{fundamentalTheoremOfCalculus}
%\renewcommand{\dirname}{notes/phy1520/}
%%\newcommand{\dateintitle}{}
%%\newcommand{\keywords}{}
%
%\input{../peeter_prologue_print2.tex}
%
%\usepackage{peeters_layout_exercise}
%\usepackage{peeters_braket}
%\usepackage{peeters_figures}
%\usepackage{siunitx}
%
%\beginArtNoToc
%
%\generatetitle{Fundamental theorem of geometric calculus}
%\label{chap:fundamentalTheoremOfCalculus}

\subsection{Hypervolume integral}
We wish to generalize the concepts of line, surface and volume integrals to hypervolumes and multivector functions, and define a hypervolume integral as

\makedefinition{Multivector integral.}{dfn:fundamentalTheoremOfCalculus:240}{
Given a hypervolume parameterized by \( k \) parameters, k-volume volume element \( d^k \Bx \), and
multivector functions \( F, G \), a k-volume integral with the vector derivative acting to the right on \( G \) is written as
\begin{equation*}
\int_V d^k\Bx \rboldpartial G,
\end{equation*}
a k-volume integral with the vector derivative acting to the left on \( F \) is written as
\begin{equation*}
\int_V F d^k\Bx \lboldpartial,
\end{equation*}
and a k-volume integral with the vector derivative acting bidirectionally on \( F, G \) is written as
\begin{equation*}
\int_V F d^k\Bx \lrboldpartial G
\equiv
\int_V \lr{ F d^k\Bx \lboldpartial} G
+
\int_V F d^k\Bx \lr{ \rboldpartial G }.
\end{equation*}
The explicit meaning of these directional acting derivative operations is given by the chain rule coordinate expansion
\begin{dmath*}
F d^k \Bx \lrboldpartial G
=
F d^k \Bx \lr{ \sum_i \Bx^i {\stackrel{ \leftrightarrow }{\partial_i}} } G
=
(\partial_i F) d^k \Bx \sum_i \Bx^i G
+
F d^k \Bx \sum_i \Bx^i (\partial_i G)
\equiv
(F d^k \Bx \lboldpartial) G
+
F d^k \Bx (\rboldpartial G),
\end{dmath*}
with \( \boldpartial \) acting on \( F \) and \( G \), but not the volume element \( d^k \Bx \), which may also be a function of the implied parameterization.
} % definition

The vector derivative
% (and gradient)
may not commute with \( F, G \) nor the volume element \( d^k \Bx \), so we are forced to use some notation to indicate what the vector derivative (or gradient) acts on.
In conventional right acting cases, where there is no ambiguity, arrows will usually be omitted, but braces may also be used to indicate the scope of derivative operators.
This bidirectional notation will also be used for the gradient, especially for volume integrals in \R{3} where the vector derivative is identical to the gradient.

Some authors use the Hestenes dot notation, with overdots or primes to indicating the exact scope of multivector derivative operators, as in
\begin{dmath}\label{eqn:fundamentalTheoremOfCalculus:260}
\dot{F} d^k \Bx \dot{\boldpartial} \dot{G} =
\dot{F} d^k \Bx \dot{\boldpartial} G
+
F d^k \Bx \dot{\boldpartial} \dot{G}.
\end{dmath}
The dot notation has the advantage of emphasizing that the action of the vector derivative (or gradient) is on the functions \( F, G \), and not on the hypervolume element \( d^k \Bx \).
However, in this book, where primed operators such as \( \spacegrad' \) are used to indicate that derivatives are taken with respect to primed \( \Bx' \) variables, a mix of dots and ticks would have been confusing.
%Over arrows also have the advantage of being visually conspicuous.

\subsection{Fundamental theorem.}
\index{fundamental theorem of geometric calculus}

The fundamental theorem of geometric calculus is a generalization of many conventional scalar and vector integral theorems, and relates a hypervolume integral to its boundary.
This is a a powerful theorem, which we will use with Green's functions to solve Maxwell's equation, but also to derive the geometric algebra form of Stokes' theorem, from which most of the familiar integral calculus results follow.

\maketheorem{Fundamental theorem of geometric calculus}{thm:fundamentalTheoremOfCalculus:1}{
Given
multivectors \(F, G \),
a parameterization \( \Bx = \Bx(u_1, u_2, \cdots) \), with hypervolume element \( d^k \Bx = d^k u I_k \), where
\( I_k = \Bx_1 \wedge \Bx_2 \wedge \cdots \wedge \Bx_k \), the hypervolume integral is related to the boundary integral by
\begin{equation*}
\int_V F d^k \Bx \lrboldpartial G = \oint_{\partial V} F d^{k-1} \Bx G,
\end{equation*}
where \( \partial V \) represents the boundary of the volume, and \( d^{k-1} \Bx \) is the hypersurface element.  The hypersurface element and boundary integral is defined for \( k > 1 \) as
\begin{equation*}
\oint_{\partial V} F d^{k-1} \Bx G
\equiv
\sum_i \int d^{k-1} u_i \evalbar{ \lr{ F \lr{ I_k \cdot \Bx^i} G }}{\Delta u_i},
\end{equation*}
where \( d^{k-1} u_i \) is the product of all \( du_j \) except for \( du_i \).
For
\( k = 1 \) the hypersurface element and associated
boundary ``integral''
is really just convenient general shorthand, and
should be taken to mean the evaluation of the \( F G \) multivector product over the range of the parameter
\begin{equation*}
\oint_{\partial V} F d^{0} \Bx G
\equiv
\evalbar{ F G }{\Delta u_1}.
\end{equation*}
} % theorem

The geometry of the hypersurface element \( d^{k-1} \Bx \) will be made more clear when we
consider the specific cases of \( k = 1, 2, 3 \), representing generalized line, surface, and volume integrals respectively.
Instead of terrorizing the reader with a general proof
\cref{thm:fundamentalTheoremOfCalculus:1},
which requires some unpleasant index gymnastics,
this book
will separately state and prove the fundamental theorem of calculus
for each of the \( k = 1, 2, 3 \) cases that are of interest for problems in \R{2} and \R{3}.
For the interested reader, a sketch of the general proof
of \cref{thm:fundamentalTheoremOfCalculus:1}
is available in \cref{chap:gagcProof}.

Before moving on to the line, surface, and volume integral cases, we will state and prove the
general Stokes' theorem in its geometric algebra form.

%}
%\EndArticle

         \subsection{Green's function for the gradient in Euclidean spaces.}
            \input{../gabookI/calculus/gradientGreensFunction.tex}
            % example:
            \input{../gabookI/calculus/biotSavartGreens.tex}
         %\subsection{Problems}
         \subsection{Helmholtz theorem}
            %
% Copyright © 2016 Peeter Joot.  All Rights Reserved.
% Licenced as described in the file LICENSE under the root directory of this GIT repository.
%
\index{Helmholtz's theorem}
In conventional electromagnetism Maxwell's equations are posed in terms of separate divergence and curl equations.  It is therefore desirable to show that the divergence and curl of a function and it's normal characteristics on the boundary of an integraion volume determine that function uniquely.  This is known as the Helmholtz theorem
\maketheorem{Helmholtz first theorem.}{thm:helmholtzDerviationMultivectorStatement:1}{
A vector \( \BM \) is uniquely determined by its
divergence
\begin{equation*}
\spacegrad \cdot \BM = s,
\end{equation*}
and curl
\begin{equation*}
\spacegrad \cross \BM = \BC,
\end{equation*}
and its value
over the boundary.
} % theorem

%It could be argued that Helmholtz's theorem is irrelavent when using the GA formalism, since we consolidate the separate divergence and curl equations into one gradient operator.
%We include a proof here regardless, since it can be performed in a compact and interesting fashion using
%%the fundamental theorem of geometric calculus
%\cref{thm:fundamentalTheoremOfCalculus:1}.

            %
% Copyright © 2016 Peeter Joot.  All Rights Reserved.
% Licenced as described in the file LICENSE under the root directory of this GIT repository.
%
%{
The conventional proof of Helmholtz's theorem uses the Green's function for the (second order) Helmholtz operator.
Armed with a vector valued Green's function for the gradient, a first order proof is also possible.
As illustrations of the geometric integration theory developed in this chapter, both
strategies will be applied here to this problem.

In either case, we start by forming an even grade multivector (gradient) equation containing both the dot and cross product contributions
\begin{equation}\label{eqn:helmholtzDerviationMultivectorSolution:60}
\spacegrad \BM
= \spacegrad \cdot \BM + I \spacegrad \cross \BM
= s + I \BC.
\end{equation}

\paragraph{First order proof.}

For the first order case, we
perform a grade one selection of \cref{lemma:greensFunctionOverview:420}, setting
\( F = \BM \) where \( G \) is the Green's function for the gradient given by
\cref{eqn:greensFunctionFirstOrderHelmholtz:900}.  The proof follows directly

\begin{equation}\label{eqn:helmholtzDerviationMultivectorSolution:820}
\begin{aligned}
M(\Bx)
&= - \int_V \lr{ G(\Bx, \Bx') \lspacegrad' } \BM(\Bx') dV' \\
&= \int_V \gpgradeone{G(\Bx, \Bx') \lr{ \rspacegrad' \BM(\Bx') }} dV'
-
\int_{\partial V} \gpgradeone{ G(\Bx, \Bx') \ncap' \BM(\Bx') } dA' \\
&=
\int_V
\inv{4 \pi \Norm{\Bx - \Bx'}^3 }
\gpgradeone{ (\Bx - \Bx') \lr{ s(\Bx') + I \BC(\Bx') }} dV' \\
&\quad -
\int_{\partial V}
\inv{4 \pi \Norm{\Bx - \Bx'}^3 }
\gpgradeone{ (\Bx - \Bx') \ncap' \BM(\Bx') } dA' \\
&=
\int_V
\inv{4 \pi \Norm{\Bx - \Bx'}^3 }
\lr{ (\Bx - \Bx') s(\Bx') - (\Bx - \Bx') \cross \BC(\Bx') } dV' \\
&\quad -
\int_{\partial V}
\inv{4 \pi \Norm{\Bx - \Bx'}^3 }
\gpgradeone{ (\Bx - \Bx') \ncap' \BM(\Bx') } dA'.
\end{aligned}
\end{equation}
If \( \BM \) is well behaved enough that the boundary integral vanishes on an infinite surface, we see that \( \BM \) is completely specified by the divergence and the curl.
In general, the divergence and the curl, must also be supplemented by the value of vector valued function on the boundary.

Observe that the boundary integral has a particularly simple form for a spherical surface or radius \( R \) centered on \( \Bx' \).
Switching to spherical coordinates \( \Br = \Bx' - \Bx = R\, \rcap(\theta, \phi) \) where \( \rcap = (\Bx' - \Bx)/\Norm{\Bx' - \Bx} \) is the outwards normal, we have
\begin{equation}\label{eqn:helmholtzDerviationMultivectorSolution:840}
\begin{aligned}
-
\int_{\partial V} &
\inv{4 \pi \Norm{\Bx - \Bx'}^3 }
\gpgradeone{ (\Bx - \Bx') \ncap' \BM(\Bx') } dA' \\
&= \int_{\partial V} \frac{\BM(\Bx')}{4 \pi \Norm{\Bx - \Bx'}^2 } dA' \\
&= \inv{4\pi} \int_{\theta = 0}^\pi \int_{\phi = 0}^{2 \pi} \BM(R, \theta, \phi) \sin\theta d\theta d\phi.
\end{aligned}
\end{equation}
This is an average of \( \BM \) over the surface of the radius-\(R\) sphere surrounding the point \( \Bx \) where the field \( \BM \) is evaluated.

\paragraph{Second order proof.}

%Observe that the Laplacian of \( \BM \) is vector valued
%
%\begin{equation}\label{eqn:helmholtzDerviationMultivectorSolution:760}
%\spacegrad^2 \BM = \spacegrad s + I \spacegrad \BC.
%\end{equation}
%
%This means that \( \spacegrad \BC \) must be a bivector \( \spacegrad \BC = \spacegrad \wedge \BC \), or that \( \BC \) has zero divergence
%
%\begin{equation}\label{eqn:helmholtzDerviationMultivectorSolution:780}
%\spacegrad \cdot \BC = 0.
%\end{equation}

Again, we use \cref{eqn:helmholtzDerviationMultivectorSolution:60}
to discover the relation between the vector \( \BM \) and its divergence and curl.
\index{delta function}
The vector \( \BM \) can be expressed at the point of interest as a convolution with the delta function at all other points in space
\index{convolution}
\begin{equation}\label{eqn:helmholtzDerviationMultivectorSolution:80}
\BM(\Bx) = \int_V dV'\, \delta(\Bx - \Bx') \BM(\Bx').
\end{equation}

\index{Laplacian}
The Laplacian representation of the delta function in \R{3} is
\begin{equation}\label{eqn:helmholtzDerviationMultivectorSolution:100}
\delta(\Bx - \Bx') = -\inv{4\pi} \spacegrad^2 \inv{\Norm{\Bx - \Bx'}},
\end{equation}
so \( \BM \) can be represented as the following convolution
\begin{equation}\label{eqn:helmholtzDerviationMultivectorSolution:120}
\BM(\Bx) = -\inv{4\pi} \int_V dV'\, \spacegrad^2 \inv{\Norm{\Bx - \Bx'}} \BM(\Bx').
\end{equation}

%As noted in \cref{eqn:helmholtzDerviationMultivector:460} the Laplacian of a vector can be factored as
%
%\begin{equation}\label{eqn:helmholtzDerviationMultivectorSolution:140}
%\spacegrad^2 \Ba
%=
%\spacegrad (\spacegrad \cdot \Ba)
%-
%\spacegrad \cross (\spacegrad \cross \Ba).
%\end{equation}
%
%Note that the last term can be written in cross product notation using \( \Bc \cdot (\Ba \wedge \Bb) = -\Bc \cross (\Ba \cross \Bb) \) if desired.

Using this relation and proceeding with a few applications of the chain rule, plus the fact that \( \spacegrad 1/\Norm{\Bx - \Bx'} = -\spacegrad' 1/\Norm{\Bx - \Bx'} \), we find
%
%I previously posted a Geometric Algebra attack on the Helmholtz theorem.  Here is
%
%Here's a third way of deriving the Helmholtz theorem inversion relation.  This is a refinement of the traditional vector algebra solution that led to \cref{eqn:helmholtzDerviationMultivector:200}, that uses a factorization of the Laplacian directly, deferring any expansion in terms of dot and cross (or wedge) products until the very end.
%
%Starting from the first line of \cref{eqn:helmholtzDerviationMultivector:160}, we have
\begin{equation}\label{eqn:helmholtzDerviationMultivectorSolution:720}
\begin{aligned}
-4 &\pi \BM(\Bx) \\
&= \int_V dV'\, \spacegrad^2 \inv{\Norm{\Bx - \Bx'}} \BM(\Bx') \\
&= \gpgradeone{\int_V dV'\, \spacegrad^2 \inv{\Norm{\Bx - \Bx'}} \BM(\Bx')} \\
&= -\gpgradeone{\int_V dV'\, \spacegrad \lr{ \spacegrad' \inv{\Norm{\Bx - \Bx'}}} \BM(\Bx')} \\
&= -\gpgradeone{\spacegrad \int_V dV' \lr{
   \spacegrad' \frac{\BM(\Bx')}{\Norm{\Bx - \Bx'}}
   -\frac{\spacegrad' \BM(\Bx')}{\Norm{\Bx - \Bx'}}
   } } \\
&= -\gpgradeone{\spacegrad \int_{\partial V} dA'\,
   \ncap \frac{\BM(\Bx')}{\Norm{\Bx - \Bx'}}
    }
   +\gpgradeone{\spacegrad \int_V dV'
   \frac{s(\Bx') + I\BC(\Bx')}{\Norm{\Bx - \Bx'}}
    } \\
&= -\gpgradeone{\spacegrad \int_{\partial V} dA'\,
   \ncap \frac{\BM(\Bx')}{\Norm{\Bx - \Bx'}}
    }
   +\spacegrad \int_V dV'\,
   \frac{s(\Bx')}{\Norm{\Bx - \Bx'}}
   +\spacegrad \cdot \int_V dV'
   \frac{I\BC(\Bx')}{\Norm{\Bx - \Bx'}}.
\end{aligned}
\end{equation}

By inserting a no-op grade selection operation in the second step, the trivector terms that would show up in subsequent steps are automatically filtered out.
%the troublesome trivector term that shows up in my first purely Geometric Algebra
%attempt is eliminated.
This leaves us with a boundary term dependent on the surface and the normal and tangential components of \( \BM \).
Added to that is a pair of volume integrals that provide the unique dependence of \( \BM \) on its divergence and curl.
When the surface is taken to infinity, which requires \( \Norm{\BM}/\Norm{\Bx - \Bx'} \rightarrow 0 \), then the dependence of \( \BM \) on its divergence and curl is unique.

In order to express final result in traditional vector algebra form, a couple transformations are required.
The first is that
\begin{equation}\label{eqn:helmholtzDerviationMultivectorSolution:800}
\gpgradeone{ \Ba I \Bb } = I^2 \Ba \cross \Bb = -\Ba \cross \Bb.
\end{equation}

For the grade selection in the boundary integral, note that
\begin{equation}\label{eqn:helmholtzDerviationMultivectorSolution:740}
\begin{aligned}
\gpgradeone{ \spacegrad \ncap \BX }
&= \gpgradeone{ \spacegrad (\ncap \cdot \BX) } + \gpgradeone{ \spacegrad (\ncap \wedge \BX) } \\
&= \spacegrad (\ncap \cdot \BX) + \gpgradeone{ \spacegrad I (\ncap \cross \BX) } \\
&= \spacegrad (\ncap \cdot \BX) - \spacegrad \cross (\ncap \cross \BX).
\end{aligned}
\end{equation}

These give
%\begin{equation}\label{eqn:helmholtzDerviationMultivectorSolution:721}
\boxedEquation{eqn:helmholtzDerviationMultivectorSolution:721}{
\begin{aligned}
\BM(\Bx)
&=
\spacegrad \inv{4\pi} \int_{\partial V} dA'\, \ncap \cdot \frac{\BM(\Bx')}{\Norm{\Bx - \Bx'}}
-
\spacegrad \cross \inv{4\pi} \int_{\partial V} dA'\, \ncap \cross \frac{\BM(\Bx')}{\Norm{\Bx - \Bx'}} \\
&-\spacegrad \inv{4\pi} \int_V dV'
\frac{s(\Bx')}{\Norm{\Bx - \Bx'}}
+\spacegrad \cross \inv{4\pi} \int_V dV'
\frac{\BC(\Bx')}{\Norm{\Bx - \Bx'}}.
\end{aligned}
}
%\end{equation}
%}

      \section{Problem solutions}
         \shipoutAnswer

\part{Electromagnetism}
   \chapter{Maxwell's equations}
      %
% Copyright © 2016 Peeter Joot.  All Rights Reserved.
% Licenced as described in the file LICENSE under the root directory of this GIT repository.
%
\section{Conventional differential form}

The differential form of Maxwell's equations, with extensions for magnetic sources, is the starting point for all the analysis in these notes.  Those equations are

\input{../ece1229-antenna/MaxwellsStatement.tex}

The magnetic sources can be considered fictional, and are included because they are useful in antenna theory to model real phenomina such as infinitesimal current loops.

\input{../ece1229-antenna/MaxwellsFieldAndSourceDescription.tex}

These fields and sources are all real valued.  In many situations it will be desirable to work with a time harmonic (frequency-domain phasor) form of Maxwell's equations.  In engineering, a time harmonic representation presumes that all sources and fields have a frequency dependence of the form
\index{time harmonic}

\begin{dmath}\label{eqn:maxwellsEquations:20}
\bcY(\Bx, t) = \Real( \BY(\Bx, \omega) e^{j\omega t} ),
\end{dmath}

where the field (or source) \( \BY(\Bx, \Bomega) \) is allowed to be complex valued, whereupon Maxwell's equations take the form

\input{../ece1229-antenna/MaxwellsTimeHarmonic.tex}

Note that the time harmonic convention typically used in physics literature presumes a frequency dependence of the form

\begin{dmath}\label{eqn:maxwellsEquations:40}
\bcY(\Bx, t) = \Real( \BY(\Bx, \omega) e^{-i\omega t} ),
\end{dmath}

which alters the sign of any imaginary originating from a time derivative.  Care is required by the reader to understand which form of frequency dependence has been assumed.

\section{GA differential form}

Geometric Algebra admits a number of alternative representations of Maxwell's equations.  The first follows from expressing the cross products all as wedge products, leaving a pair of bivector and a pair of scalar equations

\begin{subequations}
\begin{dmath}\label{eqn:maxwellsEquations:60}
\spacegrad \wedge \bcE = - I \bcM - \PD{t}{I\bcB}
\end{dmath}
\begin{dmath}\label{eqn:maxwellsEquations:80}
\spacegrad \wedge \bcH = I \bcJ + I \PD{t}{\bcD}
\end{dmath}
\begin{dmath}\label{eqn:maxwellsEquations:100}
\spacegrad \cdot \bcD = q_\txte
\end{dmath}
\begin{dmath}\label{eqn:maxwellsEquations:120}
\spacegrad \cdot \bcB = q_\txtm.
\end{dmath}
\end{subequations}

Alternatively, the duality transformation \( \Ba \wedge \Bb = -I \Ba \cdot (I \Bb) \) allows Maxwell's equations to be all written as dot products

\begin{subequations}
\begin{dmath}\label{eqn:maxwellsEquations:140}
\spacegrad \cdot (I \bcE) = \bcM + \PD{t}{\bcB}
\end{dmath}
\begin{dmath}\label{eqn:maxwellsEquations:160}
\spacegrad \cdot (I \bcH) = -\bcJ - \PD{t}{\bcD}
\end{dmath}
\begin{dmath}\label{eqn:maxwellsEquations:180}
\spacegrad \cdot \bcD = q_\txte
\end{dmath}
\begin{dmath}\label{eqn:maxwellsEquations:200}
\spacegrad \cdot \bcB = q_\txtm,
\end{dmath}
\end{subequations}

or, using the duality transformation \( \Ba \cdot \Bb = -I (\Ba \wedge (I \Bb) \), Maxwell's equations can all be written as wedge products

\begin{subequations}
\begin{dmath}\label{eqn:maxwellsEquations:220}
\spacegrad \wedge \bcE = - I \bcM - \PD{t}{I\bcB}
\end{dmath}
\begin{dmath}\label{eqn:maxwellsEquations:240}
\spacegrad \wedge \bcH = I \bcJ + I \PD{t}{\bcD}
\end{dmath}
\begin{dmath}\label{eqn:maxwellsEquations:260}
\spacegrad \wedge (I\bcD) = I q_\txte
\end{dmath}
\begin{dmath}\label{eqn:maxwellsEquations:280}
\spacegrad \wedge (I\bcB) = I q_\txtm.
\end{dmath}
\end{subequations}

Each of these forms can be useful in different circumstances, however the real power of GA in electromagnetism follows from presuming constituative relationships between the pairs of electric and magnetic fields

\begin{subequations}
\label{eqn:maxwellsEquations:300}
\begin{dmath}\label{eqn:maxwellsEquations:320}
\bcB = \mu \bcH
\end{dmath}
\begin{dmath}\label{eqn:maxwellsEquations:340}
\bcD = \epsilon \bcE,
\end{dmath}
\end{subequations}

where \( \epsilon \) is the permitivitity of the medium [\si{F/m}] (Farads/meter), and \( \mu \) is the permeability of the medium [\si{H/m}] (Henries/meter).
The permitivitity and permeability may be functions of both time and position, and model the materials that the fields are propagating through.  In general, the these may be non-isotropic tensor operators, however, unless otherwise specified, isotropic media will be assumed in these notes.

With this constitutative relationship assumed (and a bit of rescaling), the dot and wedge products of \cref{eqn:maxwellsEquations:60}, \cref{eqn:maxwellsEquations:100} can be added, as can those of \cref{eqn:maxwellsEquations:80}, \cref{eqn:maxwellsEquations:120}.  This reduces Maxwell's equations to a pair of first order coupled gradient equations

\begin{subequations}
\begin{dmath}\label{eqn:maxwellsEquations:360}
\spacegrad \bcE = \inv{\epsilon} q_\txte - I \bcM - \mu \PD{t}{(I\bcH)}
\end{dmath}
\begin{dmath}\label{eqn:maxwellsEquations:380}
\spacegrad (I \bcH) = \frac{I q_\txtm}{\mu} - \bcJ - \epsilon \PD{t}{\bcE}.
\end{dmath}
\end{subequations}



      \section{Problem solutions}
         \shipoutAnswer
   \chapter{Electrostatics}
      \section{Problem solutions}
         \shipoutAnswer
   \chapter{Magnetostatics}
      \section{Problem solutions}
         \shipoutAnswer
   \chapter{Constitutive relations}
      \section{Problem solutions}
         \shipoutAnswer
   \chapter{Boundary value conditions}
      \section{Problem solutions}
         \shipoutAnswer
%   \chapter{Time harmonic fields}
%      \section{Frequency domain}
%         \input{../frequencydomain/frequencydomainMaxwells.tex}
%      \section{Plane waves}
%         %
% Copyright © 2016 Peeter Joot.  All Rights Reserved.
% Licenced as described in the file LICENSE under the root directory of this GIT repository.
%
%\section{Plane waves}

The gradient action on the electromagnetic field is

\begin{dmath}\label{eqn:frequencydomainCore:160}
\spacegrad F_0 e^{-j \Bk \cdot \Bx}
=
\sum_{m = 1}^3 \Be_m \partial_m
F_0 e^{-j \Bk \cdot \Bx}
=
\sum_{m = 1}^3 \Be_m
F_0
\lr{ -j k_m }
e^{-j \Bk \cdot \Bx}
=
-j \Bk F_0,
\end{dmath}

so

\begin{dmath}\label{eqn:frequencydomainCore:180}
j k (1 - \kcap) F_0 = 0.
\end{dmath}

This means that the field must be of the form

%\begin{dmath}\label{eqn:frequencydomainCore:200}
\boxedEquation
{eqn:frequencydomainCore:200}
{
F = (1 + \kcap) \BE_0 e^{-j \Bk \cdot \Bx},
}
%\end{dmath}

where \( \BE_0 \) is a vector valued complex constant, and \( \kcap \cdot \BE_0 = 0 \).  The dot product constraint follows from the requirement that the \( I \BH \propto \kcap \BE_0 \) portion of the electromagnetic field is a bivector.

From \cref{eqn:frequencydomainCore:200} the interdependence of the electric and magnetic field portions of the field can be read off immediately.  Those are

\begin{subequations}
\label{eqn:frequencydomainCore:220}
\begin{dmath}\label{eqn:frequencydomainCore:221}
\BE = \BE_0 e^{-j \Bk \cdot \Bx} 
\end{dmath}
\begin{dmath}\label{eqn:frequencydomainCore:222}
I \BH = \inv{\eta} \kcap \BE_0 e^{-j \Bk \cdot \Bx},
\end{dmath}
\end{subequations}

or
\begin{dmath}\label{eqn:frequencydomainCore:380}
I \BH = \inv{\eta} \kcap \BE.
\end{dmath}

Since the \R{3} pseudoscalar can be written as

\begin{dmath}\label{eqn:frequencydomainCore:400}
I = \kcap \Ecap \Hcap,
\end{dmath}

the directions \( \kcap, \Ecap, \Hcap \) must form a right handed triple.  It is thus expected that the magnetic field is perpendicular to the propagation direction, and that the electric and magnetic fields are explicitly perpendicular, facts that are easily verified

\begin{subequations}
\label{eqn:frequencydomainCore:440}
\begin{dmath}\label{eqn:frequencydomainCore:260}
\kcap \cdot \BH
= \gpgradezero{ \kcap (-I \kcap \BE_0) } e^{-j \Bk \cdot \Bx}
= -\gpgradezero{ I \BE_0 } e^{-j \Bk \cdot \Bx}
= 0
\end{dmath}
\begin{dmath}\label{eqn:frequencydomainCore:280}
\BE \cdot \BH
=
\gpgradezero{ \BE \lr{ -\frac{I}{\eta}} \kcap \BE }
=
-\inv{\eta} \BE^2
\gpgradezero{ \kcap I }
=
0.
\end{dmath}
\end{subequations}

In conventional vector treatments of electromagnetic field theory the field relationships of \cref{eqn:frequencydomainCore:220} and the propagation directions are written out explicitly as cross products, instead of multivector equations.  Those cross product relations are obtained easily

\begin{subequations}
\label{eqn:frequencydomainCore:420}
\begin{dmath}\label{eqn:frequencydomainCore:240}
\BH
= -I \inv{\eta} \kcap \BE
= -I \inv{\eta} (\kcap \wedge \BE)
= -I \inv{\eta} I (\kcap \cross \BE)
= \inv{\eta} \kcap \cross \BE
\end{dmath}
\begin{dmath}\label{eqn:frequencydomainCore:300}
\BE
= \eta \kcap I \BH
= \eta I \kcap \wedge \BH
= \eta I^2 \kcap \cross \BH
= \eta \BH \cross \kcap
\end{dmath}
\begin{dmath}\label{eqn:frequencydomainCore:340}
\kcap
= I \Hcap \Ecap
= I (\Hcap \wedge \Ecap)
= I^2 (\Hcap \cross \Ecap)
= \Ecap \cross \Hcap.
\end{dmath}
\end{subequations}

%      \section{Problem solutions}
%         \shipoutAnswer
   \chapter{Polarization}
      \section{Problem solutions}
         \shipoutAnswer
   \chapter{Potentials}
      \section{Problem solutions}
         \shipoutAnswer
   \chapter{Green's functions}
      \section{Problem solutions}
         \shipoutAnswer
   \chapter{Wave equations}
      \section{Problem solutions}
         \shipoutAnswer
   \chapter{Radiation and scattering}
      \section{Problem solutions}
         \shipoutAnswer
%\end{itemize}



