%
% Copyright � 2016 Peeter Joot.  All Rights Reserved.
% Licenced as described in the file LICENSE under the root directory of this GIT repository.
%
%----------------------------------------------------------------------------------------
\part{Geometric Algebra.}
   \chapter{Geometric Algebra.}
      \section{Prerequisites.}
         
\section{Junk?}

\subsection{Problems}
%\makeproblem{Explicit squared norm}{problem:multivector:60}{
%   Given a coordinate representation of a vector with respect to a standard basis
%\begin{dmath}\label{eqn:multivector:240}
%   \Bx = \sum_{i = 1}^N x_i \Be_i,
%\end{dmath}
%
%show that the squared norm is
%\begin{dmath}\label{eqn:multivector:260}
%   \Norm{\Bx}^2 = \Bx \cdot \Bx = \sum_{i = 1}^N x_i^2 (\Be_i \cdot \Be_i).
%\end{dmath}
%
%Observe that for a Euclidean vector space this is the squared length in the Pythagorean sense.
%}
%
\makeproblem{Null vector}{problem:multivector:80}{
Given a two dimensional non-Euclidean vector space with basis elements satisfying
\( \gamma_0 \cdot \gamma_0 = 1 = -\gamma_1 \cdot \gamma_1 \), construct a vector that has a squared
norm of 0.  Such a vector is called a null vector.
%   \Bx = \gamma_0 + \gamma_1,
}


\subsection{basis, norm, ...}

%We will use a representation such as \( \Bv = x \Be_1 + y \Be_2 + z \Be_3 \) for such vectors, where the
%coordinates are always paired with their respective direction vectors, and will not use
%column vector of coordinates or tuples such as \( \Bv = (x, y, z)\).
%, \Bv = x \xcap + y \ycap + z \zcap, or \Bv = x \ahat_x + y \ahat_y + z \ahat_z.
\makedefinition{Coordinates.}{dfn:prerequisites:coordinates}{
%Given a basis \( B =
FIXME: define
} % definition

%\makedefinition{Basis and coordinates}{dfn:multivector:basis}{
%   If \( N \) is the dimension of a vector space \( V \), a set of \( N \) vectors \( B = \setlr{ \Ba_1, \Ba_2, \cdots , \Ba_N } \) is a basis for that vector space, if it is possible to form any vector \( \Bx \in V \) as a linear combination of those vectors \( \Ba_k \).  That is, there exists scalars \( c_k \) such that for any \( \Bx \in V \)
%
%\begin{equation*}
%   \Bx = \sum_{k = 1}^N c_k \Ba_k.
%\end{equation*}
%
%The numbers \( (c_1, c_2, \cdots, c_N ) \) are referred to as the coordinates of the vector \( \Bx \) with respect to the basis \( B \).
%}

\makedefinition{Standard basis, and dot product properties.}{dfn:multivector:standardbasis}{
   Any vector space \( V \) used in this book will be assumed to have been generated from a basis \( \setlr{ \Be_1, \Be_2, \cdots, \Be_N } \), associated with a dot product that has the properties

\begin{enumerate}
   \item \( \Be_i \cdot \Be_i = \pm 1 \).
   \item \( \Be_i \cdot \Be_j = 0 \) for any \( i \ne j \).
\end{enumerate}

Such a basis will called a standard basis.  When these dot products are always positive, the vector space is referred to as a Euclidean vector space.
}

\paragraph{FIXME: remove?}
There are many possible standard bases sets.  In \R{3}, it is conventional to refer to \( \Be_1, \Be_2, \Be_3 \) as the standard bases elements if these represent the directions of the x, y, and z directions respectively.  Unless otherwise noted \( \Be_k \) refers to the direction vector for the k-th direction in a standard basis for that space.
The only non-Euclidean vector space of interest in this book (for relativistic material), has a Minkowski dot product.  For such a space, the standard basis elements will be labeled \( \setlr{ \gamma_0, \gamma_1, \gamma_2, \gamma_3 } \), where for \( i \in [1,3] \), \( \gamma_0 \cdot \gamma_0 = \pm 1 = -\gamma_i \cdot \gamma_i \).  The positive sign convention will be used.

%GA requires the vector space to have an associated
%dot product \( \Bx \cdot \By \) that
%defines the notion of perpendicularity for the space.  We will want to extend the scalar multiplication operation of the vector
%space to complex numbers, but
%will not require a (complex) order dependent inner product \( \innerprod{\Bx}{\By} \) for our vector space.
%

\paragraph{The metric, length and normality.}

An abstract vector need not have an associated notion of length, nor a notion of perpendicularity (normality).
In abstract vector algebra, length and normality are provided by defining an associated dot product \(\Bx \cdot \By\), or inner product \(\innerprod{\Bx}{\By}\).
In GA, length and normality of two vectors are provided by a metric \(g(\Bx, \By)\).
Like the dot product where \( \Bx \cdot \By = \By \cdot \Bx\), this metric is independent of order, a property that is not generally required of the inner product.
However, unlike both the dot and inner products of abstract vector algebra, where \( \Bx \cdot \Bx \ge 0\), and \( \innerprod{\Bx}{\Bx} \ge 0\), the metric \(g(\Bx, \Bx)\) may be negative (i.e. for spacetime vectors).
If \(c \) is any real or complex number, the metric in GA is \( g(c \Bx, c \Bx) = c^2 g(\Bx, \Bx)\), unlike the inner product in complex spaces, where \( \innerprod{c \Bx}{c \Bx} = \Abs{c}^2 \innerprod{c \Bx}{c \Bx} \).
Effectively, this means that our underlying direction vectors are always real.

\subsection{Orientation}
We are familiar with the idea of an oriented line segment (a vector), a quantity that can be visualized as an arrow with direction and magnitude.
The idea of an oriented plane, volume, or hypervolume is probably less familiar.
An oriented plane segment, in addition to having a specific area and a direction in space, can be visualized as having a
circulation direction, or handedness.
In a three dimensional space, this circulation direction can be associated with one of the two possible normal directions for the plane.
An oriented volume, in addition to having a given magnitude, is considered to have an associated circulation direction along its surface.
In a three dimensional space, an oriented volume can be conceptualized as a volume with either an inwards or outwards normal.

\subsection{dot and metric original text}

Vectors are often represented with an implied basis, with tuples like \( \Bx = (x,y,z) \), or with column (or row) vectors like
\(
   \Bx =
\begin{bmatrix}
x \\
y \\
z
\end{bmatrix}
\).
The values \( x, y, z \) in these representations are called the coordinates of the vectors, but only have specific meaning once a direction and magnitude is associated with each coordinate (i.e. a basis is chosen).
In three dimensions, the simplest such basis choice (the standard basis), associates the respective coordinates with a set of mutually perpendicular (normal) directions.
This is conventionally a right handed triple of direction vectors of unit length, perhaps designated \( \xcap, \ycap, \zcap \) or \( \Be_1, \Be_2, \Be_3 \).

In GA, when working with coordinates, we generally prefer to make the basis explicit, so instead of writing a vector as a set of coordinates, these coordinates
will be explicitly paired with their associated basis vectors.
For example in \R{3} a vector with coordinates \( x, y, z \) will be written as

\begin{dmath}\label{eqn:prerequisites:280}
x \Be_1 + y \Be_2 + z \Be_3.
\end{dmath}

By convention, we understand that \( \Be_1, \Be_2, \Be_3 \) in \cref{eqn:prerequisites:280} are unit length vectors, and are all mutually perpendicular (orthonormal).
The vector space must be augmented with a dot product (or inner product) to provide a measure of length and normality.  

%\makedefinition{Inner product.}{dfn:prerequisites:innerproduct}{
%The inner product 
%} % definition

For \R{3}, the dot product satisfies the following conditions

\begin{equation}\label{eqn:prerequisites:320}
\Be_i \cdot \Be_j = \delta_{ij} \, \forall i, j \in [1,3],
\end{equation}

where \( \delta_{ij} \) is the Kronecker delta \( \delta_{ij} = 1 \) for \( i = j \) and \( \delta_{ij} = 0 \) for \( i \ne j \).
Specifying the action of the dot product on all the unit vectors, completely specifies the action of the dot product on any two vectors, provided one assumes that the dot product is a bilinear operator.
For example, given

\begin{dmath}\label{eqn:prerequisites:340}
\begin{aligned}
\Ba &= a_1 \Be_1 + a_2 \Be_2 + a_3 \Be_3 \\
\Bb &= b_1 \Be_1 + b_2 \Be_2 + b_3 \Be_3,
\end{aligned}
\end{dmath}

or \( \Ba = \sum_i a_i \Be_i, \Bb = \sum_j b_j \Be_j \), we recover the familiar coordinate description of the dot product

\begin{dmath}\label{eqn:prerequisites:360}
\Ba \cdot \Bb
=
\lr{ \sum_i a_i \Be_i } \cdot \lr{ \sum_j b_j \Be_j }
=
\sum_{i,j} a_i b_j \lr{ \Be_i \cdot \Be_j }
=
\sum_{i,j} a_i b_j \delta_{ij}
=
\sum_{i} a_i b_i.
\end{dmath}

Electromagnetism is intrinsically relativistic, and there will be circumstances where vectors with both space and time components are required.
In physics, these are called four-vectors, but we will call them spacetime vectors here to avoid confusion with \( k = 4 \) k-vectors.
Following \citep{doran2003gap}, the Dirac (matrix) notation will be used as the relativistic basis, so a spacetime vector might be written like

\begin{dmath}\label{eqn:prerequisites:300}
A = c t \gamma_0 + x \gamma_1 + y \gamma_2 + z \gamma_3.
\end{dmath}

It will be seen later that our spacetime vector representation has similar properties to Dirac matrices, but we need not refer to any specific matrix representation.

For spacetime vectors, we can also assume a dot product operation between the basis vectors.  For example, given two spacetime vectors

\begin{dmath}\label{eqn:prerequisites:380}
\begin{aligned}
A &= c t \gamma_0 + x \gamma_1 + y \gamma_2 + z \gamma_3 \\
B &= c t' \gamma_0 + x' \gamma_1 + y' \gamma_2 + z' \gamma_3,
\end{aligned}
\end{dmath}

if the action of a ``dot-product'' is known between all basis vectors \( \gamma_\mu, \mu \in [0,3] \), then it will be possible to compute the dot-product of any pair of four vectors as done above for the \R{3} example.  Special relativity constrains the properties of four-vector dot products, requiring the following of the four-vector basis

\begin{dmath}\label{eqn:prerequisites:400}
\left\{
\begin{array}{l l}
\gamma_\mu \cdot \gamma_\nu = 0 & \quad \mbox{ \( \mu \ne \nu ; \mu, \nu \in [0,3] \) } \\
\gamma_0 \cdot \gamma_0 = -\gamma_i \cdot \gamma_i = \pm 1 & \quad \mbox{ \( i \in [1,3] \) }
\end{array}
\right.
\end{dmath}

Strictly speaking, this is a specification of a metric, not a dot product, since this four vector dot product specification does not satisfy the positive definite property required by most dot product definitions (i.e. \( A \cdot A \ge 0 \)).
There is a sign ambiguity in the metric specification above.  The physics of relativity is independent of the sign convention used, but we will use the positive sign convention, consistent with field theory and most matrix representations of the Dirac matrices.
\footnote{In general relativitity, many authors will use the opposite sign convention.}

Stated explicitly, we use a metric where the basis vectors satisfy the following properties

\begin{dmath}\label{eqn:prerequisites:420}
\left\{
\begin{array}{l l}
\gamma_\mu \cdot \gamma_\nu = 0 & \quad \mbox{ \( \mu \ne \nu ; \mu, \nu \in [0,3] \) } \\
\gamma_i \cdot \gamma_i = -1& \quad \mbox{ \( i \in [1,3] \) } \\
\gamma_0 \cdot \gamma_0 = 1. &\\
\end{array}
\right.
\end{dmath}


%%%\makeproblem{}{problem:multivector:50}{
%%%The most general definition of an Euclidean norm satisfies all of the properties
%%%
%%%\begin{enumerate}
%%%   \item \( \Norm{\Bx} \ge 0 \), and \( \Norm{\Bx} = 0 \iff \Bx = 0 \).
%%%   \item \( \Norm{a \Bx} = \Abs{a} \Norm{\Bx} \).
%%%   \item \( \Norm{\Bx + \By} \le \Norm{\Bx} + \Norm{\By} \).
%%%\end{enumerate}
%%%
%%%If the coordinates of a vector with respect to the standard basis are \( x_i \) then show that the Euclidean norm defined in
%%%that the Pythagorean norm
%%%\begin{equation*}
%%%\Norm{\Bx}^2 = \sum_{i = 1}^N x_i^2,
%%%\end{equation*}
%%%
%%%satisfies these properties.
%%%} % problem
%%%

         \subsection{Vector space.}
            %
% Copyright © 2017 Peeter Joot.  All Rights Reserved.
% Licenced as described in the file LICENSE under the root directory of this GIT repository.
%

%Vectors have many generalizations in mathematics, where
%a number of disparate mathematical objects
%%, such as
%%directed ``arrows'', tuples of real or complex numbers, matrices, functions, polynomials, and quantum states
%can all be considered vectors.
%A vector space is an enumeration of the properties and operations that are common to a set of
%vector-like objects, allowing them to be treated in a unified fashion, regardless of their representation and application.
%%The definition of a vector space and some other basic ideas from linear algebra are all reviewed here.
%%This review will set the stage for the definition of a \boldTextAndIndex{multivector space}, the GA analogue of a vector space.

Two representation specific methods of vector addition and multiplication have been described.
Addition can be performed graphically, connecting vectors heads to tails, or by adding the respective coordinates.
Multiplication can be performed by changing the length of a vector represented by an arrow, or by multiplying each coordinate algebraically.
These rules can be formalized and abstracted by introducing the concept of vector space, which describes both vector addition and multiplication in a representation agnostic fashion.
\index{vector space}
\makedefinition{Vector space.}{def:prerequisites:vectorspace}{
A vector space is a set \( V = \setlr{\Bx, \By, \Bz, \cdots} \), the elements of which are called vectors, which has an addition operation designated \( + \) and a scalar multiplication operation designated by juxtaposition, where the following axioms are satisfied for all
for all vectors \( \Bx, \By, \Bz \in V \) and scalars \( a, b \in \bbR \).
\begin{tablebox}[tabularx={X|Y}]%{Vector space axioms.}
    V is closed under addition & \( \Bx + \By \in V \) \\ \hline
    V is closed under scalar multiplication & \( a \Bx \in V \) \\ \hline
    Addition is associative & \( (\Bx + \By) + \Bz = \Bx + (\By + \Bz) \) \\ \hline
    Addition is commutative & \( \By + \Bx = \Bx + \By \) \\ \hline
    There exists a zero element \( \Bzero \in V \)  & \( \Bx + \Bzero = \Bx \) \\ \hline
    For any \( \Bx \in V \) there exists a negative additive inverse \( -\Bx \in V \) & \( \Bx + (-\Bx) = \Bzero \) \\ \hline
    Scalar multiplication is distributive  & \( a( \Bx + \By ) = a \Bx + a \By \), \( (a + b)\Bx = a \Bx + b\Bx \) \\ \hline
    Scalar multiplication is associative & \( (a b) \Bx = a ( b \Bx ) \) \\ \hline
    There exists a multiplicative identity & \( 1 \Bx = \Bx \) \\ \hline
\end{tablebox}
}

One may define finite or infinite dimensional vector spaces with matrix, polynomial, complex tuple, or many other types of elements.
Some examples of general vector spaces are given in the problems below, and many more can be found in any introductory book on linear algebra.
The applications of geometric algebra to electromagnetism found in this book require only real vector spaces with dimension no greater than three.
\Cref{def:prerequisites:vectorspace} serves as a reminder, as the concept of vector space will be built upon and generalized shortly.

         \subsection{Basis, span and dimension.}
            %
% Copyright © 2017 Peeter Joot.  All Rights Reserved.
% Licenced as described in the file LICENSE under the root directory of this GIT repository.
%
\index{linear combination}
\makedefinition{Linear combination}{dfn:prerequisites:linearcombination}{
Let \( S = \setlr{ \Bx_1, \Bx_2, \cdots, \Bx_k } \) be a subset of a vector space \( V \).
A linear combination of vectors in \( S \) is any sum
\begin{equation*}
a_1 \Bx_1
+
a_2 \Bx_2
+
\cdots
+
a_k \Bx_k.
\end{equation*}
} % definition

\index{linear dependence}
\makedefinition{Linear dependence.}{dfn:prerequisites:dependence}{
Let \( S = \setlr{ \Bx_1, \Bx_2, \cdots, \Bx_k } \) be a subset of a vector space \( V \).
This set \( S \) is linearly dependent if one can construct any equation
\begin{equation*}
\Bzero =
a_1 \Bx_1
+
a_2 \Bx_2
+
\cdots
+
a_k \Bx_k,
\end{equation*}
for which not all of the coefficients \( a_i, \, 1 \le i \le k \) are zero.
} % definition

\index{linear independence}
\makedefinition{Linear independence.}{dfn:prerequisites:independence}{
Let \( S = \setlr{ \Bx_1, \Bx_2, \cdots, \Bx_k } \) be a subset of a vector space \( V \).
This set is linearly independent if there are no equations with \( a_i \ne 0,\, 1 \le i \le k\) such that
\begin{equation*}
\Bzero =
a_1 \Bx_1
+
a_2 \Bx_2
+
\cdots
+
a_k \Bx_k.
\end{equation*}
} % definition

\index{span}
\makedefinition{Span.}{dfn:prerequisites:span}{
Let \( S = \setlr{ \Bx_1, \Bx_2, \cdots, \Bx_k } \) be a subset of a vector space \( V \).
The span
of this set is the set of all linear combinations of these vectors, denoted
\begin{equation*}
\Span(S) =
\setlr{
a_1 \Bx_1
+
a_2 \Bx_2
+
\cdots
+
a_k \Bx_k}.
\end{equation*}
} % definition

\index{subspace}
\makedefinition{Subspace.}{dfn:prerequisites:subspace}{
Let \( S = \setlr{ \Bx_1, \Bx_2, \cdots, \Bx_k } \) be a subset of a vector space \( V \).
This subset is
a subspace if \( S \) is a vector space under the multiplication and addition operations of the vector space \( V \).
} % definition

\index{basis}
\index{dimension}
\makedefinition{Basis and dimension}{dfn:prerequisites:basisanddimension}{
Let \( S = \setlr{ \Bx_1, \Bx_2, \cdots, \Bx_n } \) be a linearly independent subset of \( V \).
This set is a basis if \( \Span(S) = V \).
The number of vectors \( n \) in this set is called the dimension of the space.
} % definition


         \subsection{Standard basis, length and normality.}
            %
% Copyright © 2017 Peeter Joot.  All Rights Reserved.
% Licenced as described in the file LICENSE under the root directory of this GIT repository.
%
\makedefinition{Dot product.}{dfn:prerequisites:dotproduct}{
Let \( \Bx, \By \) be vectors from a vector space \( V \).
A dot product \( \Bx \cdot \By \) is a mapping \( V \cross V \rightarrow \bbR \)
with the following properties

\begin{tcolorbox}[tab2,tabularx={X|Y},title=Dot product properties.,boxrule=0.5pt]
    Symmetric in both arguments & \( \Bx \cdot \By = \By \cdot \Bx \) \\ \hline
    Bilinear & \( (a \Bx + b \By) \cdot (a' \Bx' + b' \By' ) =
a a' (\Bx \cdot \Bx') + b b' (\By \cdot \By')
+
a b' (\Bx \cdot \By') + b a' (\By \cdot \Bx') \)
\\ \hline
    (Optional) Positive definite & \( \Bx \cdot \Bx \ge 0 \) \\ \hline
\end{tcolorbox}
} % definition

In GA it can be useful to omit the requirement for a dot product to have the positive definite property.
This has specific relevance in electrodynamics, since Maxwell's equations take their simplest form when expressed in terms of four-vector (relativistic) vector spaces.

Because the dot product is bilinear, it is
specified completely by the dot products of a set of basis elements for the space.  For example,
given a basis \( \setlr{ \Be_1, \Be_2, \cdots, \Be_N} \), and two vectors

\begin{dmath}\label{eqn:prerequisites:240}
\begin{aligned}
   \Bx &= \sum_{i = 1}^N x_i \Be_i \\
   \By &= \sum_{i = 1}^N y_i \Be_i,
\end{aligned}
\end{dmath}

the dot product of the two is

\begin{dmath}\label{eqn:prerequisites:260}
\Bx \cdot \By
=
   \lr{ \sum_{i = 1}^N x_i \Be_i } \cdot
   \lr{ \sum_{j = 1}^N y_j \Be_j }
=
   \sum_{i,j = 1}^N x_i y_j \lr{ \Be_i \cdot \Be_j }.
\end{dmath}

Such an expansion in coordinates can be written in matrix form as

\begin{dmath}\label{eqn:prerequisites:280}
\Bx \cdot \By
=
\Bx^\T G \By,
\end{dmath}

where \( G \) is the symmetric matrix with elements \( g_{ij} = \Be_i \cdot \Be_j \).  This matrix \( G \), or its elements \( g_{ij} \) is also called the metric for the space.

In this book the metric is always diagonal, with all diagonal values having an absolute value of one.

\makedefinition{Length}{dfn:prerequisites:norm}{
   The squared norm of a vector \( \Bx \) is defined as

\begin{dmath}\label{eqn:prerequisites:200}
   \Norm{\Bx}^2 = \Bx \cdot \Bx,
\end{dmath}

a quantity that need not be positive.  The length of a vector \( \Bx \) is defined as

\begin{equation*}
\Norm{\Bx} =
\sqrt{\Abs{ \Norm{\Bx}^2 }}.
\end{equation*}
}

%A vector space with an associated norm based length is called a normed vector space.  Any dot product space is also a normed vector space.

\makedefinition{Unit vector}{dfn:prerequisites:unitvector}{
   A vector \( \Bx \) is called a unit vector if its absolute squared length is one, \( \Abs{\Bx \cdot \Bx} = 1 \).
} % definition

%A unit vector \( \xcap \) may be generated from any vector \( \Bx \) that has a non-zero squared norm by computing
%
%\begin{dmath}\label{eqn:prerequisites:220}
%\xcap = \frac{\Bx}{\sqrt{\Abs{\Norm{\Bx}^2}}}.
%\end{dmath}
%
\makedefinition{Normal}{dfn:prerequisites:normal}{
   Two vectors \( \Bx, \By \) are normal, or orthogonal, if their dot product is zero, \( \Bx \cdot \By = 0 \).
}

\makedefinition{Orthonormal}{dfn:prerequisites:orthonormal}{
   Two vectors \( \Bx, \By \) are orthonormal if they are both unit vectors and normal to each other, \( \Bx \cdot \By = 0 \), \( \Abs{\Bx \cdot \Bx} = \Abs{\By \cdot \By} = 1 \).

   A set of vectors \( \setlr{ \Bx, \By, \cdots, \Bz } \) is an orthonormal set if all pairs of vectors in that set are orthonormal.
}

\makedefinition{Standard basis.}{dfn:prerequisites:standardbasis}{
   A basis
\( \setlr{ \Be_1, \Be_2, \cdots, \Be_N} \) is called a standard basis if that set is orthonormal.
} % definition

\makedefinition{Euclidean space.}{dfn:prerequisites:euclideanspace}{
   A vector space with basis
   \( \setlr{ \Be_1, \Be_2, \cdots, \Be_N} \) is called Euclidean if all the dot product pairs between the basis elements are not only orthonormal, but positive definite.  That is

\begin{equation*}
\Be_i \cdot \Be_j = \delta_{ij}.
\end{equation*}
} % definition

      \section{Definitions}
         \subsection{Multivector space.}
            %
% Copyright © 2017 Peeter Joot.  All Rights Reserved.
% Licenced as described in the file LICENSE under the root directory of this GIT repository.
%
%{
Geometric algebra takes a vector space and adds two additional operations, a vector multiplication operation, and a generalized addition operation that extends vector addition to include addition of scalars and products of vectors.
Multiplication of vectors is indicated by juxtaposition, for example, if \( \Bx, \By, \Be_1, \Be_2, \Be_3, \cdots \) are vectors, then some vector products are
\begin{dmath}\label{eqn:multivector:20}
\begin{aligned}
&\Bx \By, \Bx \By \Bx, \Bx \By \Bx \By, \\
&\Be_1 \Be_2, \Be_2 \Be_1, \Be_2 \Be_3, \Be_3 \Be_2, \Be_3 \Be_1, \Be_1 \Be_3, \\
&\Be_1 \Be_2 \Be_3, \Be_3 \Be_1 \Be_2, \Be_2 \Be_3 \Be_1, \Be_3 \Be_2 \Be_1, \Be_2 \Be_1 \Be_3, \Be_1 \Be_3 \Be_2, \\
&\Be_1 \Be_2 \Be_3 \Be_1, \Be_1 \Be_2 \Be_3 \Be_1 \Be_3 \Be_2, \cdots
\end{aligned}
\end{dmath}

Vector multiplication is constrained by a rule, called the contraction axiom, that gives a meaning to the square of a vector
\boxedEquation{eqn:multivector:120}{
\Bx \Bx \equiv \Bx \cdot \Bx.
}

The square of a vector, by this definition, is the squared length of the vector, and is a scalar.
This may not appear to be a useful way to assign meaning to the simplest of vector products, since the product and the vector live in separate spaces.
If we want a closed algebraic system that includes both vectors and their products, we have to allow for the addition of scalars, vectors, or any products of vectors.  Such a sum is called a multivector, an example of which is
\begin{dmath}\label{eqn:multivector:40}
1 + 2 \Be_1 + 3 \Be_1 \Be_2 + 4 \Be_1 \Be_2 \Be_3.
\end{dmath}
In this example, we have added a
scalar (or 0-vector) \( 1 \), to a
vector (or 1-vector) \( 2 \Be_1 \), to a
bivector (or 2-vector) \( 3 \Be_1 \Be_2 \), to a
trivector (or 3-vector) \( 4 \Be_1 \Be_2 \Be_3 \).
Geometric algebra uses vector multiplication to build up a hierarchy of geometrical objects, representing points, lines, planes, volumes and hypervolumes (in higher dimensional spaces.)

\index{scalar}
\index{0-vector}
\paragraph{Scalar.}
A scalar, which we will also call a 0-vector, is a zero-dimensional object with sign, and a magnitude.
We may geometrically interpret a scalar as a (signed) point in space.
%The sign of a scalar can be represented graphically as an arrow with a head and a tail pointing into the paper (or chalkboard),
%as illustrated in
%\cref{fig:scalarOrientation:scalarOrientationFig1} where a crossed circle represents the tail, and a solid dot represents the head.
%%\footnote{We don't usually try to represent the magnitude of a scalar graphically, but could do so by scaling the size of the cross or dot.}
%\imageFigure{../figures/GAelectrodynamics/scalarOrientationFig1}{Scalar illustration.}{fig:scalarOrientation:scalarOrientationFig1}{0.05}

\index{vector}
\index{1-vector}
\paragraph{Vector.}
A vector, which we will also call a 1-vector, is a one-dimensional object with a sign, a magnitude, and a rotational attitude within the space it is embedded.
XX
\index{bivector}
\index{2-vector}
\paragraph{Bivector.}

We now wish to define a bivector, or 2-vector, as a 2 dimensional object representing a signed plane segment with magnitude and orientation.  Formally,
assuming a vector product, the algebraic description of a bivector is

\makedefinition{Bivector.}{dfn:multivector:60}{
A bivector, or 2-vector, is a sum of products of pairs of orthogonal vectors.
Given an \( N \) dimensional vector space \( V \) with an orthonormal basis \( \setlr{ \Be_1, \Be_2, \cdots, \Be_N } \),
a general bivector can be expressed as
\begin{equation*}
\sum_{1 \le i < j \le N} B_{ij} \Be_i \Be_j,
\end{equation*}
where \( B_{ij} \) is a scalar.
The vector basis \( V \) is said to be a generator of a bivector space.
} % definition

The bivector provides a structure that can encode plane oriented quantities such as torque, angular momentum, or a general plane of rotation.
A quantity like angular momentum can be represented as a magnitude times a quantity that represents the orientation of the plane of rotation.
In conventional vector algebra we use the normal of the plane to describe this orientation, but that is problematic in higher dimensional spaces where there is no unique normal.
Use of the normal to represent a plane is also logically problematic in two dimensional spaces, which have to be extended to three dimensions to use normal centric constructs like the cross product.
A bivector representation of a plane can eliminate the requirement to utilize a third (normal) dimension, which may not be relevant in the problem, and can allow some concepts (like the cross product) to be generalized to dimensions other than three when desirable.

One of the implications of the contraction axiom \cref{eqn:multivector:120}, to be discussed in more detail a bit later, is a linear dependence between bivectors formed from orthogonal products.  For example, given any pair of unit bivectors, where \( i \ne j \) we have
\begin{dmath}\label{eqn:multivector:140}
\Be_i \Be_j + \Be_j \Be_i = 0,
\end{dmath}
This is why the sum in \cref{dfn:multivector:60} was over only half the possible pairs of \( i \ne j \) indexes.
The reader can check that the set of all bivectors is a vector space per
\cref{def:prerequisites:vectorspace}, so we will call the set of all bivectors a bivector space.
In \R{2} a basis for the bivector space is \( \setlr{ \Be_1 \Be_2 } \), whereas in \R{3} a basis for the bivector space is
\( \setlr{ \Be_1 \Be_2, \Be_2 \Be_3, \Be_3 \Be_1 } \).  The unit bivectors for two possible \R{3} bivector space bases are illustrated in
\cref{fig:unitBivectors:unitBivectorsFig}.
\imageTwoFigures
{../figures/GAelectrodynamics/unitBivectorsFig1}
{../figures/GAelectrodynamics/unitBivectorsFig2}
{Unit bivectors for \R{3}}
{fig:unitBivectors:unitBivectorsFig}
{scale=0.35}

We interpret the sign of a vector as an indication of the sense of the vector's ``head'' vs ``tail''.
For a bivector, we can interpret the sign as a representation of a
a ``top'' vs. ``bottom'', or equivalently a left or right ``handedness'', as illustrated using arrows around a plane segment in
\cref{fig:circularBivectorsIn3D:circularBivectorsIn3DFig1}.
\imageFigure{../figures/GAelectrodynamics/circularBivectorsIn3DFig1}{Circular representation of two bivectors.}{fig:circularBivectorsIn3D:circularBivectorsIn3DFig1}{0.3}
For a product like \( \Be_1 \Be_2 \), the sense of the handedness follows the path \( 0 \rightarrow \Be_1 \rightarrow \Be_1 + \Be_2 \rightarrow \Be_2 \rightarrow 0 \) around the unit square in the x-y plane.
This is illustrated for all the unit bivectors in \cref{fig:unitBivectors:unitBivectorsFig}.
In \R{3} we can use the right hand rule to visualize such a handedness.  You could say that we are using the direction of the fingers around the normal to indicate the sign of the bivector, but without actually drawing that normal.

Similar to the interpretation of the magnitude of a vector as the length of that vector, we interpret the magnitude of a bivector (to be defined more exactly later), as the area of the bivector.
Other than having a boundary that surrounds a given area, a graphical bivector representation as a plane segment need not have any specific geometry, which is illustrated in
\cref{fig:bivectorRepresentationsInPlane:bivectorRepresentationsInPlaneFig1} for a set of bivectors all representing \( \Be_1 \Be_2 \).
\imageFigure{../figures/GAelectrodynamics/bivectorRepresentationsInPlaneFig1}{Graphical representations of \( \Be_1 \Be_2 \).}{fig:bivectorRepresentationsInPlane:bivectorRepresentationsInPlaneFig1}{0.3}

An oriented plane segment can always be represented as a bivector in any number of dimensions, however, when the generating vector space has dimension \( N \ge 4 \) not all bivectors defined by \cref{dfn:multivector:60} necessarily represent oriented plane segments.
The restrictions required for a bivector to have an associated oriented plane segment interpretation in higher dimensional spaces will be defined later.

Vector addition can be performed graphically by connecting vectors head to tail, and joining the first tail to the last head.  A similar procedure can be used for bivector addition as well, but gets complicated if the bivectors lie in different planes.  Here is a simple bivector sum
\begin{dmath}\label{eqn:multivector:160}
3 \Be_1 \Be_2 - 2 \Be_1 \Be_2 + 5 \Be_1 \Be_2 = 6 \Be_1 \Be_2,
\end{dmath}
which can be interpreted as taking a 3 unit area, subtracting a 2 unit area, and adding a 5 unit area.  This sum is illustrated in
\cref{fig:bivectorAdditionInPlane:bivectorAdditionInPlaneFig1}.
An visualization of arbitrarily oriented bivector addition can be found in
\cref{fig:AdditionOfBivectors:AdditionOfBivectorsFig2}, where \( \text{red} + \text{blue} = \text{green} \).  This visualization shows that the
moral of the story is that we will almost exclusively be adding bivectors algebraically, but can interpret the sum geometrically after the fact.
\imageFigure{../figures/GAelectrodynamics/bivectorAdditionInPlaneFig1}{Graphical representation of bivector addition in plane.}{fig:bivectorAdditionInPlane:bivectorAdditionInPlaneFig1}{0.2}
\imageFigure{../figures/GAelectrodynamics/AdditionOfBivectorsFig2}{Bivector addition.}{fig:AdditionOfBivectors:AdditionOfBivectorsFig2}{0.3}
%The same can be done with bivectors, where the bivectors are also connected with compatible orientation to construct a sum.
%This is illustrated graphically in \cref{fig:AdditionOfBivectors:AdditionOfBivectorsFig1}, where a blue bivector with a right handed orientation is added to a red bivector with right handed orientation, to form a green bivector also with right handed orientation, where all orientations are with respect to the exterior of the bounding surface formed by the three bivectors.
%\imageFigure{../figures/GAelectrodynamics/AdditionOfBivectorsFig1}{Bivector addition.}{fig:AdditionOfBivectors:AdditionOfBivectorsFig1}{0.3}

\index{trivector}
\index{3-vector}
\paragraph{Trivector.}

Again, assuming a vector product

\makedefinition{Trivector.}{dfn:multivector:80}{
A trivector, or 3-vector, is a sum of products of triplets of mutually orthogonal vectors.
Given an \( N \) dimensional vector space \( V \) with an orthonormal basis \( \setlr{ \Be_1, \Be_2, \cdots, \Be_N } \), a trivector is any value
\begin{equation*}
\sum_{1 \le i < j < k \le N} T_{ijk} \Be_i \Be_j \Be_k,
\end{equation*}
where \( \T_{ijk} \) is a scalar.
The vector space \( V \) is said to generate a trivector space.
} % definition

In \R{3} all trivectors are scalar multiples of \( \Be_1 \Be_2 \Be_3 \).
Like scalars, there is no direction to such a quantity, but like scalars trivectors may be signed.  The magnitude of a trivector may be interpreted as a volume.
We will defer interpreting the sign of a trivector geometrically until we tackle integration theory.
%%%, which requires some interpretation.
%%%We can interpret the magnitude of a trivector as a volume, but what is a signed volume?
%%%One answer to this question is that we can interpret the sign of the volume as the exterior or the interior of the surface on the boundry of the volume.
%%%We will see another answer when we study integration theory, since geometric integration theory uses signed volume elements, and
%%%swapping the order of two adjacent products in the volume element toggles the sign.
%%%\footnote{In conventional integration theory,
%%%this sign change occurs when swapping rows or columns in the Jacobian, but this is masked by taking the absolute value of the Jacobian after coordinate transformation.}
%%%One possible interpretation of this sign is the interior or the exterior of the bounding surface of a volume.
%%%%This orientation can be visualized with a normal pointing into or out of the volume, or like bivectors, with a cyclic direction on the surface of the volume as in illustrated with the spherical volume of \cref{fig:orientedVolume:orientedVolumeFig1}.
%%%%\imageFigure{../figures/GAelectrodynamics/orientedVolumeFig1}{Oriented Volume}{fig:orientedVolume:orientedVolumeFig1}{0.3}
%%%%In greater than three dimensions, a trivector can have a ``direction'' in the higher dimensional space, as well as a sidedness.
%%%%As was the case with the bivector, because not all the products \( \Be_i \Be_j \Be_k \) for any set of indexes \( i, j, k \) are independent, it is possible to form a trivector as a sum over a more restricted set, such as \( \sum_{1 \le i < j < k \le N} T_{ijk} \Be_i \Be_j \Be_k \).
%%%%In particular, in three dimensions, all trivectors can be expressed as scalar multiples of \( \Be_1 \Be_2 \Be_3 \).
%%%%
\index{k-vector}
\index{grade}
\paragraph{K-vector.}
\makedefinition{K-vector and grade.}{dfn:multivector:100}{
A k-vector is a sum of products of \( k \) mutually orthogonal vectors.
Given an \( N \) dimensional vector space with an orthonormal basis \( \setlr{ \Be_1, \Be_2, \cdots, \Be_N } \),
a general k-vector can be expressed as
\begin{equation*}
\sum_{1 \le i < j \cdots < m \le N} K_{i j \cdots m} \Be_{i} \Be_{j} \cdots \Be_{m},
\end{equation*}
where \( K_{i j \cdots m} \) is a scalar, indexed by \( k \) indexes \( i, j, \cdots, m \).

The number \( k \) of orthogonal vectors that generate a k-vector is called the grade.

A 0-vector is a scalar.

The vector space \( V \) is said to generate the k-vector space.
} % definition

Illustrating by example, \( 1 \) is a 0-vector with grade 0, \( \Be_1 \) is a 1-vector with grade 1, \( \Be_1 \Be_2, \Be_2 \Be_3 \), and \( \Be_3 \Be_1 \) are 2-vectors with grade 2, and \( \Be_1 \Be_2 \Be_3 \) is a 3-vector with grade 3.

We will see that the highest grade for a k-vector in an N dimensional vector space is \( N \).

\index{multivector}
\index{multivector space}
\paragraph{Multivector space.}
\makedefinition{Multivector space.}{def:multiplication:multivectorspace}{
   Given an N dimensional (generating) vector space \( V \) 
and a vector multiplication operation represented by juxtaposition,
a multivector is a sum of k-vectors, \( k \in [ 1, N ] \).

The multivector space generated by \( V \) is a set \( M = \setlr{ x, y, z, \cdots } \) of multivectors, where the following axioms are satisfied

\begin{tablebox}[tabularx={X|Y}]{Multivector space axioms.}
    Contraction & \( \Bx^2 = \Bx \cdot \Bx, \,\forall \Bx \in V \) \\ \hline
    \( M \) is closed under addition & \( x + y \in M \) \\ \hline
    \( M \) is closed under multiplication & \( x y \in M \) \\ \hline
    Addition is associative & \( (x + y) + z = x + (y + z) \) \\ \hline
    Addition is commutative & \( y + x = x + y \) \\ \hline
    There exists a zero element \( 0 \in M \)  & \( x + 0 = x \) \\ \hline
    For all \( x \in M \) there exists a negative additive inverse \( -x \in M \) & \( x + (-x) = 0 \) \\ \hline
    Multiplication is distributive  & \( x( y + z ) = x y + x z \), \( (x + y)z = x z + y z \) \\ \hline
    Multiplication is associative & \( (x y) z = x ( y z ) \) \\ \hline
    There exists a multiplicative identity \( 1 \in M \) & \( 1 x = x \) \\ \hline
\end{tablebox}
}

(CUT)

with an orthonormal basis \( \setlr{ \Be_1, \Be_2, \cdots, \Be_N } \),
%a basis \( \setlr{ \Bx_1, \Bx_2, \cdots } \),
, such as
   \( a_0 + \sum_i a_i \Be_i + \sum_{i \ne j} a_{ij} \Be_i \Be_j + \sum_{i \ne j \ne k} a_{ijk} \Be_i \Be_j \Be_k + \cdots \), where \( a_0, a_i, a_{ij}, \cdots \) are scalars.

Compared to the vector space, \cref{def:prerequisites:vectorspace}, the multivector space

\begin{itemize}
\item specifies a rule providing the meaning of a squared vector (the contraction axiom).
\item presumes a vector multiplication operation, which is not assumed to be commutative (order matters),
\item generalizes vector addition to multivector addition,
\item generalizes scalar multiplication to multivector multiplication (of which scalar multiplication and vector multiplication are special cases),
\end{itemize}

The contraction axiom is arguably the most important of the multivector space axioms, as it allows for multiplicative closure without an infinite dimensional multivector space.
The remaining set of non-contraction axioms of a multivector space are almost that of a field, however,
%\footnote{A mathematician would call a multivector space a non-commutative ring with identity \citep{van1943modern}, and could state the multivector space definition much more compactly without listing all the properties of a ring explicitly as done above.}
%(as encountered in the study of complex inner products),
%as they describe most of the properties one
%would expect of a ``well behaved'' set of number-like quantities.
a field also requires a multiplicative inverse element for all elements of the space, which exists for some multivector subspaces, but not in general.

%These axioms may seem simple enough, especially since they are not that different from the familiar axioms of the vector space,
%but it will take considerable work to extract all their consequences.
%The subject of Geometric Algebra can be viewed as the study of the impliciations of the axioms
%of the multivector space.

%}

         \subsection{Nomenclature.}
            %
% Copyright © 2017 Peeter Joot.  All Rights Reserved.
% Licenced as described in the file LICENSE under the root directory of this GIT repository.
%
A fair amount of nomenclature and notation is unfortunately required before systematically examining the implications of the multivector space axioms that define geometric algebra.

\index{blade}
\index{grade}
\makedefinition{Blade and grade}{def:multiplication:blade}{
A product of \( k \) perpendicular vectors is called a k-blade, or a blade of grade \( k \).
A grade zero blade is a scalar.

The notation \( F \in \bigwedge^k \) is used in the literature to indicate that \( F \) is a blade of grade \( k \).
}

The maximum grade of a multivector is equal to the dimension of the generating vector space.
For example, for a multivector space generated by \R{3}, no k-vector can have grade greater than 3.

Examples of blades with grades 0, 1, 2, and 3 respectively are

\begin{dmath}\label{eqn:multivector:180}
\begin{aligned}
&1 \\
&\Be_1,\quad \Be_2,\quad \Be_3 \\
&\Be_1 \Be_2,\quad \Be_2 \Be_1,\quad \Be_1 \Be_2 + \Be_2 \Be_3 \\
&\Be_1 \Be_2 \Be_3,\quad \Be_1 \Be_3 \Be_2,\quad \Be_1 \Be_4 \Be_2
\end{aligned}
\end{dmath}

Multivectors which can be factored into perpendicular vector products, such as
\begin{dmath}\label{eqn:multiplication:220}
\Be_1 \Be_2 + 3 \Be_1 \Be_3
=
\Be_1 (\Be_2 + 3 \Be_3),
\end{dmath}

are blades.  In contrast, the following grade 2 multivectors

\begin{dmath}\label{eqn:multiplication:240}
\Be_1 \Be_2 + \Be_3 \Be_4,
\end{dmath}

and
\begin{dmath}\label{eqn:multiplication:260}
\Be_1 \Be_2 + \Be_2 \Be_3 + \Be_3 \Be_1,
\end{dmath}

which cannot be factored into two vector products, are not blades.

\index{k-vector}
\makedefinition{k-vector.}{dfn:multivector:kvector}{
A sum of k-blades is called a k-vector.
} % definition

Multivectors are therefore sums of k-vectors with different grades.

All the k-blade examples above are also k-vectors.
K-vectors with grades 2 and 3 are so pervasive that they are given special names.

\index{bivector}
\makedefinition{Bivector.}{dfn:multivector:bivector}{
A bivector, or 2-vector, is a k-vector with grade 2.
} % definition

The product \( \Be_1 \Be_2 \) is a bivector, as is \( \Be_2 \Be_3 + 3 \Be_4 \Be_1 \)
%Each of \( \Be_1 \Be_2, \Be_2 \Be_1, \Be_1 \Be_2 + \Be_2 \Be_3 \), and \( \Be_1 \Be_2 + \Be_3 \Be_4 \) are bivectors.
%All but the last of these represents an oriented plane segment.

\index{trivector}
\makedefinition{Trivector.}{dfn:multivector:trivector}{
A trivector, or 3-vector, is a k-vector with grade 3.
} % definition

%Quantities with higher grades than 3 are not generally given explicit names.
The multivector \( \Be_3 \Be_1 \Be_2 \) is a trivector, as is \( \Be_1 \Be_2 \Be_3 + 3 \Be_5 \Be_4 \Be_1 \).  The latter is not a blade.
%Each of \( \Be_1 \Be_2 \Be_3, \Be_1 \Be_3 \Be_2, \Be_1 \Be_4 \Be_2 \) are trivectors.
% , and represent oriented volumes.

\index{grade selection}
\makedefinition{Grade selection operator}{dfn:gradeselection:gradeselection}{
Given a set of k-vectors \( M_k, k \in [0,N] \), and any multivector of their sum

\begin{equation*}
M = \sum_{i = 0}^N M_i,
\end{equation*}

the grade selection operator is defined as

\begin{equation*}\label{eqn:gradeselection:40}
\gpgrade{M}{k} \equiv M_k.
\end{equation*}

Due to its importance, selection of the (scalar) zero grade is given the shorthand
\begin{equation*}
\gpgradezero{M} \equiv \gpgrade{M}{0} = M_0.
\end{equation*}
}

For example, if \( M = 3 - \Be_3 + 2 \Be_1 \Be_2 \), then
\begin{equation}\label{eqn:gradeselection:80}
\begin{aligned}
\gpgradezero{M} &= 3 \\
\gpgrade{M}{1} &= - \Be_3 \\
\gpgrade{M}{2} &= 2 \Be_1 \Be_2 \\
\gpgrade{M}{3} &= 0.
\end{aligned}
\end{equation}

\index{orthonormal blades}
\makedefinition{Orthonormal product shorthand.}{dfn:multivector:shorthand}{
Given an orthonormal basis \( \setlr{ \Be_1, \Be_2, \cdots } \), a multiple indexed quantity \( \Be_{ij\cdots k} \) should be interpretted as the product (in the same order) of the basis elements with those indexes

\begin{equation*}
\Be_{ij\cdots k} = \Be_i \Be_j \cdots \Be_k.
\end{equation*}
} % definition

For example,

\begin{equation}\label{eqn:multivector:360}
\begin{aligned}
\Be_{12} &= \Be_1 \Be_2 \\
\Be_{123} &= \Be_1 \Be_2 \Be_3 \\
\Be_{23121} &= \Be_2 \Be_3 \Be_1 \Be_2 \Be_1.
\end{aligned}
\end{equation}

\index{pseudoscalar}
\makedefinition{Pseudoscalar.}{def:multiplication:pseudoscalar}{
A blade with grade that matches the dimension of the space.
}

In a two dimensional space \( \Be_1 \Be_2 \) is a pseudoscalar, as is \( 3 \Be_2 \Be_1 \).  In a three dimensional space
\( \Be_3 \Be_1 \Be_2 \) is a pseudoscalar, as is \( - 7 \Be_3 \Be_1 \Be_2 \).
%A pseudoscalar has an implied orientation, which can be
%associated with the handedness of the underlying basis.
It is conventional to refer to

\begin{dmath}\label{eqn:definitions:320}
i = \Be_1 \Be_2,
\end{dmath}

as ``the pseudoscalar'' for a two dimensional space, and to

\begin{dmath}\label{eqn:definitions:340}
I = \Be_1 \Be_2 \Be_3,
\end{dmath}

as ``the pseudoscalar'' for a three dimensional space.



      \section{Analysis.}
         %
% Copyright © 2017 Peeter Joot.  All Rights Reserved.
% Licenced as described in the file LICENSE under the root directory of this GIT repository.
%
\paragraph{Unless otherwise stated, a Euclidean vector space with an orthonormal basis \( \setlr{\Be_1, \Be_2, \cdots } \) is assumed for the remainder of this chapter.}
Generalizations required for non-Euclidean spaces will be discussed when (if?) spacetime vectors are introduced.

         \subsection{Colinear vectors.}
            %
% Copyright © 2017 Peeter Joot.  All Rights Reserved.
% Licenced as described in the file LICENSE under the root directory of this GIT repository.
%
It was pointed out that the vector multiplication operation was not assumed to be commutative (order matters).
The only condition for which the product of two vectors is order independent, is when those vectors are colinear.

\index{commutation}
\maketheorem{Vector commutation.}{thm:multiplication:commutation}{
Given two vectors \( \Bx, \By \), if \( \By = \alpha \Bx \) for some scalar \( \alpha \), then \( \Bx \) and \( \By \) commute
\begin{equation*}
\Bx \By = \By \Bx.
\end{equation*}
} % theorem

%vu = auu.
%uv = u(au) = auu
The proof is simple.
\begin{dmath}\label{eqn:colinearVectors:380}
\begin{aligned}
\By \Bx &= \alpha \Bx \Bx \\
\Bx \By &= \Bx \alpha \Bx = \alpha \Bx \Bx.
\end{aligned}
\end{dmath}

The contraction axiom ensures that the product of two colinear vectors is a scalar.
In particular, the square of a unit vector, say \( \Bu \) is unity.
This should be highlighted explicitly, because this property will be used again and again
%\begin{equation}\label{eqn:colinearVectors:300}
\boxedEquation{eqn:multiplication:320}{
\Bu^2 = 1.
}
%\end{equation}

For example, the squares of any orthonormal basis vectors are unity \( (\Be_1)^2 = (\Be_2)^2 = (\Be_3)^3 = 1 \).

A corollary of
\cref{eqn:multiplication:320} is that we can factor \( 1 \) into
the square of any unit vector \( \Bu \)
\boxedEquation{eqn:multiplication:400}{
1 = \Bu \Bu.
}

This factorization trick will be used repeatedly in this book.

         \subsection{Normal vectors.}
            %
% Copyright © 2017 Peeter Joot.  All Rights Reserved.
% Licenced as described in the file LICENSE under the root directory of this GIT repository.
%
An interchange of the order of the factors of two normal factors results in a change of sign,
for example \( \Be_2 \Be_1 = -\Be_1 \Be_2 \).
This is a consequence of the contraction axiom, and can be demonstrated by squaring the vector
\( \Be_1 + \Be_2 \) (\cref{fig:unitSum:unitSumFig1}).

\imageFigure{../figures/GAelectrodynamics/unitSumFig1}{\( \Be_1 + \Be_2 \).}{fig:unitSum:unitSumFig1}{0.3}
By the contraction axiom, the square of this vector is \( 2 \), the squared length of the vector
\begin{dmath}\label{eqn:normalVectors:420}
\lr{ \Be_1 + \Be_2 }^2 =
\lr{ \Be_1 + \Be_2 } \cdot
\lr{ \Be_1 + \Be_2 }
=
\Be_1 \cdot \Be_1
+
\cancel{\Be_1 \cdot \Be_2}
+
\cancel{\Be_2 \cdot \Be_1}
+
\Be_2 \cdot \Be_2
=
2.
\end{dmath}

On the other hand, deferring the application of the contraction axiom until after the vectors products have been distributed gives
\begin{dmath}\label{eqn:normalVectors:80}
(\Be_1 + \Be_2)^2
= (\Be_1 + \Be_2)(\Be_1 + \Be_2)
= \Be_1^2 + \Be_2 \Be_1 + \Be_1 \Be_2 + \Be_2^2
= 1 + \Be_2 \Be_1 + \Be_1 \Be_2 + 1
= 2 + \Be_2 \Be_1 + \Be_1 \Be_2.
\end{dmath}

The right hand side of
\cref{eqn:normalVectors:80},
a mixed grade multivector with grades zero and two, must also equal \cref{eqn:normalVectors:420}.
A solution requires that the grade two components all sum to zero
\begin{dmath}\label{eqn:normalVectors:280}
\Be_2 \Be_1 + \Be_1 \Be_2 = 0,
\end{dmath}
or
%\begin{dmath}\label{eqn:normalVectors:140}
\boxedEquation{eqn:normalVectors:140}{
\Be_1 \Be_2 = -\Be_1 \Be_2.
}
%\end{dmath}

The same computation could have been performed for any two orthonormal vectors, so we conclude that any interchange of two orthonormal vectors changes the sign.
In general this is true of any normal vectors.

\index{anticommutation}
\maketheorem{Anticommutation}{thm:multiplication:anticommutationNormal}{
Let \(\Bu\), and \(\Bv\) be two normal vectors, the product of which \( \Bu \Bv \) is a bivector.
Changing the order of these products toggles the sign of the bivector.
\begin{equation*}
\Bu \Bv = -\Bv \Bu.
\end{equation*}

This sign change on interchange is called anticommutation.
} % theorem

         \subsection{2D multiplication table.}
            %
% Copyright © 2017 Peeter Joot.  All Rights Reserved.
% Licenced as described in the file LICENSE under the root directory of this GIT repository.
%
The multiplication table for the \R{2} geometric algebra can be computed with relative ease.
Many of the interesting products involve \( i = \Be_1 \Be_2 \), the unit pseudoscalar.
%\cref{eqn:normalVectors:140} % reference dead
The imaginary nature of the pseudoscalar can be demonstrated using \cref{thm:multiplication:anticommutationNormal}
\begin{dmath}\label{eqn:2dMultiplication:220}
   \lr{ \Be_1 \Be_2 }^2
   =
   (\Be_1 \Be_2)(\Be_1 \Be_2)
   =
   -(\Be_1 \Be_2)(\Be_2 \Be_1)
   =
   -\Be_1 (\Be_2^2 ) \Be_1
   =
   -\Be_1^2
   = -1.
\end{dmath}

\index{complex imaginary}
Like the (scalar) complex imaginary, the bivector \( \Be_1 \Be_2 \) also squares to \( -1 \).
The only non-trivial products left to fill in the \R{2} multiplication table are those of the unit vectors with \( i \), products that are order dependent
\begin{dmath}\label{eqn:2dMultiplication:180}
\begin{aligned}
   \Be_1 i &= \Be_1 \lr{ \Be_1 \Be_2 } \\
           &= \lr{ \Be_1 \Be_1 } \Be_2 \\
           &= \Be_2 \\
   i \Be_1 &= \lr{ \Be_1 \Be_2 } \Be_1 \\
           &= \lr{ -\Be_2 \Be_1 } \Be_1 \\
           &= -\Be_2 \lr{ \Be_1 \Be_1 } \\
           &= -\Be_2 \\
   \Be_2 i &= \Be_2 \lr{ \Be_1 \Be_2 } \\
           &= \Be_2 \lr{ -\Be_2 \Be_1 } \\
           &= -\lr{ \Be_2 \Be_2 }\Be_1 \\
           &= -\Be_1 \\
   i \Be_2 &= \lr{ \Be_1 \Be_2 } \Be_2 \\
           &= \Be_1 \lr{ \Be_2 \Be_2 } \\
           &= \Be_1.
\end{aligned}
\end{dmath}

The multiplication table for the \R{2} multivector basis can now be tabulated

%FIXME: how to reference a tcolorbox table?
% examples in http://ctan.mirrors.hoobly.com/macros/latex/contrib/tcolorbox/tcolorbox.pdf section 5.1
% requires setting up a counter variable like some of the others (theorem environments)

% various options for prettier than default table:
% https://tex.stackexchange.com/a/135421/15
% https://tex.stackexchange.com/a/298109/15
% https://tex.stackexchange.com/a/112359/15
%\captionedTable{2D Multiplication table.}{tab:2dMultiplication:10}{
%\begin{tabular}{|l||l|l|l|l|}
%\hline
%        & \( 1 \) & \( \Be_1 \) & \( \Be_2 \) & \( \Be_1 \Be_2 \) \\ \hline
%\( 1 \) & \( 1 \) & \( \Be_1 \) & \( \Be_2 \) & \( \Be_1 \Be_2 \) \\ \hline
%\( \Be_1\) & \( \Be_1 \) & \( 1 \) & \( \Be_1 \Be_2 \) & \( \Be_2 \)\\ \hline
%\( \Be_2\) & \( \Be_2 \) & \( -\Be_1 \Be_2 \) & \( 1 \) & \( -\Be_1 \)\\ \hline
%\( \Be_1 \Be_2\) & \( \Be_1 \Be_2 \) & \( -\Be_2 \) & \( \Be_1 \) & \( -1 \) \\ \hline
%\end{tabular}
%}

\begin{tablelabelbox}[tabularx={X||Y|Y|Y|Y}]{2D Multiplication table.}{label=tab:2dMultiplication:10}
        & \( 1 \) & \( \Be_1 \) & \( \Be_2 \) & \( \Be_1 \Be_2 \) \\ \hline
\( 1 \) & \( 1 \) & \( \Be_1 \) & \( \Be_2 \) & \( \Be_1 \Be_2 \) \\ \hline
\( \Be_1\) & \( \Be_1 \) & \( 1 \) & \( \Be_1 \Be_2 \) & \( \Be_2 \)\\ \hline
\( \Be_2\) & \( \Be_2 \) & \( -\Be_1 \Be_2 \) & \( 1 \) & \( -\Be_1 \)\\ \hline
\( \Be_1 \Be_2\) & \( \Be_1 \Be_2 \) & \( -\Be_2 \) & \( \Be_1 \) & \( -1 \) \\ \hline
\end{tablelabelbox}
%\begin{tcolorbox}[tab2,tabularx={X||Y|Y|Y|Y},title=2D Multiplication table.,boxrule=0.5pt]
%        & \( 1 \) & \( \Be_1 \) & \( \Be_2 \) & \( \Be_1 \Be_2 \) \\ \hline
%\( 1 \) & \( 1 \) & \( \Be_1 \) & \( \Be_2 \) & \( \Be_1 \Be_2 \) \\ \hline
%\( \Be_1\) & \( \Be_1 \) & \( 1 \) & \( \Be_1 \Be_2 \) & \( \Be_2 \)\\ \hline
%\( \Be_2\) & \( \Be_2 \) & \( -\Be_1 \Be_2 \) & \( 1 \) & \( -\Be_1 \)\\ \hline
%\( \Be_1 \Be_2\) & \( \Be_1 \Be_2 \) & \( -\Be_2 \) & \( \Be_1 \) & \( -1 \) \\ \hline
%\end{tcolorbox}

\index{pseudoscalar}
It is important to point out that the
pseudoscalar \( i \) does not commute with either basis vector, but anticommutes with both, since \( i \Be_1 = - \Be_1 i \), and \( i \Be_2 = - \Be_2 i \).
By superposition \( i \) anticommutes with any vector in the x-y plane.

More generally, if \( \Bu \) and \( \Bv \) are orthonormal, and \( \Bx \in \Span\setlr{\Bu, \Bv} \) then the bivector \( \Bu \Bv \) anticommutes with \( \Bx \), or any other vector in this plane.

%\ref{tab:2dMultiplication:10}.


         \subsection{Plane rotations.}
            %
% Copyright © 2017 Peeter Joot.  All Rights Reserved.
% Licenced as described in the file LICENSE under the root directory of this GIT repository.
%

\index{rotation}
Plotting \cref{eqn:2dMultiplication:180}, as in
\cref{fig:rotationOfe1:rotationOfe1Fig1},
 shows that multiplication by \( i \) rotates the \R{2} basis vectors by \( \pm \pi/2 \) radians,
with the
rotation direction dependent on the order of multiplication.

\mathImageTwoFigures
{../figures/GAelectrodynamics/rotationOfe1Fig1}
{../figures/GAelectrodynamics/rotationOfe2Fig1}
{Multiplication by \( \Be_1 \Be_2 \).}{fig:rotationOfe1:rotationOfe1Fig1}{scale=0.5}
{orientedAreas.nb}

Multiplying a polar vector representation
\begin{equation}\label{eqn:2dRotations:280}
   \Bx = \rho \lr{ \Be_1 \cos\theta + \Be_2 \sin\theta },
\end{equation}
by \( i \) shows that a \( \pi/2 \) rotation is induced.
\index{pseudoscalar}

Multiplying the vector from the right by \( i \) gives
\begin{equation}\label{eqn:2dRotations:300}
\begin{aligned}
\Bx i
&= \Bx \Be_1 \Be_2 \\
&= \rho \lr{ \Be_1 \cos\theta + \Be_2 \sin\theta } \Be_1 \Be_2 \\
&= \rho \lr{ \Be_2 \cos\theta - \Be_1 \sin\theta },
\end{aligned}
\end{equation}
a counterclockwise rotation of \( \pi/2 \) radians, and
multiplying the vector by \( i \) from the left gives
\begin{equation}\label{eqn:2dRotations:3}
\begin{aligned}
i \Bx
&= \Be_1 \Be_2 \Bx \\
&= \rho \Be_1 \Be_2 \lr{ \Be_1 \cos\theta + \Be_2 \sin\theta } \Be_1 \Be_2 \\
&= \rho \lr{ -\Be_2 \cos\theta + \Be_1 \sin\theta },
\end{aligned}
\end{equation}
a clockwise rotation by \( \pi/2 \) radians
(\cref{problem:2dRotations:1}).

The transformed vector \( \Bx' = \Bx \Be_1 \Be_2 = -\Be_1 \Be_2 \Bx \,(= \Bx i = -i \Bx) \) has been rotated in the direction that takes \( \Be_1 \) to \( \Be_2 \), as illustrated
in \cref{fig:rotationOfV:rotationOfVFig1}.

\mathImageFigure{../figures/GAelectrodynamics/rotationOfVFig1}{\( \pi/2\) rotation in the plane using pseudoscalar multiplication.}{fig:rotationOfV:rotationOfVFig1}{0.3}{orientedAreas.nb}

In complex number theory the complex exponential \( e^{i\theta} \) can be used as a rotation operator.
Geometric algebra puts this rotation operator into the vector algebra toolbox, by utilizing
Euler's formula
\index{Euler's formula}
\begin{equation}\label{eqn:2dRotations:1140}
e^{i\theta} = \cos\theta + i \sin\theta,
\end{equation}
which holds for this pseudoscalar imaginary representation too (\cref{problem:2dRotations:Euler}).
\index{complex exponential}
By writing \( \Be_2 = \Be_1 \Be_1 \Be_2 \),
a complex exponential can be factored directly out of the polar vector representation \cref{eqn:2dRotations:280}
\begin{equation}\label{eqn:2dRotations:940}
\begin{aligned}
\Bx
&= \rho \lr{ \Be_1 \cos\theta + \Be_2 \sin\theta } \\
&= \rho \lr{ \Be_1 \cos\theta + (\Be_1 \Be_1) \Be_2 \sin\theta } \\
&= \rho \Be_1 \lr{ \cos\theta + \Be_1 \Be_2 \sin\theta } \\
&= \rho \Be_1 \lr{ \cos\theta + i \sin\theta } \\
&= \rho \Be_1 e^{i\theta}.
\end{aligned}
\end{equation}

We end up with a complex exponential multivector factor on the right.
Alternatively, since \( \Be_2 = \Be_2 \Be_1 \Be_1 \), a complex exponential can be factored out on the left
\begin{equation}\label{eqn:2dRotations:960}
\begin{aligned}
\Bx &= \rho \lr{ \Be_1 \cos\theta + \Be_2 \sin\theta } \\
&= \rho \lr{ \Be_1 \cos\theta + \Be_2 (\Be_1 \Be_1) \sin\theta } \\
&= \rho \lr{ \cos\theta - \Be_1 \Be_2 \sin\theta } \Be_1 \\
&= \rho \lr{ \cos\theta - i \sin\theta } \Be_1 \\
&= \rho e^{-i\theta} \Be_1.
\end{aligned}
\end{equation}

Left and right exponential expressions have now been found for the polar representation
\begin{equation}\label{eqn:2dRotations:1120}
\rho \lr{ \Be_1 \cos\theta + \Be_2 \sin\theta }
= \rho e^{-i\theta} \Be_1 = \rho \Be_1 e^{i\theta}.
\end{equation}

This is essentially a recipe for rotation of a vector in the x-y plane.
Such rotations are
illustrated in \cref{fig:rotationOfX:rotationOfXFig1}.
\mathImageFigure{../figures/GAelectrodynamics/rotationOfXFig1}{Rotation in a plane.}{fig:rotationOfX:rotationOfXFig1}{0.3}{orientedAreas.nb}

This generalizes to rotations of \R{N} vectors constrained to a plane.
Given orthonormal vectors \( \Bu, \Bv \) and any vector in the plane of these two vectors (\( \Bx \in \Span\setlr{\Bu,\Bv} \)), this vector is rotated \( \theta \) radians in the direction of rotation that takes \( \Bu \) to \( \Bv \) by
\begin{equation}\label{eqn:2dRotations:1160}
\Bx' = \Bx e^{ \Bu \Bv \theta } = e^{-\Bu \Bv \theta} \Bx.
\end{equation}

The sense of rotation for the rotation \( e^{ \Bu \Bv \theta} \) is opposite that of \( e^{\Bv \Bu \theta} \), which provides a first hint that bivectors can be characterized as having an orientation, somewhat akin to thinking of a vector as having a head and a tail.

\makeexample{Velocity and acceleration in polar coordinates.}{example:2dRotations:1180}{
Complex exponential representations of rotations work very nicely for describing vectors in polar coordinates.
A radial vector can be written as
\begin{equation}\label{eqn:2dRotations:1200}
\Br = r \rcap,
\end{equation}
as illustrated in \cref{fig:radialVectorCylindrical:radialVectorCylindricalFig1}.
The polar representation of the radial and azimuthal unit vector are simply
\mathImageFigure{../figures/GAelectrodynamics/radialVectorCylindricalFig1}{Radial vector in polar coordinates.}{fig:radialVectorCylindrical:radialVectorCylindricalFig1}{0.3}{radialVectorCylindricalFig1.nb}
\begin{equation}\label{eqn:2dRotations:1220}
\begin{aligned}
\rcap &= \Be_1 e^{i\theta} =
\Be_1 \lr{ \cos\theta + \Be_1 \Be_2 \sin\theta } = \Be_1 \cos\theta + \Be_2 \sin\theta \\
\thetacap &= \Be_2 e^{i\theta} =
\Be_2 \lr{ \cos\theta + \Be_1 \Be_2 \sin\theta } = \Be_2 \cos\theta - \Be_1 \sin\theta,
\end{aligned}
\end{equation}
where \( i = \Be_{12} \) is the unit bivector for the x-y plane.  We can easily show that these unit vectors are orthogonal
\begin{equation}\label{eqn:2dRotations:1340}
\begin{aligned}
\rcap \thetacap
&= \lr{ \Be_1 e^{i \theta}} \lr{ e^{-i\theta} \Be_2} \\
&= \Be_1 \cancel{e^{i \theta} e^{-i\theta}} \Be_2 \\
&= \Be_1 \Be_2.
\end{aligned}
\end{equation}
By \cref{thm:multiplication:anticommutationNormal}, since the product of \( \rcap \thetacap \) is a bivector,
\( \rcap \) is orthogonal to \( \thetacap \).

We can find the
velocity and acceleration by taking time derivatives
\begin{equation}\label{eqn:2dRotations:1240}
\begin{aligned}
\Bv &= r' \rcap + r \rcap' \\
\Ba &= r'' \rcap + 2 r' \rcap' + r \rcap'',
\end{aligned}
\end{equation}
but to make these more meaningful want to evaluate the \( \rcap, \thetacap \) derivatives explicitly.  Those are
\begin{equation}\label{eqn:2dRotations:1260}
\begin{aligned}
\rcap' &= \lr{ \Be_1 e^{i \theta} }' =
\mathLabelBox[ labelstyle={yshift=1.2em}, linestyle={} ]
{
\Be_1 i
}
{
\(\Be_1 (\Be_1 \Be_2) = (\Be_1 \Be_1) \Be_2 \)
}
 e^{i\theta} \theta' = \Be_2 e^{i\theta} \theta' = \thetacap \omega \\
\thetacap' &= \lr{ \Be_2 e^{i \theta} }' =
\mathLabelBox[ labelstyle={below of=m\themathLableNode, below of=m\themathLableNode} ]
{
\Be_2 i
}
{
\(\Be_2 \Be_1 \Be_2
=
(-\Be_1 \Be_2) \Be_2\)
}
 e^{i\theta} \theta' = -\Be_1 e^{i\theta} \theta' = -\rcap \omega,
\end{aligned}
\end{equation}
where \( \omega = d\theta/dt \), and primes denote time derivatives.  The velocity and acceleration vectors can now be written explicitly in terms of radial and azimuthal components.  The velocity is
\begin{equation}\label{eqn:2dRotations:1280}
\Bv = r' \rcap + r \omega \thetacap,
\end{equation}
and the acceleration is
\begin{equation}\label{eqn:2dRotations:1300}
\begin{aligned}
\Ba
&= r'' \rcap + 2 r' \omega \thetacap + r (\omega \thetacap)' \\
&= r'' \rcap + 2 r' \omega \thetacap + r \omega' \thetacap - r \omega^2 \rcap,
\end{aligned}
\end{equation}
or
\begin{equation}\label{eqn:2dRotations:1320}
\Ba
= \rcap \lr{ r'' - r \omega^2 }
+ \inv{r} \thetacap \lr{ r^2 \omega }'.
\end{equation}

Using \cref{eqn:2dRotations:1220}, we also have the option of factoring out the rotation operation from the position vector or any of its derivatives
\begin{equation}\label{eqn:2dRotations:1360}
\begin{aligned}
\Br &= \lr{ r \Be_1 } e^{i \theta } \\
\Bv &= \lr{ r' \Be_1 + r \omega \Be_2 } e^{i \theta } \\
\Ba &= \lr{ \lr{ r'' - r \omega^2 } \Be_1 + \inv{r} \lr{ r^2 \omega }' \Be_2 } e^{i\theta}.
\end{aligned}
\end{equation}

In particular,
for uniform circular motion, each of the position, velocity and acceleration vectors can be represented by a vector that is fixed in space, subsequently rotated by an angle \( \theta \).
} % example
\makeproblem{\R{2} rotations.}{problem:2dRotations:1}{
Using familiar methods, such as rotation matrices, show that the counterclockwise and clockwise rotations of
\cref{eqn:2dRotations:280} are given by
\cref{eqn:2dRotations:300} and
\cref{eqn:2dRotations:3} respectively.
} % problem
\makeanswer{problem:2dRotations:1}{
The 2D rotation matrix is
\begin{equation}\label{eqn:2dRotations:1380}
R_\theta =
\begin{bmatrix}
   \cos\theta & -\sin\theta \\
   \sin\theta & \cos\theta
\end{bmatrix},
\end{equation}
so to rotate coordinates by \(\pm\pi/2\), we multiply by
\begin{equation}\label{eqn:2dRotations:1400}
R_{\pm\pi/2} = \pm
\begin{bmatrix}
   0 & -1 \\
   1 & 0
\end{bmatrix}.
\end{equation}
In particular
\begin{equation}\label{eqn:2dRotations:1420}
   R_{\pm\pi/2}
\begin{bmatrix}
   \rho \cos\theta \\
   \rho \sin\theta
\end{bmatrix}
   = \pm \pi/2
\begin{bmatrix}
   0 & -1 \\
   1 & 0
\end{bmatrix}
\begin{bmatrix}
   \rho \cos\theta \\
   \rho \sin\theta
\end{bmatrix}
=
\pm
\rho
\begin{bmatrix}
-\sin\theta \\
\cos\theta
\end{bmatrix},
\end{equation}
consistent with the results observed from left and right multiplication with the plane pseudoscalar \( \Be_1 \Be_2 \).
}
\makeproblem{Multivector Euler's formula and trig relations.}{problem:2dRotations:Euler}{
For a multivector \( x \) assume an infinite series representation of the exponential, sine and cosine functions and their hyperbolic analogues
\begin{equation*}
\begin{aligned}
e^x &= \sum_{k = 0}^\infty \frac{x^k}{k!} \\
\cos x &= \sum_{k = 0}^\infty (-1)^k \frac{x^{2k}}{(2k)!} \qquad \sin x = \sum_{k = 0}^\infty (-1)^k \frac{x^{2k+1}}{(2k+1)!} \\
\cosh x &= \sum_{k = 0}^\infty \frac{x^{2k}}{(2k)!} \qquad \sinh x = \sum_{k = 0}^\infty \frac{x^{2k+1}}{(2k+1)!} \\
\end{aligned}
\end{equation*}
\makesubproblem{}{problem:2dRotations:Euler:a}
Show that for scalar \( \theta \), and any multivector \( J \) that satisfies \( J^2 = -1 \)
\begin{equation*}
\begin{aligned}
\cosh (J \theta ) &= \cos \theta \\
\sinh (J \theta ) &= J \sin \theta.
\end{aligned}
\end{equation*}
\makesubproblem{}{problem:2dRotations:Euler:c}
Show that the trigonometric and hyperbolic Euler formulas
\begin{equation*}
\begin{aligned}
e^{ J \theta } &= \cos \theta + J \sin \theta \\
e^{ K \theta } &= \cosh \theta + K \sinh \theta,
\end{aligned}
\end{equation*}
hold for multivectors \( J, K \) satisfying \( J^2 = -1 \) and \( K^2 = 1 \) respectively.
\makesubproblem{}{problem:2dRotations:Euler:b}
Given multivectors \( X, Y \), show that \( e^{ X + Y } = e^{ X } e^{ Y } \) if \( X, Y \) commute.  That is \( X Y = Y X \).
} % problem

         \subsection{Vector product, dot product and wedge product.}
            %
% Copyright © 2017 Peeter Joot.  All Rights Reserved.
% Licenced as described in the file LICENSE under the root directory of this GIT repository.
%
\index{vector product}
The product of two colinear vectors is a scalar, and the product of two normal vectors is a bivector.
The product of two general vectors is a multivector with structure to be determined.
A powerful way to examine this structure is to compute the product of two vectors in a polar representation with respect to the plane that they span.
Let \( \ucap \) and \( \vcap \) be an orthonormal pair of vectors in the plane of \( \Ba \) and \( \Bb \), oriented in a positive rotational sense as illustrated in
\cref{fig:Parallelogram:ParallelogramFig1}.
\imageFigure{../figures/GAelectrodynamics/ParallelogramFig1}{Two vectors in a plane.}{fig:Parallelogram:ParallelogramFig1}{0.3}

With respect to the orthonormal vectors \( \ucap \) and \( \vcap \), a
a polar representation of \( \Ba, \Bb \) is

\begin{dmath}\label{eqn:products:1660}
\begin{aligned}
\Ba &= \Norm{\Ba} \ucap e^{ i_{ab} \theta_a } = \Norm{\Ba} e^{ -i_{ab} \theta_a } \ucap \\
\Bb &= \Norm{\Bb} \ucap e^{ i_{ab} \theta_b } = \Norm{\Bb} e^{ -i_{ab} \theta_b } \ucap,
\end{aligned}
\end{dmath}
where \( i_{ab} = \ucap \vcap \) is a unit pseudoscalar for the planar subspace spanned by \( \Ba \) and \( \Bb \).
The vector product of these two vectors is

\begin{dmath}\label{eqn:products:1680}
\Ba \Bb
=
\lr{ \Norm{\Ba} e^{ -i_{ab} \theta_a } \ucap } \lr{ \Norm{\Bb} \ucap e^{ i_{ab} \theta_b } }
=
 \Norm{\Ba} \Norm{\Bb}
e^{ -i_{ab} \theta_a } ( \ucap \ucap ) e^{ i_{ab} \theta_b }
=
 \Norm{\Ba} \Norm{\Bb}
e^{ i_{ab} (\theta_b - \theta_a)}.
=
 \Norm{\Ba} \Norm{\Bb}
\lr{
\cos
(\theta_b - \theta_a)
+ i_{ab}
\sin
(\theta_b - \theta_a)
}.
\end{dmath}

We see that the product of two vectors is a multivector that has only grades 0 and 2.
This can be expressed symbolically as

\begin{dmath}\label{eqn:products:1800}
\Ba \Bb
=
\gpgradezero{ \Ba \Bb }
+
\gpgradetwo{ \Ba \Bb }.
\end{dmath}

We recognize the scalar grade of the vector product as the \R{N} dot product, but the grade 2 component of the vector product is something new that requires a name.
We respectively identify and define operators for these vector grade selection operations

\index{wedge product}
\index{dot product}
\makedefinition{Dot and wedge products of two vectors.}{dfn:products:dotandwedge}{
Given two vectors \( \Ba, \Bb \in \bbR^{N} \) the dot product is identified as the scalar grade of their product
\begin{equation*}
\gpgradezero{ \Ba \Bb }
=
\Ba \cdot \Bb
.
\end{equation*}

A wedge product of the vectors is defined as a grade-2 selection operation
\begin{equation*}
\Ba \wedge \Bb \equiv \gpgradetwo{ \Ba \Bb }.
\end{equation*}

Given this notation, the product of two vectors can be written
\begin{equation*}
\Ba \Bb = \Ba \cdot \Bb + \Ba \wedge \Bb.
\end{equation*}
} % definition

\index{grade selection}
Scalar grade selection of a product of two vectors is an important new tool.
There will be many circumstances where the easiest way to compute a dot product is using scalar grade selection.

The split of a vector product into dot and wedge product components is also important.
However, to utilize it, the properties of the wedge product have to be determined.

Summarizing \cref{eqn:products:1680} with our new operators, we write
\boxedEquation{eqn:SimpleProducts2:1700}{
\begin{aligned}
\Ba \Bb &= \Norm{\Ba} \Norm{\Bb} \exp\lr{ i_{ab} (\theta_b - \theta_a) } \\
\Ba \cdot \Bb &= \Norm{\Ba} \Norm{\Bb} \cos( \theta_b - \theta_a ) \\
\Ba \wedge \Bb &= i_{ab} \Norm{\Ba} \Norm{\Bb} \sin( \theta_b - \theta_a ),
\end{aligned}
}

\makeexample{Products of two unit vectors.}{example:products:1860}{
To develop some intuition about the vector product, let's consider product of two unit vectors \( \Ba, \Bb \) in the equilateral triangle of
\cref{fig:exponentialVectorProduct:exponentialVectorProductFig1}, where

\begin{dmath}\label{eqn:products:1880}
\begin{aligned}
\Ba &= \inv{\sqrt{2}} \lr{ \Be_3 + \Be_1 } = \Be_3 \exp\lr{ \Be_{31} \pi/4 } \\
\Bb &= \inv{\sqrt{2}} \lr{ \Be_3 + \Be_2 } = \Be_3 \exp\lr{ \Be_{32} \pi/4 }.
\end{aligned}
\end{dmath}

\imageFigure{../figures/GAelectrodynamics/exponentialVectorProductFig1}{Equilateral triangle in \R{3}.}{fig:exponentialVectorProduct:exponentialVectorProductFig1}{0.3}

The product of these vectors is

\begin{dmath}\label{eqn:products:1900}
\Ba \Bb
=
\inv{2} \lr{ \Be_3 + \Be_1 } \lr{ \Be_3 + \Be_2 }
=
\inv{2} \lr{ 1 + \Be_{32} + \Be_{13} + \Be_{12 } }
=
\inv{2} + \frac{\sqrt{3}}{2} \frac{\Be_{32} + \Be_{13} + \Be_{12 } }{\sqrt{3}}.
\end{dmath}

Let the bivector factor be designated

\begin{dmath}\label{eqn:products:1920}
j = \frac{\Be_{32} + \Be_{13} + \Be_{12 } }{\sqrt{3}}.
\end{dmath}

The reader can check (\cref{problem:products:2000})
that \( j \) is a unit bivector (i.e. it squares to \( -1 \)), allowing us to write

\begin{dmath}\label{eqn:products:1940}
\Ba \Bb =
\inv{2} + \frac{\sqrt{3}}{2} j
= \cos(\pi/3) + j \sin(\pi/3)
= \exp\lr{ j \pi/3 }.
\end{dmath}

Since both vector factors were unit length, this ``complex'' exponential has no leading scalar factor contribution from \( \Norm{\Ba} \Norm{\Bb} \).

From the above analysis, we can also see that in this case, the product of two 0,2 multivectors was itself a 0,2 multivector.  We can see that by forming the product

\begin{dmath}\label{eqn:products:1960}
\Ba \Bb =
\biglr{
   \exp\lr{ -\Be_{31} \pi/4 } \Be_3 }
\biglr{
   \Be_3 \exp\lr{ \Be_{32} \pi/4 }
}
=
\exp\lr{ -\Be_{31} \pi/4 }
\exp\lr{ \Be_{32} \pi/4 },
\end{dmath}
so
\begin{dmath}\label{eqn:products:1980}
\exp\lr{ \Be_{13} \pi/4 }
\exp\lr{ \Be_{32} \pi/4 } = \exp\lr{ j \pi/3 }.
\end{dmath}

Composition of two rotation operators has resulted in another rotation operator, but that result is in a different rotational plane, and through a different rotation angle.
} % example

\index{polar representation}
Two wedge product properties can be immediately deduced from the polar representation of \cref{eqn:SimpleProducts2:1700}

\begin{enumerate}
\item \( \Bb \wedge \Ba = - \Ba \wedge \Bb \).
\item \( \Ba \wedge (\alpha \Ba) = 0, \quad \forall \alpha \in \bbR \).
\end{enumerate}

We have now had a few hints that the wedge product might be related to the cross product.  Given two vectors \( \Ba, \Bb \) both the wedge and the cross product contain a \( \Norm{\Ba} \Norm{\Bb} \sin \Delta \theta \) factor, and both the wedge and cross product are antisymmetric operators.
The cross product is a bilinear operator \( (\Ba + \Bb) \cross (\Bc + \Bd) = \Ba \cross \Bc + \Ba \cross \Bc + \cdots \).  To see whether this is the case for the wedge product, let's examine the coordinate expansion of the wedge product.  Let

\begin{dmath}\label{eqn:products:1160}
\begin{aligned}
\Ba &= \sum_i a_i \Be_i \\
\Bb &= \sum_i b_i \Be_i.
\end{aligned}
\end{dmath}

The product of these vectors is

\begin{dmath}\label{eqn:products:1360}
\Ba \Bb
=
\lr{ \sum_i a_i \Be_i } \lr{ \sum_j b_j \Be_j }
=
\sum_{ij} a_i b_j \Be_i \Be_j
=
\sum_{i = j} a_i b_j \Be_i \Be_j
+
\sum_{i \ne j} a_i b_j \Be_i \Be_j.
\end{dmath}

Since \( \Be_i \Be_i = 1 \), we see again that the scalar component of the product is the dot product \( \sum_i a_i b_i \).
The remaining grade 2 components are the wedge product, for which the coordinate expansion can be simplified further

\begin{dmath}\label{eqn:products:1460}
\Ba \wedge \Bb
=
\sum_{i \ne j} a_i b_j \Be_i \Be_j
=
\sum_{i < j} a_i b_j \Be_i \Be_j
+
\sum_{j < i} a_i b_j \Be_i \Be_j
=
\sum_{i < j} a_i b_j \Be_i \Be_j
+
\sum_{i < j} a_j b_i \Be_j \Be_i
%=
%\sum_{i < j} a_i b_j \Be_i \Be_j
%+
%\sum_{i < j} a_j b_i (-\Be_i \Be_j)
=
\sum_{i < j} (a_i b_j - a_j b_i) \Be_i \Be_j.
\end{dmath}

\index{determinant!wedge product}
The scalar factors can be written as \( 2 x 2 \) determinants
\boxedEquation{eqn:products:1320}{
\Ba \wedge \Bb
=
\sum_{i < j}
\begin{vmatrix}
a_i & a_j \\
b_i & b_j
\end{vmatrix}
\Be_i \Be_j.
}

It is now straightforward to show that the wedge product is distributive and bilinear (\cref{problem:products:bilinear}).
It is also simple to use \cref{eqn:products:1320} to show that \( \Bb \wedge \Ba = -\Ba \wedge \Bb \) and \( \Ba \wedge \Ba = 0 \).

For \R{2} there is only one term in \cref{eqn:products:1320}

\begin{dmath}\label{eqn:products:1720}
\Ba \wedge \Bb
=
\begin{vmatrix}
a_1 & a_2 \\
b_1 & b_2
\end{vmatrix}
\Be_1 \Be_2.
\end{dmath}

\index{cross product}
We are used to writing the cross product as a \( 3 x 3 \) determinant, which can also be done with the coordinate expansion of the
\R{3} wedge product

\begin{equation}\label{eqn:products:1740}
\Ba \wedge \Bb
=
\sum_{ ij \in \setlr{ 12, 13, 23 } }
\begin{vmatrix}
a_i & a_j \\
b_i & b_j
\end{vmatrix}
\Be_i \Be_j
=
\begin{vmatrix}
\Be_2 \Be_3 & \Be_3 \Be_1 & \Be_1 \Be_2 \\
a_1 & a_2 & a_3 \\
b_1 & b_2 & b_3 \\
\end{vmatrix}.
\end{equation}

Let's summarize the wedge product properties and relations we have found so far, comparing the \R{3} wedge product to the cross product

\begin{tcolorbox}[tab2,tabularx={X||Y|Y},title=Cross product and \R{3} wedge product comparison.,boxrule=0.5pt]
Property & Cross product & Wedge product
\\ \hline
Same vectors & \( \Ba \cross \Ba = 0 \) & \( \Ba \wedge \Ba = 0 \)
\\ \hline
Antisymmetry & \( \Bb \cross \Ba = -\Ba \cross \Bb \) & \( \Bb \wedge \Ba = -\Ba \wedge \Bb \)
\\ \hline
Linear & \( \Ba \cross (\alpha \Bb) = \alpha (\Ba \cross \Bb) \) &
\( \Ba \wedge (\alpha \Bb) = \alpha (\Ba \wedge \Bb) \)
\\ \hline
Distributive
& \( \Ba \cross (\Bb + \Bc) = \Ba \cross \Bb + \Ba \cross \Bc \)
& \( \Ba \wedge (\Bb + \Bc) = \Ba \wedge \Bb + \Ba \wedge \Bc \)
\\ \hline
Determinant expansion
&
\(
\Ba \cross \Bb
=
\begin{vmatrix}
\Be_1 & \Be_2 & \Be_3 \\
a_1 & a_2 & a_3 \\
b_1 & b_2 & b_3 \\
\end{vmatrix}
\)
&
\(
\Ba \wedge \Bb
=
\begin{vmatrix}
\Be_2 \Be_3 & \Be_3 \Be_1 & \Be_1 \Be_2 \\
a_1 & a_2 & a_3 \\
b_1 & b_2 & b_3 \\
\end{vmatrix}
\)
\\ \hline
Polar form &
\( \ncap_{ab} \Norm{\Ba} \Norm{\Bb} \sin( \theta_b - \theta_a )  \) &
\( i_{ab} \Norm{\Ba} \Norm{\Bb} \sin( \theta_b - \theta_a )  \)
\\ \hline
\end{tcolorbox}

All the wedge properties except the determinant expansion above are valid in any dimension.
Comparing \cref{eqn:products:1740} to the determinant representation of the cross product, and referring to
\cref{eqn:dual:1580}, shows that
the \R{3} wedge product is related to the cross product by a duality transformation \( i_{ab} = I \ncap_{ab} \),
or
\boxedEquation{eqn:SimpleProducts2:1620}{
\Ba \wedge \Bb = I ( \Ba \cross \Bb ).
}

The direction of the cross product \( \Ba \cross \Bb \) is normal to the plane represented by the bivector \( \Ba \wedge \Bb \).  The magnitude of both (up to a sign) is the area of the parallelogram spanned by the two vectors.

\makeexample{Wedge and cross product relationship.}{example:products:2000}{
To take some of the abstraction from \cref{eqn:SimpleProducts2:1620} let's consider a specific example.  Let
\begin{dmath}\label{eqn:products:2020}
\begin{aligned}
\Ba &= \Be_1 + 2 \Be_2 + 3 \Be_3 \\
\Bb &= 4 \Be_1 + 5 \Be_2 + 6 \Be_3.
\end{aligned}
\end{dmath}
The reader should check that the cross product of these two vectors is
\begin{dmath}\label{eqn:products:2040}
\Ba \cross \Bb = -3 \Be_1 - 6 \Be_2 - 3 \Be_3.
\end{dmath}
By direct computation, we find that the wedge and the cross products are related by a \R{3} pseudoscalar factor
\begin{dmath}\label{eqn:products:2060}
\Ba \wedge \Bb
=
\lr{ \Be_1 + 2 \Be_2 + 3 \Be_3 } \wedge \lr{ 4 \Be_1 + 5 \Be_2 + 6 \Be_3 }
=
5 \Be_{12} + 6 \Be_{13}
+ 8 \Be_{21} + 12 \Be_{23}
+ 12 \Be_{31} + 15 \Be_{32}
=
3 \Be_{21} + 6 \Be_{31} + 3 \Be_{32}
=
-3 \Be_{12} - 6 \Be_{13} - 3 \Be_{23}
=
\Be_{123} (-3 \Be_{3}) + \Be_{132}(- 6 \Be_{2}) + \Be_{231}(- 3 \Be_{1})
=
\Be_{123} (-3 \Be_{3} - 6 \Be_{2} - 3 \Be_{1})
=
I (\Ba \cross \Bb).
\end{dmath}
} % example

The relationship between the wedge and cross products allows us to express the
\R{3} vector product as a multivector combination of the dot and cross products
\boxedEquation{eqn:SimpleProducts2:1640}{
\Ba \Bb = \Ba \cdot \Bb + I(\Ba \cross \Bb).
}

This is a very important relationship.

In particular, for electromagnetism, \cref{eqn:SimpleProducts2:1640} can
be used to combine (the scalar) Guass's law with (the vector) Maxwell-Faraday equation, and to
combine (the scalar) Gauss's law for magnetism with (the vector) Ampere-Maxwell equation.
Such dot plus cross product sums will yield
a pair of multivector equations that can be further merged.  The resulting multivector equation will be
called
Maxwell equation (singular), and will be the starting point of all our electromagnetic analysis.

\index{colinear vectors!wedge}
\makeproblem{Wedge product of colinear vectors.}{problem:SimpleProducts2:wedgecolinear}{
Given \( \Bb = \alpha \Ba \), use
\cref{eqn:products:1320} to show that the wedge product of any pair of colinear vectors is zero.
} % problem

\makeproblem{Wedge product antisymmetry.}{problem:SimpleProducts2:1560}{
Prove that the wedge product is antisymmetric using using \cref{eqn:products:1320}.
} % problem

\makeproblem{Wedge product distributivity and linearity.}{problem:products:bilinear}{
For vectors \( \Ba, \Bb, \Bc \) and \( \Bd \), and scalars \( \alpha, \beta \) use
\cref{eqn:products:1320} to show that

\makesubproblem{}{problem:products:bilinear:a}

the wedge product is distributive
\begin{equation*}
(\Ba + \Bb) \wedge (\Bc + \Bd) =
\Ba \wedge \Bc
+
\Ba \wedge \Bd
+
\Bb \wedge \Bc
+
\Bb \wedge \Bd,
\end{equation*}
\makesubproblem{}{problem:products:bilinear:b}
and show that the wedge product is bilinear
\begin{equation*}
(\alpha \Ba) \wedge (\beta \Bb)
=
(\alpha \beta) (\Ba \wedge \Bb).
\end{equation*}
Note that these imply the wedge product also has the cross product filtering property \( \Ba \wedge (\Bb + \alpha \Ba) = \Ba \wedge \Bb \).
} % problem

%Answer (partial)
%Swapping \( \Ba \) and \( \Bb \),
%
%\begin{dmath}\label{eqn:products:1820}
%\Bb \wedge \Ba
%=
%\sum_{i < j}
%\begin{vmatrix}
%b_i & b_j \\
%a_i & a_j
%\end{vmatrix}
%\Be_i \Be_j
%=
%-\sum_{i < j}
%\begin{vmatrix}
%a_i & a_j \\
%b_i & b_j \\
%\end{vmatrix}
%\Be_i \Be_j
%=
%-\Ba \wedge \Bb,
%\end{dmath}
%
%proves that
%\cref{eqn:products:1780} holds for any vectors, and not just when they are normal.
%Because of this antisymmetry we also have
%
%\begin{dmath}\label{eqn:products:1840}
%\Ba \wedge \Ba = 0.
%\end{dmath}
%

\makeproblem{Unit bivector.}{problem:products:2000}{
Verify by explicit multiplication that the bivector of \cref{eqn:products:1920} squares to \( -1 \).
} % problem


         \subsection{Reverse.}
            %
% Copyright © 2017 Peeter Joot.  All Rights Reserved.
% Licenced as described in the file LICENSE under the root directory of this GIT repository.
%
%
% Copyright � 2016 Peeter Joot.  All Rights Reserved.
% Licenced as described in the file LICENSE under the root directory of this GIT repository.
%
\index{reverse}
\makedefinition{Reverse}{dfn:reverse:1}{

Let \( A \) be a multivector with j multivector factors,
\( A = B_1 B_2 \cdots B_j \),
not necessarily normal.
The reverse \( A^\dagger \), or reversion, of this multivector \( A \) is
\begin{equation*}
A^\dagger = B_j^\dagger B_{j-1}^\dagger \cdots B_1^\dagger.
\end{equation*}

Scalars and vectors are their own reverse, and
the reverse of a sum of multivectors is the sum of the reversions of its summands.
} % definition

Examples:

\begin{dmath}\label{eqn:reverseDefined:21}
\begin{aligned}
\lr{ 1 + 2 \Be_{12} + 3 \Be_{321} }^\dagger &= 1 + 2 \Be_{21} + 3 \Be_{123} \\
\lr{ (1 + \Be_1)(\Be_{23} - \Be_{12} }^\dagger &= (\Be_{32} + \Be_{12})(1 + \Be_1).
\end{aligned}
\end{dmath}




The reverse of a k-blade has useful closed form expression

\maketheorem{Reverse of k-blade.}{thm:reverse:kBlade}{
The reverse of a k-blade \( A_k = \Ba_1 \Ba_2 \cdots \Ba_k \) is given by
\begin{equation*}
A_k^\dagger = (-1)^{k(k-1)/2} A_k.
\end{equation*}
} % theorem

This can be proven by successive interchange of the factors

\begin{dmath}\label{eqn:scalarPermutation:81}
\begin{aligned}
A_k^\dagger
&= \Ba_k \Ba_{k-1} \cdots \Ba_1 \\
&= (-1)^{k-1} \Ba_1 \Ba_k \Ba_{k-1} \cdots \Ba_2 \\
&= (-1)^{k-1} (-1)^{k-2} \Ba_1 \Ba_2 \Ba_k \Ba_{k-1} \cdots \Ba_3 \\
&\qquad \vdots \\
&= (-1)^{k-1} (-1)^{k-2} \cdots (-1)^1 \Ba_1 \Ba_2 \cdots \Ba_k,
&= (-1)^{k(k-1)/2} \Ba_1 \Ba_2 \cdots \Ba_k. \qquad \qedmarker
\end{aligned}
\end{dmath}

A special, but important case, is the reverse of the \R{3} pseudoscalar, which is negated by reversion

\boxedEquation{eqn:reverse:103}{
I^\dagger = -I.
}


         \subsection{Imaginary nature of the \R{3} pseudoscalar.}
            % depends on reverse
            %
% Copyright © 2016 Peeter Joot.  All Rights Reserved.
% Licenced as described in the file LICENSE under the root directory of this GIT repository.
%

As with the use of the symbol \( i \) for the \R{2} pseudoscalar, it is not a coincidence that \( I \) was used 
for the 
\R{3} pseudoscalar.  It is also true that 
\( I = \Be_1 \Be_2 \Be_3 \) behaves like a complex imaginary with \( I^2 = -1 \).  This follows 
directly from repeated anticommutation

\begin{dmath}\label{eqn:projectionAndRejection:1140}
I^2
=
(\Be_1 \Be_2 \Be_3)(\Be_1 \Be_2 \Be_3)
=
\Be_1 \Be_2 (\Be_3 \Be_1) \Be_2 \Be_3
=
\Be_1 \Be_2 (-\Be_1 \Be_3) \Be_2 \Be_3
=
-\Be_1 \Be_2 \Be_1 (\Be_3 \Be_2) \Be_3
=
-\Be_1 \Be_2 \Be_1 (-\Be_2 \Be_3) \Be_3
=
+\Be_1 \Be_2 \Be_1 \Be_2 (\Be_3 \Be_3)
=
(\Be_1 \Be_2)^2
=
-1.
\end{dmath}

%\makeproblem{\R{3} pseudoscalar square}{problem:gradeselection:R3PseudoscalarSquare}{
%With the \R{3} pseudoscalar of \cref{eqn:definitions:340} show that \( I^2 = -1 \).
%} % problem
%
%\makeanswer{problem:gradeselection:R3PseudoscalarSquare}{
%
%\begin{dmath}\label{eqn:gaTutorial:160}
%I^2
%=
%...
%=
%-1,
%\end{dmath}
%
%as expected, showing that this quantity also has characteristics of an imaginary number.
%} % answer

         \subsection{Multivector dot product.}
            %
% Copyright © 2017 Peeter Joot.  All Rights Reserved.
% Licenced as described in the file LICENSE under the root directory of this GIT repository.
%

In general the product of two k-vectors is a multivector, with a selection of different grades.
For example, the product of two bivectors may have grades 0, 2, or 4

\begin{dmath}\label{eqn:generalizedDot:601}
\Be_{12} \lr{ \Be_{21} + \Be_{23} + \Be_{34} }
=
1 + \Be_{13} + \Be_{1234}.
\end{dmath}

Similarly,
the product of a vector and bivector generally has grades 1 and 3

\begin{dmath}\label{eqn:generalizedDot:621}
\Be_1 \lr{ \Be_{12} + \Be_{23} }
=
\Be_2 + \Be_{123}.
\end{dmath}

We've identified the vector dot product with scalar grade selection of their vector product, the selection of the lowest grade of their product.
This motivates the definition of a general multivector dot product

\index{multivector dot product}
\makedefinition{Multivector dot product}{dfn:generalizedDot:100}{
The dot (or inner) product of two multivectors
\( A = \sum_{i = 0}^N \gpgrade{A}{i}, B = \sum_{i = 0}^N \gpgrade{B}{i} \)
is defined as
\begin{equation*}
A \cdot B \equiv
\sum_{i,j = 0}^N \gpgrade{ A_i B_j }{\Abs{i - j}}.
\end{equation*}
}

If \( A, B \) are k-vectors with equal grade, then the dot product is just the scalar selection of their product

\begin{dmath}\label{eqn:generalizedDot:580}
A \cdot B = \gpgradezero{ A B },
\end{dmath}

and if \( A, B \) are a k-vectors with grades \( r \ne s \) respectively, then their dot product is a single grade selection

\begin{dmath}\label{eqn:generalizedDot:581}
A \cdot B = \gpgrade{ A B }{\Abs{r - s}}.
\end{dmath}


            \subsubsection{Problems.}
               %
% Copyright © 2017 Peeter Joot.  All Rights Reserved.
% Licenced as described in the file LICENSE under the root directory of this GIT repository.
%
\makeproblem{Index permuation in vector selection.}{problem:generalizedDot:distributionUnitVectorsa}{
Prove \cref{eqn:generalizedDot:1020}.
} % problem

\makeproblem{Dot product of unit vector with unit bivector.}{problem:generalizedDot:distributionUnitVectorsb}{
Prove \cref{eqn:generalizedDot:980}.
} % problem

         \subsection{Permuation within scalar selection.}
            As scalar selection is at the heart of the
generalized dot product, it is worth knowing
some of the ways that such a selection operation can be manipulated.

\maketheorem{Permutation of multivector products in scalar selection.}{theorem:scalarPermutation:1}{
The factors within a scalar grade selection of a pair of multivector products may be permuted or may be cyclically permuted
\begin{equation*}
\begin{aligned}
\gpgradezero{A B} &= \gpgradezero{B A} \\
\gpgradezero{A B \cdots Y Z} &= \gpgradezero{Z A B \cdots Y}.
\end{aligned}
\end{equation*}
} % problem

It is suffient to prove just the two multivector permutation case.
One simple, but inelegant method, is to first expand the pair of multivectors in coordinates.  Let

\begin{equation}\label{eqn:scalarSelectionPermutation:40}
\begin{aligned}
A &= a_0 + \sum_i a_i \Be_i + \sum_{i < j} a_{ij} \Be_{ij} + \cdots \\
B &= b_0 + \sum_i b_i \Be_i + \sum_{i < j} b_{ij} \Be_{ij} + \cdots
\end{aligned}
\end{equation}

Only the products of like unit blades can contribute scalar components to the sum, so the
the scalar selection of the products must have the form

\begin{dmath}\label{eqn:scalarSelectionPermutation:60}
\gpgradezero{ A B }
=
a_0 b_0 + \sum_i a_i b_i \Be_i^2 + \sum_{i < j} a_{ij} b_{ij} \Be_{ij}^2 + \cdots
\end{dmath}

This sum is also clearly equal to \( \gpgradezero{ B A } \), completing the proof.

         \subsection{Multivector wedge product.}
            %
% Copyright © 2017 Peeter Joot.  All Rights Reserved.
% Licenced as described in the file LICENSE under the root directory of this GIT repository.
%
We've identified the vector wedge product with scalar grade selection of their vector product, the selection of the highest grade of their product.
Looking back to the multivector products of \cref{eqn:generalizedDot:601}, and \cref{eqn:generalizedDot:621} as motivation,
a generalized wedge product can be defined that selects the highest grade terms of a given multivector product

\index{multivector wedge product}
\makedefinition{Multivector wedge product.}{dfn:generalizedWedge:480}{
For the multivectors \( A, B \) defined in \cref{dfn:generalizedDot:100}, the wedge (or outer) product is defined as
\begin{equation*}
A \wedge B
\equiv
\sum_{i,j = 0}^N \gpgrade{ A_i B_j }{i + j}.
\end{equation*}
} % definition

If \( A, B \) are a k-vectors with grades \( r, s \) respectively, then their wedge product is a single grade selection

\begin{dmath}\label{eqn:generalizedWedge:560}
A \wedge B = \gpgrade{ A B }{ r + s}.
\end{dmath}

The most important example of the generalized wedge is the wedge product of a vector with a 2-blade

\maketheorem{Wedge of three vectors.}{thm:generalizedWedge:vectorTwoBlade}{
The wedge product of three vectors is associative
\begin{equation*}
(\Ba \wedge \Bb) \wedge \Bc = \Ba \wedge (\Bb \wedge \Bc),
\end{equation*}
so can be written simply as \( \Ba \wedge \Bb \wedge \Bc \).
} % theorem

The proof follows directly from the definition

\begin{dmath}\label{eqn:generalizedWedge:580}
(\Ba \wedge \Bb) \wedge \Bc
=
\gpgradethree{ (\Ba \wedge \Bb) \Bc }
=
\gpgradethree{ (\Ba \Bb -\Ba \cdot \Bb) \Bc }
=
\gpgradethree{ \Ba \Bb \Bc }.
\end{dmath}

Similarly

\begin{dmath}\label{eqn:generalizedWedge:600}
\Ba \wedge (\Bb \wedge \Bc)
=
\gpgradethree{ \Ba (\Bb \wedge \Bc) }
=
\gpgradethree{ \Ba (\Bb \Bc - \Bb \cdot \Bc) }
=
\gpgradethree{ \Ba \Bb \Bc },
\end{dmath}

which proves the theorem.
It is simple to show that the wedge of three vectors is completely antisymmetric (any interchange of vectors changes the sign), and that cyclic permutation \( \Ba \rightarrow \Bb \rightarrow \Bc \rightarrow \Ba \) of the vectors leaves it unchanged
(\cref{problem:generalizedWedge:tripleWedgeProperties}).
These properties are also common to the triple product of \R{3} vector algebra, a fact that is associated with the fact that there is also a determinant structure to the triple wedge product, which can be shown by direct expansion in coordinates

\begin{dmath}\label{eqn:generalizedWedge:620}
\Ba \wedge \Bb \wedge \Bc
=
\gpgradethree{ a_i b_j c_k \Be_i \Be_j \Be_k }
=
\sum_{i \ne j \ne k}
a_i b_j c_k \Be_i \Be_j \Be_k
=
\sum_{i < j < k}
\begin{vmatrix}
a_i & a_j & a_k \\
b_i & b_j & b_k \\
c_i & c_j & c_k \\
\end{vmatrix}
\Be_{i j k}.
\end{dmath}

This shows that the \R{3} wedge of three vectors is triple product times the pseudoscalar

\boxedEquation{eqn:generalizedWedge:640}{
\Ba \wedge \Bb \wedge \Bc
=
\lr{ \Ba \cdot (\Bb \cross \Bc) } I.
}

Note that the wedge of \( n \) vectors is also associative.
A full proof is possible by induction, which won't be done here.
Instead, as a hint of how to proceed if desired,
consider the coordinate expansion of a trivector wedged with a vector

\begin{dmath}\label{eqn:generalizedWedge:660}
(\Ba \wedge \Bb \wedge \Bc) \wedge \Bd
=
\sum_{i \ne j \ne k, l}
\gpgrade{
a_i b_j c_k
\Be_i \Be_j \Be_k
d_l \Be_l
}{4}
=
\sum_{i \ne j \ne k \ne l}
a_i b_j c_k d_l
\Be_i \Be_j \Be_k \Be_l.
\end{dmath}

This can be rewritten with any desired grouping \( ((\Ba \wedge \Bb) \wedge \Bc) \wedge \Bd = (\Ba \wedge \Bb) \wedge ( \Bc \wedge \Bd) = \Ba \wedge (\Bb \wedge \Bc \wedge \Bd) = \cdots \).
Observe that this can also be put into a determinant form like that of
\cref{eqn:generalizedWedge:620}.
Whenever the number of vectors matches the dimension of the underlying vector space, this will be a single determinant of all the coordinates of the vectors multiplied by the unit pseudoscalar for the vector space.


            \subsubsection{Problems.}
               %
% Copyright © 2017 Peeter Joot.  All Rights Reserved.
% Licenced as described in the file LICENSE under the root directory of this GIT repository.
%
\makeproblem{Properties of the wedge of three vectors.}{problem:generalizedWedge:tripleWedgeProperties}{
\makesubproblem{}{problem:generalizedWedge:tripleWedgeProperties:a}
Show that the wedge product of three vectors is completely antisymmetric.
\makesubproblem{}{problem:generalizedWedge:tripleWedgeProperties:b}
Show that the
wedge product of three vectors \( \Ba \wedge \Bb \wedge \Bc \) is invariant with respect to cyclic permutation.
} % problem

               %
% Copyright © 2016 Peeter Joot.  All Rights Reserved.
% Licenced as described in the file LICENSE under the root directory of this GIT repository.
%

\makeproblem{\R{4} wedge of a non-blade with itself.}{problem:gradeselection:r4nonzerobivectorwedgewithself}{
While the wedge product of a blade with itself is always zero, this is not generally true of the wedge products of arbitrary k-vectors in higher dimensional spaces.
To demonstrate this, show that the wedge of the bivector
\( B = \Be_1 \Be_2 + \Be_3 \Be_4 \) with itself is non-zero.
Why is this bivector not a blade?
%, show that \( B \wedge B \ne 0 \).
} % problem

         \subsection{Duality.}
           %
% Copyright © 2017 Peeter Joot.  All Rights Reserved.
% Licenced as described in the file LICENSE under the root directory of this GIT repository.
%

\index{dual}
\makedefinition{Dual}{dfn:definitions:dual}{
The dual of a k-vector \( x \) is an (n-k)vector of the same magnitude containing all the basis vectors not in \( x \).
For the spaces used in this book the dual of any multivector can be found by multiplication with a pseudoscalar for a subspace containing that multivector.
} % definition

Such a pseudoscalar multiplication is referred to as a duality transformation, and can often be interpreted as an operation that produces a normal.
\footnote{Our definition of dual is somewhat loose mathematically.  There are tricky aspects to consider for non-Euclidean spaces that will be ignored here.}

The dual vectors to the \R{2} basis vectors are those same vectors rotated by \( \pi/2 \)

\begin{dmath}\label{eqn:definitions:360}
\begin{aligned}
\Be_1 \Be_{12} &= \Be_2 \\
\Be_2 \Be_{12} &= -\Be_1,
\end{aligned}
\end{dmath}

with an inverse duality transformation given by the multiplication with \( \Be_{12}^{-1} = \Be_{21} \)

\begin{dmath}\label{eqn:definitions:440}
\begin{aligned}
\Be_2 \Be_{21} &= \Be_1 \\
-\Be_1 \Be_{21} &= \Be_2.
\end{aligned}
\end{dmath}

The \R{3} duals to the basis vectors are bivectors

\begin{dmath}\label{eqn:definitions:380}
\begin{aligned}
\Be_1 \Be_{123} &= \Be_{23} \\
\Be_2 \Be_{123} &= \Be_{31} \\
\Be_3 \Be_{123} &= \Be_{12},
\end{aligned}
\end{dmath}

whereas the duals to those bivectors with respect to the pseudoscalar \( I^{-1} = \Be_{321} \) are the original basis vectors

\begin{dmath}\label{eqn:definitions:400}
\begin{aligned}
\Be_{23} \Be_{321} &= \Be_1 \\
\Be_{31} \Be_{321} &= \Be_2 \\
\Be_{12} \Be_{321} &= \Be_3.
\end{aligned}
\end{dmath}

In a sense that can be defined more precisely once the general dot product operator is defined, the dual to a given k-vector \( A \) represents a \((N-k)\)-vector object that is normal to \( A \).

The dual of any scalar is a pseudoscalar, whereas the dual of a pseudoscalar is a scalar.

A duality transformation can also be applied to multivectors.
For example in \R{2}, given \( M = 1 + i\), its dual is

\begin{dmath}\label{eqn:dual:460}
M i = i - 1.
\end{dmath}

Because this particular multivector had a complex structure the duality operation can be interpreted as a rotation of a vector.
How to geometrically interpret the duality transformation of a general multivector is not obvious.

%When working with multivector integrals it will be useful to consider the differential volume element a volume weighted pseudoscalar.

         \subsection{Projection and rejection.}
            %
% Copyright © 2017 Peeter Joot.  All Rights Reserved.
% Licenced as described in the file LICENSE under the root directory of this GIT repository.
%
Let's now look at how
\cref{eqn:SimpleProducts2:1440},
the dot and wedge product decomposition of the vector product, can be applied to compute vector projection and rejection, which are defined as

\index{projection}
\index{rejection}
\makedefinition{Projection and rejection}{dfn:projectionAndRejection:projectionandrejection}{
Given a vector \( \Bx \) and vector \( \Bu \) the projection of \( \Bx \) onto the direction of \( \Bu \) is defined as

\begin{equation*}
\Proj_\Bu(\Bx) = (\Bx \cdot \ucap) \ucap,
\end{equation*}

where \( \ucap = \Bu/\Norm{\Bu} \).  The rejection of \( \Bu \) from \( \Bx \) is defined as the component of \( \Bx \) that is
normal to \( \Bu \)

\begin{equation*}
\Rej{\Bu}{\Bx} = \Bx - \Proj_\Bu(\Bx).
\end{equation*}
} % definition

It is possible to factor unity as \( 1 = \ucap \ucap \) to derive a useful GA representation of the projective and rejective components of a vector

\begin{dmath}\label{eqn:SimpleProducts2:680}
\Bx =
\Bx \ucap \ucap
=
\lr{ \Bx \ucap } \ucap
=
\Biglr{ \Bx \cdot \ucap + \Bx \wedge \ucap } \ucap
=
\lr{ \Bx \cdot \ucap } \ucap + \lr{ \Bx \wedge \ucap } \ucap.
\end{dmath}

The vector \( \Bx \) is split nicely into its projection and rejective components in a complementary fashion

\begin{subequations}
\label{eqn:projectionAndRejection:1080}
\begin{dmath}\label{eqn:projectionAndRejection:900}
\Proj_\Bu(\Bx) = \lr{ \Bx \cdot \ucap } \ucap
\end{dmath}
\begin{dmath}\label{eqn:projectionAndRejection:910}
\Rej{\Bu}{\Bx} = \lr{ \Bx \wedge \ucap } \ucap.
\end{dmath}
\end{subequations}

By construction,
\( \lr{ \Bx \wedge \ucap } \ucap \) must be a vector, despite any appearance of a multivector nature.

The utility of this multivector rejection formula is not for hand or computer algebra calculations, where it will generally be faster and simpler to compute \( \Bx - (\Bx \cdot \ucap) \ucap \).  Instead this will come in handy as a new abstract algebraic tool.

When it is desirable to perform this calculation explicitly, it can be done more effieciently using a no-op grade selection operation.
In particular, a vector can be written as it's own grade-1 selection

\begin{dmath}\label{eqn:projectionAndRejection:920}
\Bx = \gpgradeone{ \Bx },
\end{dmath}

so the rejection can be reexpressed
using \cref{dfn:gradeselection:100}
as a generalized bivector-vector dot product

\begin{equation}\label{eqn:projectionAndRejection:940}
\Rej{\Bu}{\Bx}
= \gpgradeone{ \lr{ \Bx \wedge \ucap } \ucap }
= \lr{ \Bx \wedge \ucap } \cdot \ucap.
\end{equation}

To help establish some confidence with these new additions to our toolbox, here are a
pair of illustrative examples using
\cref{eqn:projectionAndRejection:910}, and
\cref{eqn:projectionAndRejection:940} respectively.

\makeexample{\R{2} rejection from the x direction.}{example:projectionAndRejection:1}{
Let \( \Bx = a \Be_1 + b \Be_2 \) and \( \Bu = \Be_1 \) for which the wedge is \( \Bx \wedge \ucap = b \Be_2 \Be_1 \).
Using \cref{eqn:projectionAndRejection:910} the rejection of \( \Bu \) from \( \Bx \) is
\begin{dmath}\label{eqn:projectionAndRejection:1000}
\Rej{\Bu}{\Bx}
=
\lr{ \Bx \wedge \ucap } \ucap
=
(b \Be_2 \Be_1 )\Be_1
=
b \Be_2 (\Be_1 \Be_1)
=
b \Be_2,
\end{dmath}

as expected.
} % example

This example provides some guidance about what is happening geometrically in
\cref{eqn:projectionAndRejection:910}.
The wedge operation produces a pseudoscalar for the plane spanned by \( \setlr{\Bx, \Bu} \) that is scaled as \( \sin\theta \) where \( \theta \) is the angle between \( \Bx \) and \( \Bu \).  When that pseudoscalar is multiplied by \( \ucap \), \( \ucap \) is rotated in the plane by \( \pi/2 \) radians towards \( \Bx \), yielding the normal component of the vector \( \Bx \).

Here's a slightly less trivial \R{3} example

\makeexample{An \R{3} rejection.}{example:projectionAndRejection:r3rejection}{
Let \( \Bx = a \Be_2 + b \Be_3 \) and \( \ucap = ( \Be_1 + \Be_2 )/\sqrt{2} \) for which the
wedge product is

\begin{dmath}\label{eqn:projectionAndRejection:1040}
\Bx \wedge \ucap = \inv{\sqrt{2}}
\begin{vmatrix}
\Be_{23} & \Be_{31} & \Be_{12} \\
0 & a & b \\
1 & 1 & 0
\end{vmatrix}
=
\inv{\sqrt{2}}
\lr{
\Be_{23}(-b) - \Be_{31}(-b) + \Be_{12} (-a)
}
=
\inv{\sqrt{2}}
\lr{
b (\Be_{32} + \Be_{31} ) + a \Be_{21}
}.
\end{dmath}

Using \cref{eqn:projectionAndRejection:940} the rejection of \( \Bu \) from \( \Bx \) is

\begin{dmath}\label{eqn:projectionAndRejection:1060}
(\Bx \wedge \ucap) \cdot \ucap
=
\inv{2}
\lr{ b (\Be_{32} + \Be_{31} ) + a \Be_{21} } \cdot ( \Be_1 + \Be_2 )
\end{dmath}

Each of these bivector-vector dot products has the form \( \Be_{rs} \cdot \Be_t = \gpgradeone{ \Be_{rst} } \) which is zero whenever the indexes \( r,s, t\) are unique, and is a vector whenever one of indexes are repeated (\( r = t \), or \( s = t \)).  This leaves

\begin{dmath}\label{eqn:projectionAndRejection:1100}
(\Bx \wedge \ucap) \cdot \ucap
=
\inv{2}
\lr{ b \Be_3 + a \Be_2 + b \Be_3 - a \Be_1 }
=
b \Be_3 + \frac{a}{2}( \Be_2 - \Be_1 ).
\end{dmath}

The reader should confirm that this equals \( \Bx - (\Bx \cdot \ucap) \ucap \).
} % example

In the GA literature the projection and rejection operations are usually written using the vector inverse

\index{vector inverse}
\makedefinition{Vector inverse.}{dfn:projectionAndRejection:vectorinverse}{
Define the inverse of a vector \( \Bx \), when it exists, as
\begin{equation*}
\Bx^{-1} = \frac{\Bx}{\Norm{\Bx}^2}.
\end{equation*}

This inverse satisfies \( \Bx^{-1} \Bx = \Bx \Bx^{-1} = 1 \).
} % definition

The vector inverse may not exist in a non-Euclidean vector space where \( \Bx^2 \) can be zero for non-zero vectors \( \Bx \).

In terms of the vector inverse, the projection and rejection operations with respect to \( \Bu \) can be written without any reference to the unit vector \( \ucap = \Bu/\Norm{\Bu} \) that lies along that vector

\boxedEquation{eqn:projectionAndRejection:1120}{
\begin{aligned}
\Proj_\Bu(\Bx) &= \lr{ \Bx \cdot \Bu } \inv{\Bu} \\
\Rej{\Bu}{\Bx} &=
\lr{ \Bx \wedge \Bu } \inv{\Bu}
=
\lr{ \Bx \wedge \Bu } \cdot \inv{\Bu}.
\end{aligned}
}

Using this notation, an example of projection and rejection with respect to a direction vector \( \Bu \) is illustrated in
\cref{fig:projectionAndRejection:projectionAndRejectionFig1}.

\imageFigure{../figures/GAelectrodynamics/projectionAndRejectionFig1}{Projection and rejection illustrated.}{fig:projectionAndRejection:projectionAndRejectionFig1}{0.3}

It was claimed in the definition of rejection that the rejection is normal to the projection.  This can be shown trivially without any use of GA (\cref{problem:projectionAndRejection:rejectionnormality}).  This also follows naturally using the grade selection operator, which illustrates another useful new GA technique

\begin{dmath}\label{eqn:SimpleProducts2:720}
\Rej{\Bu}{\Bx} \cdot \Proj_\Bu(\Bx)
=
\gpgradezero{ \Rej{\Bu}{\Bx} \Proj_\Bu(\Bx) }
=
\gpgradezero{ \lr{ \Bx \wedge \ucap } \ucap \lr{ \Bx \cdot \ucap } \ucap }
=
\lr{ \Bx \cdot \ucap } \gpgradezero{ \lr{ \Bx \wedge \ucap } \ucap^2 }
=
\lr{ \Bx \cdot \ucap } \gpgradezero{ \Bx \wedge \ucap }.
\end{dmath}

Since the scalar grade of a wedge product, a (grade-2) bivector, is zero,
this demonstrates that the projection and rejection are normal.

Finally, since the \R{3} wedge product is related to the cross product through \cref{eqn:SimpleProducts2:1620},
it is reasonable to ask how the wedge product form of the rejection operator can be
expressed using the cross product.
Using the grade selection form of \cref{eqn:projectionAndRejection:940}, we find

\begin{dmath}\label{eqn:projectionAndRejection:1160}
\Rej{\Bu}{\Bx}
%=
%\lr{ \Bx \wedge \ucap } \cdot \ucap
=
\gpgradeone{ \lr{ \Bx \wedge \ucap } \ucap }
=
\gpgradeone{ I \lr{ \Bx \cross \ucap } \ucap }
=
\gpgradeone{ I
\biglr{
   \lr{ \Bx \cross \ucap } \cdot \ucap
+
   \lr{ \Bx \cross \ucap } \wedge \ucap
}
}
=
\lr{ \lr{ \Bx \cross \ucap } \cdot \ucap }
\cancel{ \gpgradeone{ I } }
+
\gpgradeone{ I^2
   \lr{ \Bx \cross \ucap } \cross \ucap
}
\end{dmath}

A grade 1 selection from the pseudoscalar is zero, since the pseudoscalar is grade 3.
Application of \cref{eqn:SimpleProducts2:1620} a second time to the remaining vector term produces a (scalar) \( I^2 = -1 \) factor, so after reordering the cross product, the rejection is

\begin{dmath}\label{eqn:projectionAndRejection:1180}
\Rej{\Bu}{\Bx}
=
   \ucap \cross \lr{ \Bx \cross \ucap }.
\end{dmath}

This is a result that may already be familiar.
% implicitly used in the front cover of Jackson
Without GA, one can prove that \cref{eqn:projectionAndRejection:1180} matches the rejection definition by choosing a coordinate system so that \( \ucap \) is aligned with one of the standard basis vectors.
It is also possible to prove this using the identity \( \Ba \cross ( \Bb \cross \Bc ) = (\Ba \cdot \Bc) \Bb - (\Ba \cdot \Bb) \Bc \) (\citep{jackson1975cew}).
A very compact proof using coordinates and using tensor contraction methods is also possible (see: \citep{landau1951classical}).
These methods are all arguably less elegant than this coordinate free GA approach.

\makeproblem{Rejection normality}{problem:projectionAndRejection:rejectionnormality}{
Prove, without any use of GA, that \( \Bx - \Proj_\Bu(\Bx) \) is normal to \( \Bu \), as claimed in
\cref{dfn:projectionAndRejection:projectionandrejection}.
} % problem

%\makeproblem{Prove \ref{dfn:projectionAndRejection:r3pcommutation}.}{problem:projectionAndRejection:1160}{
%} % problem


         \subsection{Normal factorization of the wedge product.}
            %
% Copyright © 2017 Peeter Joot.  All Rights Reserved.
% Licenced as described in the file LICENSE under the root directory of this GIT repository.
%
A general bivector has the form
\begin{dmath}\label{eqn:normalFactorization:1800}
B = \sum_{i \ne j} a_{ij} \Be_{ij},
\end{dmath}
which is not necessarily a blade.
%For example the bivector \( \Be_1 \Be_2 + \Be_3 \Be_4 \) cannot be factored into any product of normal vectors.
On the other hand, a wedge product is always a blade
\footnote{In \R{3} any bivector is also a blade \citep{ablamowicz2004lectures:chapter1}}

\index{wedge factorization}
\maketheorem{Wedge product normal factorization}{thm:SimpleProducts2:wnormalfactorize}{
The wedge product of any two non-colinear vectors \( \Ba, \Bb \) always has a orthogonal (2-blade) factorization
\begin{equation*}
\Ba \wedge \Bb = \Bu \Bv, \quad \Bu \cdot \Bv = 0.
\end{equation*}
} % theorem

This can be proven by construction.
Pick \( \Bu = \Ba \) and \( \Bv = \Rej{\Ba}{\Bb} \), then
\begin{equation}\label{eqn:normalFactorization:1840}
\begin{aligned}
\Ba \Rej{\Ba}{\Bb}
&= \cancel{\Ba \cdot \Rej{\Ba}{\Bb}} + \Ba \wedge \Rej{\Ba}{\Bb} \\
&= \Ba \wedge \lr{ \Bb - \frac{\Bb \cdot \Ba}{\Norm{\Ba}^2} \Ba } \\
&= \Ba \wedge \Bb,
\end{aligned}
\end{equation}
since \( \Ba \wedge (\alpha \Ba) = 0 \) for any scalar \( \alpha \).

The significance of \cref{thm:SimpleProducts2:wnormalfactorize} is that the square of any wedge product is negative
\begin{dmath}\label{eqn:normalFactorization:1820}
(\Bu \Bv)^2
=
(\Bu \Bv) (-\Bv \Bu)
=
-\Bu (\Bv^2) \Bu
=
- \Abs{\Bu}^2 \Abs{\Bv}^2,
\end{dmath}
which in turn means that exponentials with wedge product arguments can be used as rotation operators.

\makeproblem{\R{3} bivector factorization.}{problem:normalFactorization:1}{
Find some orthogonal factorizations for the \R{3} bivector \( \Be_{12} + \Be_{23} + \Be_{31} \).
} % problem

\makeanswer{problem:normalFactorization:1}{
\begin{equation*}
\begin{aligned}
\Be_{12} + \Be_{23} + \Be_{31}
&= \lr{ \Be_1 + \Be_2 - 2 \Be_3 } \frac{ \Be_2 - \Be_1 }{2} \\
&= \frac{ \Be_3 - \Be_2 }{2} \lr{ 2 \Be_1 - \Be_2 - \Be_3 }.
\end{aligned}
\end{equation*}
Each set of factors above can be interpretted as the edges of two different rectangular representations of the bivector, for which the total area is fixed.  The span of either set of factors describes the plane that the bivector represents.
%The respective parallelogram representations of these bivector factorizations are illustrated in
%\cref{fig:bivectorFactorization:bivectorFactorizationFig1}, showing the bivectors face on, and from a side view that shows they are coplanar.
%
%% \imageTwoFigures{path1}{path2}{fancy plots}{fig:blah}{scale=0.3}
%\imageTwoFigures
%{../figures/GAelectrodynamics/bivectorFactorizationFig2}
%{../figures/GAelectrodynamics/bivectorFactorizationFig1}
%{Two equivalent bivector factorizations.}
%{fig:bivectorFactorization:bivectorFactorizationFig1}{scale=0.3}
} % answer

         \subsection{The wedge product as an oriented area.}
            
To prove \cref{thm:SimpleProducts2:wnormalfactorize}, first assume that there is an orthonormal basis \( \setlr{\ucap, \vcap} \) for the planar subspace \( P = \Span\setlr{ \Ba, \Bb } \), for which

\begin{dmath}\label{eqn:SimpleProducts2:1840}
\begin{aligned}
\Ba &= (\Ba \cdot \ucap) \ucap + (\Ba \cdot \vcap) \vcap \\
\Bb &= (\Bb \cdot \ucap) \ucap + (\Bb \cdot \vcap) \vcap.
\end{aligned}
\end{dmath}

The wedge of \( \Ba, \Bb \) in terms of this basis is

\begin{dmath}\label{eqn:SimpleProducts2:1860}
\Ba \wedge \Bb
=
\gpgradetwo{
   \lr{
   (\Ba \cdot \ucap) \ucap + (\Ba \cdot \vcap) \vcap
   }
   \lr{
   (\Bb \cdot \ucap) \ucap + (\Bb \cdot \vcap) \vcap
   }
}
=
\gpgradetwo{
\cancel{
   (\Ba \cdot \ucap) (\Bb \cdot \ucap) \ucap^2
}
+
\cancel{
   (\Ba \cdot \vcap) (\Bb \cdot \vcap) \vcap^2
}
+
\lr{
      (\Ba \cdot \ucap)
   (\Bb \cdot \vcap)
   -
   (\Ba \cdot \vcap) (\Bb \cdot \ucap)
}
\ucap \vcap
}
=
\lr{
      (\Ba \cdot \ucap)
   (\Bb \cdot \vcap)
   -
   (\Ba \cdot \vcap) (\Bb \cdot \ucap)
}
\ucap \vcap.
\end{dmath}

Such a basis allows for the most compact (single term) coordinate representation of the wedge product

\begin{dmath}\label{eqn:SimpleProducts2:1880}
\Ba \wedge \Bb
=
\begin{vmatrix}
   \Ba \cdot \ucap & \Ba \cdot \vcap \\
   \Bb \cdot \ucap & \Bb \cdot \vcap
\end{vmatrix}
\ucap \vcap.
\end{dmath}

The wedge product is therefore the (possibly signed) area of the parallelopiped formed by the vectors \( \Ba, \Bb \), multiplied by a unit pseudoscalar for the subspace of the plane \( P \).  Provided the area of this parallelopiped is non-zero, which is always the case for non-colinear vectors, there are clearly many possible normal factorizations for the wedge product.

\subsection{General rotation.}

\Cref{eqn:SimpleProducts2:180} showed that the \R{2} pseudoscalar anticommutes with any vector \( \Bx \in \bbR^{2} \),

\begin{dmath}\label{eqn:SimpleProducts2:1760}
\Bx i = -i \Bx.
\end{dmath}

The higher dimensional generalization of this result is

\maketheorem{Commutation rules for wedge products.}{thm:SimpleProducts2:1780}{
Given a planar subspace formed by the span of two non-colinear vectors \( S = \Span \setlr{ \Ba, \Bb } \), any vector \( \Bx \in S \) anticommutes with the wedge product \( \Ba \wedge \Bb \)

\begin{equation*}
\Bx (\Ba \wedge \Bb) = - (\Ba \wedge \Bb) \Bx.
\end{equation*}

Moreover, any vector \( \Bn \) normal to this plane (\( \Bn \cdot \Ba = \Bn \cdot \Bb = 0 \)) commutes with this wedge product
\begin{equation*}
\Bn (\Ba \wedge \Bb) = (\Ba \wedge \Bb) \Bn.
\end{equation*}
} % theorem

A simple inelegant way to prove this is to specify a coordinate system for which \( \Ba, \Bb \) both lie in the \( x,y \) plane.  Then \( \Ba \wedge \Bb = \alpha i \) for some \( \alpha \), and the anticommutation part of the theorem follows from
%the \R{2} result
\cref{eqn:SimpleProducts2:1760}.  For the normal commutation part of the theorem, pick any vector normal to the \(x, y\) plane, say \( \Be_3\), for which we have

\begin{dmath}\label{eqn:SimpleProducts2:1780}
\Be_3  (\Ba \wedge \Bb)
=
\Be_3 \alpha i
=
\alpha \Be_3 \Be_1 \Be_2
=
\alpha (-\Be_1 \Be_3) \Be_2
=
-\alpha \Be_1 (\Be_3 \Be_2)
=
-\alpha \Be_1 (-\Be_2 \Be_3)
= (\Ba \wedge \Bb) \Be_3.
\end{dmath}

In dimensions with more normals, say \( \Be_4, \cdots \), the steps of \cref{eqn:SimpleProducts2:1780} can be repeated.  The general normal commuation result follows by superposition.

         \subsection{General rotation.}
            
\Cref{eqn:SimpleProducts2:180} showed that the \R{2} pseudoscalar anticommutes with any vector \( \Bx \in \bbR^{2} \),

\begin{dmath}\label{eqn:SimpleProducts2:1760}
\Bx i = -i \Bx,
\end{dmath}

and that the complex exponential ``conjugate'' anticommutes with any vector

\begin{dmath}\label{eqn:generalRotation:1820}
\Bx e^{i\theta}
=
e^{-i\theta} \Bx.
\end{dmath}

The higher dimensional generalization of these results are

\maketheorem{Wedge and exponential commutation and conjugation rules.}{thm:SimpleProducts2:1780}{
Given two
non-colinear vectors \( \Ba, \Bb \), let
the planar subspace formed by their span be designated
\( S = \Span \setlr{ \Ba, \Bb } \).

\begin{enumerate}[(a)]
\item
Any vector \( \Bp \in S \) anticommutes with the wedge product \( \Ba \wedge \Bb \)

\begin{equation*}
\Bp (\Ba \wedge \Bb) = - (\Ba \wedge \Bb) \Bp.
\end{equation*}
\item
Any vector \( \Bn \) normal to this plane (\( \Bn \cdot \Ba = \Bn \cdot \Bb = 0 \)) commutes with this wedge product
\begin{equation*}
\Bn (\Ba \wedge \Bb) = (\Ba \wedge \Bb) \Bn.
\end{equation*}
\end{enumerate}
\item
Given a ``complex'' exponential \( e^{ \Ba \wedge \Bb } \), commutation
of vectors \( \Bp \in S \) result in a ``conjugation'' of the exponential argument, whereas normal vectors \( \Bn \) commute with such an exponential, as follows

\begin{equation*}
\begin{aligned}
\Bp e^{\Ba \wedge \Bb} &= e^{-\Ba \wedge \Bb} \Bp \\
\Bn e^{\Ba \wedge \Bb} &= e^{\Ba \wedge \Bb} \Bn \\
\end{aligned}
\end{equation*}
} % theorem

The proof relies on the fact that a normal factorization of the wedge product is possible.
If \( \Bp \) is one of those factors, then the other is uniquely determined by the multivector equation \( \Ba \wedge \Bb = \Bp \Bq \), for which we must have \( \Bq = \inv{\Bx}(\Ba \wedge \Bb) \in S \) and \( \Bp \cdot \Bq = 0 \)
\footnote{The identities required to show that \( \Bq \) above has no trivector grades, and to evaluate it explicitly in terms of \( \Ba, \Bb, \Bx \), will be derived later.}
.
Then

\begin{dmath}\label{eqn:wedgeProductArea:1780}
\Bp (\Ba \wedge \Bb)
= \Bp (\Bp \Bq)
= \Bp (-\Bq \Bp)
= -(\Bp \Bq) \Bp
=
-(\Ba \wedge \Bb) \Bp.
\end{dmath}

Any normal \( \Bn \) must also be perpendicular to the factors \( \Bp, \Bq \), with \( \Bn \cdot \Bp = \Bn \cdot \Bq = 0 \), so

\begin{dmath}\label{eqn:wedgeProductArea:1800}
\Bn (\Ba \wedge \Bb)
= \Bn (\Bp \Bq)
= (-\Bp \Bn) \Bq
= -\Bp (-\Bq \Bn)
= (\Bp \Bq) \Bn
=
(\Ba \wedge \Bb) \Bn.
\end{dmath}

For the complex exponentials, introduce a unit pseudoscalar for the plane \( i = \pcap \qcap \) satisfying \( i^2 = -1 \) and a scalar rotation angle \( \theta = \ifrac{ (\Ba \wedge \Bb) }{i} \), then for vectors \( \Bp \in S \)

\begin{dmath}\label{eqn:generalRotation:1840}
\Bp e^{ \Ba \wedge \Bb }
=
\Bp e^{ i \theta }
=
\Bp \lr{ \cos\theta + i \sin\theta }
=
\lr{ \cos\theta - i \sin\theta } \Bp
=
e^{-i\theta} \Bp
=
e^{- \Ba \wedge \Bb} \Bp,
\end{dmath}

and for vectors \( \Bn \) normal to \( S \)

\begin{dmath}\label{eqn:generalRotation:1860}
\Bn e^{ \Ba \wedge \Bb }
=
\Bn e^{ i \theta }
=
\Bn \lr{ \cos\theta + i \sin\theta }
=
\lr{ \cos\theta + i \sin\theta } \Bn
=
e^{i\theta} \Bn
=
e^{\Ba \wedge \Bb} \Bn,
\end{dmath}

which completes the proof.

The point of this somewhat abstract seeming theorem is to prepare for the statement of a general \R{N} rotation, which is

\makedefinition{General rotation}{dfn:generalRotation:generalrotation}{
Let \( B = \setlr{ \pcap, \qcap } \) be an orthonormal basis for a planar subspace with unit pseudoscalar \( i = \pcap \qcap, i^2 = -1\).  The rotation of a vector \( \Bx \) through an angle \( \theta \) with respect to this plane is

\begin{equation*}
R_\theta(\Bx) = e^{ - i \theta/2 } \Bx e^{ i\theta/2 }.
\end{equation*}

Here the rotation sense is that of the \( \pi/2 \) rotation from \( \pcap \) to \( \qcap \) in the subspace \( S = \Span B \).
} % definition

This statement did not make any mention to a normal direction which is not unique for dimensions higher than 3, nor defined for two dimensions.
Instead the rotational sense is defined by the ordering of the factors in the bivector \( i \).
Let's consider the action of this definition and verify that it has the desired semantics.

Let \( \Bx = \Ba + \Bn \), where \( \Ba \in S \) and \( \Bn \cdot \Bp = 0 \forall \Bp \in S \).  Then

\begin{dmath}\label{eqn:generalRotation:1880}
R_\theta(\Bx)
=
e^{ - i \theta/2 } \Bx e^{ i\theta/2 }
=
e^{ - i \theta/2 } \lr{ \Ba + \Bn } e^{ i\theta/2 }
=
\Ba e^{ i\theta } +
\Bn e^{ - i \theta/2 } e^{ i\theta/2 }
=
\Ba e^{ i\theta } + \Bn.
\end{dmath}

This rotation operation rotates components of the vector that lie in the planar subspace \( S \) by \( \theta \), and leaves any normal components of the vector unchanged.  An illustration can be found in \cref{fig:Rotation:RotationFig1}.

\imageFigure{../figures/GAelectrodynamics/RotationFig1}{Rotation through plane of pseudoscalar.}{fig:Rotation:RotationFig1}{0.3}

         \subsection{Symmetric and antisymmetric vector sums.}
            %
% Copyright © 2017 Peeter Joot.  All Rights Reserved.
% Licenced as described in the file LICENSE under the root directory of this GIT repository.
%
\maketheorem{Symmetric and antisymmetric vector products.}{thm:symmetricAndAntiSymmetricVectorSums:symmetricAndAnti}{
\begin{enumerate}
\item The dot product of vectors \( \Bx, \By \) can be written as
\begin{equation*}
\Bx \cdot \By = \inv{2}\lr{ \Bx \By + \By \Bx }.
\end{equation*}

This sum, including all permutations of the products of \( \Bx \) and \( \By \) is called a completely symmetric sum.
\item The wedge product of vectors \( \Bx, \By \) can be written as
\begin{equation*}
\Bx \wedge \By = \inv{2}\lr{ \Bx \By - \By \Bx }.
\end{equation*}

This sum, including all permutations of the products \( \Bx \) and \( \By \), with a sign change for any interchange, is called a completely antisymmetric sum.
\end{enumerate}
} % theorem

These identities highlight the symmetric and antisymmetric nature of the respective dot and wedge products in a coordinate free form, and will be useful in the manipulation of various identities.
The proof follows by direct compuation after first noting that the respect vector products are

\begin{subequations}
\label{eqn:symmetricAndAntiSymmetricVectorSums:660}
\begin{dmath}\label{eqn:symmetricAndAntiSymmetricVectorSums:640}
\Bx \By = \Bx \cdot \By + \Bx \wedge \By
\end{dmath}
\begin{dmath}\label{eqn:symmetricAndAntiSymmetricVectorSums:680}
\By \Bx
= \By \cdot \Bx + \By \wedge \Bx
= \Bx \cdot \By - \Bx \wedge \By.
\end{dmath}
\end{subequations}

In \cref{eqn:symmetricAndAntiSymmetricVectorSums:680} the interchange utilized the respective symmetric and antisymmetric nature of the dot and wedge products.

Adding and subtracting \cref{eqn:symmetricAndAntiSymmetricVectorSums:660} proves the result.

%Some authors will use \cref{eqn:SimpleProducts2:620} as the definitions of the dot and wedge products instead of defining them in terms of grade selection.
%Grade selection is preferred here since it allows for a generalization of the wedge product to multiple vectors in higher degree spaces in a particularily simple way, and also allows for the generalization of the dot and wedge products with higher order geometric structures to be discussed.

         \subsection{Reflection.}
            %
% Copyright © 2017 Peeter Joot.  All Rights Reserved.
% Licenced as described in the file LICENSE under the root directory of this GIT repository.
%
\index{reflection}

Geometrically the reflection of a vector \( \Bx \) across a line directed along \( \Bu \) is the difference of the projection and rejection

\begin{dmath}\label{eqn:SimpleProducts2:900}
\Bx'
= \lr{ \Bx \cdot \Bu }\Bu - \lr{ \Bx \wedge \Bu } \inv{\Bu }
= \lr{ \Bx \cdot \Bu - \Bx \wedge \Bu } \inv{\Bu }
\end{dmath}

Using the symmetric and antisymmetric sum representations of the dot and wedge products from
\cref{thm:symmetricAndAntiSymmetricVectorSums:symmetricAndAnti}
the reflection can be expressed as vector products

\begin{dmath}\label{eqn:reflection:n}
\Bx'
= \inv{2} \lr{ \cancel{\Bx \Bu} + \Bu \Bx - \cancel{\Bx \Bu} + \Bu \Bx } \inv{\Bu },
\end{dmath}

yeilding a remarkably simple form in terms of vector products

\boxedEquation{eqn:SimpleProducts2:920}{
\Bx' = \Bu \Bx \inv{\Bu}.
}

An illustration of the geometry of reflection is provided in \cref{fig:reflection:reflectionFig1}.

\imageFigure{../figures/GAelectrodynamics/reflectionFig1}{Reflection}{fig:reflection:reflectionFig1}{0.3}

         \subsection{Linear systems.}
            %
% Copyright © 2017 Peeter Joot.  All Rights Reserved.
% Licenced as described in the file LICENSE under the root directory of this GIT repository.
%
\index{linear system}
\index{wedge product!linear solution}
Various types of linear systems can be solved using the wedge product.
An illustrative example is that of the intersection of two lines as illustrated in \cref{fig:intersectionOfLines:intersectionOfLinesFig1}.

\imageFigure{../figures/GAelectrodynamics/intersectionOfLinesFig1}{Intersection of two lines.}{fig:intersectionOfLines:intersectionOfLinesFig1}{0.3}

In parametric form, the lines in this problem are

\begin{dmath}\label{eqn:SimpleProducts2:1000}
\begin{aligned}
\Br_1(s) &= \Ba_0 + s( \Ba_1 - \Ba_2 ) \\
\Br_2(t) &= \Bb_0 + t( \Bb_1 - \Bb_2 ),
\end{aligned}
\end{dmath}

so the solution, if it exists, is found at the point satisfying the equality

\begin{dmath}\label{eqn:SimpleProducts2:1020}
\Ba_0 + s( \Ba_1 - \Ba_2 ) = \Bb_0 + t( \Bb_1 - \Bb_2 ).
\end{dmath}

With
\begin{dmath}\label{eqn:SimpleProducts2:1040}
\begin{aligned}
\Bu_1 &= \Ba_1 - \Ba_2 \\
\Bu_2 &= \Bb_1 - \Bb_2 \\
\Bd &= \Ba_0 - \Bb_0,
\end{aligned}
\end{dmath}

the desired equation to solve is

\begin{dmath}\label{eqn:SimpleProducts2:1060}
\Bd + s \Bu_1 = t \Bu_2.
\end{dmath}

Solving for \( s \) or \( t \) is possible by
wedging both sides with one of \( \Bu_1 \) or \( \Bu_2 \)

\begin{dmath}\label{eqn:SimpleProducts2:1080}
\begin{aligned}
\Bd \wedge \Bu_1 &= t \Bu_2 \wedge \Bu_1 \\
\Bd \wedge \Bu_2 + s \Bu_1 \wedge \Bu_2 &= 0,
\end{aligned}
\end{dmath}

In \R{2} these equations have a solution if \( \Bu_1 \wedge \Bu_2 \ne 0 \), and in \R{N} these have solutions if the bivectors on each sides of the equations describe the same plane.
Put another way, these have solutions when \( s \) and \( t \) are scalars and

\begin{dmath}\label{eqn:SimpleProducts2:1100}
\begin{aligned}
s &= \frac{\Bu_2 \wedge \Bd}{\Bu_1 \wedge \Bu_2} \\
t &= \frac{\Bu_1 \wedge \Bd}{\Bu_1 \wedge \Bu_2}.
\end{aligned}
\end{dmath}

In
\R{2}
with
\begin{dmath}\label{eqn:solutionOfLinearSystem:1120}
\begin{aligned}
\Bu_1 &= u_{11} \Be_1 + u_{12} \Be_2 \\
\Bu_2 &= u_{21} \Be_1 + u_{22} \Be_2 \\
\Bd &= d_{1} \Be_1 + d_{2} \Be_2,
\end{aligned}
\end{dmath}

the wedge products in \cref{eqn:SimpleProducts2:1100}
can be expressed explicitly as a (unit bivector scaled) determinants

\begin{equation}\label{eqn:solutionOfLinearSystem:1140}
%\begin{aligned}
s =
\frac{
\begin{vmatrix}
u_{21} & u_{22} \\
d_1 & d_2
\end{vmatrix}
\Be_{12}
}
{
\begin{vmatrix}
u_{11} & u_{12} \\
u_{21} & u_{22} \\
\end{vmatrix}
\Be_{12}
}
%=
%\frac{
%\begin{vmatrix}
%u_{21} & u_{22} \\
%d_1 & d_2
%\end{vmatrix}
%}
%{
%\begin{vmatrix}
%u_{11} & u_{12} \\
%u_{21} & u_{22} \\
%\end{vmatrix}
%}
\qquad
t =
\frac{
\begin{vmatrix}
u_{11} & u_{12} \\
d_1 & d_2
\end{vmatrix}
\Be_{12}
}
{
\begin{vmatrix}
u_{11} & u_{12} \\
u_{21} & u_{22} \\
\end{vmatrix}
\Be_{12}
}
%=
%\frac{
%\begin{vmatrix}
%u_{11} & u_{12} \\
%d_1 & d_2
%\end{vmatrix}
%}
%{
%\begin{vmatrix}
%u_{11} & u_{12} \\
%u_{21} & u_{22} \\
%\end{vmatrix}
%}
.
%\end{aligned}
\end{equation}

Once the unit bivectors \( \Be_{12} \) are cancelled \cref{eqn:solutionOfLinearSystem:1140} is the Cramer's rule solution of the problem.  Cramer's rule is seen to follow directly from the use of the wedge product to eliminate factors that are not of interest.
In a similar way, the use of the wedge product for a 3D intersection problem with three variables, will lead directly to the Cramer's rule solution.

\makeproblem{Intersection of a line and plane.}{problem:solutionOfLinearSystem:1}{
Let a line be parameterized by
\begin{equation*}
\Br(a) = \Bp + a \Ba,
\end{equation*}
and a plane be parameterized by
\begin{equation*}
\Br(b,c) = \Bq + b \Bb + c \Bc.
\end{equation*}
\makesubproblem{}{problem:solutionOfLinearSystem:1:a}
State the vector equation to be solved, and its solution for \( a \) in terms of a ratio of wedge products.
\makesubproblem{}{problem:solutionOfLinearSystem:1:b}
State the conditions for which the solution exist in \R{3} and \R{N}.
\makesubproblem{}{problem:solutionOfLinearSystem:1:c}
In terms of coordinates in \R{3} write out the ratio of wedge products as determinants and compare to the Cramer's rule solution.
} % problem

         %\subsection{Orientation.}
         %   %
% Copyright © 2017 Peeter Joot.  All Rights Reserved.
% Licenced as described in the file LICENSE under the root directory of this GIT repository.
%
\subsection{Orientation.  figure out where to put this}
Geometric algebra provides a mathematical representation for geometrical objects of each dimension in the space.
In a three dimensional space, there are representations for all of

\begin{itemize}
\item
oriented (signed) points with magnitude
\item
oriented line segments,
\item
oriented planes,
\item
oriented volumes,
\end{itemize}

and in higher dimensional spaces, it will be possible to represent higher dimensonal oriented hypervolumes.


   \chapter{Vector calculus}
      \section{Reciprocal frames}
         %
% Copyright � 2016 Peeter Joot.  All Rights Reserved.
% Licenced as described in the file LICENSE under the root directory of this GIT repository.
%
%{
%\input{../blogpost.tex}
%\renewcommand{\basename}{reciprocal}
%%\renewcommand{\dirname}{notes/phy1520/}
%\renewcommand{\dirname}{notes/ece1228-electromagnetic-theory/}
%%\newcommand{\dateintitle}{}
%%\newcommand{\keywords}{}
%
%\input{../peeter_prologue_print2.tex}
%
%\usepackage{peeters_layout_exercise}
%\usepackage{peeters_braket}
%\usepackage{peeters_figures}
%\usepackage{siunitx}
%%\usepackage{mhchem} % \ce{}
%%\usepackage{macros_bm} % \bcM
%%\usepackage{macros_qed} % \qedmarker
%%\usepackage{txfonts} % \ointclockwise
%
%\beginArtNoToc
%
%\generatetitle{Reciprocal frame vectors}
%%\chapter{reciprocal frame vectors}
%%\label{chap:reciprocal}
%
The end goal of this chapter is to be able to integrate multivector functions along curves and surfaces, known collectively as manifolds.
For our purposes, a manifold is defined by a parameterization, such as the vector valued function \( \Bx(a,b) \) where \( a, b\) are scalar parameters.  With one parameter the vector traces out a curve, with two a surface, three a volume, and so forth.
The respective partial derivatives of such a parameterized vector define a local basis for the surface at the point at which the partials are evaluated.
The span of such a basis is called the tangent space, and the partials that constitute it are not necessarily orthonormal, or even normal.

Unfortunately, in order to work with the curvilinear non-orthonormal bases that will be encountered in general integration theory, some
additional tools are required.

\begin{itemize}
\item
We introduce a reciprocal frame which partially generalizes the notion of normal to non-orthonormal bases.
\item
We will borrow the upper and lower index (tensor) notation from relativistic physics that is useful for the intrinsically non-orthonormal spaces encountered in that study, as this notation works well to define the reciprocal frame.
\end{itemize}

\index{reciprocal frame}
\makedefinition{Reciprocal frame}{dfn:reciprocal:frame}{
Given a basis for a subspace \( \setlr{ \Bx_1, \Bx_2, \cdots \Bx_n } \), where the vectors \( \Bx_i \) are not necessarily orthonormal, the reciprocal frame is defined as the set of vectors \( \setlr{ \Bx^1, \Bx^2, \cdots \Bx^n } \) satisfying

\begin{dmath*}
\Bx_i \cdot \Bx^j = {\delta_i}^j,
\end{dmath*}

where the vector \( \Bx^j \) is not the j-th power of \( \Bx \), but is a superscript index, the conventional way of denoting a reciprocal frame vector, and \( {\delta_i}^j \) is the Kronecker delta.
} % definition

This definition introduces mixed index variables for the first time in this text, which may be unfamiliar.  These are most often used in tensor algebra, where any expression that has pairs of upper and lower indexes implies a sum, and is called the summation (or Einstein) convention.  For example:

\begin{dmath}\label{eqn:reciprocal:400}
\begin{aligned}
a_i b^i &\equiv \sum_i a_i b^i \\
{A^{i}}_j B_i C^j &\equiv \sum_{i,j} {A^{i}}_j B_i C^j.
\end{aligned}
\end{dmath}

Summation convention will not be used explicitly in this text, as it deviates from normal practises in electrical engineering\footnote{Generally, when summation convention is used, explicit summation is only used explicitly when upper and lower indexes are not perfectly matched, but summation is still implied.  Readers of texts that use summation convention can check for proper matching of upper and lower indexes to ensure that the expressions make sense.  Such matching is the reason a mixed index Kronecker delta has been used in the definition of the reciprocal frame.}.

The most important application of a reciprocal frame is for the computation of the coordinates of a vector with respect to a non-orthonormal frame.
Let a vector \( \Ba \) have coordinates \( a^i \) with respect to a basis \( \setlr{ \Bx_i } \)

\begin{dmath}\label{eqn:reciprocal:20}
\Ba = \sum_j a^j \Bx_j,
\end{dmath}

where \( j \) is an index not a power\footnote{In tensor algebra, any index that is found in matched upper and lower index pairs, is known as a dummy summation index, whereas an index that is unmatched is known as a free index.  For example, in \( a^j b_{ij} \) (summation implied) \( j \) is a summation index, and \( i \) is a free index.  We are free to make a change of variables of any summation index, so for the same example we can write
\( a^k b_{ik} \).  These index tracking conventions are obvious when summation symbols are included explicitly, as we will do.}.

Dotting with the reciprocal frame vectors \( \Bx^i \) provides these coordinates \( a^i \)

\begin{dmath}\label{eqn:reciprocal:40}
\Ba \cdot \Bx^i
= \lr{\sum_j a^j \Bx_j} \cdot \Bx^i
= \sum_j a^j {\delta_j}^i
= a^i.
\end{dmath}

The vector can also be expressed with coordinates taken with respect to the reciprocal frame.  Let those coordinates be \( a_i \), so that

\begin{dmath}\label{eqn:reciprocal:60}
\Ba = \sum_i a_i \Bx^i.
\end{dmath}

Dotting with the basis vectors \( \Bx_i \) provides the reciprocal frame relative coordinates \( a_i \)

\begin{dmath}\label{eqn:reciprocal:80}
\Ba \cdot \Bx_i
= \lr{\sum_j a_j \Bx^j} \cdot \Bx_i
= \sum_j a_j {\delta^j}_i
= a_i.
\end{dmath}

We can summarize \cref{eqn:reciprocal:40} and \cref{eqn:reciprocal:80} by stating that a vector can be expressed in terms of coordinates relative to either the original or reciprocal basis as follows

\begin{equation}\label{eqn:reciprocal:420}
\Ba
= \sum_j \lr{ \Ba \cdot \Bx^j } \Bx_j
= \sum_j \lr{ \Ba \cdot \Bx_j } \Bx^j.
\end{equation}

In tensor algebra the basis is generally implied\footnote{
In tensor algebra, a vector, identified by the coordinates \( a^i \) is called a contravariant vector.
When that vector is identified by the coordinates \( a_i \) it is called a covariant vector.  These labels relate to how the coordinates transform with respect to norm preserving transformations.
We have no need of this nomenclature, since we never transform coordinates in isolation, but will always transform the coordinates along with their associated basis vectors.}.

%When doing tensor algebra manipulations, you'll generally have the freedom to swap any pairs of upper and lower indexes as done above.

An example of a 2D oblique Euclidean basis and a corresponding reciprocal basis is plotted in \cref{fig:obliqueReciprocal:obliqueReciprocalFig2}.
Also plotted are the superposition of the projections required to arrive at a given point \( (4,2) \)) along the \( \Be_1, \Be_2 \) directions or the \( \Be^1, \Be^2 \) directions.
In this plot, neither of the reciprocal frame vectors \( \Be^i \) are normal to the corresponding basis vectors \( \Be_i \).
When one of \( \Be_i \) is increased(decreased) in magnitude, there will be a corresponding decrease(increase) in the magnitude of \( \Be^i \), but if the orientation is remained fixed, the corresponding direction of the reciprocal frame vector stays the same.

\imageFigure{../figures/GAelectrodynamics/obliqueReciprocalFig2}{Oblique and reciprocal bases.}{fig:obliqueReciprocal:obliqueReciprocalFig2}{0.5}

How are the reciprocal frame vectors computed?  While these vectors have a natural GA representation, this is not intrinsically a GA problem, and can be solved with standard linear algebra, using a matrix inversion.
For example, given a 2D basis \( \setlr{ \Bx_1, \Bx_2 } \), the reciprocal basis can be assumed to have a coordinate representation in the original basis

\begin{dmath}\label{eqn:reciprocal:100}
\begin{aligned}
\Bx^1 &= a \Bx_1 + b \Bx_2 \\
\Bx^2 &= c \Bx_1 + d \Bx_2.
\end{aligned}
\end{dmath}

Imposing the constraints of \cref{dfn:reciprocal:frame} leads to a pair of 2x2 linear systems that are easily solved to find
\begin{dmath}\label{eqn:reciprocal:120}
\begin{aligned}
\Bx^1 &= \inv{ (\Bx_1)^2 (\Bx_2)^2 - \lr{ \Bx_1 \cdot \Bx_2}^2 } \lr{ (\Bx_2)^2 \Bx_1 - \lr{ \Bx_1 \cdot \Bx_2 } \Bx_2 } \\
\Bx^2 &= \inv{ (\Bx_1)^2 (\Bx_2)^2 - \lr{ \Bx_1 \cdot \Bx_2}^2 } \lr{ (\Bx_1)^2 \Bx_2 - \lr{ \Bx_1 \cdot \Bx_2 } \Bx_1 } \\
\end{aligned}
\end{dmath}

The reader may notice that for \R{3} the denominator is related to the norm of the cross product \( \Bx_1 \cross \Bx_2 \).
More generally, this can be expressed as the square of the bivector \( \Bx_1 \wedge \Bx_2 \)

\begin{dmath}\label{eqn:reciprocal:140}
-\lr{\Bx_1 \wedge \Bx_2 }^2
=
-\lr{\Bx_1 \wedge \Bx_2 } \cdot \lr{\Bx_1 \wedge \Bx_2 }
=
-\lr{ \lr{\Bx_1 \wedge \Bx_2 } \cdot \Bx_1 } \cdot \Bx_2
=
(\Bx_1)^2 (\Bx_2)^2 - \lr{\Bx_1 \cdot \Bx_2}^2.
\end{dmath}

Additionally, the numerators are each dot products of \( \Bx_1, \Bx_2 \) with that same bivector

\begin{dmath}\label{eqn:reciprocal:160}
\begin{aligned}
\Bx^1 &= \frac{\Bx_2 \cdot \lr{ \Bx_1 \wedge \Bx_2 } }{ \lr{ \Bx_1 \wedge \Bx_2}^2 } \\
\Bx^2 &= \frac{\Bx_1 \cdot \lr{ \Bx_2 \wedge \Bx_1 } }{ \lr{ \Bx_1 \wedge \Bx_2}^2 },
\end{aligned}
\end{dmath}

or

%\begin{dmath}\label{eqn:reciprocal:180}
\boxedEquation{eqn:reciprocal:180}{
\begin{aligned}
\Bx^1 &= \Bx_2 \cdot \inv{ \Bx_1 \wedge \Bx_2 } \\
\Bx^2 &= \Bx_1 \cdot \inv{ \Bx_2 \wedge \Bx_1 }.
\end{aligned}
}
%\end{dmath}

Geometrically, dotting with the bivector of the plane is a hybrid rotation and scaling operation.
For example, for \R{2} with \( \Bx_1 = a_1 \Be_1 + a_2 \Be_2, \Bx_2 = b_1 \Be_1 + b_2 \Be_2 \), that pseudoscalar for this basis is

\begin{dmath}\label{eqn:reciprocal:260}
\Bx_1 \wedge \Bx_2
=
\lr{ a_1 \Be_1 + a_2 \Be_2 } \wedge \lr{ b_1 \Be_1 + b_2 \Be_2 }
=
\lr{ a_1 b_2 - a_2 b_1 } \Be_{12}.
\end{dmath}

This has inverse
\begin{dmath}\label{eqn:reciprocal:280}
\inv{\Bx_1 \wedge \Bx_2 }
=
\inv{ a_1 b_2 - a_2 b_1 } \Be_{21}.
\end{dmath}

So for the \R{2} the reciprocal frame is just

\begin{dmath}\label{eqn:reciprocal:300}
\begin{aligned}
\Bx^1 &= \Bx_2 \frac{\Be_{21}}{ a_1 b_2 - a_2 b_1 } \\
\Bx^2 &= \Bx_1 \frac{\Be_{12}}{ a_1 b_2 - a_2 b_1 }
\end{aligned}
\end{dmath}

The vector \( \Bx^1 \) is obtained by rotating \( \Bx_2 \) by \( -\pi/2 \), and rescaling it.
The vector \( \Bx^2 \) is similarly obtained by a scaling and a rotation of \( \Bx_1 \) by \( \pi/2 \).

Generalizing \cref{eqn:reciprocal:180} is almost possible by inspection.
Given
a subspace spanned by a three vector basis \( \setlr{ \Bx_1, \Bx_2, \Bx_3 } \) the reciprocal frame vectors can be written as dot products

\begin{dmath}\label{eqn:reciprocal:320}
\begin{aligned}
\Bx^1 &= \lr{ \Bx_2 \wedge \Bx_3 } \cdot \lr{ \Bx^3 \wedge \Bx^2 \wedge \Bx^1 } \\
\Bx^2 &= \lr{ \Bx_3 \wedge \Bx_1 } \cdot \lr{ \Bx^1 \wedge \Bx^3 \wedge \Bx^2 } \\
\Bx^3 &= \lr{ \Bx_1 \wedge \Bx_2 } \cdot \lr{ \Bx^2 \wedge \Bx^1 \wedge \Bx^3 } \\
\end{aligned}
\end{dmath}

Each of those trivector terms equals \( - \Bx^1 \wedge \Bx^2 \wedge \Bx^3 \) and can be related to the (known) pseudoscalar \( \Bx_1 \wedge \Bx_2 \wedge \Bx_3 \) by observing that

\begin{dmath}\label{eqn:reciprocal:340}
\lr{ \Bx^1 \wedge \Bx^2 \wedge \Bx^3 } \cdot \lr{ \Bx_3 \wedge \Bx_2 \wedge \Bx_1 }
=
\Bx^1 \cdot \lr{ \Bx^2 \cdot \lr{ \Bx^3 \cdot \lr{ \Bx_3 \wedge \Bx_2 \wedge \Bx_1 } }}
=
\Bx^1 \cdot \lr{ \Bx^2 \cdot \lr{ \Bx_2 \wedge \Bx_1 } }
=
\Bx^1 \cdot \Bx_1
=
1,
\end{dmath}

which means that

\begin{dmath}\label{eqn:reciprocal:360}
-\Bx^1 \wedge \Bx^2 \wedge \Bx^3
= -\inv{ \Bx_3 \wedge \Bx_2 \wedge \Bx_1 }
= \inv{ \Bx_1 \wedge \Bx_2 \wedge \Bx_3 },
\end{dmath}

and

\boxedEquation{eqn:reciprocal:380}{
\begin{aligned}
\Bx^1 &= \lr{ \Bx_2 \wedge \Bx_3 } \cdot \inv{ \Bx_1 \wedge \Bx_2 \wedge \Bx_3 } \\
\Bx^2 &= \lr{ \Bx_3 \wedge \Bx_1 } \cdot \inv{ \Bx_1 \wedge \Bx_2 \wedge \Bx_3 } \\
\Bx^3 &= \lr{ \Bx_1 \wedge \Bx_2 } \cdot \inv{ \Bx_1 \wedge \Bx_2 \wedge \Bx_3 }
\end{aligned}
}

Geometrically, this trivector division is a duality transformation within the subspace spanned by the three vectors \( \Bx_1, \Bx_2, \Bx_3 \), also scaling the result so that the \( \Bx_i \cdot \Bx^j = {\delta_i}^j \) condition is satisfied.

It should be clear how to generalize the reciprocal basis calculation formulas of
\cref{eqn:reciprocal:180} and \cref{eqn:reciprocal:380} to higher dimensions if desired.
%}
%\EndNoBibArticle

         \subsection{Problems}
            \input{2dreciprocalMatrixCalculation.tex}
            \input{2subspaceR3reciprocalExample.tex}
      \section{Curvilinear coordinates}
         %
% Copyright © 2017 Peeter Joot.  All Rights Reserved.
% Licenced as described in the file LICENSE under the root directory of this GIT repository.
%
\index{curvilinear coordinates}
Curvilinear coordinates can be defined for any subspace spanned by a parameterized vector into that space.
%Consider a continuous subspace parameterized by a two parameter vector function \( \Bx = \Bx(u_1, u_2) \) that is differentiable with respect to either parameter
As an example, consider a two parameter planar subspace of parameterized by the following continuous vector function

\begin{dmath}\label{eqn:curvilinearDefined:480}
\Bx(u_1, u_2) = u_1 \Be_1 \frac{\sqrt{3}}{2} \cosh\lr{ \Atanh(1/2) + \Be_{12} u_2 },
\end{dmath}

where \( u_1 \in [0,1] \) and \( u_2 \in [0, \pi/2] \).
This parameterization spans the first quadrant of the ellipse with semi-major axis length 1, and semi-minor axis length \( 1/2 \)
\footnote{
A parameterization of an elliptic area may or may not not be of much use in electrodynamics.  It does, however, provide a fairly simple but non-trivial example of a non-orthonormal parameterization.}
Contours for this parameterization are plotted in \cref{fig:ellipticalContours:ellipticalContoursFig1}.
The radial contours are for fixed values of \( u_2 \) and the elliptical contours fix the value of \( u_1 \), and depict a set of ellipic curves
with a semi-major/major axis ratio of \( 1/2 \).

\imageFigure{../figures/GAelectrodynamics/ellipticalContoursFig1}{Contours for an elliptical region.}{fig:ellipticalContours:ellipticalContoursFig1}{0.3}

We define a curvilinear basis associated with each point in the region by the partials

\begin{dmath}\label{eqn:curvilinearDefined:80}
\begin{aligned}
\Bx_{1} &= \PD{u_1}{\Bx} \\
\Bx_{2} &= \PD{u_2}{\Bx}.
\end{aligned}
\end{dmath}

For our the function \cref{eqn:curvilinearDefined:480} our curvilinear basis elements are

\begin{dmath}\label{eqn:curvilinearDefined:520}
\begin{aligned}
\Bx_{1} &= \Be_1 \frac{\sqrt{3}}{2} \cosh\lr{ \Atanh(1/2) + \Be_{12} u_2 } \\
\Bx_{2} &= u_1 \Be_2 \frac{\sqrt{3}}{2} \sinh\lr{ \Atanh(1/2) + \Be_{12} u_2 }.
\end{aligned}
\end{dmath}

We form vector valued differentials for each parameter

\begin{dmath}\label{eqn:curvilinearDefined:500}
\begin{aligned}
d\Bx_{1} &= \Bx_1 du_1 \\
d\Bx_{2} &= \Bx_2 du_2.
\end{aligned}
\end{dmath}

For \cref{eqn:curvilinearDefined:480},
the values of these differentials \( d\Bx_1, d\Bx_2 \) with \( du_1 = du_2 = 0.1 \) are plotted
in
\cref{fig:ellipticalContours:ellipticalContoursFig2}
for the points
\( (u_1, u_2) = (0.7, 5 \pi/20), (0.9, 3 \pi/20), (1.0, 5 \pi/20) \)
in
(dark-thick) red, blue and purple respectively.

\imageFigure{../figures/GAelectrodynamics/ellipticalContoursFig2}{Differentials for an elliptical parameterization.}{fig:ellipticalContours:ellipticalContoursFig2}{0.3}

In this case and in general there is no reason to presume that there is any orthonormality constraint on the basis \( \setlr{ \Bx_{1}, \Bx_{2} } \) for a given two parameter subspace.

Should we wish to calculate the reciprocal frame
for \cref{eqn:curvilinearDefined:480}
, we would find
(\cref{problem:curvilinearDefined:560}) that

\begin{dmath}\label{eqn:curvilinearDefined:540}
\begin{aligned}
\Bx^{1} &= \Be_1 \sqrt{3} \sinh\lr{ \Atanh(1/2) + \Be_{12} u_2 } \\
\Bx^{2} &= \frac{\Be_2}{u_1} \sqrt{3} \cosh\lr{ \Atanh(1/2) + \Be_{12} u_2 }.
\end{aligned}
\end{dmath}

These are plotted (scaled by \( da = 0.1 \) so they fit in the image nicely) in \cref{fig:ellipticalContours:ellipticalContoursFig2} using thin light arrows.

When evaluating surface integrals, we will form
oriented (bivector) area elements from the wedge product of the differentials

\begin{dmath}\label{eqn:curvilinearDefined:60}
d^2 \Bx \equiv d\Bx_{1} \wedge d\Bx_{2}.
\end{dmath}

This absolute value of this area element \( \sqrt{-(d^2 \Bx)^2} \) is the area of the parallelogram spanned by \( d\Bx_1, d\Bx_2 \).
In this example, all such area elements lie in the \( x-y \) plane, but that need not be the case.

Also note that we will only perform integrals for those parametrizations for which the area element \( d^2 \Bx \) is non-zero.

%If the spacing between the contours is made small enough, the boundaries of each partition will define a planar region at the point of evaluation.
%All points in the interior will be accessible by a combination of the vectors formed from the partials of \( \Bx \) at that point.

\makeproblem{Elliptic curvilinear and reciprocal basis.}{problem:curvilinearDefined:560}{
From \cref{eqn:curvilinearDefined:480}, compute the
curvilinear coordinates \cref{eqn:curvilinearDefined:520}, and the reciprocal frame vectors \cref{eqn:curvilinearDefined:540}.
Check using scalar grade selection that \( \Bx^i \cdot \Bx_j = {\delta^i}_j \).
Hints: Given \( \mu = \Atanh(1/2) \),
\begin{itemize}
\item \( \cosh( \mu + i \theta ) \Be_2 = \Be_2 \cosh( \mu - i \theta ) \).
\item \( \Real\lr{ \cosh( \mu - i \theta ) \sinh( \mu + i \theta ) } = 2/3 \).
\end{itemize}
} % problem

\paragraph{fixme:}
don't introduce the idea of tangent space until a 3D example.
Remove the \R{3} reference above, and keep this first example planar.

At the point of evaluation, the span of these differentials is called the tangent space.
In this particular case the tangent space at all points in the region is the entire x-y plane.
These partials locally span the tangent space at a given point on the surface.

\subsubsection{Curved two parameter surfaces.}

Continuing to illustrate by example, let's now consider a non-planar two parameter surface

\begin{dmath}\label{eqn:curvilinearDefined:560}
\Bx(u_1, u_2) =
(u_1-u_2)^2
\Be_1
+ (1-(u_2)^2 ) \Be_2
+ u_1 u_2 \Be_3.
\end{dmath}

The curvilinear basis elements are
\begin{dmath}\label{eqn:curvilinearDefined:580}
\begin{aligned}
\Bx_1 &= 2 (u_1 - u_2) \Be_1 + u_2 \Be_3 \\
\Bx_2 &= 2 (u_2 - u_1) \Be_1 - 2 u_2 \Be_2 + u_1 \Be_3.
\end{aligned}
\end{dmath}

These vectors and two examples of the oriented plane (rescaled to fit) formed by \( \Bx_1 \wedge \Bx_2 \) is plotted in
\cref{fig:2dmanifold:2dmanifoldFig1}.
This plane is called the tangent space at the point in question, and has been evaluated at \( (u_1, u_2) = (0.5,0.5), (0.35, 0.75) \).

\imageFigure{../figures/GAelectrodynamics/2dmanifoldFig1}{Two parameter manifold.}{fig:2dmanifold:2dmanifoldFig1}{0.3}

%\imageFigure{../figures/GAelectrodynamics/twoParameterDifferentialFieldFig1}{Curvilinear coordinates along a two parameter surface.}{fig:twoParameterDifferentialField:twoParameterDifferentialFieldFig1}{0.3}


         %
% Copyright � 2017 Peeter Joot.  All Rights Reserved.
% Licenced as described in the file LICENSE under the root directory of this GIT repository.
%
While the reciprocal frame can be computed explcitly, it can also be computed very simply by computing the gradient of the parameters themselves.  Two theorems relate the gradient and the reciprocal frame vectors.

\maketheorem{Gradient definition of reciprocal frame vectors}{thm:curvilinearGradient:1}{

Given a curvilinear basis \( \setlr{ \Bx_k } \), the reciprocal frame vectors are

\begin{dmath*}
\Bx^i = \spacegrad u_i.
\end{dmath*}
} % theorem

\maketheorem{Curvilinear representation of the gradient}{thm:curvilinearGradient:2}{

Given an n-parameter representation of a vector that spans an n-dimensional space

\begin{dmath*}
\Bx = \Bx(u_1, \cdots, u_n),
\end{dmath*}

the curvilinear representation of the gradient is

\begin{dmath*}
\spacegrad = \sum_i \Bx^i \PD{u_i}{}.
\end{dmath*}

It is often convienent to write this as

\begin{dmath*}
\spacegrad = \sum_{i=1}^n \Bx^i \partial_i,
\end{dmath*}

or the same with sums over mixed indexes implied.

} % theorem

The proof of both are both just applications of the chain rule.  Assuming \cref{thm:curvilinearGradient:1} to be true, then the dot products of the reciprocal frame vectors with the curvilinear basis vectors are

\begin{dmath}\label{eqn:curvilinearGradient:20}
\Bx^i \cdot \Bx_j
=
(\spacegrad u_i) \cdot \PD{u_j}{\Bx}
=
\sum_{r,s=1}^n
\lr{ \Be_r \PD{x_r}{u_i} } \cdot \lr{ \Be_s \PD{u_j}{x_s} }
=
\sum_{r,s=1}^n (\Be_r \cdot \Be_s)
\PD{x_r}{u_i} \PD{u_j}{x_s}
=
\sum_{r,s=1}^n \delta_{rs}
\PD{x_r}{u_i} \PD{u_j}{x_s}
=
\sum_{r=1}^n
\PD{x_r}{u_i} \PD{u_j}{x_r}
=
\PD{u_i}{u_j}
=
\delta_{ij}.
\end{dmath}

This shows that \( \Bx^i = \spacegrad u_i \) has the properties required of the reciprocal frame, proving the theorem.

The curvilinear representation of the gradient follows from the gradient representation of the reciprocal frame, and the chain rule.  The sum in \cref{thm:curvilinearGradient:2} expands as

\begin{dmath}\label{eqn:curvilinearGradient:40}
\sum_{i=1}^n
\Bx^i \PD{u_i}{F}
=
\sum_{i=1}^n
(\spacegrad u_i) \PD{u_i}{F}
=
\sum_{i,j=1}^n
\Be_j \PD{x_j}{u_i}
\PD{u_i}{F}
=
\sum_{j=1}^n
\Be_j
\PD{x_j}{F}
=
\spacegrad F,
\end{dmath}

which proves the result.

Note that the gradient representation of the reciprocal frame is mainly useful for theoretical reasons (i.e. the proof of the curvilinear representation of the gradient).  In many cases it will likely be more difficult to compute the reciprocal frame vectors using the gradient of the parameters than

An excellent (and more detailed) discussion of the relationships of the reciprocal frame and the gradient can be found in \citep{aMacdonaldVAGC}.


Many of the concepts are illuminated nicely by considering some examples.

         \subsection{Cylindrical coordinates.}
            %
% Copyright © 2017 Peeter Joot.  All Rights Reserved.
% Licenced as described in the file LICENSE under the root directory of this GIT repository.
%
%\index{cylindrical coordinates}
\index{polar coordinates}
\index{curvilinear coordinates}
One of the simplest curvilinear coordinate systems are polar coordinates (cylindrical coordinates in a plane.)

FIXME: Wolfgang: ``picture.''

The parameterization associated with such a space is

\begin{dmath}\label{eqn:2Dcylindrical:100}
\Bx(\rho, \phi) = \rho \Be_1 \exp\lr{ \Be_{12} \phi }.
\end{dmath}

The curvilinear coordinate basis is therefore

\begin{subequations}
\label{eqn:2Dcylindrical:120}
\begin{dmath}\label{eqn:2Dcylindrical:140}
\Bx_\rho
= \PD{\rho}{} \lr{ \rho \Be_1 \exp\lr{ \Be_{12} \phi } }
= \Be_1 \exp\lr{ \Be_{12} \phi }
\end{dmath}
\begin{dmath}\label{eqn:2Dcylindrical:160}
\Bx_\phi
= \PD{\phi}{} \lr{ \rho \Be_1 \exp\lr{ \Be_{12} \phi } }
= \rho
\Be_1 \Be_{12} \exp\lr{ \Be_{12} \phi }
= \rho
\Be_2 \exp\lr{ \Be_{12} \phi }.
\end{dmath}
\end{subequations}

\index{reciprocal basis}
Noting that this is a normal set of vectors, the reciprocal basis can be found by inspection

\begin{dmath}\label{eqn:2Dcylindrical:180}
\begin{aligned}
\Bx^\rho &= \Be_1 \exp\lr{ \Be_{12} \phi } \\
\Bx^\phi &= \inv{\rho} \Be_2 \exp\lr{ \Be_{12} \phi }.
\end{aligned}
\end{dmath}

\index{gradient}
For completeness, it's worth verifying that the gradient representation of the reciprocal frame provides this same result.
The \( x, y \) variables are related to \( \rho, \phi \) through

\begin{dmath}\label{eqn:2Dcylindrical:620}
\begin{aligned}
x &= r \cos\phi \\
y &= r \sin\phi.
\end{aligned}
\end{dmath}

Rearranging slightly to facilitate evaluation of the \( x, y \) partials

\begin{dmath}\label{eqn:2Dcylindrical:500}
\begin{aligned}
\rho^2 &= x^2 + y^2 \\
\tan\phi &= y/x,
\end{aligned}
\end{dmath}

we can evaluate the components of the gradients by implicit differentiation

\begin{dmath}\label{eqn:2Dcylindrical:520}
\begin{aligned}
2 \rho \PD{x}{\rho} &= 2 x \\
2 \rho \PD{y}{\rho} &= 2 y \\
\inv{\cos^2\phi} \PD{x}{\phi} &= -\frac{y}{x^2} \\
\inv{\cos^2\phi} \PD{y}{\phi} &= \inv{x},
\end{aligned}
\end{dmath}

The gradients are
\begin{subequations}
\label{eqn:2Dcylindrical:540}
\begin{dmath}\label{eqn:2Dcylindrical:560}
\spacegrad \rho
= \inv{\rho} (\cos\phi, \sin\phi)
= \Be_1 e^{\Be_{12} \phi}
= \Bx^\rho
\end{dmath}
\begin{dmath}\label{eqn:2Dcylindrical:580}
\spacegrad \phi
=
\cos^2 \phi \lr{ -\frac{y}{x^2}, \inv{x} }
=
\inv{\rho} ( -\sin\phi, \cos\phi )
=
\frac{\Be_2}{\rho} ( \cos\phi + \Be_{12} \sin\phi )
=
\frac{\Be_2}{\rho} e^{ \Be_{12} \phi }
=
\Bx^\phi,
\end{dmath}
\end{subequations}

which is consistent with the result found by inspection as desired.

In this particular parameterization, it is convenient to define a locally orthonormal coordinate basis \( \setlr{ \rhocap, \phicap } \)

\begin{dmath}\label{eqn:2Dcylindrical:200}
\begin{aligned}
\rhocap &= \Bx_\rho = \Be_1 \exp\lr{ \Be_{12} \phi } \\
\phicap &= \inv{\rho} \Bx_\phi = \Be_2 \exp\lr{ \Be_{12} \phi },
\end{aligned}
\end{dmath}

so that \( \Bx^\rho = \Bx_\rho = \rhocap \), \( \Bx_\phi = \rho \rhocap \), and \( \Bx^\phi = \rhocap/\rho \), and the gradient is

\begin{dmath}\label{eqn:2Dcylindrical:600}
\spacegrad
=
\Bx^\rho \PD{\rho}{}
+ \Bx^\phi \PD{\phi}{}
=
\rhocap \PD{\rho}{}
+\inv{\rho} \phicap \PD{\phi}{}.
\end{dmath}

The volume element for this subspace is
\begin{dmath}\label{eqn:2Dcylindrical:220}
d\Bx_\rho \wedge d\Bx_\phi
=
d\rho d\phi
\Bx_\rho \wedge \Bx_\phi
=
d\rho d\phi
\gpgradetwo{
\Bx_\rho \Bx_\phi
}
=
d\rho d\phi
\gpgradetwo{
\Be_1 \exp\lr{ \Be_{12} \phi } \rho
\Be_2 \exp\lr{ \Be_{12} \phi }
}
\end{dmath}

To evaluate this we use \cref{thm:SimpleProducts2:1780}, property (c), and change the order of a pair of vector and complex exponentials, performing the required conjugation of that exponential

\begin{dmath}\label{eqn:2Dcylindrical:640}
d\Bx_\rho \wedge d\Bx_\phi
=
\rho d\rho d\phi
\gpgradetwo{
\Be_1 \Be_2 \exp\lr{ -\Be_{12} \phi }
\exp\lr{ \Be_{12} \phi }
}
=
\rho d\rho d\phi \Be_{12}.
\end{dmath}

Observe that the (oriented) volume of a circular region of radius \( r \) in this space has the expected result

\begin{dmath}\label{eqn:2Dcylindrical:360}
\int d\Bx_\rho \wedge d\Bx_\phi
=
\int_0^r \rho d\rho \int_0^{2\pi} d\phi \Be_{12}
= \pi r^2 \Be_{12}.
\end{dmath}

Given a vector \( \Bv = \Be_1 f(\rho, \phi) + \Be_2 g(\rho, \phi) \), the cylindrical representation \( \Bv = \Bv_\rho \rhocap + \Bv_\phi \phicap \) can be found by computing the dot products

\begin{subequations}
\label{eqn:2Dcylindrical:420}
\begin{dmath}\label{eqn:2Dcylindrical:440}
\Bv \cdot \rhocap
=
\gpgradezero{ (\Be_1 f + \Be_2 g) \Be_1 e^{\Be_{12} \phi} }
=
f \cos\phi + g \sin\phi
\end{dmath}
\begin{dmath}\label{eqn:2Dcylindrical:460}
\Bv \cdot \phicap
=
\gpgradezero{ (\Be_1 f + \Be_2 g) \Be_2 e^{\Be_{12} \phi} }
=
g \cos\phi - f \sin\phi,
\end{dmath}
\end{subequations}

so
\begin{dmath}\label{eqn:2Dcylindrical:480}
\Bv = \lr{ f \cos\phi + g \sin\phi } \rhocap + \lr{ g \cos\phi - f \sin\phi } \phicap.
\end{dmath}


         \subsection{Spherical coordinates.}
            %
% Copyright � 2017 Peeter Joot.  All Rights Reserved.
% Licenced as described in the file LICENSE under the root directory of this GIT repository.
%

The spherical vector parameterization admits a compact GA representation.
From the coordinate representation, some factoring gives

\begin{dmath}\label{eqn:curvilinearspherical:20}
\Bx
= r \lr{ \Be_1 \sin\theta \cos\phi + \Be_2 \sin\theta \sin\phi + \Be_3 \cos\theta }
= r \lr{ \sin\theta \Be_1 (\cos\phi + \Be_{12} \sin\phi ) + \Be_3 \cos\theta }
= r \lr{ \sin\theta \Be_1 e^{\Be_{12} \phi } + \Be_3 \cos\theta }
= r \Be_3 \lr{ \cos\theta + \sin\theta \Be_3 \Be_1 e^{\Be_{12} \phi } }.
\end{dmath}

With
\begin{dmath}\label{eqn:curvilinearspherical:40}
\begin{aligned}
i &= \Be_{12} \\
j &= \Be_{31} e^{i \phi},
\end{aligned}
\end{dmath}

this is

\begin{dmath}\label{eqn:curvilinearspherical:60}
\Bx = r \Be_3 e^{j \theta}.
\end{dmath}

The curvilinear basis vectors can now be computed

\begin{subequations}
\label{eqn:curvilinearspherical:80}
\begin{dmath}\label{eqn:curvilinearspherical:100}
\Bx_r = \Be_3 e^{j \theta}
\end{dmath}
\begin{dmath}\label{eqn:curvilinearspherical:120}
\Bx_\theta
= \Be_3 j e^{j \theta}
= \Be_3 \Be_{31} e^{i\phi} e^{j \theta}
= \Be_1 e^{i\phi} e^{j \theta}
\end{dmath}
\begin{dmath}\label{eqn:curvilinearspherical:140}
\Bx_\phi
=
\PD{\phi}{} \lr{
r \Be_3 \sin\theta \Be_{31} e^{i \phi}
}
=
r \sin\theta \Be_1 \Be_{12} e^{i \phi}
=
r \sin\theta \Be_2 e^{i \phi}.
\end{dmath}
\end{subequations}

These are all mutually normal, which can be verified by computing dot products.
With that asserted, orthornomalizing the curvilinear basis is now possible by inspection

\begin{dmath}\label{eqn:curvilinearspherical:240}
\begin{aligned}
\rcap &= \Bx_r \\
\thetacap &= \inv{r} \Bx_\theta \\
\phicap &= \inv{r \sin\theta} \Bx_\phi,
\end{aligned}
\end{dmath}

or

\begin{dmath}\label{eqn:curvilinearspherical:260}
\begin{aligned}
\Bx^r &= \rcap \\
\Bx^\theta &= \inv{r} \thetacap \\
\Bx^\phi &= \inv{r \sin\theta} \phicap.
\end{aligned}
\end{dmath}

In particular, this shows that the spherical representation of the gradient is
\begin{dmath}\label{eqn:curvilinearspherical:280}
\spacegrad
=
\Bx^r \PD{r}{}
+ \Bx^\theta \PD{\theta}{}
+ \Bx^\phi \PD{\phi}{}
=
\rcap \PD{r}{}
+\inv{r} \thetacap \PD{\theta}{}
+\inv{r \sin\theta} \PD{\phi}{}.
\end{dmath}

The spherical (oriented) volume element can also be computed in a compact fashion, without having to evaluate a very messy Jacobian determinant

\begin{dmath}\label{eqn:curvilinearspherical:300}
\Bx_r \wedge \Bx_\theta \wedge \Bx_\phi
=
\gpgradethree{
\Bx_r \Bx_\theta \Bx_\phi
}
=
\gpgradethree{
\Be_3 e^{j \theta}
r \Be_1 e^{i\phi} e^{j \theta}
r \sin\theta \Be_2 e^{i \phi}
}
=
r^2 \sin\theta
\gpgradethree{
\Be_3 e^{j \theta}
\Be_1 e^{i\phi} e^{j \theta}
\Be_2 e^{i \phi}
}
=
r^2 \sin\theta \Be_{123}
.
\end{dmath}

The final reduction is left as a problem for the student.
It is left to the student to evaluate whether this method is easier or more difficult than the conventional volume element Jacobean determinant expansion

\begin{dmath}\label{eqn:curvilinearspherical:320}
dV =
dr d\theta d\phi\,
\frac{\partial( x_1, x_2, x_3)}{\partial(r, \theta, \phi)}
=
dr d\theta d\phi\,
\begin{vmatrix}
\sin\theta \cos\phi & \sin\theta \sin\phi & \cos\theta \\
r \cos\theta \cos\phi & r \cos\theta \sin\phi & -r \sin\theta \\
-r \sin\theta \sin\phi & r \sin\theta \cos\phi & 0 \\
\end{vmatrix}.
\end{dmath}

It is easily argued that both volume element calculation methods are best performed by a computer algebra system.


         \subsection{Toroidal coordinates.}
            %
% Copyright � 2012 Peeter Joot.  All Rights Reserved.
% Licenced as described in the file LICENSE under the root directory of this GIT repository.
%
\index{toroid}
\index{differential form}
%\imageFigure{../figures/gabook/toriodalSegment}{Toroidal parameterization.}{fig:toriodalSegment}{0.5}
\mathImageFigure{../figures/GAelectrodynamics/toroidFig1}{Toroidal parameterization.}{fig:toriodalSegment}{0.3}{gaToroid.nb}
Here is a 3D example of a parameterization with a non-orthogonal curvilinear basis, that of a
toroidal subspace specified by two angles and a radial distance to the center of the toroid, as illustrated in \cref{fig:toriodalSegment}.

The position vector on the surface of a toroid of radius \( \rho \) within the torus can be stated directly
\begin{subequations}
\begin{align}\label{eqn:torusCenterOfMassParameterization:1}
\Bx(\rho, \theta, \phi) &= e^{-j\theta/2} \left( \rho \Be_1 e^{ i \phi } + R \Be_3 \right) e^{j \theta/2} \\
i &= \Be_1 \Be_3 \\
j &= \Be_3 \Be_2
\end{align}
\end{subequations}

It happens that the unit bivectors \(i\) and \(j\) used in this construction happen
to have the
quaternion-ic properties \(i j = -j i\), and \(i^2 = j^2 = -1\) which can be verified easily.

The curvilinear basis is found (\cref{problem:toriodalProblem:1}) to be
\begin{subequations}\label{eqn:torusCenterOfMassParameterization:3}
\begin{align}
\Bx_\rho &= \PD{\rho}{\Bx} = e^{-j\theta/2} \Be_1 e^{ i \phi } e^{j \theta/2} \\
\Bx_\theta &= \PD{\theta}{\Bx}
%&= e^{-j\theta/2} \left( \rho \inv{2} \left( -\Be_3 \Be_2 \Be_1 e^{ i \phi } + \Be_1 e^{ i \phi } \Be_3 \Be_2 \right) + R \Be_2 \right) e^{j \theta/2} \\
= e^{-j\theta/2} \left( R + \rho \sin\phi \right) \Be_2 e^{j \theta/2} \\
\Bx_\phi &= \PD{\phi}{\Bx} = e^{-j\theta/2} \rho \Be_3 e^{ i \phi } e^{j \theta/2}.
\end{align}
\end{subequations}

The oriented
volume element can be computed using a trivector selection operation, which conveniently wipes out a number of the interior exponentials
%\begin{align}\label{eqn:torusCenterOfMassParameterization:4}
\begin{equation}\label{eqn:torusCenterOfMassParameterization:4}
\PD{\rho}{\Bx} \wedge \PD{\theta}{\Bx} \wedge \PD{\phi}{\Bx}
=
\rho \left( R + \rho \sin\phi \right) \gpgradethree{ e^{-j\theta/2} \Be_1 e^{ i \phi } \Be_2 \Be_3 e^{ i \phi } e^{j \theta/2} }.
%\end{align}
\end{equation}

Note that \(\Be_1\) commutes with \(j = \Be_3 \Be_2\), so also with \(e^{-j\theta/2}\).
Also \(\Be_2 \Be_3 = -j\) anticommutes with \(i\), so
there is a conjugate commutation effect \(e^{i\phi} j = j e^{-i\phi}\).  This gives
\begin{equation}\label{eqn:torusCenterOfMassParameterization:28}
\begin{aligned}
\gpgradethree{ e^{-j\theta/2} \Be_1 e^{ i \phi } \Be_2 \Be_3 e^{ i \phi } e^{j \theta/2} }
&=
-\gpgradethree{ \Be_1 e^{-j\theta/2} j e^{ -i \phi } e^{ i \phi } e^{j \theta/2} } \\
&=
-\gpgradethree{ \Be_1 e^{-j\theta/2} j e^{j \theta/2} } \\
&=
-\gpgradethree{ \Be_1 j } \\
&=
I.
\end{aligned}
\end{equation}

Together the trivector grade selection reduces almost magically to just
\begin{equation}\label{eqn:torusCenterOfMassParameterization:5}
\PD{\rho}{\Bx} \wedge \PD{\theta}{\Bx} \wedge \PD{\phi}{\Bx}
=
\rho \left( R + \rho \sin\phi \right) I.
\end{equation}

\todo{Show this with Mathematica too.}

Thus the (scalar) volume element is
\begin{align}\label{eqn:torusCenterOfMassParameterization:6}
dV = \rho \left( R + \rho \sin\phi \right) d\rho d\theta d\phi.
\end{align}

As a check, it should be the case that the
volume of the complete torus using this volume element has the
expected \(V = (2 \pi R) (\pi r^2)\) value.

That volume is
\begin{align}\label{eqn:torusCenterOfMassParameterization:7}
V = \int_{\rho=0}^r \int_{\theta=0}^{2\pi} \int_{\phi=0}^{2\pi} \rho \left( R + \rho \sin\phi \right) d\rho d\theta d\phi.
\end{align}

The sine term conveniently vanishes over the \(2\pi\) interval, leaving just
\begin{align}\label{eqn:torusCenterOfMassParameterization:8}
V = \inv{2} r^2 R (2 \pi)(2 \pi),
\end{align}

as expected.


         \subsection{Problems}
            %
% Copyright � CCYY Peeter Joot.  All Rights Reserved.
% Licenced as described in the file LICENSE under the root directory of this GIT repository.
%
\makeproblem{Spherical coordinate basis orthogonality.}{problem:sphericaldot:1}{
\index{spherical coordinates}
Using scalar selection, show that the spherical curvilinear basis of \cref{eqn:curvilinearspherical:80} are all mutually orthogonal.
} % problem

\makeanswer{problem:sphericaldot:1}{
Computing the various dot products is made easier by noting that \( \Be_3 \) and \( e^{i \phi } \) commute, whereas \( e^{j\theta } \Be_3 = \Be_3 e^{-j\theta}, \Be_1 e^{i\phi} = e^{-i\phi} \Be_1, \Be_2 e^{i\phi} = e^{-i\phi} \Be_2 \) (since \( \Be_3 j \), \( \Be_1 i \) and \( \Be_2 i \) all anticommute.)  Also note that
\begin{equation}\label{eqn:sphericaldot:240}
\begin{aligned}
j \phicap
&= \Be_{31} e^{i\phi} \Be_2 e^{i\phi} \\
&= \Be_{312} e^{-i\phi} e^{i\phi} \\
&= I.
\end{aligned}
\end{equation}
The dot products, working with the normalized vectors, are
\begin{subequations}
\label{eqn:sphericaldot:160}
\begin{equation}\label{eqn:sphericaldot:180}
\begin{aligned}
\rcap \cdot \thetacap
&=
\gpgradezero{
\rcap \rcap j
} \\
&=
\gpgradezero{
j
} \\
&= 0
\end{aligned}
\end{equation}
\begin{equation}\label{eqn:sphericaldot:200}
\begin{aligned}
\rcap \cdot \phicap
&=
\gpgradezero{
\Be_3 e^{j \theta} \phicap
} \\
&=
\gpgradezero{
\Be_3 \lr{ \cos\theta + j \sin\theta } \phicap
} \\
&=
\cos\theta
\gpgradezero{
\Be_3
\phicap
}
+
\sin\theta
\gpgradezero{
\Be_3 j \phicap
}
\\
&=
\cos\theta
\gpgradezero{
\Be_{32} \cos\phi + \Be_{13} \sin\phi
}
+
\sin\theta
\gpgradezero{
\Be_{12}
}
\\
&=
0
\end{aligned}
\end{equation}
\begin{equation}\label{eqn:sphericaldot:220}
\begin{aligned}
\thetacap \cdot \phicap
&=
\gpgradezero{
\rcap j \phicap
} \\
&=
\gpgradezero{
\rcap I
} \\
&=
0.
\end{aligned}
\end{equation}
\end{subequations}
} % answer

            %
% Copyright � CCYY Peeter Joot.  All Rights Reserved.
% Licenced as described in the file LICENSE under the root directory of this GIT repository.
%
\makeproblem{Spherical volume element pseudoscalar.}{problem:volumeselection:1}{
Using geometric algebra, perform the reduction of the grade three selection made in the final step of \cref{eqn:curvilinearspherical:300}.
} % problem
\makeanswer{problem:volumeselection:1}{
\begin{equation}\label{eqn:volumeselection:17}
\begin{aligned}
\Be_{31} e^{i\phi} \Be_2 e{i\phi}
&=
\Be_{31} e^{i\phi} e^{-i\phi} \Be_2 \\
&=
\Be_{312} \\
&= I.
\end{aligned}
\end{equation}
} % answer


%gabook: 31.1
%Also: Stokes chapter.  Lots of examples there that should really be separated out from the stokes core content
%(now included here).
      \section{Green's theorem}
         %
% Copyright © 2013 Peeter Joot.  All Rights Reserved.
% Licenced as described in the file LICENSE under the root directory of this GIT repository.
%
Given a two parameter (\(u,v\)) surface parameterization, the curvilinear coordinate representation of a vector \(\Bf\) has the form

\begin{dmath}\label{eqn:stokesTheoremGeometricAlgebra:1640}
\Bf = f_u \Bx^u + f_v \Bx^v + f_\perp \Bx^\perp.
\end{dmath}

We assume that the vector space is of dimension two or greater but otherwise unrestricted, and need not have an Euclidean basis.  Here \(f_\perp \Bx^\perp\) denotes the rejection of \(\Bf\) from the tangent space at the point of evaluation.  Green's theorem relates the integral around a closed curve to an ``area'' integral on that surface

\maketheorem{Green's Theorem}{thm:stokesTheoremGeometricAlgebra:1660}{
\index{Green's theorem}
\begin{equation*}
\ointctrclockwise \Bf \cdot d\Bl
=
\iint \lr{
-\PD{v}{f_u}
+\PD{u}{f_v}
}
du dv
\end{equation*}
}

Following the arguments used in \citep{schwartz1987pe} for Stokes theorem in three dimensions, we first evaluate the loop integral along the differential element of the surface at the point \(\Bx(u_0, v_0)\) evaluated over the range \((du, dv)\), as shown in the infinitesimal loop of \cref{fig:loopIntegralInfinitesimal:loopIntegralInfinitesimalFig1}.

\imageFigure{../figures/gabook/loopIntegralInfinitesimalFig1}{Infinitesimal loop integral}{fig:loopIntegralInfinitesimal:loopIntegralInfinitesimalFig1}{0.35}

Over the infinitesimal area, the loop integral decomposes into

\begin{dmath}\label{eqn:stokesTheoremGeometricAlgebra:1700}
\ointctrclockwise \Bf \cdot d\Bl
=
\int \Bf \cdot d\Bx_1
+\int \Bf \cdot d\Bx_2
+\int \Bf \cdot d\Bx_3
+\int \Bf \cdot d\Bx_4,
\end{dmath}

where the differentials along the curve are

\begin{dmath}\label{eqn:stokesTheoremGeometricAlgebra:1600}
\begin{aligned}
d\Bx_1 &= \evalbar{ \PD{u}{\Bx} }{v = v_0} du \\
d\Bx_2 &= \evalbar{ \PD{v}{\Bx} }{u = u_0 + du} dv \\
d\Bx_3 &= -\evalbar{ \PD{u}{\Bx} }{v = v_0 + dv} du \\
d\Bx_4 &= -\evalbar{ \PD{v}{\Bx} }{u = u_0} dv.
\end{aligned}
\end{dmath}

It is assumed that the parameterization change \((du, dv)\) is small enough that this loop integral can be considered planar (regardless of the dimension of the vector space).  Making use of the fact that \(\Bx^\perp \cdot \Bx_\alpha = 0\) for \(\alpha \in \setlr{u,v}\), the loop integral is

\begin{dmath}\label{eqn:stokesTheoremGeometricAlgebra:1620}
\ointctrclockwise \Bf \cdot d\Bl
=
\int
\lr{
f_u \Bx^u + f_v \Bx^v + f_\perp \Bx^\perp
}
\cdot
\Bigl(
\Bx_u(u, v_0) du - \Bx_u(u, v_0 + dv) du
+\Bx_v(u_0 + du, v) dv - \Bx_v(u_0, v) dv
\Bigr)
=
\int
f_u(u, v_0) du - f_u(u, v_0 + dv) du
+
f_v(u_0 + du, v) dv - f_v(u_0, v) dv
\end{dmath}

With the distances being infinitesimal, these differences can be rewritten as partial differentials

\begin{dmath}\label{eqn:stokesTheoremGeometricAlgebra:1860}
\ointctrclockwise \Bf \cdot d\Bl
=
\iint \lr{
-\PD{v}{f_u}
+\PD{u}{f_v}
}
du dv.
\end{dmath}

We can now sum over a larger area as in \cref{fig:loopIntegralInfinitesimalSum:loopIntegralInfinitesimalSumFig2}

\imageFigure{../figures/gabook/loopIntegralInfinitesimalSumFig2}{Sum of infinitesimal loops}{fig:loopIntegralInfinitesimalSum:loopIntegralInfinitesimalSumFig2}{0.35}

All the opposing oriented loop elements cancel, so the integral around the complete boundary of the surface \(\Bx(u, v)\) is given by the \(u,v\) area integral of the partials difference.

We will see that Green's theorem is a special case of the Stokes theorem.  This observation will also provide a geometric interpretation of the right hand side area integral of \cref{thm:stokesTheoremGeometricAlgebra:1660}, and allow for a coordinate free representation.

\paragraph{Special case:}

An important special case of Green's theorem is for a Euclidean two dimensional space where the vector function is

\begin{dmath}\label{eqn:stokesTheoremGeometricAlgebra:1720}
\Bf = P \Be_1 + Q \Be_2.
\end{dmath}

Here Green's theorem takes the form

\boxedEquation{eqn:stokesTheoremGeometricAlgebra:1710}{
\ointctrclockwise P dx + Q dy
=
\iint \lr{
\PD{x}{Q}
-\PD{y}{P}
}
dx dy.
}

         %\subsection{Problems}
      \section{Stokes' theorem}
         \subsection{Statement}
            %
% Copyright © 2016 Peeter Joot.  All Rights Reserved.
% Licenced as described in the file LICENSE under the root directory of this GIT repository.
%

Stokes' theorem is fairly easy to state, but takes a fair amount of work to understand and unpack its implications.

%
% Copyright © 2013 Peeter Joot.  All Rights Reserved.
% Licenced as described in the file LICENSE under the root directory of this GIT repository.
%
An important consequence of the fundamental theorem of geometric calculus is the
geometric algebra generalization of Stokes' theorem.  This form of Stokes' theorem is equivalent to the same from the theory of differential forms.
Stokes' theorem in differential forms and geometric algebra is more general and powerful than Stokes' theorem from conventional vector calculus which only relates
surface integrals to the line integral around the bounding surface.

\maketheorem{Stokes' Theorem}{thm:stokesTheoremGeometricAlgebra:1740}{
Given a \(k\) volume element \(d^k \Bx \) and an s-blade \( F, s < k \)
\begin{equation*}%\label{eqn:stokesTheoremTheStatement:120}
\int_V d^k \Bx \cdot (\boldpartial \wedge F) = \int_{\partial V} d^{k-1} \Bx \cdot F.
\end{equation*}
%Here the volume integral is over a \(k\) dimensional hypervolumesurface (manifold).  The derivative operator \(\boldpartial\) is called the vector derviative and is the projection of the gradient onto the tangent space of the manifold.  Integration over the boundary of \(V\) is indicated by \( \partial V \).
}

We will see that most of the well known scalar and vector integral theorems are consequences of \cref{thm:stokesTheoremGeometricAlgebra:1740}.

To prove the theorem, set \( F = 1 \) in \cref{thm:fundamentalTheoremOfCalculus:1}, and require that \( G \) is an s-blade, with grade \( s < k \).  We select the \( k-(s+1) \) grade, the lowest grade of \( d^k \Bx (\boldpartial \wedge G) \) from
both sides of
\cref{thm:fundamentalTheoremOfCalculus:1}.

For the grade selection of the hypervolume integral we have
\begin{dmath}\label{eqn:stokesTheoremTheStatement:100}
\gpgrade{ \int_V d^k \Bx \boldpartial G }{k-(s+1)}
=
\gpgrade{
\int_V d^k \Bx (\boldpartial \cdot G )
+
\int_V d^k \Bx (\boldpartial \wedge G )
}{k-(s-1)},
\end{dmath}
however, the lowest grade of \( d^k \Bx (\boldpartial \cdot G ) \) is \( k -(s-1) = k - s + 1 > k - (s+1) \), so the divergence integral is zero.  This leaves
\begin{dmath}\label{eqn:stokesTheoremTheStatement:110}
\int_V d^k \Bx \cdot (\boldpartial \wedge G )
= \int_{\partial V} \gpgrade{d^{k-1} \Bx G}{k-(s+1)}
= \int_{\partial V} d^{k-1} \Bx \cdot G,
\end{dmath}
proving the theorem.

%%%\paragraph{FIXME: (rewrite) old proof using gagc.}
%%%The vector derivative is defined by
%%%\begin{equation}\label{eqn:stokesTheoremTheStatement:1400}
%%%\boldpartial = \Bx^i \partial_i = \sum_i \Bx_i \PD{u^i}{}.
%%%\end{equation}
%%%
%%%where \( \Bx^i \) are reciprocal frame vectors dual to the tangent vector basis \( \Bx_i \) associated with the parameters \( u^1, u^2, \cdots \).
%%%%These will be defined in more detail in the next section.
%%%Once the volume element, vector product and the other concepts are defined, the proof of
%%%Stokes theorem is really just a statement that
%%%\boxedEquation{eqn:stokesTheoremGeometricAlgebra:2840}{
%%%\int_V d^k \Bx \cdot (\Bx^i \partial_i \wedge F) =
%%%\int_V \lr{ d^k \Bx \cdot \Bx^i } \cdot \partial_i F.
%%%}
%%%
%%%This dot product expansion applies to any degree blade and volume element provided the degree of the blade is less than that of the volume element (i.e. \(s < k\)).  That magic follows directly from \cref{thm:stokesTheoremGeometricAlgebra:1420}.


This dot product defines the oriented surface ``area'' elements associated with the ``volume'' element \( d^k \Bx \).  That area element can be obtained from the mnemonic

\begin{dmath}\label{eqn:stokesTheoremCore:1561}
\sum_i d^k \Bx \cdot \Bx^i,
\end{dmath}

with each of the ith differentials evaluated.  This will be made clear by example.


         \subsection{One parameter specialization of Stokes' theorem.}
            %
% Copyright © 2016 Peeter Joot.  All Rights Reserved.
% Licenced as described in the file LICENSE under the root directory of this GIT repository.
%
\index{differential form}
An example parameterization with one parameter, and the corresponding differential with respect to that parameter, is plotted in
\cref{fig:oneParameterDifferential:oneParameterDifferentialFig1}, for a parameterization over \( [a, b] \in [0,1]\otimes[0,1] \).

\imageFigure{../figures/GAelectrodynamics/oneParameterDifferentialFig1}{One parameter manifold.}{fig:oneParameterDifferential:oneParameterDifferentialFig1}{0.3}

The differential with respect to the parameter \( a \) is

\begin{equation}\label{eqn:stokesTheoremCore:20}
d\Bx_a = \PD{a}{\Bx} da = \Bx_a da.
\end{equation}

On this curve the projection of the gradient has just one component

\begin{dmath}\label{eqn:stokesTheoremCore:40}
\boldpartial
=
\sum_\mu \Bx^\mu (\Bx_\mu \cdot \spacegrad)
=
\Bx^a \PD{a}{}
\equiv
\Bx^a \partial_a.
\end{dmath}

Please see \citep{aMacdonaldVAGC} for a full justification of the curvilinear coordinate representation of the vector derivative (or the gradient).
That text also discusses pertinent issues with the connectivity of the manifold ignored here.

Stokes' theorem for a one parameter manifold can only be expressed for scalar fields.
That is

\begin{dmath}\label{eqn:stokesTheoremCore:60}
\int d\Bx \cdot (\boldpartial \wedge \psi)
=
\int d\Bx \cdot \boldpartial \psi
=
\int da \PD{a}{ \psi }
= \evalbar{\psi}{\Delta a}.
\end{dmath}

Observe that the vector derivative can be replaced by the gradient since \( d\Bx \cdot \boldpartial = d\Bx \cdot \spacegrad \).
This is the case since dotting the
gradient with a differential element \( d\Bx \) on this curve, no component of the gradient that isn't colinear to the curve makes no contribution.

\index{Stokes' theorem}
That means that Stokes' theorem for a one parameter curve is exactly the fundamental theorem of calculus for line integrals

%\begin{dmath}\label{eqn:stokesTheoremCore:80}
\boxedEquation{eqn:stokesTheoremCore:80}{
\int_{\Ba}^{\Bb} d\Bx \cdot \spacegrad \psi = \psi(\Bb) - \psi(\Ba).
}
%\end{dmath}

         \subsection{Two parameter specialization of Stokes' theorem.}
            %
% Copyright © 2016 Peeter Joot.  All Rights Reserved.
% Licenced as described in the file LICENSE under the root directory of this GIT repository.
%

An example parameterization with two parameters, and the corresponding differentials with respect to those parameters, is plotted in
\cref{fig:twoParameterDifferential:twoParameterDifferentialFig1}.

\imageFigure{../figures/GAelectrodynamics/twoParameterDifferentialFig1}{Two parameter manifold differentials.}{fig:twoParameterDifferential:twoParameterDifferentialFig1}{0.4}

Given parameters \( a, b \), the differentials along each of the parameterization directions are

\begin{dmath}\label{eqn:stokesTheoremCore:100}
\begin{aligned}
d\Bx_a &= \PD{a}{\Bx} da = \Bx_a da \\
d\Bx_b &= \PD{b}{\Bx} db = \Bx_b db.
\end{aligned}
\end{dmath}

The ``volume'' element for this parameterization (a surface area element) is

\begin{equation}\label{eqn:stokesTheoremCore:120}
d^2 \Bx
=
d\Bx_a \wedge
d\Bx_b
=
da db (\Bx_a \wedge \Bx_b).
\end{equation}

The vector derivative, the projection of the gradient onto the surface at the point of integration (also called the tangent space), now has two components

\begin{dmath}\label{eqn:stokesTheoremCore:200}
\boldpartial
=
\sum_\mu \Bx^\mu (\Bx_\mu \cdot \spacegrad)
=
\Bx^a \PD{a}{}
+
\Bx^b \PD{b}{}
\equiv
\Bx^a \partial_a
+
\Bx^b \partial_b.
\end{dmath}

The Stokes integral can be evaluated over this volume element for either scalar fields \( \psi \) or vector fields \( \Bf \), and takes the form

\begin{subequations}
\label{eqn:stokesTheoremCore:140}
\begin{equation}\label{eqn:stokesTheoremCore:160}
\int_A d^2 \Bx \cdot (\boldpartial \wedge \psi) =
\int_A (d^2 \Bx \cdot \boldpartial) \psi
=
\int_{\partial A} d^1 \Bx \psi
\end{equation}
\begin{equation}\label{eqn:stokesTheoremCore:180}
\int_A d^2 \Bx \cdot (\boldpartial \wedge \Bf) =
\int_A (d^2 \Bx \cdot \boldpartial) \cdot \Bf
=
\int_{\partial A} d^1 \Bx \cdot \Bf.
\end{equation}
\end{subequations}

To extract the full meaning of this the boundary differential \( d^1 \Bx \) must be computed.  This has the same structure for a vector or scalar field

\begin{dmath}\label{eqn:stokesTheoremCore:220}
\begin{aligned}
\int_A d^2 \Bx \cdot (\boldpartial \wedge \Bf)
&=
\int_A (d^2 \Bx \cdot \boldpartial) \cdot \Bf \\
&=
\sum_\mu \int_A (d^2 \Bx \cdot \Bx^\mu) \cdot \partial_\mu \Bf \\
&=
\sum_\mu \int_A da db  \lr{ \Bx_a \wedge \Bx_b ) \cdot \Bx^\mu } \cdot \partial_\mu \Bf \\
&=
\sum_\mu \int_A da db  \lr{ \Bx_a {\delta_b}^\mu - \Bx_b {\delta_a}^\mu } \cdot \partial_\mu \Bf \\
&=
\int_A da db  \lr{ \Bx_a \cdot \PD{b}{ \Bf} - \Bx_b \cdot \PD{a}{\Bf} }
\end{aligned}
\end{dmath}

While \( \Bx_a, \Bx_b \) both depend on both parameters \( a, b \), the differential form immediately above is still a perfect integral in the variables of the partials since \( \Bx_a \) is computed with \( b \) held fixed, and \( \Bx_b \) is computed with \( a \) held fixed.  Proceeding with the integrals that match the respective partials, this gives

\begin{dmath}\label{eqn:stokesTheoremCore:240}
\int_A d^2 \Bx \cdot (\boldpartial \wedge \Bf)
=
\int
da \Bx_a \cdot \evalbar{\Bf}{\Delta b}
-\int
db \Bx_b \cdot \evalbar{\Bf}{\Delta a}
=
\int
d\Bx_a \cdot \evalbar{\Bf}{\Delta b}
-\int
d\Bx_b \cdot \evalbar{\Bf}{\Delta a}.
\end{dmath}

This shows that the boundary differential \( d^1 \Bx \) in \cref{eqn:stokesTheoremCore:140} is given by

\begin{dmath}\label{eqn:stokesTheoremCore:260}
d^1 \Bx = d\Bx_a - d\Bx_b,
\end{dmath}

where it is implied that the field in question is evaluated at the boundaries of the parameter that has been eliminated by this first integration.  This boundary integral can be interpretted as the integral around a contour, as indicated in
\cref{fig:twoParameterDifferentialBoundary:twoParameterDifferentialBoundaryFig2}.

\imageFigure{../figures/GAelectrodynamics/twoParameterDifferentialBoundaryFig2}{Contour for two parameter surface boundary.}{fig:twoParameterDifferentialBoundary:twoParameterDifferentialBoundaryFig2}{0.4}

Additionally, as with the single parameter case, a substitution of the gradient does not change the result, since any component of the gradient that lies outside of the tangent space on the surface has a zero dot product with the surface volume element \( d^2 \Bx \).
This allows the two parameter Stokes integrals to be written as

%\begin{dmath}\label{eqn:stokesTheoremCore:280}
\boxedEquation{eqn:stokesTheoremCore:280}{
\begin{aligned}
\int_A d^2 \Bx \cdot \spacegrad \psi &= \ointclockwise d\Bx \psi \\
\int_A d^2 \Bx \cdot (\spacegrad \wedge \Bf) &= \ointclockwise d\Bx \cdot \Bf.
\end{aligned}
}
%\end{dmath}

It can be shown that this two parameter Stokes integral is equivalent to Green's theorem.

         \subsection{Three parameter specialization of Stokes' theorem.}
            \input{../stokesTheorem/threeparameter.tex}
         \subsection{Using scalar volume elements}
            %
% Copyright © 2016 Peeter Joot.  All Rights Reserved.
% Licenced as described in the file LICENSE under the root directory of this GIT repository.
%

In \R{3} the area and volume elements of \cref{eqn:twoparameter:140}, and \cref{eqn:threeparameter:1481} can be reexpressed as scalars, recovering a number of the integral calculus identities that are more familiar than the wedge product variants above.

The pseudoscalar volume element can be written

\begin{dmath}\label{eqn:scalarVolumeElement:1621}
d^3 \Bx = I dV,
\end{dmath}

and the (oriented) area elements can be written as

\begin{dmath}\label{eqn:scalarVolumeElement:1641}
d^2 \Bx \ncap = I dA,
\end{dmath}

or
\begin{dmath}\label{eqn:scalarVolumeElement:1661}
d^2 \Bx = I \ncap dA.
\end{dmath}

For \( \psi \in \bigwedge^0, \Bf \in \bigwedge^1, B \in \bigwedge^2 \), this gives

\begin{subequations}
\label{eqn:scalarVolumeElement:1681}
\begin{equation}\label{eqn:scalarVolumeElement:1701}
I \int_A dA \ncap \wedge \spacegrad \psi = \ointclockwise d\Bx \psi
\end{equation}
\begin{equation}\label{eqn:scalarVolumeElement:1721}
I \int_A dA \ncap \wedge \spacegrad \wedge \Bf = \ointclockwise d\Bx \cdot \Bf
\end{equation}
\begin{equation}\label{eqn:scalarVolumeElement:1741}
\int_V dV \spacegrad \psi = \int_{\partial V} dA \ncap \psi
\end{equation}
\begin{equation}\label{eqn:scalarVolumeElement:1761}
\int_V dV \spacegrad \wedge \Bf = \int_{\partial V} dA \ncap \wedge \Bf
\end{equation}
\begin{equation}\label{eqn:scalarVolumeElement:1781}
\int dV \spacegrad \wedge B = \int_{\partial V} dA \ncap \wedge B
\end{equation}
\end{subequations}

It is straightforward to re-express all the wedge products above in their dual forms.
With \( B = I \Bf \), that is

\begin{subequations}
\label{eqn:scalarVolumeElement:1801}
\begin{equation}\label{eqn:scalarVolumeElement:1821}
\int_A dA \ncap \cross \spacegrad \psi = \ointctrclockwise d\Bx \psi
\end{equation}
\begin{equation}\label{eqn:scalarVolumeElement:1841}
\int_A dA \ncap \cdot (\spacegrad \cross \Bf) = \ointctrclockwise d\Bx \cdot \Bf
\end{equation}
\begin{equation}\label{eqn:scalarVolumeElement:1861}
\int_V dV \spacegrad \psi = \int_{\partial V} dA \ncap \psi
\end{equation}
\begin{equation}\label{eqn:scalarVolumeElement:1881}
\int_V dV \spacegrad \cross \Bf = \int_{\partial V} dA \ncap \cross \Bf
\end{equation}
\begin{equation}\label{eqn:scalarVolumeElement:1901}
\int dV \spacegrad \cdot \Bf = \int_{\partial V} dA \ncap \cdot \Bf.
\end{equation}
\end{subequations}

Each of the cross product terms above can also be put into dual forms, giving

\begin{subequations}
\label{eqn:scalarVolumeElement:1801c}
\begin{equation}\label{eqn:scalarVolumeElement:1821c}
\int_A dA \ncap \cdot \lr{ I \spacegrad \psi } = \ointclockwise d\Bx \psi
\end{equation}
\begin{equation}\label{eqn:scalarVolumeElement:1841c}
\int_A dA \ncap \cdot (\spacegrad \cdot B) = \ointctrclockwise d\Bx \cdot (I B)
\end{equation}
\begin{equation}\label{eqn:scalarVolumeElement:1881c}
\int_V dV \spacegrad \cdot B = \int_{\partial V} dA \ncap \cdot B.
\end{equation}
\end{subequations}

Note that all of
\cref{eqn:scalarVolumeElement:1861}, \cref{eqn:scalarVolumeElement:1901}, and \cref{eqn:scalarVolumeElement:1881c} all have the same form

%\begin{equation}\label{eqn:scalarVolumeElement:1881d}
\boxedEquation{eqn:scalarVolumeElement:1881d}{
\int_V dV \spacegrad \cdot A = \int_{\partial V} dA \ncap \cdot A.
}
%\end{equation}
\index{divergence theorem}

This is also true for pseudoscalar grades, which can be demonstrated by multiplying both sides of \cref{eqn:scalarVolumeElement:1741} with \( I \).
This implies that \cref{eqn:scalarVolumeElement:1881d} is valid for any \R{3} multivector, generalizing the conventional divergence theorem over a 3D volume to all spatial grades.

         \subsection{Problems}
            %
% Copyright � CCYY Peeter Joot.  All Rights Reserved.
% Licenced as described in the file LICENSE under the root directory of this GIT repository.
%
\makeproblem{Stokes' theorem relation to Green's theorem}{problem:stokesAndGreens:1}{
Show that Stokes' theorem, in its two parameter form, applied to a vector field recovers Green's theorem.
\index{Green's theorem}
\index{Stokes' theorem}
} % problem

\makeanswer{problem:stokesAndGreens:1}{

To demonstrate this, expand the LHS of the Stokes identity

\begin{dmath}\label{eqn:stokesAndGreens:20}
\int_A d^2 \Bx \cdot (\boldpartial \wedge \Bf) = \ointclockwise d\Bx \cdot \Bf.
\end{dmath}

Assuming \( u, v\) parameterization

\begin{dmath}\label{eqn:stokesAndGreens:40}
\int_A d^2 \Bx \cdot (\boldpartial \wedge \Bf)
=
\int_A (d\Bx_u \wedge d\Bx_v) \cdot (\boldpartial \wedge \Bf)
=
\int_A ((d\Bx_u \wedge d\Bx_v) \cdot \Bx^u) \cdot \partial_u \Bf
+
\int_A ((d\Bx_u \wedge d\Bx_v) \cdot \Bx^v) \cdot \partial_v \Bf
=
-\int_A du dv \Bx_v \cdot \partial_u \Bf
+
\int_A du dv \Bx_u \cdot \partial_v \Bf
=
-\int_A du dv \Bx_v \cdot \partial_u \Bf
+
\int_A du dv \lr{
-\Bx_v \cdot \partial_u \Bf
+
\Bx_u \cdot \partial_v \Bf
}.
\end{dmath}

The coordinate expansion of \( \Bf \) with respect to the tangent space coordinates is

\begin{dmath}\label{eqn:stokesAndGreens:60}
\Bf = \Bx^u f_u + \Bx^v f_v + \Bf_\perp
\end{dmath}

where \( \Bf_\perp \) lies in normal to the tangent space at the point in question.
Because \( \Bx_v \) is computed with \( u \) held fixed and \( \Bx_u \) computed with \( v \) held fixed, the area integrand can be written

\begin{dmath}\label{eqn:stokesAndGreens:80}
-\Bx_v \cdot \partial_u \Bf
+
\Bx_u \cdot \partial_v \Bf
=
-\PD{u}{}\lr{ \Bx_v \cdot \Bf }
+\PD{v}{}\lr{ \Bx_u \cdot \Bf }
=
-\PD{u}{f_v}
+\PD{v}{f_u},
\end{dmath}

which gives
\begin{dmath}\label{eqn:stokesAndGreens:100}
\int_A du dv \lr{ -\PD{u}{f_v}
+\PD{v}{f_u}
}
=
\ointclockwise d\Bx \cdot \Bf,
\end{dmath}

which recovers \cref{thm:stokesTheoremGeometricAlgebra:1660} as desired.
} % answer

            %
% Copyright © 2016 Peeter Joot.  All Rights Reserved.
% Licenced as described in the file LICENSE under the root directory of this GIT repository.
%

\makeproblem{\R{3} dual forms of Stokes' theorem.}{problem:stokesTheoremCoreProblems:1}{
Prove
\makesubproblem{}{problem:stokesTheoremCoreProblems:1:a}
\cref{eqn:scalarVolumeElement:1681},
\makesubproblem{}{problem:stokesTheoremCoreProblems:1:b}
\cref{eqn:scalarVolumeElement:1801},
\makesubproblem{}{problem:stokesTheoremCoreProblems:1:c}
and \cref{eqn:scalarVolumeElement:1801c}.
} % problem

\makeanswer{problem:stokesTheoremCoreProblems:1}{

The volume elements are
\makeSubAnswer{}{problem:stokesTheoremCoreProblems:1:a}
\begin{subequations}
\label{eqn:stokesTheoremCoreProblems:20}
\begin{dmath}\label{eqn:stokesTheoremCoreProblems:40}
d^2 \Bx \cdot \spacegrad
=
dA \gpgradeone{ I \ncap \spacegrad }
=
dA I \ncap \wedge \spacegrad
\end{dmath}
\begin{dmath}\label{eqn:stokesTheoremCoreProblems:60}
d^2 \Bx \cdot (\spacegrad \wedge \BA)
=
dA \gpgradezero{ I \ncap \spacegrad \BA }
=
dA I \ncap \wedge \spacegrad \wedge \BA
\end{dmath}
\begin{dmath}\label{eqn:stokesTheoremCoreProblems:80}
d^3 \Bx \cdot \spacegrad \phi
=
dV \gpgradetwo{ I \spacegrad \phi }
=
dV I \spacegrad \phi
\end{dmath}
\begin{dmath}\label{eqn:stokesTheoremCoreProblems:100}
d^3 \Bx \cdot (\spacegrad \wedge \BA)
=
dV \gpgradeone{ I (\spacegrad \wedge \BA) }
=
dV I \spacegrad \wedge \BA
\end{dmath}
\begin{dmath}\label{eqn:stokesTheoremCoreProblems:120}
d^3 \Bx \cdot (\spacegrad \wedge B)
=
dV \gpgradezero{ I (\spacegrad \wedge B) }
=
dV I \spacegrad \wedge B.
\end{dmath}
\end{subequations}

The corresponding boundary forms are
\begin{subequations}
\label{eqn:stokesTheoremCoreProblems:140}
\begin{equation}\label{eqn:stokesTheoremCoreProblems:160}
d\Bx \psi
\end{equation}
\begin{dmath}\label{eqn:stokesTheoremCoreProblems:180}
d\Bx \cdot \BA
\end{dmath}
\begin{dmath}\label{eqn:stokesTheoremCoreProblems:200}
d^2 \Bx \psi
=
dA I \ncap \psi
\end{dmath}
\begin{dmath}\label{eqn:stokesTheoremCoreProblems:220}
d^2 \Bx \cdot \BA
=
dA \gpgradeone{ I \ncap \BA }
=
dA I \ncap \wedge \BA
\end{dmath}
\begin{dmath}\label{eqn:stokesTheoremCoreProblems:240}
d^2 \Bx \cdot B
=
dA \gpgradezero{ I \ncap B }
=
dA I \ncap \wedge B.
\end{dmath}
\end{subequations}

Assembling these pieces back into the integrals proves the relationships.

\makeSubAnswer{}{problem:stokesTheoremCoreProblems:1:b}

To show \cref{eqn:scalarVolumeElement:1841} note that

\begin{dmath}\label{eqn:stokesTheoremCoreProblems:260}
I (\Ba \wedge \Bb \wedge \Bc)
=
\gpgradezero{ I \Ba \wedge \Bb \wedge \Bc }
=
\gpgradezero{ I \Ba (\Bb \wedge \Bc) -
I \Ba \cdot (\Bb \wedge \Bc)
}
=
\gpgradezero{ I \Ba I(\Bb \cross \Bc) }
=
- \Ba \cdot (\Bb \cross \Bc).
\end{dmath}

To show \cref{eqn:scalarVolumeElement:1901} note that

\begin{dmath}\label{eqn:stokesTheoremCoreProblems:280}
\Ba \wedge (I \BA)
=
\Ba \wedge (I \BA)
=
\gpgradethree{ \Ba I \BA }
=
\gpgradethree{ I \Ba \cdot \BA }
=
I (\Ba \cdot \BA).
\end{dmath}

\makeSubAnswer{}{problem:stokesTheoremCoreProblems:1:c}

For vector \( \Ba \), these transformations all follow from

\begin{dmath}\label{eqn:stokesTheoremCoreProblems:300}
\Ba \cross \Bf
=
\gpgradeone{ -I \Ba \wedge \Bf}
=
\gpgradeone{ -I \Ba \Bf}
=
-\gpgradeone{ \Ba I \Bf}
=
-\Ba \cdot (I \Bf)
=
\Ba \cdot B.
\end{dmath}

} % answer


      \section{Fundamental theorem of geometric calculus}
         \subsection{Fundamental Theorem of Geometric Calculus}
            %
% Copyright � 2016 Peeter Joot.  All Rights Reserved.
% Licenced as described in the file LICENSE under the root directory of this GIT repository.
%
%{
%\input{../blogpost.tex}
%\renewcommand{\basename}{fundamentalTheoremOfCalculus}
%\renewcommand{\dirname}{notes/phy1520/}
%%\newcommand{\dateintitle}{}
%%\newcommand{\keywords}{}
%
%\input{../peeter_prologue_print2.tex}
%
%\usepackage{peeters_layout_exercise}
%\usepackage{peeters_braket}
%\usepackage{peeters_figures}
%\usepackage{siunitx}
%
%\beginArtNoToc
%
%\generatetitle{Fundamental theorem of geometric calculus}
%\label{chap:fundamentalTheoremOfCalculus}

\subsection{Hypervolume integral}
We wish to generalize the concepts of line, surface and volume integrals to hypervolumes and multivector functions, and define a hypervolume integral as

\makedefinition{Multivector integral.}{dfn:fundamentalTheoremOfCalculus:240}{
Given a hypervolume parameterized by \( k \) parameters, k-volume volume element \( d^k \Bx \), and
multivector functions \( F, G \), a k-volume integral with the vector derivative acting to the right on \( G \) is written as
\begin{equation*}
\int_V d^k\Bx \rboldpartial G,
\end{equation*}
a k-volume integral with the vector derivative acting to the left on \( F \) is written as
\begin{equation*}
\int_V F d^k\Bx \lboldpartial,
\end{equation*}
and a k-volume integral with the vector derivative acting bidirectionally on \( F, G \) is written as
\begin{equation*}
\int_V F d^k\Bx \lrboldpartial G
\equiv
\int_V \lr{ F d^k\Bx \lboldpartial} G
+
\int_V F d^k\Bx \lr{ \rboldpartial G }.
\end{equation*}
The explicit meaning of these directional acting derivative operations is given by the chain rule coordinate expansion
\begin{dmath*}
F d^k \Bx \lrboldpartial G
=
F d^k \Bx \lr{ \sum_i \Bx^i {\stackrel{ \leftrightarrow }{\partial_i}} } G
=
(\partial_i F) d^k \Bx \sum_i \Bx^i G
+
F d^k \Bx \sum_i \Bx^i (\partial_i G)
\equiv
(F d^k \Bx \lboldpartial) G
+
F d^k \Bx (\rboldpartial G),
\end{dmath*}
with \( \boldpartial \) acting on \( F \) and \( G \), but not the volume element \( d^k \Bx \), which may also be a function of the implied parameterization.
} % definition

The vector derivative
% (and gradient)
may not commute with \( F, G \) nor the volume element \( d^k \Bx \), so we are forced to use some notation to indicate what the vector derivative (or gradient) acts on.
In conventional right acting cases, where there is no ambiguity, arrows will usually be omitted, but braces may also be used to indicate the scope of derivative operators.
This bidirectional notation will also be used for the gradient, especially for volume integrals in \R{3} where the vector derivative is identical to the gradient.

Some authors use the Hestenes dot notation, with overdots or primes to indicating the exact scope of multivector derivative operators, as in
\begin{dmath}\label{eqn:fundamentalTheoremOfCalculus:260}
\dot{F} d^k \Bx \dot{\boldpartial} \dot{G} =
\dot{F} d^k \Bx \dot{\boldpartial} G
+
F d^k \Bx \dot{\boldpartial} \dot{G}.
\end{dmath}
The dot notation has the advantage of emphasizing that the action of the vector derivative (or gradient) is on the functions \( F, G \), and not on the hypervolume element \( d^k \Bx \).
However, in this book, where primed operators such as \( \spacegrad' \) are used to indicate that derivatives are taken with respect to primed \( \Bx' \) variables, a mix of dots and ticks would have been confusing.
%Over arrows also have the advantage of being visually conspicuous.

\subsection{Fundamental theorem.}
\index{fundamental theorem of geometric calculus}

The fundamental theorem of geometric calculus is a generalization of many conventional scalar and vector integral theorems, and relates a hypervolume integral to its boundary.
This is a a powerful theorem, which we will use with Green's functions to solve Maxwell's equation, but also to derive the geometric algebra form of Stokes' theorem, from which most of the familiar integral calculus results follow.

\maketheorem{Fundamental theorem of geometric calculus}{thm:fundamentalTheoremOfCalculus:1}{
Given
multivectors \(F, G \),
a parameterization \( \Bx = \Bx(u_1, u_2, \cdots) \), with hypervolume element \( d^k \Bx = d^k u I_k \), where
\( I_k = \Bx_1 \wedge \Bx_2 \wedge \cdots \wedge \Bx_k \), the hypervolume integral is related to the boundary integral by
\begin{equation*}
\int_V F d^k \Bx \lrboldpartial G = \oint_{\partial V} F d^{k-1} \Bx G,
\end{equation*}
where \( \partial V \) represents the boundary of the volume, and \( d^{k-1} \Bx \) is the hypersurface element.  The hypersurface element and boundary integral is defined for \( k > 1 \) as
\begin{equation*}
\oint_{\partial V} F d^{k-1} \Bx G
\equiv
\sum_i \int d^{k-1} u_i \evalbar{ \lr{ F \lr{ I_k \cdot \Bx^i} G }}{\Delta u_i},
\end{equation*}
where \( d^{k-1} u_i \) is the product of all \( du_j \) except for \( du_i \).
For
\( k = 1 \) the hypersurface element and associated
boundary ``integral''
is really just convenient general shorthand, and
should be taken to mean the evaluation of the \( F G \) multivector product over the range of the parameter
\begin{equation*}
\oint_{\partial V} F d^{0} \Bx G
\equiv
\evalbar{ F G }{\Delta u_1}.
\end{equation*}
} % theorem

The geometry of the hypersurface element \( d^{k-1} \Bx \) will be made more clear when we
consider the specific cases of \( k = 1, 2, 3 \), representing generalized line, surface, and volume integrals respectively.
Instead of terrorizing the reader with a general proof
\cref{thm:fundamentalTheoremOfCalculus:1},
which requires some unpleasant index gymnastics,
this book
will separately state and prove the fundamental theorem of calculus
for each of the \( k = 1, 2, 3 \) cases that are of interest for problems in \R{2} and \R{3}.
For the interested reader, a sketch of the general proof
of \cref{thm:fundamentalTheoremOfCalculus:1}
is available in \cref{chap:gagcProof}.

Before moving on to the line, surface, and volume integral cases, we will state and prove the
general Stokes' theorem in its geometric algebra form.

%}
%\EndArticle

         \subsection{Green's function for the gradient in Euclidean spaces.}
            %\input{../gabookI/calculus/gradientGreensFunction.tex}
            %
% Copyright � 2016 Peeter Joot.  All Rights Reserved.
% Licenced as described in the file LICENSE under the root directory of this GIT repository.
%
\index{Green's function}

\subsection{Definition.}
Green's functions allow for convolution solution of linear differential operators.  Given an operator

\begin{equation}\label{eqn:gradientGreensFunctionEuclidean:20}
L' F(\Bx') = M(\Bx'),
\end{equation}

the operator \( L \) can be inverted, yielding a solution
\begin{equation}\label{eqn:gradientGreensFunctionEuclidean:40}
F(\Bx) = \int_V dV' G(\Bx, \Bx') M(\Bx'),
\end{equation}

where \( G(\Bx, \Bx') \) is called the Green's function.

\subsection{Green's functions for electrodynamics.}

There are a few Green's functions of interest for electrodynamics, the Green's function for the Laplacian\( \spacegrad^2 \) , the gradient \( \spacegrad \) , the Helmholtz operator \(\spacegrad^2 + k^2 \), and a for the (linear) factors of the Helmholtz operator \( \spacegrad \pm j k \).  The Green's functions for the operators that are linear in the gradient are multivector functions and will be applied to multivector functions.

\index{Laplacian!Green's function}
\index{Green's function!Laplacian representation}

\maketheorem{Laplacian Green's function.}{thm:gradientGreensFunctionEuclidean:60}{

The Green's function for the \R{3} Laplacian is
\begin{equation*}
G(\Bx, \Bx') =
-\inv{4 \pi} \inv{\Norm{\Bx - \Bx'}}.
\end{equation*}
} % theorem

This representation is proven in \cref{chap:greensFunctionLaplacian}.

\index{gradient!Green's function}
\index{Green's function!gradient representation}
%The Green's function for the Laplacian of the Green's function is well known, but there are advantages to utilizing the Green's function for first order gradient equations directly.

\maketheorem{Green's function for the gradient}{thm:gradientGreensFunctionEuclidean:1}{

%A sufficient convergence constraint for \( F \) over the infinite sphere is
For a
multivector function \( F \) in a Euclidean space that tends to zero at infinity at least as fast as
\begin{equation*}
\lim_{\Bx' \rightarrow \infty} \frac{\Norm{F(\Bx')}}{\Norm{\Bx - \Bx'}^{n-1}} \rightarrow 0,
\end{equation*}

the
first order multivector gradient equation
\begin{equation*}
\spacegrad' F(\Bx') = M(\Bx'),
\end{equation*}

has an inverse
\begin{equation*}
F(\Bx) = \int_V dV' G(\Bx,\Bx') M(\Bx').
\end{equation*}

Here \( G \) is the Green's function for the gradient over an infinite spherical boundary, satisfying
\begin{equation*}
   \spacegrad G = \spacegrad \cdot G = \delta(\Bx - \Bx'),
\end{equation*}

and is given by
\begin{equation*}
   G(\Bx, \Bx') = \inv{S_n} \frac{\Bx - \Bx'}{\Norm{\Bx-\Bx'}^n}.
\end{equation*}

Here \( n \) is the dimension of the space, \( S_n \) is the area of the unit sphere.
} % theorem

Conventions for the sign of Green's functions or the parameters in the convolution integral often vary.
The sign convention used here is that of \citep{doran2003gap}.

%This result applies not only to gradient equations in Euclidean spaces, but also to multivector (or even just vector) fields \( F \), instead of the usual scalar functions that we usually apply Green's functions to.
Note that the statement that
\( \spacegrad G = \spacegrad \cdot G \), or \( \rspacegrad G = G \lspacegrad \), implies that the curl of the Green's function must be zero.

The proof of this theorem will be left to \cref{chap:gradientGreensFunctionProof}.

\maketheorem{Green's function for the Helmholtz operator.}{thm:gradientGreensFunctionEuclidean:3}{
The Green's function for the Helmholze operator \( \spacegrad^2 + k^2 \) is
\begin{equation*}
G(\Bx, \Bx') = -\frac{e^{\pm j k \Norm{ \Bx - \Bx' } }}{ 4 \pi \Norm{\Bx - \Bx'}}.
\end{equation*}
With a time harmonic dependence \( e^{j \omega t} \) we want the negative sign variation for causal solutions, or
%\( r = \Bx - \Bx' \), that is
\begin{equation*}
G(\Bx, \Bx') = -\frac{e^{-j k \Norm{\Bx - \Bx'}}}{ 4 \pi \Norm{\Bx - \Bx'}}.
\end{equation*}
} % theorem

%}

         %\subsection{Problems}
         \subsection{Helmholtz theorem}
            %
% Copyright © 2016 Peeter Joot.  All Rights Reserved.
% Licenced as described in the file LICENSE under the root directory of this GIT repository.
%
\index{Helmholtz's theorem}
In conventional electromagnetism Maxwell's equations are posed in terms of separate divergence and curl equations.  It is therefore desirable to show that the divergence and curl of a function and it's normal characteristics on the boundary of an integraion volume determine that function uniquely.  This is known as the Helmholtz theorem
\maketheorem{Helmholtz first theorem.}{thm:helmholtzDerviationMultivectorStatement:1}{
A vector \( \BM \) is uniquely determined by its
divergence
\begin{equation*}
\spacegrad \cdot \BM = s,
\end{equation*}
and curl
\begin{equation*}
\spacegrad \cross \BM = \BC,
\end{equation*}
and its value
over the boundary.
} % theorem

%It could be argued that Helmholtz's theorem is irrelavent when using the GA formalism, since we consolidate the separate divergence and curl equations into one gradient operator.
%We include a proof here regardless, since it can be performed in a compact and interesting fashion using
%%the fundamental theorem of geometric calculus
%\cref{thm:fundamentalTheoremOfCalculus:1}.

            %
% Copyright © 2016 Peeter Joot.  All Rights Reserved.
% Licenced as described in the file LICENSE under the root directory of this GIT repository.
%
%{
The conventional proof of Helmholtz's theorem uses the Green's function for the (second order) Helmholtz operator.
Armed with a vector valued Green's function for the gradient, a first order proof is also possible.
As illustrations of the geometric integration theory developed in this chapter, both
strategies will be applied here to this problem.

In either case, we start by forming an even grade multivector (gradient) equation containing both the dot and cross product contributions
\begin{equation}\label{eqn:helmholtzDerviationMultivectorSolution:60}
\spacegrad \BM
= \spacegrad \cdot \BM + I \spacegrad \cross \BM
= s + I \BC.
\end{equation}

\paragraph{First order proof.}

For the first order case, we
perform a grade one selection of \cref{lemma:greensFunctionOverview:420}, setting
\( F = \BM \) where \( G \) is the Green's function for the gradient given by
\cref{eqn:greensFunctionFirstOrderHelmholtz:900}.  The proof follows directly

\begin{equation}\label{eqn:helmholtzDerviationMultivectorSolution:820}
\begin{aligned}
M(\Bx)
&= - \int_V \lr{ G(\Bx, \Bx') \lspacegrad' } \BM(\Bx') dV' \\
&= \int_V \gpgradeone{G(\Bx, \Bx') \lr{ \rspacegrad' \BM(\Bx') }} dV'
-
\int_{\partial V} \gpgradeone{ G(\Bx, \Bx') \ncap' \BM(\Bx') } dA' \\
&=
\int_V
\inv{4 \pi \Norm{\Bx - \Bx'}^3 }
\gpgradeone{ (\Bx - \Bx') \lr{ s(\Bx') + I \BC(\Bx') }} dV' \\
&\quad -
\int_{\partial V}
\inv{4 \pi \Norm{\Bx - \Bx'}^3 }
\gpgradeone{ (\Bx - \Bx') \ncap' \BM(\Bx') } dA' \\
&=
\int_V
\inv{4 \pi \Norm{\Bx - \Bx'}^3 }
\lr{ (\Bx - \Bx') s(\Bx') - (\Bx - \Bx') \cross \BC(\Bx') } dV' \\
&\quad -
\int_{\partial V}
\inv{4 \pi \Norm{\Bx - \Bx'}^3 }
\gpgradeone{ (\Bx - \Bx') \ncap' \BM(\Bx') } dA'.
\end{aligned}
\end{equation}
If \( \BM \) is well behaved enough that the boundary integral vanishes on an infinite surface, we see that \( \BM \) is completely specified by the divergence and the curl.
In general, the divergence and the curl, must also be supplemented by the value of vector valued function on the boundary.

Observe that the boundary integral has a particularly simple form for a spherical surface or radius \( R \) centered on \( \Bx' \).
Switching to spherical coordinates \( \Br = \Bx' - \Bx = R\, \rcap(\theta, \phi) \) where \( \rcap = (\Bx' - \Bx)/\Norm{\Bx' - \Bx} \) is the outwards normal, we have
\begin{equation}\label{eqn:helmholtzDerviationMultivectorSolution:840}
\begin{aligned}
-
\int_{\partial V} &
\inv{4 \pi \Norm{\Bx - \Bx'}^3 }
\gpgradeone{ (\Bx - \Bx') \ncap' \BM(\Bx') } dA' \\
&= \int_{\partial V} \frac{\BM(\Bx')}{4 \pi \Norm{\Bx - \Bx'}^2 } dA' \\
&= \inv{4\pi} \int_{\theta = 0}^\pi \int_{\phi = 0}^{2 \pi} \BM(R, \theta, \phi) \sin\theta d\theta d\phi.
\end{aligned}
\end{equation}
This is an average of \( \BM \) over the surface of the radius-\(R\) sphere surrounding the point \( \Bx \) where the field \( \BM \) is evaluated.

\paragraph{Second order proof.}

%Observe that the Laplacian of \( \BM \) is vector valued
%
%\begin{equation}\label{eqn:helmholtzDerviationMultivectorSolution:760}
%\spacegrad^2 \BM = \spacegrad s + I \spacegrad \BC.
%\end{equation}
%
%This means that \( \spacegrad \BC \) must be a bivector \( \spacegrad \BC = \spacegrad \wedge \BC \), or that \( \BC \) has zero divergence
%
%\begin{equation}\label{eqn:helmholtzDerviationMultivectorSolution:780}
%\spacegrad \cdot \BC = 0.
%\end{equation}

Again, we use \cref{eqn:helmholtzDerviationMultivectorSolution:60}
to discover the relation between the vector \( \BM \) and its divergence and curl.
\index{delta function}
The vector \( \BM \) can be expressed at the point of interest as a convolution with the delta function at all other points in space
\index{convolution}
\begin{equation}\label{eqn:helmholtzDerviationMultivectorSolution:80}
\BM(\Bx) = \int_V dV'\, \delta(\Bx - \Bx') \BM(\Bx').
\end{equation}

\index{Laplacian}
The Laplacian representation of the delta function in \R{3} is
\begin{equation}\label{eqn:helmholtzDerviationMultivectorSolution:100}
\delta(\Bx - \Bx') = -\inv{4\pi} \spacegrad^2 \inv{\Norm{\Bx - \Bx'}},
\end{equation}
so \( \BM \) can be represented as the following convolution
\begin{equation}\label{eqn:helmholtzDerviationMultivectorSolution:120}
\BM(\Bx) = -\inv{4\pi} \int_V dV'\, \spacegrad^2 \inv{\Norm{\Bx - \Bx'}} \BM(\Bx').
\end{equation}

%As noted in \cref{eqn:helmholtzDerviationMultivector:460} the Laplacian of a vector can be factored as
%
%\begin{equation}\label{eqn:helmholtzDerviationMultivectorSolution:140}
%\spacegrad^2 \Ba
%=
%\spacegrad (\spacegrad \cdot \Ba)
%-
%\spacegrad \cross (\spacegrad \cross \Ba).
%\end{equation}
%
%Note that the last term can be written in cross product notation using \( \Bc \cdot (\Ba \wedge \Bb) = -\Bc \cross (\Ba \cross \Bb) \) if desired.

Using this relation and proceeding with a few applications of the chain rule, plus the fact that \( \spacegrad 1/\Norm{\Bx - \Bx'} = -\spacegrad' 1/\Norm{\Bx - \Bx'} \), we find
%
%I previously posted a Geometric Algebra attack on the Helmholtz theorem.  Here is
%
%Here's a third way of deriving the Helmholtz theorem inversion relation.  This is a refinement of the traditional vector algebra solution that led to \cref{eqn:helmholtzDerviationMultivector:200}, that uses a factorization of the Laplacian directly, deferring any expansion in terms of dot and cross (or wedge) products until the very end.
%
%Starting from the first line of \cref{eqn:helmholtzDerviationMultivector:160}, we have
\begin{equation}\label{eqn:helmholtzDerviationMultivectorSolution:720}
\begin{aligned}
-4 &\pi \BM(\Bx) \\
&= \int_V dV'\, \spacegrad^2 \inv{\Norm{\Bx - \Bx'}} \BM(\Bx') \\
&= \gpgradeone{\int_V dV'\, \spacegrad^2 \inv{\Norm{\Bx - \Bx'}} \BM(\Bx')} \\
&= -\gpgradeone{\int_V dV'\, \spacegrad \lr{ \spacegrad' \inv{\Norm{\Bx - \Bx'}}} \BM(\Bx')} \\
&= -\gpgradeone{\spacegrad \int_V dV' \lr{
   \spacegrad' \frac{\BM(\Bx')}{\Norm{\Bx - \Bx'}}
   -\frac{\spacegrad' \BM(\Bx')}{\Norm{\Bx - \Bx'}}
   } } \\
&= -\gpgradeone{\spacegrad \int_{\partial V} dA'\,
   \ncap \frac{\BM(\Bx')}{\Norm{\Bx - \Bx'}}
    }
   +\gpgradeone{\spacegrad \int_V dV'
   \frac{s(\Bx') + I\BC(\Bx')}{\Norm{\Bx - \Bx'}}
    } \\
&= -\gpgradeone{\spacegrad \int_{\partial V} dA'\,
   \ncap \frac{\BM(\Bx')}{\Norm{\Bx - \Bx'}}
    }
   +\spacegrad \int_V dV'\,
   \frac{s(\Bx')}{\Norm{\Bx - \Bx'}}
   +\spacegrad \cdot \int_V dV'
   \frac{I\BC(\Bx')}{\Norm{\Bx - \Bx'}}.
\end{aligned}
\end{equation}

By inserting a no-op grade selection operation in the second step, the trivector terms that would show up in subsequent steps are automatically filtered out.
%the troublesome trivector term that shows up in my first purely Geometric Algebra
%attempt is eliminated.
This leaves us with a boundary term dependent on the surface and the normal and tangential components of \( \BM \).
Added to that is a pair of volume integrals that provide the unique dependence of \( \BM \) on its divergence and curl.
When the surface is taken to infinity, which requires \( \Norm{\BM}/\Norm{\Bx - \Bx'} \rightarrow 0 \), then the dependence of \( \BM \) on its divergence and curl is unique.

In order to express final result in traditional vector algebra form, a couple transformations are required.
The first is that
\begin{equation}\label{eqn:helmholtzDerviationMultivectorSolution:800}
\gpgradeone{ \Ba I \Bb } = I^2 \Ba \cross \Bb = -\Ba \cross \Bb.
\end{equation}

For the grade selection in the boundary integral, note that
\begin{equation}\label{eqn:helmholtzDerviationMultivectorSolution:740}
\begin{aligned}
\gpgradeone{ \spacegrad \ncap \BX }
&= \gpgradeone{ \spacegrad (\ncap \cdot \BX) } + \gpgradeone{ \spacegrad (\ncap \wedge \BX) } \\
&= \spacegrad (\ncap \cdot \BX) + \gpgradeone{ \spacegrad I (\ncap \cross \BX) } \\
&= \spacegrad (\ncap \cdot \BX) - \spacegrad \cross (\ncap \cross \BX).
\end{aligned}
\end{equation}

These give
%\begin{equation}\label{eqn:helmholtzDerviationMultivectorSolution:721}
\boxedEquation{eqn:helmholtzDerviationMultivectorSolution:721}{
\begin{aligned}
\BM(\Bx)
&=
\spacegrad \inv{4\pi} \int_{\partial V} dA'\, \ncap \cdot \frac{\BM(\Bx')}{\Norm{\Bx - \Bx'}}
-
\spacegrad \cross \inv{4\pi} \int_{\partial V} dA'\, \ncap \cross \frac{\BM(\Bx')}{\Norm{\Bx - \Bx'}} \\
&-\spacegrad \inv{4\pi} \int_V dV'
\frac{s(\Bx')}{\Norm{\Bx - \Bx'}}
+\spacegrad \cross \inv{4\pi} \int_V dV'
\frac{\BC(\Bx')}{\Norm{\Bx - \Bx'}}.
\end{aligned}
}
%\end{equation}
%}

      \section{Problem solutions}
         \shipoutAnswer

\part{Electromagnetism}
   \chapter{Statics}
      \makeexample{Electrostatic and magnetostatics.}{example:vectorproduct:electrostatics}{

With no magnetic current, no magnetic sources, and no time derivatives, Maxwell's equations in simple media take the form

\begin{dmath}\label{eqn:vectorproduct:120}
\begin{aligned}
\spacegrad \cdot \BB &= 0 \\
\spacegrad \cross \BB &= \mu \BJ \\
\spacegrad \cross \BE &= 0 \\
\spacegrad \cdot \BE &= \frac{\rho}{\epsilon}.
\end{aligned}
\end{dmath}

For electrostatic conditions \( \BJ = 0 \), so using \cref{eqn:vectorproduct:100} the first and last equations can be combined into a single first order homogeneous multivector gradient equation

\begin{equation}\label{eqn:vectorproduct:140}
\spacegrad \BB
=
\spacegrad \cdot \BB +I (\spacegrad \cross \BB )
=
0.
\end{equation}

The electric gradient equation is

\begin{equation}\label{eqn:vectorproduct:160}
\spacegrad \BE
=
\spacegrad \cdot \BE +I (\spacegrad \cross \BE )
=
\frac{\rho}{\epsilon}.
\end{equation}

Maxwell's equations are reduced to two multivector equations with this transformation
\begin{dmath}\label{eqn:vectorproduct:180}
\begin{aligned}
\spacegrad \BE &= \frac{\rho}{\epsilon} \\
\spacegrad \BB &= 0.
\end{aligned}
\end{dmath}

For magnetostatics \( \rho = 0 \), and the same assembly of Maxwell's equations gives

\begin{dmath}\label{eqn:vectorproduct:220}
\begin{aligned}
\spacegrad \BB &= I \mu \BJ \\
\spacegrad \BE &= 0.
\end{aligned}
\end{dmath}

It will be seen later that it is actually more natural to express magnetic fields as a bivector \( I \BB \).  Using \( I^2 = -1 \) (\cref{problem:gradeselection:R3PseudoscalarSquare}) the magnetostatic equation takes the form

\begin{dmath}\label{eqn:vectorproduct:240}
\spacegrad (I \BB) = - \mu \BJ.
\end{dmath}

Both the electrostatic and magnetostatic equations can be solved directly using the Green's function for the gradient, producing the Coulomb integral for the electric field and Biot-Savart's law for the magnetic field.
Before demonstrating this, the concepts required to attack multivector integrals must be formulated.
} % example
 % find a proper home for this
   \chapter{Maxwell's equations}
      %
% Copyright © 2016 Peeter Joot.  All Rights Reserved.
% Licenced as described in the file LICENSE under the root directory of this GIT repository.
%
\section{Conventional differential form}

The differential form of Maxwell's equations, with extensions for magnetic sources, is the starting point for all the analysis in these notes.  Those equations are

\input{../ece1229-antenna/MaxwellsStatement.tex}

The magnetic sources can be considered fictional, and are included because they are useful in antenna theory to model real phenomina such as infinitesimal current loops.

\input{../ece1229-antenna/MaxwellsFieldAndSourceDescription.tex}

These fields and sources are all real valued.  In many situations it will be desirable to work with a time harmonic (frequency-domain phasor) form of Maxwell's equations.  In engineering, a time harmonic representation presumes that all sources and fields have a frequency dependence of the form
\index{time harmonic}

\begin{dmath}\label{eqn:maxwellsEquations:20}
\bcY(\Bx, t) = \Real( \BY(\Bx, \omega) e^{j\omega t} ),
\end{dmath}

where the field (or source) \( \BY(\Bx, \Bomega) \) is allowed to be complex valued, whereupon Maxwell's equations take the form

\input{../ece1229-antenna/MaxwellsTimeHarmonic.tex}

Note that the time harmonic convention typically used in physics literature presumes a frequency dependence of the form

\begin{dmath}\label{eqn:maxwellsEquations:40}
\bcY(\Bx, t) = \Real( \BY(\Bx, \omega) e^{-i\omega t} ),
\end{dmath}

which alters the sign of any imaginary originating from a time derivative.  Care is required by the reader to understand which form of frequency dependence has been assumed.

\section{GA differential form}

Geometric Algebra admits a number of alternative representations of Maxwell's equations.  The first follows from expressing the cross products all as wedge products, leaving a pair of bivector and a pair of scalar equations

\begin{subequations}
\begin{dmath}\label{eqn:maxwellsEquations:60}
\spacegrad \wedge \bcE = - I \bcM - \PD{t}{I\bcB}
\end{dmath}
\begin{dmath}\label{eqn:maxwellsEquations:80}
\spacegrad \wedge \bcH = I \bcJ + I \PD{t}{\bcD}
\end{dmath}
\begin{dmath}\label{eqn:maxwellsEquations:100}
\spacegrad \cdot \bcD = q_\txte
\end{dmath}
\begin{dmath}\label{eqn:maxwellsEquations:120}
\spacegrad \cdot \bcB = q_\txtm.
\end{dmath}
\end{subequations}

Alternatively, the duality transformation \( \Ba \wedge \Bb = -I \Ba \cdot (I \Bb) \) allows Maxwell's equations to be all written as dot products

\begin{subequations}
\begin{dmath}\label{eqn:maxwellsEquations:140}
\spacegrad \cdot (I \bcE) = \bcM + \PD{t}{\bcB}
\end{dmath}
\begin{dmath}\label{eqn:maxwellsEquations:160}
\spacegrad \cdot (I \bcH) = -\bcJ - \PD{t}{\bcD}
\end{dmath}
\begin{dmath}\label{eqn:maxwellsEquations:180}
\spacegrad \cdot \bcD = q_\txte
\end{dmath}
\begin{dmath}\label{eqn:maxwellsEquations:200}
\spacegrad \cdot \bcB = q_\txtm,
\end{dmath}
\end{subequations}

or, using the duality transformation \( \Ba \cdot \Bb = -I (\Ba \wedge (I \Bb) \), Maxwell's equations can all be written as wedge products

\begin{subequations}
\begin{dmath}\label{eqn:maxwellsEquations:220}
\spacegrad \wedge \bcE = - I \bcM - \PD{t}{I\bcB}
\end{dmath}
\begin{dmath}\label{eqn:maxwellsEquations:240}
\spacegrad \wedge \bcH = I \bcJ + I \PD{t}{\bcD}
\end{dmath}
\begin{dmath}\label{eqn:maxwellsEquations:260}
\spacegrad \wedge (I\bcD) = I q_\txte
\end{dmath}
\begin{dmath}\label{eqn:maxwellsEquations:280}
\spacegrad \wedge (I\bcB) = I q_\txtm.
\end{dmath}
\end{subequations}

Each of these forms can be useful in different circumstances, however the real power of GA in electromagnetism follows from presuming constituative relationships between the pairs of electric and magnetic fields

\begin{subequations}
\label{eqn:maxwellsEquations:300}
\begin{dmath}\label{eqn:maxwellsEquations:320}
\bcB = \mu \bcH
\end{dmath}
\begin{dmath}\label{eqn:maxwellsEquations:340}
\bcD = \epsilon \bcE,
\end{dmath}
\end{subequations}

where \( \epsilon \) is the permitivitity of the medium [\si{F/m}] (Farads/meter), and \( \mu \) is the permeability of the medium [\si{H/m}] (Henries/meter).
The permitivitity and permeability may be functions of both time and position, and model the materials that the fields are propagating through.  In general, the these may be non-isotropic tensor operators, however, unless otherwise specified, isotropic media will be assumed in these notes.

With this constitutative relationship assumed (and a bit of rescaling), the dot and wedge products of \cref{eqn:maxwellsEquations:60}, \cref{eqn:maxwellsEquations:100} can be added, as can those of \cref{eqn:maxwellsEquations:80}, \cref{eqn:maxwellsEquations:120}.  This reduces Maxwell's equations to a pair of first order coupled gradient equations

\begin{subequations}
\begin{dmath}\label{eqn:maxwellsEquations:360}
\spacegrad \bcE = \inv{\epsilon} q_\txte - I \bcM - \mu \PD{t}{(I\bcH)}
\end{dmath}
\begin{dmath}\label{eqn:maxwellsEquations:380}
\spacegrad (I \bcH) = \frac{I q_\txtm}{\mu} - \bcJ - \epsilon \PD{t}{\bcE}.
\end{dmath}
\end{subequations}



      \section{Potentials}
         %
% Copyright � 2017 Peeter Joot.  All Rights Reserved.
% Licenced as described in the file LICENSE under the root directory of this GIT repository.
%
%{
\index{potential}
\index{multivector potential}

For both electrostatics and magnetostatics, where Maxwell's equations are both a pair of gradients, we were able to require that the respective scalar and vector potentials were both gradients.
For electrodynamics where Maxwell's equation is
\begin{dmath}\label{eqn:potentialSection:1800}
\lr{ \spacegrad + \inv{c} \PD{t}{} } F = J,
\end{dmath}
it seems more reasonable to demand a different structure of the potential, say
\begin{dmath}\label{eqn:potentialSection:1820}
F = \lr{ \spacegrad - \inv{c} \PD{t}{} } A,
\end{dmath}
where \( A \) is a multivector potential that may contain all grades, with structure to be determined.
If such a multivector potential can be found, then Maxwell's equation is reduced to a single wave equation
\begin{dmath}\label{eqn:potentialSection:1840}
\lr{ \spacegrad^2 - \inv{c^2} \PDSq{t}{} } A = J,
\end{dmath}
which can be thought of as one wave equation for each multivector grade of the multivector source \( J \).

Some thought shows that the guess \cref{eqn:potentialSection:1820} is not quite right, as it allows for the invalid possibility that \( F \) has scalar or pseudoscalar grades.
While it is possible to impose constraints (a gauge choice) on potential \( A \) that ensure
\( F \) has only the vector and bivector grades,
in general,
a grade selection filter must be imposed
\boxedEquation{eqn:potentialSection:1860}{
F
=
\gpgrade{ \lr{ \spacegrad - \inv{c} \PD{t}{} } A }{1,2}.
}

(cut)


      \section{Boundary value conditions}
         %
% Copyright © 2017 Peeter Joot.  All Rights Reserved.
% Licenced as described in the file LICENSE under the root directory of this GIT repository.
%
%{
\index{boundary values}
%
% Copyright � 2018 Peeter Joot.  All Rights Reserved.
% Licenced as described in the file LICENSE under the root directory of this GIT repository.
%
\maketheorem{Boundary value relations.}{thm:boundarySurfaceSources:480}{
The difference in the normal and tangential components of the electromagnetic field spanning a surface on which there are
a surface current or surface charge or current densities \( J_\txte = J_{\textrm{es}} \delta(n), J_\txtm = J_{\textrm{ms}} \delta(n) \)
can be related to those surface sources as follows
%\label{eqn:boundarySurfaceSources:420}
\begin{equation*}
\begin{aligned}
\gpgrade{\ncap (F_2 - F_1) }{0,1} &= J_{\textrm{es}} \\
\gpgrade{\ncap (G_2 - G_1) }{2,3} &= I J_{\textrm{ms}},
\end{aligned}
\end{equation*}
where \( F_k = \BD_k + I \BH_k/c, G_k = \BE_k + I c \BB_k, k = 1,2 \) are the fields in the
where \( \ncap = \ncap_2 = -\ncap_1 \) is the outwards facing normal in the second medium.
In terms of the conventional constituent fields, these may be written
%\label{eqn:boundarySurfaceSources:460}
\begin{equation*}
\begin{aligned}
\ncap \cdot \lr{ \BD_2 - \BD_1 } &= \rho_\txts \\
\ncap \cross \lr{ \BH_2 - \BH_1 } &= \BJ_\txts \\
\ncap \cdot \lr{ \BB_2 - \BB_1 } &= \rho_{\textrm{ms}} \\
\ncap \cross \lr{ \BE_2 - \BE_1 } &= -\BM_\txts.
\end{aligned}
\end{equation*}
} % theorem


\Cref{fig:ps3Problem1Pillbox:ps3Problem1PillboxFig1} illustrates a surface where we seek to find the fields above the surface (region 2), and below the surface (region 1).
These fields will be determined by integrating Maxwell's equation over the pillbox configuration, allowing the height \( n \) of that pillbox above or below the surface to tend to zero,
and the area of the pillbox top to also tend to zero.
\pmathImageFigure{../figures/GAelectrodynamics/}{pillboxIntegrationVolumeFig1}{Pillbox integration volume.}{fig:ps3Problem1Pillbox:ps3Problem1PillboxFig1}{0.3}{pillboxIntegrationVolumeFig1.nb}

\begin{proof}
We will work with \cref{thm:dielectric:20}, Maxwell's equations in media, in their frequency domain form
\begin{equation}\label{eqn:boundarySurfaceSources:480}
\begin{aligned}
\gpgrade{ \spacegrad F }{0,1} + j k \BD &= J_{\textrm{es}} \delta(n) \\
\gpgrade{ \spacegrad G }{2,3} + j k I c \BB &= I J_{\textrm{ms}} \delta(n),
\end{aligned}
\end{equation}
and integrate these over the pillbox volume in the figure.  That is
\begin{equation}\label{eqn:boundarySurfaceSources:500}
\begin{aligned}
\int dV\, \gpgrade{ \spacegrad F }{0,1} + j k \int dV\, \BD &= \int dn dA\, J_{\textrm{es}} \delta(n) \\
\int dV\, \gpgrade{ \spacegrad G }{2,3} + j k I c \int dV\, \BB &= I \int dn dA\, J_{\textrm{ms}} \delta(n).
\end{aligned}
\end{equation}
The gradient integrals can be evaluated with \cref{thm:volumeintegral:100}.  Evaluating the delta functions picks leaves an area integral on the surface.  Additionally, we assume that we are making the pillbox volume small enough that we can employ the mean value theorem for the \( \BD, \BB \) integrals
\begin{equation}\label{eqn:boundarySurfaceSources:520}
\begin{aligned}
\int_{\partial V} dA\, \gpgrade{ \ncap F }{0,1} + j k \Delta A \lr{ n_1 \tilde{\BD}_1 + n_2 \tilde{\BD}_2 } &= \Delta A J_{\textrm{es}} \\
\int_{\partial V} dA\, \gpgrade{ \ncap G }{2,3} + j k I c \Delta A \lr{ n_1 \tilde{\BB}_1 + n_2 \tilde{\BB}_2} &= I \Delta A J_{\textrm{ms}}.
\end{aligned}
\end{equation}
We now let \( n_1, n_2 \) tend to zero, which kills off the \( \BD, \BB \) contributions, and also kills off the side wall contributions in the first pillbox surface integral.  This leaves
\begin{equation}\label{eqn:boundarySurfaceSources:540}
\begin{aligned}
\gpgrade{ \ncap_2 F_2 }{0,1} + \gpgrade{ \ncap_1 F_1 }{0,1} &= J_{\textrm{es}} \\
\gpgrade{ \ncap_2 G_2 }{2,3} + \gpgrade{ \ncap_1 G_1 }{2,3} &= J_{\textrm{ms}}.
\end{aligned}
\end{equation}
Inserting \( \ncap = \ncap_2 = -\ncap_1 \) completes the first part of the proof.

Expanding the grade selection operations, we find
\begin{equation}\label{eqn:boundarySurfaceSources:440}
\begin{aligned}
\ncap \cdot (\BD_2 - \BD_1) &= \rho_s \\
I \ncap \wedge \lr{ \BH_2/c - \BH_1/c } &= - \BJ_s/c \\
\ncap \wedge (\BE_2 - \BE_1) &= -I \BM_s \\
I c \ncap \cdot (\BB_2 - \BB_1) &= I c \rho_{ms},
\end{aligned}
\end{equation}
and expansion of the wedge's as cross's using \cref{eqn:SimpleProducts2:1620} completes the proof.
\end{proof}
%It is somewhat remarkable that the
%crazy jumble of dot products, cross products and field components in the conventional statement of the boundary conditions, follows directly from the evaluation of the product of the normal with the multivector fields.

In the special case where there are surface charge and current densities along the interface surface, but the media is uniform (\(\epsilon_1 = \epsilon_2, \mu_1 = \mu_2\)), then the field and current relationship has a particularly simple form \citep{chappell2014geometric}
\begin{equation}\label{eqn:boundarySurfaceSources:421}
\ncap (F_2 - F_1) = J_s.
\end{equation}

\makeproblem{Uniform media with currents and densities.}{problem:boundarySurfaceSources:1}{
Prove that \cref{eqn:boundarySurfaceSources:421} holds when \( \epsilon_1 = \epsilon_2, \mu_1 = \mu_2 \).
} % problem
%}

      \section{Problem solutions}
         \shipoutAnswer
   \chapter{Electrostatics}
      \section{Problem solutions}
         \shipoutAnswer
   \chapter{Magnetostatics}
            % example:
            \input{../gabookI/calculus/biotSavartGreens.tex}
      \section{Problem solutions}
         \shipoutAnswer
   \chapter{Constitutive relations}
      \section{Problem solutions}
         \shipoutAnswer
%   \chapter{Time harmonic fields}
%      \section{Frequency domain}
%         \input{../frequencydomain/frequencydomainMaxwells.tex}
%      \section{Plane waves}
%         %
% Copyright © 2016 Peeter Joot.  All Rights Reserved.
% Licenced as described in the file LICENSE under the root directory of this GIT repository.
%
%\section{Plane waves}

The gradient action on the electromagnetic field is

\begin{dmath}\label{eqn:frequencydomainCore:160}
\spacegrad F_0 e^{-j \Bk \cdot \Bx}
=
\sum_{m = 1}^3 \Be_m \partial_m
F_0 e^{-j \Bk \cdot \Bx}
=
\sum_{m = 1}^3 \Be_m
F_0
\lr{ -j k_m }
e^{-j \Bk \cdot \Bx}
=
-j \Bk F_0,
\end{dmath}

so

\begin{dmath}\label{eqn:frequencydomainCore:180}
j k (1 - \kcap) F_0 = 0.
\end{dmath}

This means that the field must be of the form

%\begin{dmath}\label{eqn:frequencydomainCore:200}
\boxedEquation
{eqn:frequencydomainCore:200}
{
F = (1 + \kcap) \BE_0 e^{-j \Bk \cdot \Bx},
}
%\end{dmath}

where \( \BE_0 \) is a vector valued complex constant, and \( \kcap \cdot \BE_0 = 0 \).  The dot product constraint follows from the requirement that the \( I \BH \propto \kcap \BE_0 \) portion of the electromagnetic field is a bivector.

From \cref{eqn:frequencydomainCore:200} the interdependence of the electric and magnetic field portions of the field can be read off immediately.  Those are

\begin{subequations}
\label{eqn:frequencydomainCore:220}
\begin{dmath}\label{eqn:frequencydomainCore:221}
\BE = \BE_0 e^{-j \Bk \cdot \Bx} 
\end{dmath}
\begin{dmath}\label{eqn:frequencydomainCore:222}
I \BH = \inv{\eta} \kcap \BE_0 e^{-j \Bk \cdot \Bx},
\end{dmath}
\end{subequations}

or
\begin{dmath}\label{eqn:frequencydomainCore:380}
I \BH = \inv{\eta} \kcap \BE.
\end{dmath}

Since the \R{3} pseudoscalar can be written as

\begin{dmath}\label{eqn:frequencydomainCore:400}
I = \kcap \Ecap \Hcap,
\end{dmath}

the directions \( \kcap, \Ecap, \Hcap \) must form a right handed triple.  It is thus expected that the magnetic field is perpendicular to the propagation direction, and that the electric and magnetic fields are explicitly perpendicular, facts that are easily verified

\begin{subequations}
\label{eqn:frequencydomainCore:440}
\begin{dmath}\label{eqn:frequencydomainCore:260}
\kcap \cdot \BH
= \gpgradezero{ \kcap (-I \kcap \BE_0) } e^{-j \Bk \cdot \Bx}
= -\gpgradezero{ I \BE_0 } e^{-j \Bk \cdot \Bx}
= 0
\end{dmath}
\begin{dmath}\label{eqn:frequencydomainCore:280}
\BE \cdot \BH
=
\gpgradezero{ \BE \lr{ -\frac{I}{\eta}} \kcap \BE }
=
-\inv{\eta} \BE^2
\gpgradezero{ \kcap I }
=
0.
\end{dmath}
\end{subequations}

In conventional vector treatments of electromagnetic field theory the field relationships of \cref{eqn:frequencydomainCore:220} and the propagation directions are written out explicitly as cross products, instead of multivector equations.  Those cross product relations are obtained easily

\begin{subequations}
\label{eqn:frequencydomainCore:420}
\begin{dmath}\label{eqn:frequencydomainCore:240}
\BH
= -I \inv{\eta} \kcap \BE
= -I \inv{\eta} (\kcap \wedge \BE)
= -I \inv{\eta} I (\kcap \cross \BE)
= \inv{\eta} \kcap \cross \BE
\end{dmath}
\begin{dmath}\label{eqn:frequencydomainCore:300}
\BE
= \eta \kcap I \BH
= \eta I \kcap \wedge \BH
= \eta I^2 \kcap \cross \BH
= \eta \BH \cross \kcap
\end{dmath}
\begin{dmath}\label{eqn:frequencydomainCore:340}
\kcap
= I \Hcap \Ecap
= I (\Hcap \wedge \Ecap)
= I^2 (\Hcap \cross \Ecap)
= \Ecap \cross \Hcap.
\end{dmath}
\end{subequations}

%      \section{Problem solutions}
%         \shipoutAnswer
   \chapter{Polarization}
      %
% Copyright © 2017 Peeter Joot.  All Rights Reserved.
% Licenced as described in the file LICENSE under the root directory of this GIT repository.
%
\paragraph{Real Phasor representation.}

A real time dependent field, represented in terms of a complex vector valued phasor \( \tilde{\BA} \), is formed by taking the real part of the product of that phasor with the phase exponential

\begin{dmath}\label{eqn:ellipticalWaves:20}
\bcA
= \Real\lr{ \tilde{\BA} e^{j \Bk \cdot \Bx -j \omega t} }
=
\BA_r \cos\lr{ \Bk \cdot \Bx - \omega t }
- \BA_i \sin\lr{ \Bk \cdot \Bx - \omega t }.
\end{dmath}

In the complex representation above, the imaginary \( j \) is not interpreted geometrically, but like the unit pseudoscalar \( I = \Be_1 \Be_2 \Be_3 \), squares to \( -1 \) and commutes with all grades.  It is therefore possible to express the field using the pseudoscalar as the imaginary

\begin{dmath}\label{eqn:ellipticalWaves:40}
\bcA
=
\inv{2} \BA_r
\lr{ e^{I \phi} + e^{-I\phi} }
   - \inv{2 I} \BA_i
   \lr{ e^{I \phi} - e^{-I\phi} }
=
\inv{2}\lr{ \BA_r + I \BA_i } e^{I \phi}
+
\inv{2}\lr{ \BA_r - I \BA_i } e^{-I \phi}
=
\inv{2} \lr{ \BA e^{I \phi} + \lr{ \BA e^{I \phi} }^\dagger }
,
\end{dmath}

where the phase angle was written as

\begin{dmath}\label{eqn:ellipticalWaves:400}
\phi = \Bk \cdot \Bx - \omega t,
\end{dmath}

and where the field magnitude and orientation has been specified by a ``complex'' (grade-1,3) multivector

\begin{dmath}\label{eqn:ellipticalWaves:120}
\BA = \BA_r + I \BA_i,
\end{dmath}

and its reverse \( \BA^\dagger \).
This has the structure of a real-part operation, where the real part is represented by half the multivector plus its reverse.  This is in fact one way of expressing the vector grade selection operation for the grade-1,3 multivector \( \BA e^{I\phi} \), which can also be considered a phasor representation

\begin{dmath}\label{eqn:ellipticalWaves:80}
\BA e^{I \phi}
=
\lr{ \BA_r + I \BA_i }
\lr{ \cos\phi + I \sin\phi }
=
\lr{ \BA_r + I \BA_i }
\lr{ \cos\phi + I \sin\phi }
=
\BA_r \cos\phi - \BA_i \sin\phi
+ I \BA_i \cos\phi + I \BA_r \sin\phi.
\end{dmath}

Adding this to its reverse (which negates the sign of the pseudoscalar, but not the vector), eliminates all the bivector components of this multivector phasor representation.  It is now possible to represent the field completely in terms of real vectors and a vector grade selection operation

\begin{dmath}\label{eqn:ellipticalWaves:100}
\bcA = \gpgradeone{ \BA e^{I \phi } }.
\end{dmath}

\paragraph{Electromagnetic plane wave.}

The electromagnetic field, with \( \BE = \BE_r + j \BE_i \), where \( \BE_r \cdot \kcap = \BE_i \cdot \kcap = 0 \), for a plane wave is

\begin{dmath}\label{eqn:ellipticalWaves:140}
   F = \Real \lr{ \lr{ 1 + \kcap } \BE e^{j \phi} }.
\end{dmath}

(I have a derivation of this elsewhere, but there is also one in \citep{doran2003gap})

The real representation, with multivector phasor \( \BE = \BE_r + I \BE_i \), is

\begin{dmath}\label{eqn:ellipticalWaves:160}
F
=
\lr{ 1 + \kcap } \gpgradeone{ \BE e^{I \phi } }
=
\inv{2} \lr{ 1 + \kcap } \lr{ \BE e^{I \phi } + \BE^\dagger e^{-I \phi } }.
\end{dmath}

Note that this is not equal to \(
\inv{2} \lr{
   \lr{ 1 + \kcap } \BE e^{I \phi }
   +
\lr{ \lr{ 1 + \kcap } \BE e^{I \phi } }^\dagger } \), since \( \lr{ \kcap \BE }^\dagger = -\kcap \BE^\dagger \).
% because \( \kcap \) is normal to both the \( \BE_r \) and \( \BE_i \) vectors.

Should the electric and magnetic fields be desired explicitly, they can be obtained by the grade selection, with

\begin{equation}\label{eqn:ellipticalWaves:220}
F = \bcE + I \eta \bcH
=
\gpgradeone{ \BE e^{I \phi } } +
\gpgradetwo{ \kcap \BE e^{I \phi } },
\end{equation}

where this split into electric (vector) and magnetic (bivector) field components was facilitated by
the fact that
\( \kcap \gpgradeone{ \BE e^{I \phi } } = \gpgradetwo{ \kcap \BE e^{I \phi } } \) [exercise].

\paragraph{Circular waves}

The use of the 3D pseudoscalar above to express the sine and cosines was arbitrary, and isn't the only option.  Another obvious choice is the pseudoscalar for the plane normal to the propagation direction.  One such unit pseudoscalar is \( i = I \kcap \), for which \( i^\dagger = -i \), and \( i^2 = -1 \) (as was also the case with the 3D pseudoscalar).
With \( \BE = \BE_r + i \BE_i \), the electromagnetic field can be represented as

\begin{dmath}\label{eqn:ellipticalWaves:240}
F
=
\lr{ 1 + \kcap } \gpgradeone{ \BE e^{i \phi } }.
\end{dmath}

Observe that for this choice of pseudoscalar, the grade selection is a no-op, so the electromagnetic field is real, and is just

\begin{dmath}\label{eqn:ellipticalWaves:260}
F
=
\lr{ 1 + \kcap } \BE e^{i \phi }.
\end{dmath}

For example, with \( \BE = E_0 \Be_1 \), and \( \kcap = \Be_3 \), \( i = \kcap I = \Be_1 \Be_2 \), this is

\begin{dmath}\label{eqn:ellipticalWaves:280}
F
=
\lr{ 1 + \kcap } \BE e^{i \phi }
=
E \lr{ 1 + \Be_3 } \Be_1 \lr{ \cos\phi + \Be_1 \Be_2 \sin\phi }
=
E \lr{ 1 + \Be_3 } \lr{ \Be_1 \cos\phi - \Be_2 \sin\phi }.
\end{dmath}

\paragraph{Linear polarized waves.}

The example above was of a circularly polarized state.  The linear polarized plane wave states can be obtained by superposition.  For example, again with \( \kcap = \Be_3, i = \Be_1 \Be_2 \), linear plane electric field configurations with cosine and sine phase follow from

\begin{dmath}\label{eqn:ellipticalWaves:300}
\begin{aligned}
   \inv{2} E_0 \Be_1 \lr{ e^{i \phi} + e^{-i\phi} } &= E_0 \Be_1 \cos\phi \\
   \inv{2} E_0 \Be_1 \lr{ e^{i \phi} - e^{-i\phi} } &= E_0 \Be_2 \sin\phi.
\end{aligned}
\end{dmath}

\paragraph{Elliptically polarized waves.}

While a circle can be parameterized as

\begin{dmath}\label{eqn:ellipticalWaves:320}
\Br(\phi)
=
r \Be_1 e^{i \phi}
=
\Be_1 \lr{ \cos\phi + i \sin\phi }
=
\Be_1 \cos\phi + \Be_2 \sin\phi,
\end{dmath}

an ellipse can be parameterized as
\begin{dmath}\label{eqn:ellipticalWaves:340}
\Br(\phi)
=
a \Be_1 \cos\phi + b \Be_2 \sin\phi.
\end{dmath}

If \( a, b \) are the semi-major/minor axes of the ellipse (i.e. \( a > b \)),
and \( \Ba = a \Be_1 e^{i\alpha} \) is the vectorial representation of the semimajor axis (not necessarily placed along \( \Be_1 \)),
and \( e = \sqrt{1 - (b/a)^2} \) is the eccentricity of the ellipse,
then an elliptic parameterization can be written
\citep{hestenes1999nfc}
in the compact form

\begin{dmath}\label{eqn:ellipticalWaves:360}
\Br(\phi)
=
e \Ba \cosh( \Atanh(b/a) + i \phi).
\end{dmath}

This is also real and has only vector grades, so the electromagnetic field for a general elliptic wave has the form

\begin{dmath}\label{eqn:ellipticalWaves:380}
F
=
e \lr{ 1 + \kcap } \BE_a
\cosh\lr{ \Atanh\lr{ \frac{\Norm{\BE_b}}{\Norm{\BE_a}}} + I \kcap \phi},
\end{dmath}

where \( \BE_a(\BE_b) \) are the electric field components lying along the semi-major(minor) axes, and the propagation direction \( \kcap \) is normal to both \( \BE_a \) and \(\BE_b\).

\paragraph{Problems.}

\makeproblem{}{problem:ellipticalWaves:1}{
Given \( \BE = \BE_r + I \BE_i \), and \( \kcap \cdot \BE_r = \kcap \cdot \BE_i = 0 \), show that
\( \kcap \gpgradeone{ \BE e^{I \phi } } = \gpgradetwo{ \kcap \BE e^{I \phi } } \).
Also show that \( \gpgradetwo{ \kcap \BE e^{I \phi } } \) can be expanded as an antisymmetric sum of the multivector \( \kcap \BE e^{I\phi} \) and its reverse.
} % problem

\makeanswer{problem:ellipticalWaves:1}{
\begin{dmath}\label{eqn:ellipticalWaves:180}
\gpgradetwo{ \kcap \BE e^{I \phi } }
=
\gpgradetwo{ \kcap \lr{ \BE_r + I \BE_i} e^{I \phi } }
=
\gpgradetwo{ \kcap \lr{ \BE_r \cos\phi - \BE_i \sin\phi + I \BE_i \cos\phi + I \BE_r \sin\phi } }
=
\kcap \wedge \BE_r \cos\phi - \kcap \wedge \BE_i \sin\phi
=
\kcap \BE_r \cos\phi - \kcap \BE_i \sin\phi
=
\kcap \lr{ \BE_r \cos\phi - \kcap \BE_i \sin\phi }
=
\kcap \gpgradeone{ \BE e^{I\phi} }.
\end{dmath}

For the second part, we have

\begin{dmath}\label{eqn:ellipticalWaves:200}
\inv{2} \lr{ \kcap \BE e^{I \phi } - \lr{ \kcap \BE e^{I \phi } }^\dagger }
=
\inv{2} \lr{ \kcap \BE e^{I \phi } - e^{-I \phi } \BE^\dagger \kcap }
=
\inv{2} \lr{ \kcap \BE e^{I \phi } + e^{-I \phi } \kcap \BE^\dagger }
=
\frac{\kcap}{2} \lr{ \BE e^{I \phi } + \BE^\dagger e^{-I \phi } }
=
\frac{\kcap}{2} \lr{ \lr{\BE_r + I \BE_i} \lr{ \cos\phi + I \sin\phi } + \lr{\BE_r - I \BE_i} \lr{ \cos\phi - I \sin\phi } }
=
\frac{\kcap}{2} \lr{
   \BE_r \cos\phi - \BE_i \sin\phi + I \lr{ \BE_i \cos\phi + \BE_r \sin\phi }
+  \BE_r \cos\phi - \BE_i \sin\phi - I \lr{ \BE_i \cos\phi + \BE_r \sin\phi }
}
=
\kcap \lr{ \BE_r \cos\phi - \BE_i \sin\phi }
=
\kcap \gpgradeone{ \BE e^{I \phi} }
=
\gpgradetwo{ \kcap \BE e^{I \phi} }
.
\end{dmath}

} % answer

      \section{Problem solutions}
         \shipoutAnswer
   \chapter{Green's functions}
      \section{Problem solutions}
         \shipoutAnswer
   \chapter{Wave equations}
      \section{Problem solutions}
         \shipoutAnswer
   \chapter{Radiation and scattering}
      \section{Problem solutions}
         \shipoutAnswer
%</TEMPORARY COMMENT OUT FOR QUICKER BUILD.>
