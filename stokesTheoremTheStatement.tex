%
% Copyright © 2013 Peeter Joot.  All Rights Reserved.
% Licenced as described in the file LICENSE under the root directory of this GIT repository.
%
An important consequence of the fundamental theorem of geometric calculus is the
geometric algebra generalization of Stokes' theorem.  This form of Stokes' theorem is equivalent to the same from the theory of differential forms.
Stokes' theorem in differential forms and geometric algebra is more general and powerful than Stokes' theorem from conventional vector calculus which only relates
surface integrals to the line integral around the bounding surface.

\maketheorem{Stokes' Theorem}{thm:stokesTheoremGeometricAlgebra:1740}{
Given a \(k\) volume element \(d^k \Bx \) and an s-blade \( F, s < k \)
\begin{equation*}%\label{eqn:stokesTheoremTheStatement:120}
\int_V d^k \Bx \cdot (\boldpartial \wedge F) = \int_{\partial V} d^{k-1} \Bx \cdot F.
\end{equation*}
%Here the volume integral is over a \(k\) dimensional hypervolumesurface (manifold).  The derivative operator \(\boldpartial\) is called the vector derviative and is the projection of the gradient onto the tangent space of the manifold.  Integration over the boundary of \(V\) is indicated by \( \partial V \).
}

We will see that most of the well known scalar and vector integral theorems are consequences of \cref{thm:stokesTheoremGeometricAlgebra:1740}.

To prove the theorem, set \( F = 1 \) in \cref{thm:fundamentalTheoremOfCalculus:1}, and require that \( G \) is an s-blade, with grade \( s < k \).  We select the \( k-(s+1) \) grade, the lowest grade of \( d^k \Bx (\boldpartial \wedge G) \) from
both sides of
\cref{thm:fundamentalTheoremOfCalculus:1}.

For the grade selection of the hypervolume integral we have
\begin{dmath}\label{eqn:stokesTheoremTheStatement:100}
\gpgrade{ \int_V d^k \Bx \boldpartial G }{k-(s+1)}
=
\gpgrade{
\int_V d^k \Bx (\boldpartial \cdot G )
+
\int_V d^k \Bx (\boldpartial \wedge G )
}{k-(s-1)},
\end{dmath}
however, the lowest grade of \( d^k \Bx (\boldpartial \cdot G ) \) is \( k -(s-1) = k - s + 1 > k - (s+1) \), so the divergence integral is zero.  This leaves
\begin{dmath}\label{eqn:stokesTheoremTheStatement:110}
\int_V d^k \Bx \cdot (\boldpartial \wedge G )
= \int_{\partial V} \gpgrade{d^{k-1} \Bx G}{k-(s+1)}
= \int_{\partial V} d^{k-1} \Bx \cdot G,
\end{dmath}
proving the theorem.

%%%\paragraph{FIXME: (rewrite) old proof using gagc.}
%%%The vector derivative is defined by
%%%\begin{equation}\label{eqn:stokesTheoremTheStatement:1400}
%%%\boldpartial = \Bx^i \partial_i = \sum_i \Bx_i \PD{u^i}{}.
%%%\end{equation}
%%%
%%%where \( \Bx^i \) are reciprocal frame vectors dual to the tangent vector basis \( \Bx_i \) associated with the parameters \( u^1, u^2, \cdots \).
%%%%These will be defined in more detail in the next section.
%%%Once the volume element, vector product and the other concepts are defined, the proof of
%%%Stokes theorem is really just a statement that
%%%\boxedEquation{eqn:stokesTheoremGeometricAlgebra:2840}{
%%%\int_V d^k \Bx \cdot (\Bx^i \partial_i \wedge F) =
%%%\int_V \lr{ d^k \Bx \cdot \Bx^i } \cdot \partial_i F.
%%%}
%%%
%%%This dot product expansion applies to any degree blade and volume element provided the degree of the blade is less than that of the volume element (i.e. \(s < k\)).  That magic follows directly from \cref{thm:stokesTheoremGeometricAlgebra:1420}.
