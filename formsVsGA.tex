%
% Copyright � 2019 Peeter Joot.  All Rights Reserved.
% Licenced as described in the file LICENSE under the root directory of this GIT repository.
%
%{
\input{../latex/blogpost.tex}
\renewcommand{\basename}{formsVsGA}
%\renewcommand{\dirname}{notes/phy1520/}
\renewcommand{\dirname}{notes/ece1228-electromagnetic-theory/}
%\newcommand{\dateintitle}{}
%\newcommand{\keywords}{}

\input{../latex/peeter_prologue_print2.tex}

\usepackage{peeters_layout_exercise}
\usepackage{peeters_braket}
\usepackage{peeters_figures}
\usepackage{siunitx}
\usepackage{verbatim}
%\usepackage{mhchem} % \ce{}
%\usepackage{macros_bm} % \bcM
%\usepackage{macros_qed} % \qedmarker
%\usepackage{txfonts} % \ointclockwise

\beginArtNoToc

\generatetitle{Differential forms vs geometric algebra}
%\chapter{Differential forms vs geometric algebra}

I can partially answer this question.

In particular, it is possible to relate geometric calculus and differential forms by introducing a parameterization.  To consider these relations, consider a vector surface those span is controlled by two parameters
\begin{dmath}\label{eqn:formsVsGA:20}
\Bx = \Bx(a , b).
\end{dmath}
In geometric calculus we introduce differentials that span the tangent plane at the point of evaluation
\begin{dmath}\label{eqn:formsVsGA:40}
\begin{aligned}
   dx_a &= \PD{a}{\Bx}\, da \\
   dx_b &= \PD{b}{\Bx}\, db,
\end{aligned}
\end{dmath}
so the area element for this parameterization is
\begin{dmath}\label{eqn:formsVsGA:60}
\begin{aligned}
   d^2 \Bx &= dx_a \wedge dx_b \\
           &= \PD{a}{\Bx} \PD{b}{\Bx}\, da db.
\end{aligned}
\end{dmath}
To relate this to differential forms, introduce an
orthonormal basis \( \Be_\mu \cdot \Be_\nu = 0, \Be_\mu^2 = \pm 1\).  In this basis, the coordinate expansion (summation implied) of the vector \( \Bx \) is
\begin{dmath}\label{eqn:formsVsGA:80}
   \Bx = \Be_\mu x^\mu.
\end{dmath}
The coordinate expansion of the geometric area element is
\begin{dmath}\label{eqn:formsVsGA:100}
\begin{aligned}
   d^2 \Bx &=
   \PD{a}{x^\mu} \wedge \PD{b}{x^\nu} \Be_\mu \wedge \Be_\nu\, da db \\
           &=
   \sum_{\mu < \nu}
   \lr{
      \PD{a}{x^\mu} \PD{b}{x^\nu} -
      \PD{a}{x^\nu} \PD{b}{x^\mu}
   }
      \Be_\mu \wedge \Be_\nu\, da db \\
           &=
   \sum_{\mu < \nu}
      \Be_\mu \Be_\nu
\begin{vmatrix}
   \PD{a}{x^\mu} & \PD{a}{x^\nu} \\
   \PD{b}{x^\mu} & \PD{b}{x^\nu}
\end{vmatrix}
      \, da db \\
           &=
   \sum_{\mu < \nu}
      \Be_\mu \Be_\nu
      \PD{(a,b)}{(x^\mu, x^\nu)}
      \, da db.
\end{aligned}
\end{dmath}
Each element of this sum includes a product of a pseudoscalar, a Jacobian determinant, and a scalar two parameter differential.

Now consider a two parameter differential for the same vector.  Recall that a differential (1-form) of a scalar function, again assuming two parameters, has the characteristic
\begin{dmath}\label{eqn:formsVsGA:120}
   df  =
   \PD{a}{f} \, da +
   \PD{b}{f} \, db.
\end{dmath}
In particular, we may compute the differentials of the coordinate functions
\begin{dmath}\label{eqn:formsVsGA:140}
\begin{aligned}
   dx^\mu &= \PD{a}{x^\mu} \, da + \PD{b}{x^\mu} \, db \\
   dx^\nu &= \PD{a}{x^\nu} \, da + \PD{b}{x^\nu} \, db,
\end{aligned}
\end{dmath}
from which we can compute a 2-form
\begin{dmath}\label{eqn:formsVsGA:160}
\begin{aligned}
   dx^\mu \wedge dx^\nu
   &= \lr{ \PD{a}{x^\mu} \, da + \PD{b}{x^\mu} \, db } \wedge \lr{ \PD{a}{x^\nu} \, da + \PD{b}{x^\nu} \, db } \\
   &= \PD{a}{x^\mu} \PD{b}{x^\nu} \, da \wedge db + \PD{b}{x^\mu} \PD{a}{x^\nu} \, db \wedge da \\
   &=
\begin{vmatrix}
   \PD{a}{x^\mu} & \PD{a}{x^\nu} \\
   \PD{b}{x^\mu} & \PD{b}{x^\nu}
\end{vmatrix}
   \, da \wedge db \\
   &=
      \PD{(a,b)}{(x^\mu, x^\nu)}
   \, da \wedge db.
\end{aligned}
\end{dmath}
We have almost the same structure as with geometric algebra, however, in differential forms, the antisymmetry of the surface area element is encoded in the 2-form \( da \wedge db \) whereas in geometric calculus the required antisymmetry is encoded in a unit bivector.

Should we restrict our attention to a strictly planar subspace, the mapping between the two formalisms becomes more striking.  We now have
\begin{dmath}\label{eqn:formsVsGA:180}
\begin{aligned}
   d^2 \Bx &= \Be_1 \Be_2 \PD{(a,b)}{(x^1, x^2)} \, da db \\
   dx^1 \wedge dx^2 &= \PD{(a,b)}{(x^1, x^2)} \, da \wedge db.
\end{aligned}
\end{dmath}
That is, we can relate the formalisms by the mapping
\begin{dmath}\label{eqn:formsVsGA:200}
   \Be_1 \Be_2 \, da db \leftrightarrow da \wedge db.
\end{dmath}
The \( 1-form \) has something of a vectorial nature, whereas the 2-form has a bivector nature.

It kind of looks like the \( \sqrt{\pm \begin{vmatrix} g \end{vmatrix} } \) term is probably related to the Jacobian determinant for the chosen parameterization, but I'll let somebody else elaborate on that.

%}
%\EndArticle
\EndNoBibArticle
